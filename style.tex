
% Use the full page with 1 cm margins.

\usepackage[
  paper=letterpaper,
  nofoot,nomarginpar,
  includehead,
  margin=1cm
]{geometry}

% Use TeX Gyre Schola for text and math.

\usepackage[T1]{fontenc}
\usepackage{unicode-math}
\setmainfont{TeX Gyre Schola}
\setmathfont{TeX Gyre Schola Math}
\setsansfont{TeX Gyre Heros}

\usepackage{titlesec}
\usepackage{titletoc}
\usepackage{fancyhdr}
\usepackage{enumitem}
\usepackage{hyperref}
\usepackage{graphicx}
\usepackage{xcolor}
\usepackage{tabularx}
\usepackage{tabulary}
\usepackage{booktabs}
\usepackage{multicol}
\usepackage{microtype}
\usepackage{multirow}
\usepackage{pdflscape}

% Basic spacing.

\setlength{\columnsep}{2em}

\setlength{\parindent}{0em}
\setlength{\parskip}{0.8ex plus 0.1ex}
\newcommand{\numberspace}{\phantom{0}}

\newsavebox{\fitwidthbox}
\newenvironment{fitwidth}[1]{%
    \def\fitwidth{#1}%
    \begin{lrbox}{\fitwidthbox}%
}{%
    \end{lrbox}%
    \resizebox{\fitwidth}{!}{\usebox{\fitwidthbox}}%
}

% Headers

\newcommand{\therunningtitle}{}
\newcommand{\runningtitle}[1]{\renewcommand{\therunningtitle}{#1}}

\pagestyle{fancy}
\setlength{\headsep}{1em}
\fancyhead[L]{\textbf{\therunningtitle}}
\fancyhead[C]{}
\fancyhead[R]{\textbf{Page \thepage}}
\fancyfoot[L,C,R]{}
\setlength{\footskip}{4pt}
\renewcommand{\headruleskip}{0.1em}
\addtolength{\headheight}{\headruleskip}

% Line weights

\setlength{\arrayrulewidth}{0.4pt}
\setlength{\columnseprule}{0.4pt}
\renewcommand{\headrulewidth}{0.4pt}

% Lists

\setlist[itemize]{
  label=--,
  leftmargin=1.5em,
  rightmargin=1.5em,
  nosep,
  parsep=\parskip
}
\setlist[enumerate]{
  leftmargin=1.5em,
  rightmargin=1.5em,
  nosep,
  parsep=\parskip
}

\newcommand{\itemparagraph}[1]{{\bfseries #1}}

% Sections

\setcounter{secnumdepth}{3}

\newcommand{\beforesectionnumber}{}
\newcommand{\aftersectionnumber}{}
\newcommand{\beforesubsectionnumber}{}
\newcommand{\aftersubsectionnumber}{}
\newcommand{\beforesubsubsectionnumber}{}
\newcommand{\aftersubsubsectionnumber}{}

\titleformat{\section}
    [block]
    {\filcenter\large\bfseries}
    {\beforesectionnumber\thesection\aftersectionnumber}
    {\wordsep}
    {\MakeUppercase}
\titlespacing{\section}{0pt}{2.4ex plus .1ex minus .1ex}{1.3ex minus .1ex}

\titleformat{\subsection}
    {\filright\large\bfseries}
    {\beforesubsectionnumber\thesubsection\aftersubsectionnumber}
    {\wordsep}
    {}
\titlespacing{\subsection}{0pt}{1.8ex plus .1ex minus .1ex}{0.3ex minus .1ex}

\titleformat{\subsubsection}
    {\filright\bfseries}
    {\beforesubsubsectionnumber\thesubsubsection\aftersubsubsectionnumber}
    {\wordsep}
    {}
\titlespacing{\subsubsection}{0pt}{1.2ex plus .1ex minus .1ex}{0.3ex minus .1ex}

\newcommand{\afterparagraphtitle}{}
\titleformat{\paragraph}[runin]{\bfseries}{}{0.5em}{}[\afterparagraphtitle]
\titlespacing{\paragraph}{0em}{0em}{\wordsep}

\newcommand{\advancedrules}{
    \vspace{1ex}
    \pagebreak[1]
    \begin{center}
        \bfseries{\large Advanced Rules}
    \end{center}
}

\newcommand{\trainingnote}[1]{
  \vspace{1ex}
  \begin{center}
    \framebox{\begin{minipage}{0.9\linewidth} #1\end{minipage}}
  \end{center}
  \vspace{1ex}
}

% Style for the standard title page

\let\originaltitle\title
\renewcommand{\title}[1]{\originaltitle{\bfseries\Large\MakeUppercase{#1}}}
\let\originalauthor\author
\renewcommand{\author}[1]{\originalauthor{\itshape\normalsize{#1}}}
\let\originaldate\date
\renewcommand{\date}[1]{\originaldate{\normalsize{#1}}}

% Abbreviations

\newcommand{\AirSup}{\textit{Air Superiority}}
\newcommand{\AirStr}{\textit{Air Strike}}
\newcommand{\DF}{\textit{Desert Falcons}}
\newcommand{\EOTG}{\textit{Eagles of the Gulf}}
\newcommand{\TSOH}{\textit{The Speed of Heat}}
\newcommand{\AirPow}{\textit{Air Power}}
\newcommand{\APJ}{\textit{Air Power Journal}}

\newcommand{\FP}  {\text{FP}}
\newcommand{\FPs} {\text{FPs}}
\newcommand{\HFP} {\text{HFP}}
\newcommand{\HFPs}{\text{HFPs}}
\newcommand{\VFP} {\text{VFP}}
\newcommand{\VFPs}{\text{VFPs}}
\newcommand{\CC}  {\text{CC}}

\newcommand{\onethird}{$1/3$}
\newcommand{\onehalf}{$1/2$}
\newcommand{\twothirds}{$2/3$}

\renewcommand{\deg}{\ensuremath{^\circ}}


%%%%%%%%%%%%%%%%%%%%%%%%%%%%%%%%%%%%%%%%

% Aids.

% We use this flag to change formatting according to whether we are in
% the main text or the aids.

\newif\ifaids
\aidsfalse

%%%%%%%%%%%%%%%%%%%%%%%%%%%%%%%%%%%%%%%%

% Floats

\newenvironment{onecolumntable}[1][tp]{
    \begin{table}[#1]
    \ifaids
        \centering
        \begin{minipage}{0.5\linewidth}
    \fi
    \centering
    
}{
    \ifaids
        \end{minipage}
    \fi
    \end{table}
}

\newenvironment{twocolumntable}[1][tp]{
    \begin{table*}[#1]
    \centering
}{
    \end{table*}
}

\newenvironment{fullwidthtable}[1][tp]{
    \begin{table*}[#1]
    \centering
}{
    \end{table*}
}

\newenvironment{onecolumnfigure}[1][tp]{
    \begin{figure}[#1]
    \ifaids
        \centering
        \begin{minipage}{0.5\linewidth}
    \fi
    \centering
}{
    \ifaids
        \end{minipage}
    \fi
    \end{figure}
}

\newenvironment{twocolumnfigure}[1][tp]{
    \begin{figure*}[#1]
    \centering
}{
    \end{figure*}
}

% Align floats in their own page or column to the top.
\makeatletter
\setlength{\@fptop}{0pt}
\setlength{\@fpbot}{0pt plus 1fil}
\makeatother

%%%%%%%%%%%%%%%%%%%%%%%%%%%%%%%%%%%%%%%%

% Table and figure captions. 

% Some tables appear in both the main text
% and the play aids. Others only in the play aids.

\newcommand{\tablecaption}[2]{
    \ifaids
        \par
        {\normalsize Table~\ref{#1}: #2}
        \par
    \else
        \caption{\normalsize #2}
        \label{#1}
    \fi
    \medskip
}

\newcommand{\tablecaptionforaidsonly}[2]{
    \par
    \caption{#2}
    \label{#1}
    \medskip
}
\newcommand{\tablecaptioncontinued}[2]{
    \par
    Table~\ref{#1} (continued): #2
    \par
    \medskip
}

\newcommand{\figurecaption}[2]{
    \ifaids
        \par
        \medskip
        Figure~\ref{#1}: #2
    \else
        \caption{#2}
        \label{#1}
    \fi
}

%%%%%%%%%%%%%%%%%%%%%%%%%%%%%%%%%%%%%%%%

% Tables

\newcommand{\minitable}[3][c]{\begin{tabular}[#1]{@{}#2@{}}#3\end{tabular}}

\newcommand*\vertical{\rotatebox{90}}

% New tabularx columns

\newcolumntype{P}{X}%
\newcolumntype{L}{>{\raggedright\arraybackslash}X}%
\newcolumntype{R}{>{\raggedleft\arraybackslash}X}%
\newcolumntype{C}{>{\centering\arraybackslash}X}%
\newcolumntype{q}[1]{>{\centering\arraybackslash\hspace{0pt}}p{#1}}

\newenvironment{tablenote}[1]{%
    \tabularx{#1}{L}%
    \vspace{\parskip}%
}{%
    \endtabularx%
}

%%%%%%%%%%%%%%%%%%%%%%%%%%%%%%%%%%%%%%%%

% Figures

\usepackage{tikz}
\usetikzlibrary{calc}
\usetikzlibrary{arrows.meta}

\newenvironment{tikzfigure}[1]{%
    \begin{fitwidth}{#1}%
    \begin{tikzpicture}[>={Stealth[length=1.5mm,inset=0pt]}]%
    \scriptsize%
}{%
    \end{tikzpicture}%
    \end{fitwidth}%
}

\newcommand{\drawaircraftcounter}[5]{
    \begin{athex}{#1}{#2}
        \begin{scope}[rotate=#3-45]
            \draw[fill=white] (-0.3,-0.3) rectangle (0.3,0.3);
            \node [anchor=center, transform shape] {
                \includegraphics[angle=45,width=0.6cm]{plan-views/#4.png}
            };
            \node [anchor=south east,rotate=#3-45] at (+0.3,-0.3) {\tiny #5};
        \end{scope}
    \end{athex}
}

\newcommand{\drawdashedcounter}[3]{
    \begin{athex}{#1}{#2}
        \begin{scope}[rotate=#3-45]
            \draw[dashed,fill=white] (-0.3,-0.3) rectangle (0.3,0.3);
        \end{scope}
    \end{athex}
}

\newcommand{\drawarrowcounter}[3]{
    \begin{athex}{#1}{#2}
        \begin{scope}[rotate=#3-45]
            \draw[fill=white] (-0.3,-0.3) rectangle (0.3,0.3);
            \draw[very thick, ->] (-0.2,-0.2) -- (0.2,0.2);
        \end{scope}
    \end{athex}
}


% The number is sqrt(0.75) to three decimal places. It is the distance between column centers of a hex grid in which the difference between row centers is 1.0.
\newcommand{\hexxfactor}{0.866}

\newenvironment{athex}[3][]{
    \begin{scope}[shift={(#2*\hexxfactor,#3)},#1]
}{
    \end{scope}
}
\newcommand{\miniathex}[4][]{\begin{athex}[#1]{#2}{#3}#4\end{athex}}

\newcommand{\drawhex}[3][]{
    \begin{athex}{#2}{#3}
        \begin{scope}[xscale=\hexxfactor]
            \draw [color=black!50,#1] (+0.667,+0.000) -- (+0.333,+0.500) -- (-0.333,+0.500) -- (-0.667,+0.000) -- (-0.333,-0.500) -- (+0.333,-0.500) -- cycle;
        \end{scope}
    \end{athex}
}

\newcommand{\drawdottedhex}[3][]{
    \begin{athex}{#2}{#3}
        \begin{scope}[xscale=\hexxfactor,color=black!25,#1]
            \draw (+0.667,+0.000) -- (+0.333,+0.500) -- (-0.333,+0.500) -- (-0.667,+0.000) -- (-0.333,-0.500) -- (+0.333,-0.500) -- cycle;
        \end{scope}
        \begin{scope}[color=black!25,#1]
            \draw[fill]     (0,0) circle (0.02);
            \draw[fill]  (30:0.5) circle (0.02);
            \draw[fill]  (90:0.5) circle (0.02);
            \draw[fill] (150:0.5) circle (0.02);
            \draw[fill] (210:0.5) circle (0.02);
            \draw[fill] (270:0.5) circle (0.02);
            \draw[fill] (330:0.5) circle (0.02);
        \end{scope}
    \end{athex}
}

\newcommand{\drawpositions}[3][]{
    \begin{athex}{#2}{#3}
        \begin{scope}[color=black!25,#1]
            \draw[fill]     (0,0) circle (0.02);
            \draw[fill]  (30:0.5) circle (0.02);
            \draw[fill]  (90:0.5) circle (0.02);
            \draw[fill] (150:0.5) circle (0.02);
            \draw[fill] (210:0.5) circle (0.02);
            \draw[fill] (270:0.5) circle (0.02);
            \draw[fill] (330:0.5) circle (0.02);
        \end{scope}
    \end{athex}
}

\newcommand{\drawmegahex}[2]{
    \begin{athex}{#1}{#2}
        \begin{scope}[yscale=5,xscale=5*\hexxfactor,ultra thick,color=black!25]
            \draw (+0.667,+0.000) -- (+0.333,+0.500) -- (-0.333,+0.500) -- (-0.667,+0.000) -- (-0.333,-0.500) -- (+0.333,-0.500) -- cycle;
        \end{scope}
    \end{athex}
}

\newcommand{\cliphexgrid}[4]{
    \clip (#1*\hexxfactor-0.667*\hexxfactor,#2-0.5-0.01) rectangle (#3*\hexxfactor+0.667*\hexxfactor,#4+0.5+0.01);
}

\newcommand{\drawhexgrid}[4]{
    \cliphexgrid{#1}{#2}{#3}{#4}
    \foreach \x in {0,2,...,#3} {
        \foreach \y in {0,1,...,#4} {
            \miniathex{0.0}{0.0}{\drawhex{\x}{\y}}
        }
    }
    \foreach \x in {1,3,...,#3} {
        \foreach \y in {0,1,...,#4} {
            \miniathex{0.0}{0.5}{\drawhex{\x}{\y}}
        }
    }
}

\newcommand{\drawpositiongrid}[4]{
    \cliphexgrid{#1}{#2}{#3}{#4}
    \foreach \x in {0,2,...,#3} {
        \foreach \y in {0,1,...,#4} {
            \miniathex{0.0}{0.0}{\drawpositions{\x}{\y}}
            \miniathex{1.0}{0.5}{\drawpositions{\x}{\y}}
        }
    }
}

\newcommand{\drawdotathex}[3][]{
    \miniathex{#2}{#3}{\draw [fill,color=black,#1] circle (0.1);}
}

\newcommand{\drawpositionathex}[3][]{
    \miniathex{#2}{#3}{\draw [fill,color=black,#1] circle (0.02);}
}

\newlength{\standardhexwidth}
\setlength{\standardhexwidth}{8.3mm}
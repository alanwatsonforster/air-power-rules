\section{The Game Turn}

This chapter describes game turns and how they are divided into phases to regulate play.

In {\AirPow}, a scenario is played out in game turns (often just called turns). Each game turn represents 12 seconds of real time.  A game turn is a tool used to regularize the movement and combat actions of the playing pieces. Within each turn, all aircraft and weapons in flight will get to move a scale distance equal to that which could be moved in 12 seconds of real time.

\subsection{Phases Within A Turn}

More than movement takes place in a turn. Air combat is a confusing affair in which opposing pilots are hotly engaged in sorting, tracking and attacking their enemies. This all happens in a continuous, rapid, dynamic and fluid manner. To keep the game manageable, the various actions pilots are concerned with, such as sighting the enemy or using radar etc., have been defined and given specific times during the game turn, called “phases”, in which they will be attended to. Each aircraft in play may participate in each phase dependent on the ability of the aircraft to function in those phases (i.e., an aircraft without a radar, would not participate in the radar phase). The order in which the phases are accomplished is termed the “Sequence of Play” (SOP).

The {\AirPow} SOP is shown below and must be followed exactly in each game turn. However, phases which are not applicable to the scenario or current situation can be skipped over to speed play. Often, the scenarios are simply air combat ones not involving ground units. In those scenarios, the AAA, SAM, and Ground Unit interaction phases can be ignored.

\paragraph{SEQUENCE OF PLAY}

\begin{enumerate}
    \item AAA Interaction Phase.
    \item SAM Interaction Phase.
    \item Stalled Aircraft Phase.
    \item Visual Sighting Phase.
    \item Aircraft Decisions Phase.
    \item Order of Flight Determination Phase.
    \item Flight Phase.
    \item Air To Air Missile Phase.
    \item Air Radar Lock-on Phase.
    \item Ground Unit Interaction Phase.
    \item Aircraft Admln Phase.
    \item End of Turn Admin Phase.
\end{enumerate}  

There are reasons for the specific order of the phases. Visual sighting comes before the flight and missile launch phases because the rules require targets to be sighted before attacks can be made on them. The sighting phase also comes before the order of flight phase since determining which aircraft moves first is largely dependent on who sees who.

\paragraph{EXPANDED SEQUENCE OF PLAY.} In the play aids to these rules, there is an expanded SOP chart which fully details each of the action’s players will be concerned with in each phase of the game-turn.
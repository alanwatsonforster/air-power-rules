\section{Air to Air Gun and Rocket Combat}

This chapter details how to conduct gun and rockets attacks against other aircraft.

\subsection{Air to Air Gunnery}

An aircraft equipped with guns may conduct one or two gun attacks per game turn. It may attack at any point during its move, but at least one FP (HFP or VFP) must be expended before each shot. The Internal Gun Data section on the Combat Characteristics panel of the ADC provides basic information about internal guns on an aircraft. If the aircraft has been fitted with a gun pod, the Gun Pod Weapons Table shows the basic information.

\paragraph{Range.} All aircraft guns have a range of two hexes unless noted otherwise on the ADC. The Aircraft Gun Range Diagram shows the field of fire for fixed or pod mounted guns. When firing at a target at a different altitude, each two full levels of altitude difference equals one hex of range.

\paragraph{Gun Attack Procedure.} The Roll To Hit entry (in the Internal Gun Data section of the ADC) for the aircraft shows the basic die roll number required to hit the target at ranges 0, 1, and 2. Roll the die, and modify the result as required. Compare the final result to the hit numbers. A hit is achieved if the modified roll is less than or equal to the hit number.

    
\paragraph{Die Roll Modifiers.} The roll to hit is modified by a variety of circumstances. The possible modifiers are summarized here and in the play aid tables:

\begin{itemize}

    \itemparagraph{Target Size.} The Size number on the ADC is used directly as a die roll modifier.

    \itemparagraph{Snap Shot.} If the attack was a Snap Shot, apply a modifier of $+1$.

    \itemparagraph{Deflection (Angle-Off).} The best shots occur when an attacker is directly behind his opponent; at other angles, the target aircraft is harder to hit. Consult the Angle-Off Table and apply the listed modifiers (see 9.2)

    \itemparagraph{Firer Damage.} See chapter 10 for the effects of aircraft damage on firing aircraft.

    \itemparagraph{Gunsight Effects.} Apply the die roll modifier if the firing aircraft is currently turning or has faced by turning during the game-turn at one of the rates listed. The modifier for the highest turn rate used up to that point or carried over into that game-turn up to the instant of attack must be used. If an ET turn rate was used, attacks are not normally allowed (see recovery period below).

    \itemparagraph{Steady State Gunsight Tracking.} See the advanced SSGT rules below. Apply the appropriate modifier for tracking time on the target.

    \itemparagraph{Radar Ranging.} See the Advanced Ranging rules below. If ranging is successful, apply the appropriate modifier.

    \itemparagraph{Pilot Quality (chapter 18).} Better trained pilots shoot better while poorly trained or inexperienced pilots shoot worse. Apply modifiers for pilot quality/characteristics.

\end{itemize}

\paragraph{Restrictions On Gun Attacks.} Gun attacks are restricted as follows:

\begin{itemize}
    \item An aircraft may not fire at unspotted aircraft.
    \item A climbing aircraft may not fire at an aircraft at a lower altitude.
    \item A diving aircraft may not fire at an aircraft at a higher altitude.
    \item An aircraft flying level may fire at a target in the same hex only if it is at the same altitude.
    \item An aircraft may fire at a target in another hex only if it is at the same altitude level or at an adjacent altitude level.
    \item An aircraft may not fire while in, or just after having faced from, an ET turn (see recovery period).
    \item Aircraft performing rolling maneuvers may not fire until they expend an FP doing something other than prep-moving for a roll or executing a roll.
\end{itemize}

\paragraph{Recovery Period Exception.} Aircraft may fire after using ET turns and/or need only apply turn rate modifiers for turning done after a recovery period has elapsed. The “recovery period” is completed if an aircraft has expended at least half its FPs (round down) while turning at a rate less than ET and/or while wings level and not turning, not maneuvering or not prep-moving for maneuvers prior to firing. The recovery period represents the time it takes for the gunsight or pilot to recover from the effects of high G forces.

\paragraph{Snap Shots.} A snap shot is a short gun burst. It uses half the normal ammunition, but has a lower chance of hitting and a lower damage rating. If multiple guns fire and a Snap Shot is used, all guns fire Snap Shots.

\paragraph{Head On Gun Attacks.} If both the attacker and target directly face each other with the attacker on the 180 degree line shown in the Angle-Off Diagram, the attack is a Head-on Attack. A target aircraft waiting to move (or having already finished its move) may return fire in response to head-on attacks provided it does not exceed the 2 shot per game turn limit, the higher/lower target restrictions, and the ET turn rate prohibition.

\paragraph{Ammunition.} The Internal Gun Data section shows the number of shots allowed for the aircraft gun. Each shot represents 2 seconds of firing and expends 1 ammo point. A snap shot (a 1 second burst) may be taken instead and uses up a half point of ammo.

\paragraph{Gun Pods.} A gun pod places aircraft machine guns or cannon in an external, detachable container. If the Station Limits section of the ADC permits, an aircraft may carry a gun pod as part of its external load. The External Stores Table shows the types of gun pods available. An aircraft may carry more than one gun pod; if it does, they must be of the same type. If loaded on wing stations, they must be carried symmetrically in pairs; each pod must be on the same weapon station as the pod on the opposite wing.

\paragraph{Multiple Guns Firing.} If the aircraft is firing both internal guns and pods or more than one pod, a single die roll is used for each attack. The roll is compared to all the to hit numbers of each of the guns firing. Of those that hit, the highest damage rating available is used, and it is increased by $+1$ for each additional gun of those fired that hit. 

\subsection{Angle-Off}

\paragraph{Concept.} Angle-off (another term for deflection) is the angle “off” the target's tail. It is used to define eligibility for attack modifiers and other functions. The Angle-Off Diagrams show the various angles of approach to an aircraft. Two diagrams are used: one for target aircraft in a hex; the other for target aircraft on a hex side. In each case, the target aircraft defines the Line of Flight. The Line of Flight extending ahead of the target aircraft is the 180° Line; the Line of Flight extending behind the target aircraft is the 0° Line.

For example, an aircraft directly behind the target aircraft and facing in the same direction has a zero-deflection shot (0° angle-off). An aircraft making a head-on attack is facing in the opposite direction to the target (180° of angle-off).

\paragraph{Angle-Off Arcs.} Angle-off is described in 30° arcs to the left or right of the target per the hex grid. The Angle Off Diagrams illustrate the arcs relative to the target aircraft. An attacker will be clearly in an arc or directly on one of the lines defining the border of two arcs. If it is on one of the borderlines, it is in the arc it would fall into if the faster of the target or attacker were moved forward one hex. If the attacker would remain on the line (if moved forward), it is in the arc that benefits the attacker. In the case of same hex (range 0/same altitude) attacks, the aircraft is in the angle off arc that equates to its heading difference from the target at the instant it fires. For diving and climbing same hex attacks against lower or higher targets respectively, both angle-off and a $+2$ vertical attack modifier apply.

\paragraph{Other Angle-Off Arc Functions.} The Angle-Off Diagrams in the play aids are also used to define radar and visual spotting arcs, aircraft restricted and blind sighting arcs, jamming arcs, and for determining missile attack modifiers.

The following illustrations show examples of angle-off arc determinations:

\begin{figure}
    \centering
    \includegraphics[width=0.9\linewidth]{figures/figure-I.pdf}
\end{figure}

\begin{figure}
    \centering
    \includegraphics[width=0.9\linewidth]{figures/figure-J.pdf}
\end{figure}

\begin{figure}
    \centering
    \includegraphics[width=0.9\linewidth]{figures/figure-K.pdf}
\end{figure}

\advancedrules

\subsection{Air to Air Rocketry}

Before the advent of guided missiles, aircraft designers worked to provide Interceptors with a weapon that out ranged the defensive guns of long ranged bombers and that had sufficient power to knock the bomber down. The solution most came up with was to utilize clusters of unguided rockets fired from retractable packs or pods attached to the wings. The rockets were to be fired in shotgun like blasts. The concept was never tested in battle and it never proved satisfactory in practice. Intercept geometry was difficult to attain and the rockets proved to be inaccurate. As soon as guided missiles became available, air to air rocketry faded from the scene. Nevertheless, they were, for a short time at least, the primary anti-bomber weapon of the early 1950's Cold War period.

\paragraph{Air To Air Rocket Factors.} The ADC shows if an aircraft can carry rockets (and if so, how many). Each factor represents 10 to 15 rockets fired in volley.

\paragraph{Range.} To declare a rocket attack, the firing aircraft must have a target in its limited radar arc and be within four hexes (count each two full altitude levels as one additional hex of range). Rocket attacks may not be done at range zero.

Rocket attacks are distinct from air to air gun attacks. Rockets or guns may be fired, but not both in one game turn. Only one rocket attack is allowed per game turn; it may be fired at any point in the aircraft's move.

\paragraph{Procedure.} The attacking aircraft declares its target and indicates the number of rocket factors being fired. The Air To Air Rocketry Table is consulted. At the intersection of the range column and the rocket factors row is the base die roll to hit. Roll the ten-sided die and modify the result as appropriate; if the result is equal to or less than the base die roll to hit, the attack has produced a hit. The rocket factors entry also shows the Attack Rating for that number of rockets being fired. This Attack Rating is used on the Damage Tables if a hit occurs.

\paragraph{Rocket Attack Modifiers.} The roll to hit is modified as for air to air gunnery, however only the following modifiers apply:

\begin{itemize}
    \item Target Size.
    \item Deflection (Angle-Off).
    \item Gunsight Effects.
    \item SSGT.
    \item Radar Ranging.
    \item Collision Course Attack (CCA) Technology
\end{itemize}

\paragraph{Collision Course Attack (CCA) Technology.} Aircraft designers fitted some American and Canadian fighters with auto-pilot guidance systems which utilized an early computer linked to the interceptor's radar. The computer figured rocket ballistics and if the radar was tracking a target, it could guide the fighter to a release point and automatically launch the rockets. This concept freed the fighter from having to use pursuit curves to get in to gunnery parameters and was called “collision course” guidance as the interceptor could now theoretically attack from any angle.

To qualify for the CCA to hit modifier, the attacking aircraft must start a game-turn with an air to air lock-on to the target and must not do TT or greater turns, any maneuvers except slides, or utilize climbs or dives of more than one altitude level up to the point of executing the rocket attack. If it meets this criteria, a $-2$ is applied to the hit roll.

\paragraph{Rocket Damage Modifiers.} Rockets, because of their large warheads, receive $-2$ to the damage table roll just like direct missile hits.

\paragraph{Rocket Attack Restrictions.} Rocket attacks are restricted as follows:

\begin{itemize}

    \item An aircraft may not fire rockets at unspotted aircraft.

    \item A climbing aircraft may not fire rockets at an aircraft at a lower altitude.

    \item A diving aircraft may not fire rockets at an aircraft at a higher altitude.

    \item An aircraft may not fire at a target at zero range.

    \item An aircraft may fire at a target in another hex only if it is at the same altitude level or at an adjacent altitude level.

    \item An aircraft may not fire while in, or just after having faced from, an HT, BT, or ET turn.

    \item An aircraft may not fire while prepping for or executing other than slide maneuvers.

\end{itemize}

\paragraph{Air To Ground Rockets in The Air-To-Air Role.} Aircraft equipped with air to ground rockets or rocket pods may fire them in the air to air role. The conversion is as follows:

\begin{itemize}
    \item Each 10 single RKs are equal to one air-to-air rocket factor.	
    \item One small rocket pod (≤7 rockets) is equal to one air-to-air rocket factor.
    \item One medium or large rocket pod is equal to two air-to-air rocket factors.
\end{itemize}

\paragraph{Air to Air rockets in the Air to Ground Role.} Each point of aerial rockets equals 2 soft attack strength factors and one hard attack strength factor.

\subsection{Additional Gun and Rocket Attack Modifiers}

\paragraph{Steady State Gunsight Tracking (SSGT).} Gunsights are optimized for rear quarter attacks. Any aircraft attacking from the 60° or less angle-off arc may track its target and achieve improved probabilities for hits. The firing aircraft must expend FPs while on a tracking line (see Tracking diagram). For each 1/3 (round down) of the aircraft's speed (in full FPs) expended on a tracking line, modify the Roll To Hit by $-1$. A maximum modifier of $-2$ ls allowed for SSGT; tracking may not begin until the firing aircraft is within six hexes range of the target.

\paragraph{Radar Ranging (RR).} An aircraft with radar ranging uses its radar to compute precise range and modify gunsight position for best hit probabilities. The Radar line of the Internal Gun Data section of the ADC shows the type of RR available (if any). There are three types:

\begin{itemize}

\item RE (Regular). The attacker must be in SSGT in order to get RR benefits.

\item CA (Computer Assisted). The attacker may use CA Radar Ranging with or without SSGT when firing from the 90° angle-off or less.

\item IG (Integrated Gun Ballistics). The attacker may fire from any angle, with or without SSGT.

\end{itemize}

\paragraph{RR Procedure.} Ranging is automatic if the firer already has a radar lock-on to the target and meets the arc/SSGT requirements. Otherwise, once the arc/SSGT requirements are met, roll the die. If the result is less than or equal to the radar lock-on number listed in the radar section of the data card, ranging is successful and the ranging modifier is applied.

Having previous radar contact or lock-ons is not a prerequisite for ranging nor are previous contacts and locks lost when ranging. Radar ranging, once achieved is maintained for any second shots at the same target in that game-turn. If not achieved for the first shot, it may be rolled for again prior to the second shot at the same target. Radar ranging does not carry forward to the next game turn or to different targets in the same game turn.

\subsection{Formation Restrictions on Gun and Rocket Combat}

\paragraph{Close Formations.} The wingmen aircraft in a close formation may not fire cannons or rockets at air to air targets. They are too busy holding formation with the formation leader.

\paragraph{Tactical Formations.} There are no restrictions on wingmen of Tactical formations.

\subsection{Nuclear Rockets (AIR-2 Genie)}

In an attempt to compensate for the general inaccuracy of air to air rocketry, the AIR-2 Genie was developed by the USAF. It was large, unwieldly, but featured a nuclear warhead.

\paragraph{Genie Launch.} To launch a Genie, an aircraft must have a radar lock-on to the target and end its move wings level (not turning or maneuvering). Roll for a successful launch in the Air to Air Missile Launch Phase. A die roll less than or equal to the launch number of the Genie indicates a successful launch, otherwise the Genie fails due to a dud motor or warhead and is removed from play.

\paragraph{Genie Flight.} The Genie is unguided. It does not turn or maneuver. It simply flies forward for its entire movement expending its FPs as HFPs unless the firer climbed or dived on the turn of launch, in which case the Genie must also climb or dive expending the same proportion of FPs as VFPs that the launcher did. Each VFP expended must gain or lose a full two levels of altitude.

\paragraph{Genie Scatter.} The Genie wasn't very accurate, it had a big warhead though. After completing the Genie's flight, and after all aircraft have moved for the turn, the Genie’s position will be shifted randomly by rolling a die twice and consulting the scatter diagram below. The first roll indicates scatter direction. The second roll's result is halved (drop fractions) and that is the number of hexes the Genie is shifted in the direction previously determined. See Genie Scatter diagram in play aids.

\paragraph{Nuclear Attack.} After shifting the Genie, roll the die; on a 10 the warhead is a dud. On anything else, it explodes creating a nuclear blast zone that extends out to a range of six hexes in every direction (count two altitude levels as 1 hex). All aircraft, friendly or enemy, in the same position as an exploding Genie are vaporized along with their crews. Other aircraft elsewhere in the zone are automatically hit. The attack rating is 12 minus 2 for each hex of range from the point of detonation.

For example, an aircraft four hexes away is attacked with a rating of 4 ($12 - (2 \times 4\ \mathrm{hexes}) = 4$).

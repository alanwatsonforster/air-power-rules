%!TEX root = ./rules-working.tex
%LTeX: enabled=false

\rulechapter{Strafing and Air to Ground Rockets}

\section{Strafing}
\label{rule:strafing}

An aircraft may attack a ground target with its guns. This is called strafing. An aircraft may strafe if it is on a line of approach to the target, within strafing range, and in level or diving flight. Aiming is required. The attack is resolved the instant it is announced.

\paragraph{Strafing Range.} Aircraft cannons have a range of 4 hexes; aircraft machine guns have a range of 3 hexes. Two altitude levels of difference between the firing aircraft and target equals one hex of range. A strafing attack may use up either one regular shot, or a snap shot worth of ammo.

\paragraph{Altitude Requirements.} A strafing aircraft in level flight may not be more than one level above the target. A strafing aircraft in diving flight must spend at least one VFP to lose altitude while on the line of approach either before or after firing. An aircraft in climbing flight may not strafe.

\paragraph{Strafing Final Attack Strength.} The FAS of a strafing attack is determined by summing up all the air to ground attack ratings of the internal guns and/or gun pods being fired by the aircraft in the attack. If using a snap shot, divide the result by two (retain fractions). The final sum is the FAS.

\paragraph{Die Roll Modifiers.} A number of factors affect strafing runs. Refer to \changedin{1B}{1B-tables}{the Air to Ground Modifiers Table}{Table \ref{table:air-to-ground-attack-modifiers}} and apply any required modifiers to the attack roll. Note; the bombsight modifier does apply when strafing as does the gunsight modifier if any turns were done prior to strafing and a recovery period has not passed.

\paragraph{Hard Target Effects.} If the strafer's air to ground gun's or pod's attack rating is asterisked once (*), its rating is halved when strafing HARD targets. If asterisked twice (**) the gun or pod has no effect against HARD targets. If the strafer's air to ground gun's or pod's attack rating is underlined, it is extra effective against HARD targets and gets a \minus{1} modifier to its attack roll.

\paragraph{Multiple Strafing Attacks.} Aircraft are allowed to fire twice in a game-turn while in their strafing run provided at least one FP is expended prior to and between each shot. The two shots may be taken at the same target, different targets in the same hex, or at different targets in different hexes. Treat the entire strafing run as the one ground attack allowed for the turn even though up to two separate targets may be attacked.

Aiming always ceases as soon as the aircraft commences firing, however, the aiming modifiers may be retained for both shots and over into the next game-turn for additional shots as long as all the shots are still at the original target. If strafing is shifted to a different target in the same or different hex, aiming must be done anew.

\section{Air to Ground Rocket Attacks}

%!TEX root = ./rules-working.tex
%LTeX: enabled=true
\addedin{1B}{1B-tables}{
\begin{onecolumntablefloat}
\begin{onecolumntable}

\tablecaption{table:air-to-ground-rocket-modifiers}{Air-to-Ground Rocket Modifiers.}
\begin{tabularx}{0.4\linewidth}{LR}
\toprule
Condition&Modifier\\
\midrule
\multicolumn{2}{c}{Slant Range}\\
\midrule
1--4            &\plus{0}\\
5--6            &\plus{1}\\
7               &\plus{2}\\
8               &\plus{3}\\
9--10           &\plus{5}\\
\midrule
In TFF          &\plus{2}\\
\bottomrule
\end{tabularx}
\end{onecolumntable}
\end{onecolumntablefloat}
}


The parameters for making an air to ground rocket attack are the same as those for strafing except that rockets (\changedin{2W}{2W-rpt}{RP or RK class}{RP, RK, or RPT} weapons) may be fired from a range of up to nine hexes. The attack is resolved the instant it is announced. Multiple rocket attacks in a single game turn are not allowed.

\paragraph{Slant Range Modifiers.} Additional modifiers for a rocket attack are based on slant range. Slant range is the horizontal range plus the altitude difference counting each two levels difference as another hex of range.

\paragraph{Die Roll Modifiers.} A number of factors affect rocket attack runs. Refer to \changedin{1B}{1B-tables}{the Air to Ground Modifiers Table}{Tables \ref{table:air-to-ground-attack-modifiers} and \ref{table:air-to-ground-rocket-modifiers}} and apply any required modifiers to the attack roll. Note: the bombsight modifier does apply when rocketing as does the gunsight modifier if any turns were done prior to strafing and a recovery period has not passed.

\paragraph{Altitude Requirements.} Diving rocket attacks use the rocket release point table. Level rocket attacks are allowed if the firer is at the same level, or no more than one altitude level above the target. Aircraft in T-level flight may only fire rockets at targets at the same altitude and must add a +2 modifier to their attack due to the grazing angle.

\paragraph{Rocket Attack Restrictions.} All rockets fired in a single attack must be the same size: \changedin{1C}{1C-apj-23-errata}{small, medium, large, or heavy}{57mm, 78mm, etc}. Rocket pods of different size rockets may not be fired at the same time. Different sized rocket pods may be fired at the same time if all rockets fired are the same size.

Single rockets may not be fired at the same time as rocket pods. Any number of single rockets (of the same size) may be fired in the same attack.

\paragraph{Safe Carriage Speed Limits.} A streamlined rocket pod (indicated by an asterisk on the weapons tables) may be carried safely at speeds up to 8.0. An unstreamlined rocket pod may be carried safely at speeds up to 6.0. Single rockets mounted on an aircraft may be carried at any speed. An aircraft which exceeds the rocket pod's carriage speed causes the pod to become misaligned or unreliable: reduce FAS of all subsequent rocket attacks by 1/2.

\paragraph{Rocket Final Attack Strengths.} For an individual rocket pod or single rocket, use the listed strength. For multiple pods, use the sum of their strengths. For volleys of single rockets, use 2/3s the sum of all rockets fired (round fractions up).

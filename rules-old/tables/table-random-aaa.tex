%!TEX root = ../rules-working.tex
%LTeX: enabled=false
\addedin{1B}{1B-tables}{
\begin{onecolumntablefloat}
\begin{onecolumntable}
\tablecaption{table:random-aaa}{Random AAA Fire.}

\begin{tabularx}{0.9\linewidth}{lL}
\toprule
\multicolumn{2}{c}{Aimed Fire Attempt Roll}\\
\midrule
Range to Target A/C&Die Roll to Fire\\
\midrule
Short           & 8 or less\\
Medium          & 6 or less\\
Long            & 4 or less\\
\bottomrule
\end{tabularx}
\addedin{1C}{1C-apj-23-errata}{
\begin{tablenote}{0.9\linewidth}
Modifiers:
\begin{itemize}
    \item \plus{1} if aircraft already fired at
    \item \minus{2} if aircraft last target to move
\end{itemize}
\end{tablenote}
}

\begin{tabularx}{0.9\linewidth}{C}
\toprule
Plotted Fire Procedure\\
\midrule
\multicolumn{1}{@{}l@{}}{
\begin{tablenote}{0.9\linewidth}
\begin{itemize}
    \item Target A/C is one nearest heavy AAA unit.
    \itemdeletedin{2A}{2A-random-plotted-fire}{Roll one die referencing random plotted fire diagram.}
    \itemaddedin{2A}{2A-random-plotted-fire}{Roll the die once and consult Figure \ref{figure:random-aaa} to determine the direction the plotted fire hex will be from the aircraft. A zero indicates the plotted hex is the aircraft's hex. Otherwise, roll the die again and halve the result dropping fractions to determine how many hexes in that direction the plotted fire hex lies.}
    \item Roll die twice; subtract 2d roll from first for altitude difference in levels from target altitude.
\end{itemize}
\end{tablenote}
}\\[-0.5ex]
\bottomrule
\end{tabularx}

\end{onecolumntable}
\end{onecolumntablefloat}
}

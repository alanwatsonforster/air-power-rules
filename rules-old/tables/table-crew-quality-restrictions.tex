%!TEX root = ../rules-working.tex
%LTeX: enabled=false

\begin{onecolumntablefloat}
\begin{onecolumntable}
\tablecaption{table:crew-quality-restrictions}{Pilot Quality Restrictions}
\begin{tabularx}{\linewidth}{X}
\toprule

\medskip
Pilot Quality Flight Restrictions
\medskip

\begin{itemize}
    \item Green:
        \begin{enumerate}
            \item No ET turns\deletedin{2A}{2A-snap}{, no Snap turning}.
            \item No T-level flight, no Viff maneuvers.
            \item No VTOL flight, no Vert.\ Rev.\ maneuvers.
            \item May not use High Pitch Rate aircraft abilities.
            \item May not engage attacking missiles.
            \item Risks disorientation for rolling maneuvers.
            \item Risks disorientation for Vert.\ climbs/dives.
            \itemdeletedin{1B}{1B-apj-23-errata}{$-2$ die roll modifier for GLOC.}
            \itemaddedin{1B}{1B-apj-34-qa}{$-2$ die roll modifier for GLOC.}
        \end{enumerate}
    \item Novice:
        \begin{enumerate}
            \item No Vertical Reverse maneuvers.
            \item May not use High Pitch Rate aircraft abilities.
            \item Risks disorientation for Vertical rolls.
            \item $-1$ die roll modifier for GLOC.
        \end{enumerate}
\end{itemize}

\medskip
{Pilot Damage Control Restrictions}
\medskip

\begin{itemize}
    \item Green: May do damage control only if in a multi-crew aircraft and other crewmember is Reg.\ or Vet. In this case damage control is as for Novice.
    \item Novice: Must perform damage control for two turns in a row to complete unless in multi-crew aircraft and other crewmember is Reg.\ or Veteran. In this case damage control is done normally.
\end{itemize}
\\
\bottomrule
\end{tabularx}
\end{onecolumntable}
\end{onecolumntablefloat}

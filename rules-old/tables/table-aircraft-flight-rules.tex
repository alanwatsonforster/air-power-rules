%!TEX root = ../rules-working.tex
%LTeX: enabled=false

\begin{twocolumntablefloat}
\begin{twocolumntable}

\newcommand{\heading}[1]{\medskip\par\textbf{\MakeUppercase{#1}}\par\smallskip}
\newcommand{\subheading}[1]{\smallskip\par\textbf{#1}\par\smallskip}

\tablecaption{table:aircraft-flight-rules-summary}{Aircraft Flight Rules Summary}
\footnotesize
\begin{tabularx}{\linewidth}{P}
\toprule
\begin{multicols}{2}

\heading{Accel/Decel}
\begin{enumerate}[nosep]
    \item Each 2.0 accel accumulated = \plus{0.5} speed normally.
    \item Each 1.5 accel = \plus{0.5} speed for Rapid Accel aircraft.
    \item If speed ≥ Mach 1, each 3.0 accel = \plus{0.5} speed for normal aircraft and each 2.0 = \plus{0.5} for Rapid Accel aircraft.
    \item Each 2.0 Decel accumulated = \minus{0.5} speed always.
\end{enumerate}

\heading{Level Flight}
\begin{enumerate}[nosep]
    \item All FPs are HFPs. An aircraft may descend one altitude level freely at any point in its move.
\end{enumerate}

\heading{Turning Flight}
\begin{enumerate}[nosep]
    \item Turn Drag decel based on highest turn rate used in game turn, incur it only once per game turn even if aircraft faced more often than once.
    \item \changedin{2A}{2A-sustained}{Extra facings in a game turn constitute sustained turns. 1.0 decel is incurred for each dacing change after the first.}{The second and subsequent facing change in a game turn constitute sustained turns. 1.0 DP are incurred for each 30 degrees of facing change in sustained turns (0.5 DP for LBR and 1.5 DP for HBR aircraft).}
    \item TT, HT, BT, ET turns require start speed of 0.5, 1.0, 1.5, and 2.0 > minimum respectively to perform.
    \item Low Roll Rate aircraft take 1 FP of flight to enter a left or right bank before turning and 2 FPs of flight to reverse bank.
    \item High Roll Rate aircraft may instantly switch from one angle of bank to another; others require 1 FP of flight to reverse.
    \item No attacks of weapon launches allowed during or after an ET turn until a Recovery Period passes.
    \item[--] A recovery period = half of the aircraft's flight (round up) while not ET turning and not doing rolls or prep-moving for them.
\end{enumerate}

\deletedin{2A}{2A-snap}{
\heading{Snap Turning}
\begin{enumerate}[nosep]
    \item Aircraft must be capable of BT turn rate.
    \item One allowed per game-turn; costs one HFP; allows immediate facing change of 30 degrees or of 60 degrees if turn chart = 60 or 90 without moving forward.
    \item One HFP prep required is wings not level or if speed ≥ to High Transonic. If both cases apply, two preps required.
    \item Incur Decel as for BT turn unless aircraft used ET rate.
    \item Unless ET follows a snap turn; the snap counts as a BT turn for purposes of combat and weapon launch modifiers until a recovery period passes.
    \item Risky Snap turns may be tried if aircraft is capable of HT turn but roll for a departure on facing (1 to 4).    
\end{enumerate}
}

\heading{FP Expenditure Restrictions}
\begin{enumerate}[nosep]
    \item If going from level to climbing or diving flight; the first FP expended must be an HFP.
    \item If going from dive to climb or climb to dive; FPs = to half the aircraft's speed (round down) must be expended as HFPs before using VFPs. High Pitch Rate aircraft need only expend FPs = to {\onethird} speed (round down) in this case.
    \item If continuing to climb or dive from previous turn; HFPs and VFPs may be mixed in any order.
\end{enumerate}

\vfill\null\columnbreak

\heading{Speedbrake Usage}
\changedin{2A}{2A-spbr}{
\begin{enumerate}[nosep]
    \item FPs up to amount listed on the ADC may be eliminated.
    \item Eliminated FPs may not be used for any turns or other maneuver/combat/proportional move requirements.
    \item 1.0 decel is incurred for each 0.5 FP eliminated.
\end{enumerate}
}{
\begin{enumerate}[nosep]
    \item DPs up to the maximum listed on the ADC may be incurred.
    \item If the aircraft is supersonic, the maximum is increased by 1 DP.
\end{enumerate}
}

\heading{Climbing Flight}

\subheading{Zoom Climbs}
\begin{enumerate}[nosep]
    \item At least one, but up to {\twothirds} of FPs may be VFPs.
    \item If CCC rate for power setting ≤ 2.0, then each VFP can gain 1 altitude level only.
    \item If CCC rate for power setting > 2, each VFP can gain 1 or 2 altitude levels.
    \itemaddedin{2A}{2A-super-climbs}{If CCC rate for power setting ≥ 6.0, one of the VFPs can gain 1, 2, or 3 altitude levels.}
    \itemdeletedin{2A}{2A-zoom-climbs}{If this is the first turn of climbing flight, 1.0 decel is incurred per level climbed.}
    \itemdeletedin{2A}{2A-zoom-climbs}{If this is the second or subsequent turn of climbing flight, 1.5 decel is incurred per level climbed.}
    \itemaddedin{2A}{2A-zoom-climbs}{1.0 DP per level climbed.}
    \item ET turn rates not allowed in zoom climbs.
\end{enumerate}

\subheading{Sustained Climbs}
\begin{enumerate}[nosep]
    \item Start speed must be at least 1.0 > minimum speed.
    \item If start speed is less than climb speed, then halve CCC value (retain fractions).
    \item If CCC value is < than 1.0, only one VFP may be used in game-turn and it gains only the fractional altitude level.
    \item If CCC value ≥ 1.0 but ≤ 2.0, up to 2/3s of the FPs may be VFPs. The first VFP gains any listed fraction (or 1 if no fractions listed), and the rest gain one altitude level each.
    \item If CCC value is > 2.0, up to 2/3s of the FPs may be VFPs. The first VFP gains 1.0 level \plus{} any fraction and the rest may gain 1.0 or 2.0 altitude levels each.
    \itemaddedin{2A}{2A-super-climbs}{If CCC value ≥ 6.0, one of the VFPs after the first can gain 1, 2, or 3 altitude levels.}
    \item If enough VFPs exist, an aircraft may climb more levels than listed on the CCC. However, the extra levels climbed cause decel as if zoom climbing.
    \item Only EZ turns and Slide maneuvers allowed.
    \item 0.5 decel is incurred for each level up to the CCC limit. Extra levels incur decel as for zoom climbing\addedin{2A}{2A-zoom-climbs}{\ at 1.0 DP per level climbed.}.
\end{enumerate}

\subheading{Vertical Climbs}
\begin{enumerate}[nosep]
    \item Previous game-turn must have involved climbing flight.
    \item Exception; High Pitch Rate aircraft may enter vertical climbs from level flight if speed < 4.0.
    \item On first turn of vertical climb, {\onethird} of FPs must be HFPs. If vertical climb continued, not more than {\onethird} of FPs may be HFPs and up to all may be VFPs.
    \item Each VFP may gain 1.0 or 2.0 altitude levels each.
    \item Each level climbed causes \changedin{2A}{2A-vertical-climbs}{2.0}{1.5} decel points.
    \item No turns or maneuvers except Vertical Rolls allowed.
    \item Diving flight may not follow Vertical climbs.
    \item Exception, High Pitch Rate aircraft may enter Steep Dives or Unloaded Dives the turn after.
    \item Exception, normal aircraft may use a Half-Roll and Dive maneuver to enter Steep Dives after a Vertical Climb.
\end{enumerate}


\end{multicols}
\\
\bottomrule
\end{tabularx}
\end{twocolumntable}
\end{twocolumntablefloat}

\begin{twocolumntablefloat}
\begin{twocolumntable}

\newcommand{\heading}[1]{\medskip\par\textbf{\MakeUppercase{#1}}\par\smallskip}
\newcommand{\subheading}[1]{\smallskip\par\textbf{#1}\par\smallskip}

\tablecaptioncontinued{table:aircraft-flight-rules-summary}{Aircraft Flight Rules Summary}
\footnotesize
\begin{tabularx}{\linewidth}{P}
\toprule
\begin{multicols}{2}

\heading{Diving Flight}

\subheading{Steep Dives}

\begin{enumerate}[nosep]
    \item At least one FP must be and up to 2/3s FPs may be VFPs.
    \item Each VFP may Lose 1.0 or 2.0 altitude levels.
    \itemdeletedin{2A}{2A-steep-dives}{Each level dived gains 0.5 accel on the first turn of Diving.}
    \itemdeletedin{2A}{2A-steep-dives}{If this is the second or subsequent turn of
    continuous Diving, each level dive gains 1.0 accel.}
    \itemaddedin{2A}{2A-steep-dives}{1.0 AP per level.}
\end{enumerate}

\subheading{Unloaded Dives}

\changedin{XX}{XX-unloaded-dives}{
\begin{enumerate}[nosep]
    \item \changedin{2A}{2A-unloaded-dives}{All FPs are HFPs.}{One or two FPs are VFPs. The rest are HFPs.}
    \item \changedin{2A}{2A-unloaded-dives}{At least 1 HFP must be expended with the aircraft “unloaded”. More than 1 and up to all may be expended “unloaded”.}{The first VFP may only be used after half of the HFPs have been expended with the aircraft unloaded. The second VPF may only be used after all of the HFPs have been expended with the aircraft unloaded.}
    \item \changedin{2A}{2A-unloaded-dives}{Each HFP expended while unloaded moves the aircraft forward one hex/hexside and loses it one altitude level.}{The aircraft loses 1 level on each VFP.}
    \item \changedin{2A}{2A-unloaded-dives}{The aircraft gains accel as if Steep Diving.}{The aircraft gains 1.0 AP per level lost.}
    \item The aircraft may not make any attacks, guide weapons or aim while unloaded.
    \item \changedin{2A}{2A-unloaded-dives}{FPs done while unloaded may not be used for turning or prep-moving.}{Unloaded FPs may not be used for turning flight or for preparing or executing any maneuver except a slide.}
    \item All unloaded \changedin{2A}{2A-unloaded-dives}{HFPs}{FPs} done in a single game-turn must be done in one continuous string.
\end{enumerate}
}{
\begin{enumerate}[nosep]
    \item All FPs are HFPs.
    \item At least half of the HFPs (round down) must be expended with the aircraft “unloaded”. All unloaded HFPs must be consecutive.
    \item The aircraft loses on altitude level after half of the HFPs (round down) have expended unloaded. If all of the HFPS are expended unloaded, the aircraft loses another altitude level after the last HFP.
    \item The aircraft gains 1.0 AP per level lost.
    \item The aircraft may not make any attacks, aim, track targets, launch or guide weapons or use radar while unloaded and until it completes a recovery period.
    \item Unloaded FPs may not be used for turning flight or for preparing or executing any maneuver except a slide.
\end{enumerate}
}

\subheading{Vertical Dives}

\begin{enumerate}[nosep]
    \item Previous game turn must have involved diving flight.
    \item Exceptions: a vertical dive may be entered from level flight using a Half Roll and Dive maneuver. If start speed ≤ 4.0, it may also be entered from a zoom or sustained climb by using a Half Roll and Dive maneuver.
    \item On first turn of vertical diving, 1/3 of FPs must be HFPs. If vertical dive continued, no more than 1/3 of FPs may be HFPs and up to all may be VFPs.
    \item Each VFP must lose 2.0 or 3.0 altitude levels.
    \item Each altitude dived gains 1.0 accel.
    \item No turns or maneuvers except vertical rolls allowed.
    \item Climbing flight may never follow vertical dives.
    \item Level flight may follow if A/C's new start speed is 3.0 or less for High Pitch Rate aircraft, or 2.0 or less for others.
    \item[--] If case 8 does not apply, diving flight must follow vertical dive.
    \item When Steep or Unloaded dives follow a vertical dive; at least half an aircraft's FPs (round down) must be expended as VFPs or Unloaded HFPs; except High Pitch Rate aircraft need only expend 1/3 FPs as VFPs or unloaded HFPs.
\end{enumerate}

\heading{Stalled Flight}

\begin{enumerate}[nosep]
    \item Aircraft does not move or change facing.
    \item Altitude lose = start speed (round 0.5 up) + 1.0; increase loss by 1.0 per additional turn of stalled flight.
    \item Aircraft gains accel as it steep diving and by power.
    \item Aircraft may recover to level or diving flight including immediately entering a vertical dive.
\end{enumerate}

\heading{Departed Flight}

\begin{enumerate}[nosep]
    \item Stay in same hex; randomly change facing left or right.
    \item Roll die to find number of facing changes in that direction.
    \item Altitude loss = start speed (round 0.5 up) + 2.0; increase altitude loss by 2.0 per additional turn of departed flight.
    \item Power has no effect, all accel/decel = 0 whole departed.
    \item Recovery is via recovery roll (\minusafter{6} including modifiers).
    \item Upon recovery aircraft must enter diving flight (vertical dives allowed). High Pitch Rate aircraft may recover to level flight.
    \item Upon recovery, start speed reverts to higher of Minimum speed or speed at which departure occurred.
\end{enumerate}

\heading{Aircraft Maneuvers}

\subheading{Slides}

\begin{enumerate}[nosep]
    \item Expend two HFPs to prep for slide. One HFP to execute.
    \item 1 slide allowed if speed ≤ 9.0, two if speed > 9.0 but at least 4 FPs must be expended between execution of first and start of preps for second.
    \item One slide causes no decel; two slides cause 1.0 decel.
\end{enumerate}

\subheading{Lag/Displacement Rolls}

\begin{enumerate}[nosep]
    \item Expend \changedin{2A}{2A-roll-preparatory-fps}{one HFP}{FPs equal to {\onethird} of speed (round down)} to prep for rolls. One HFP to execute.
    \item Shift in direction of roll (see \changedin{1B}{1B-figures}{diagram}{Figures~\ref{figure:displacement-roll-maneuvers} and \ref{figure:lag-roll-maneuvers}}) and optionally face 30 degrees in direction opposite to roll.
    \item A displacement roll from a hexside shifts the aircraft to a hexside as in a slide and not sideways as depicted for the lag roll. Decel for these rolls varies, see ADC.
\end{enumerate}

\subheading{Vertical Rolls}

\begin{enumerate}[nosep]
    \item Aircraft must be in vertical climb or dive and must have just expended a VFP.
    \item Change facing left or right up to 180 degrees.
    \item Decel cost varies; see ADC.
    \item Multiple vertical rolls allowed in a single game turn but each must occur after separate VFP expenditures.
\end{enumerate}

\subheading{Barrel Rolls}

\begin{enumerate}[nosep]
    \item Executed as 2 or more consecutive Lag/Displacement rolls.
    \item If done in level flight, 1 altitude level may be gained or lost upon executing last roll at no additional FP code.
    \itemaddedin{1C}{1C-apj-39-qa}{If done in climbing or diving flight, 1 altitude level may be gained or lost upon executing each roll after the first at no additional FP cost.}
    \item Altitude changes that occur in a diving or climbing B-Roll may be in lieu of, or in conjunction with altitude changes done via VFP expenditure.
    \item Incur 2.0 decel per level gained in climbing Barrel Roll, and gain 0.5 accel per altitude level lost in a Barrel Roll.
\end{enumerate}

\subheading{Half Roll and Dive}

\begin{enumerate}[nosep]
    \item Declare at start of move, perform normal Vertical dive except no vertical rolls allowed until last FP expended and then only if it was a VFP.
    \item Allow vertical dive entry from level flight, ot if speed ≤ 4.0 allows entry from zoom/sustained climbs.
    \item Allows steep dive entry from vertical climbs, with normal turning allowed.
    \item No attacks or weapon launches allowed that turn.
\end{enumerate}

\begin{itemize}[nosep]
    \item For purposes of weapons launch modifiers and gunsights, rolls count as BT turns until recovery period met.
    \item Incur 1.0 extra decel for each roll over one executed in a signle game-turn.
\end{itemize}

\heading{VIFF Maneuvers (VIFF Capable aircraft only)}

\subheading{VIFF Sidestep}

\begin{enumerate}[nosep]
    \item Executed as a slide except no prep-moves required but those imposed by altitude and supersonic speed.
    \item Multiple sidesteps allowed so long as 1 HFP expended in forward flight between execution of each sidestep.
    \item Each costs two HFPs to execute and each causes 2.0 decel.
\end{enumerate}

\subheading{VIFF Assisted Turn}

\begin{enumerate}[nosep]
    \item Reduce listed turn requirements by one (90 is best allowed).
    \item Treat aircraft as High Bleed Rate, incur 2.0 to use.
\end{enumerate}

\subheading{VIFF Vertical Pitch}

\begin{enumerate}[nosep]
    \item Treat as Half Roll and Dive except aircraft may go from vertical climb direct to vertical dive, incur 2.0 decel.
\end{enumerate}

\subheading{VIFF Pop-up}

\begin{enumerate}[nosep]
    \item Allows gain of one Altitude Level from level flight once per turn.
    \item Costs 1 HFP, incurs 2.0 decel, aircraft must be wings level.
\end{enumerate}

\end{multicols}
\\
\bottomrule
\end{tabularx}
\end{twocolumntable}
\end{twocolumntablefloat}

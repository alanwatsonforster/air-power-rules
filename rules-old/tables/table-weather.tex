%!TEX root = ../rules-working.tex
%LTeX: enabled=false

\begin{twocolumntablefloat}
\begin{twocolumntable}
\tablecaption{table:climate-zones}{Weather Generation: Geographic Area.}
\begin{tabularx}{0.7\linewidth}{l*{4}{C}}
\toprule
Sky Type     &Far North/South    &Middle\break Regions &Desert\break Areas &Tropical\break Areas  \\
\midrule
Clear        &\phantom{0}1 to 4\phantom{0} &\phantom{0}1 to 5\phantom{0} &\phantom{0}1 to 8\phantom{0} &\phantom{0}1 to 6\phantom{0} \\
Broken       &\phantom{0}5 to 7\phantom{0} &\phantom{0}6 to 8\phantom{0} &\phantom{0}9 to 10           &7                            \\
Overcast     &\phantom{0}8 to 10\phantom{} &\phantom{0}9 to 10\phantom{} &NA                           &\phantom{0}8 to 10\phantom{} \\
Contrail Alt.&20                 &25             &30           &25              \\
\bottomrule
\end{tabularx}
\begin{tablenote}{0.7\linewidth}
\deletedin{1B}{1B-tables}{(roll one die for sky condition)}
\end{tablenote}
\end{twocolumntable}
\end{twocolumntablefloat}

\begin{twocolumntablefloat}
\begin{twocolumntable}
\tablecaption{table:haze-and-clouds}{Weather Generation: Actual Weather.}
\begin{tabularx}{0.7\linewidth}{lLLL}
\toprule
Die Roll    &Clear      &Broken         &Overcast               \\
\midrule
    1	    &LO Hz      &LO Hz, 3 Str.  &LO Hz, 3 Str., 1 Dns.   \\
    2       &ML Hz      &ML Hz, 1 Dns.  &LO Hz, 3 Str., 1 Dns.   \\
    3	    &1 Str.     &ML Hz, 1 Dns.  &LO Hz, 2 Str., 2 Dns.   \\
    4	    &1 Str.     &1 Str., 1 Dns. &1 Str., 3 Dns.         \\
    5	    &2 Str.     &2 Str., 1 Dns. &2 Str., 2 Dns.         \\
    6	    &---        &3 Str,         &2 Str., 2 Dns.          \\
    7       &---        &4 Str.	        &2 Str., 2 Dns.         \\
    8  	    &---        &HI Hz, 1 Str.	&ML Hz., 1 Dns.	        \\
    9	    &---        &LO Hz, 1 Str.  &2 Dns.                 \\
   10	    &---	    &2 Str.	        &1 Dns.	                \\
\bottomrule
\end{tabularx}
\begin{tablenote}{0.7\linewidth}
\deletedin{1B}{1B-tables}{(roll die to determine haze and cloud layers)}
\end{tablenote}
\end{twocolumntable}
\end{twocolumntablefloat}

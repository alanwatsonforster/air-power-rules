%!TEX root = ./rules-working.tex
%LTeX: enabled=false

\rulechapter{Anti-Radiation Missiles}
\label{rule:arms}

Anti-radiation missiles (ARMs) are used to attack enemy radars. ARMs may only attack radars operating in a frequency detectable to the ARM's seeker head. The types of seeker heads available to an ARM vary from simple single frequency types to sophisticated multiple frequency ones. The exact types of seekers associated with each kind of ARM are detailed in the ARM Data Table.

\paragraph{Launching ARMs.} ARMs may be launched at one of three times during a game turn depending on the mode of launch. There are three modes not all of which are available to all ARMs. The modes are:

\begin{enumerate}

    \item[1)]\itemparagraph{Self Defense Mode:} The ARM is launched in SAM interaction phase in response to a SAM TTR lock-on held or obtained against the ARM equipped aircraft. Aiming is not required but few ARMs are allowed this mode.

    \item[2)]\itemparagraph{Straight Shot Mode:} The ARM is launched at any point in the aircraft's flight after aiming requirements are met. The missile may not climb after launch. Firer must be in level or diving flight.

    \item[3)]\itemparagraph{Lofted Shot Mode:} The ARM is launched in the Air to Air Missile Launch Phase using an extended range loft profile. Firer must use climbing flight that turn.       

\end{enumerate}

% Also: AS-4, and AS-16

Note: All ARMs may use straight or lofted shots except the AGM-122 Sidearm which can only do straight shots.

\paragraph{ARM Launch Procedure.} Firing ARMs is a three step process. First, a target must be detected. Second, a fix must be obtained on the target radar. Third, a successful launch roll must be made while in the listed range envelope of the ARM. ARMs may be fired at unsighted radars.

\begin{itemize}

    \item\itemparagraph{Target Detection:} Any EWR or TTR radar that is active at the end of the SAM interaction phase, or any FCR which is used with AAA plotted fire or that is in use to allow aimed fire at night or adverse weather that is in the ARM seeker's lock-on arc is automatically detected for the entire turn. Any FCR that wishes to participate in aimed fire, and which is in the seeker’s lock-on arc, is detected from the moment the AAA unit it is with commences tracking a target until the end of the turn.  

    %Exception:  Blind Fired capable ARMs and NAV Lofted ARMs can be launched without having a detected target.  

    \item\itemparagraph{Target Fixing:} ARM missiles are not aimed like other weapons. ARMs usually require the aircraft to perform a range and bearing check to provide the ARM seeker with a fix on the radar.

    To perform a range and bearing check, the aircraft must first maneuver or turn so that the detected radar unit is in the ARM’s limited arc, regardless of the ARM's normal lock-on arc, and must then steep dive for a number of FPs equal to half its speed while keeping the target radar in its limited arc.

    Some ARMS do not require range and bearing checks at all. They are noted on the ARM tables and may acheive a fix against detected radars anywhere in their lock-on arc any time after spending FPs in wings-level, non-maneuvering flight (climbs and dives okay) equal to the normal aiming requirements for the type of bombsight they have.
    
    % whil keeping the target EWR, TTR, or FCR within the ARM seeker’s lock-on vertical limits.  

    % to the normal aiming requirements for the type of bombsight the aircraft possess,

    Self Defense Capable ARMs get an automatic fix on detected radar SAM units which obtain lock-ons on their carrying aircraft. 

    % Self Defense Capable ARMs, NAV-Lofted ARMs, and Blind Fired ARMs do not require range and bearing checks. 

\end{itemize}

%Note:  The electronics on some aircraft can fix the location of the radar without performing the above range and bearing check (e.g., F-4G w/APR-38 RHAW and F-16C Block 52D w/ASQ-213 HTS Pod).

\paragraph{Fix Duration.} Once a fix is achieved, the ARM is ready to be fired. Firing may be delayed for as long as to the air to air missile launch phase of the next game-turn provided the fix remains valid. To remain valid. the aircraft must fly so that the targeted radar remains in the ARM's lock-on arc until launch and no other attacks, air to air or air to ground of any sort are made. If a fix is invalidated, it must be reachieved again before ARMs may be launched.

\paragraph{Launch Limits.} One or two ARMs may be launched at a time.  Regardless of when in the game-turn the ARMs are launched, they count as the aircraft's one air to ground attack and no other air to ground weapons may be used that turn. Generally, if two ARMs are fired, they must be at the same target. However, ARMs not requiring range and bearing checks may be fired at separate targets.

\section{ARM Flight}

ARMs determine start speeds and fly exactly as air to air missiles, except rolling maneuvers are not allowed since their targets are stationary. Also, most ARMs are not given a turn ability rating except in the case of a few advanced weapons. Instead of turning, ARMs fly forward until intercepting a line of approach to their target, at which time they are allowed to turn up to 60 degrees to fly down the line to the target. They may also utilize one snap turn in their flight to help reach a line of approach.

Self-Defense Mode capable ARMs have a turn ability which is used normally to allow them to turn after launch to attack radar sites off to either side of the firing aircraft.

\paragraph{ARM Flight Restrictions.} There are certain restrictions on ARM flight for each mode of launch. They are as follows:

\begin{itemize}

    \item\itemparagraph{STRAIGHT SHOT MODE:} Firing aircraft must be in level or diving flight, not turning or maneuvering at the time of launch and within the listed range parameters. Launch is allowed after the expenditure of any FP in the aircraft's move in which these criteria are met. When fired, the ARM is immediately flown its full move, and then the aircraft finishes its move.

    Shoot-out Exception: If the launching or other aircraft is under SAM attack, and the ARM was fired at the guiding SAM unit, then a shoot-out is resolved exactly as for air to air missile shoot outs. During their flight ARMs may not climb if at or above a target's altitude.

    \item\itemparagraph{LOFTED MODE:} By lofting an ARM the maximum launch range parameter may be doubled. The launch aircraft must utilize a zoom or sustained climb on the turn of launch. The ARMs are launched in the air to air missile launch phase and begin moving in the following game turn (which is considered their first for TOF purposes).

    Note: Since aircraft may not do diving and climbing flight in a game turn, it will take two game turns to set up a lofted shot if the aircraft must do a range and bearing check first.

    Lofted ARMs are restricted as for straight shots except that they are allowed to climb as per normal missile flight rules while in the loft phase of their flight. The loft phase can consist of up to the first half of their TOF in game-turns (rounded down). After that, or if in any game-turn the missile does not climb or it expands an FP to lose altitude or opts to lose altitude In forward flight, the loft phase immediately ends and the missile is no longer allowed to climb. During the loft phase, there is no restriction on how high the ARM may climb except for its available FPs and the altitude level 100 maximum limit.

    \item\itemparagraph{SELF DEFENSE MODE:} This mode is treated exactly as a straight shot except that the ARMs are fired in the SAM Interaction Phase. They move when it is the launch aircraft's time to move and will turn onto the first LOA it reaches. If not involved in a shoot-out, they move immediately after the launch aircraft does. Otherwise, the shoot-out procedure is followed.

\end{itemize}

% NAV-LOFTING:  If the scenario specifies that a radar is fixed by reconnaissance prior to play, ARMs may be nav-lofted at it.  Detection of the radar is not required, nor is a fix or range and bearing check; instead, the offset aimpoint procedure must be completed prior to launch. The subsequent launch and flight of the missile is handled as for a normal lofted ARM.  However, at some point during flight, the radar must become active within the ARM’s seeker limits for a normal attack.  If the radar fails to activate, the ARM will impact the “fixed” location, but will attack any radar present at 2/3 strength (it never corrected its flight using its seeker).

% \paragraph{BLIND FIRE ARMs.}  ARMs designated as being able to be blind-fired may be launched as a straight shot ARM without having detected or fixed a target, provided the launch aircraft is in level flight and not turning or maneuvering.   

% Ater launch, the ARM flies straight and level until it detects a radar, at which point it rolls a die.  On a 6 or less it fixes the radar and may fly towards and attack it normally; otherwise it con-tinues to fly straight until it encounters another radar.  Blind-fired ARMs may not attempt more than two fix rolls per turn, and both must be at separate radars.  If an ARM detects more than one radar simul-taneously, it must attempt to fix the closest first.  If an ARM fails to fix a radar, it may attempt to fix it again the next turn.

% OFF BEARING LAUNCH: ARMs marked as having Off Bearing Launch capability may be fired without the missile detecting the target, pro-vided the launching aircraft has a RWR-D or better AND has detected the target radar with it.  To fire an Off Bearing ARM, the launching aircraft must first fix the target radar, though it is not required to have the target in the missile's seeker arc (in effect, it may fix while facing away from the target).  It may then launch as for a straight shot ARM.  After launch, the missile expends 1 HFP and then must conduct a maximum turn at its allowed turn rate, as for Air-to-Air Missiles, until the target radar is in its seeker arc.  Thereafter, it guides normally. ARMs may not be lofted in Off Bearing Mode.

\paragraph{LOA Intercept.} During flight, all ARMs must intercept the first available line of approach to their intended target. Once established on it, they may not leave it unless the target shuts down and the ARM is capable of switching targets.

\paragraph{ARM Attacks.} ARMs always use their full attack strength against fixed radar targets. \addedin{1C}{1C-original-play-aids}{ARMs always attack land based radar equipped units (whether soft or hard targets) with their soft attack strength and naval targets with their hard attack strength. }ARMs attack only the primary target in the hex. Against vehicle units, the most damage a single ARM can do is 2D. ARM attacks are resolved the instant the missile reaches the target's hex and altitude. If multiple missiles hit at the same time, each is resolved as a separate attack. ARMs which do not reach their target in a game-turn remain in play unless their TOF is up or no available targets exist anymore.

\paragraph{Target Radar Loss.}  When an ARM attacks a radar equipped unit, there is the chance the radar will be destroyed even if the unit is not. This can disable some SAM units. To simulate this, target radars are destroyed if:

\begin{itemize}
    \item the die roll of an ARM attack was even and the level of damage was “D” or more.

    \item the die roll of an ARM attack was even, and the target was suppressed, AND an additional die roll results in a 3 or less.
\end{itemize}

Note: No modifiers apply to the attack die roll for ARMs.

\paragraph{Operative Radars.} For purposes of ARM detection and attacks, radars are considered to be on and operating (even if jammed) as follows:

\begin{itemize}

    \item Regular EWR - On continuously from start of game due to its mission unless shutdown in response to a recognized ARM attack.

    \item SAM EWR/TTR - Silent until the first pass-down received from a regular EWR or until the player declares it on, then continuous after that unless subsequently shutdown.

    \item AAA FCR - On at start of AAA planning phase until end of game turn if used for plotted fire. If used for visual aimed AAA fire, on only from time of firing until end of game-turn. If needed so that AAA unit may utilize aimed fire at night and/or adverse weather, on continuously as for EWRs.

\end{itemize}

\paragraph{Radar Shutdowns.} Radars may attempt to shutdown to avoid ARM attacks. A radar may attempt to shutdown in the SAM interaction phase if alerted to ARMs (EWRs excepted below), or in the flight phase if it recognizes an ARM launch against itself.

\paragraph{ARM Alert.} All radar units in play on one side are alerted to ARM attacks the instant one of them is attacked by ARMs or recognizes an ARM launch. Operating EWR radars are not allowed to shutdown on the basis of an ARM alert. They are allowed to shutdown only upon recognizing an ARM launch against themselves.

\paragraph{ARM Attack Recognition.} To be able to recognize an ARM launch, the radar equipped unit must be the target of the ARM and have the launch aircraft either:

\begin{itemize}
    \item visually sighted.

    \item or detected if an EWR.

    \item or locked up if a TTR.

    \item or have fired at it already this turn if an FCR using aimed fire.
\end{itemize}

Note: FCRs used for all weather purposes and/or plotted fire, may never recognize an ARM launch (they may still be alerted to them).

ARM attack recognition is not automatic. A die roll as explained below must be made and only one attempt at the instant of ARM launch is allowed.

\paragraph{SHUTDOWN Procedure.} An alerted radar wishing to shutdown in the SAM Interaction Phase, or a target radar meeting the recognition criteria in the flight phase wishing to shutdown must roll a die. On a 6 or less, the radar is immediately shutdown and goes off the air. This is done before the ARM moves.

\paragraph{REACTIVATION Procedure.} Shutdown radars may attempt to reactivate in the SAM Interaction phase of any game-turn after the one in which they shutdown. A die roll ls required as above. On a 6 or less, the radar is switched back on and may be operated normally.

\paragraph{Shutdown Effects.} Shutdown radars lose all contacts, lock-ons, and SAMs under radar guidance (optical back-ups excepted). SAM units with both EWRs and TTRs in them are required to shutdown and reactivate both simultaneously (only one die roll required).

\paragraph{ARM Target Switching.} Whenever the original target of ARM shuts down, one of the following events occurs:

\begin{itemize}

    \item If the ARM has no switch or no memory capability, it is removed from play (see ARM Data Table).

    \item If the ARM has switch capability, it is allowed to continue flying straight ahead until intercepting the line of approach to another detectable and active radar, at which time it may pivot up to 60° to get on that line of approach. ARMs with a turn ability may use that to change course to get on a line of approach to a new target.
    
    Note: Normally, only one switch is allowed.  Some ARMs may be allowed more (see ARM Data Table notes).

    \item If the ARM also has target memory capability and was already on a line of approach, it may opt to continue down the original line until the target is reached, at which time it attacks at 1:2 odds regardless of the ARM's regular attack strength.

    Note: If the memory option is taken, the switch option is no longer allowed to an ARM.

    \item If all targetable radars are shut down, all ARMs in flight except those with memory are removed from play. The memory capable ones may fly one more turn waiting for new radars to activate. If at the end of the additional turn, no targets are available, then the memory ARM is removed also. Regardless of its capabilities, no ARM may ever fly longer than its listed TOF.

\end{itemize}


\trainingnote{You may now play Training scenario six and all scenarios except those involving Guided or Smart air to ground weapons.}

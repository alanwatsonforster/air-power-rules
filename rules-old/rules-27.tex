%!TEX root = ./rules-working.tex
%LTeX: enabled=false

\rulechapter{Air-to-Ground Guided Weapons}


As air defenses became more sophisticated, incorporating radar-guided guns and SAMs, aircraft were supplied with guided weapons to improve stand-off capability and enhance survivability. The following sections describe the various guided weapon developments.

\section{Command Guided Rockets}

Command-guidance provides a flying weapon with course corrections from an external controller. There are two types of CG rocket weapons.

\begin{enumerate}

    \item[1)]\itemparagraph{Radio Command Guidance (RCG).} The pilot or the weapons officer of the launching aircraft visually tracks the weapon (which carries a flare to make it highly visible) and provides guidance commands to the weapon through the use of a joystick and a radio data link. Accuracy depends on the skill of the controller.

    \item[2)]\itemparagraph{Automatic Command Guided (ACG).} The pilot or weapons officer of the launch aircraft tracks the target (not the weapon) in his sights; the weapon is automatically tracked by an optical sensor which provides guidance commands through the use of a radio data link. The crew of the aircraft merely had to keep the sights on the target until impact.

\end{enumerate}

\paragraph{Launching CG Weapons.} The aircraft must steep dive on a LOA to the target and be within the rocket's minimum and maximum launch ranges (counting altitude). The rocket may be launched after any expenditure of FP while wings level (not turning or maneuvering). Only one guided rocket may be launched in an attack. 

Note: Aiming need not be completed but if not, the $+3$ attack modifier applies and all bombsights degrade to manual. A die roll is required for successful launch.

\paragraph{CG Rocket Flight.} Upon launch, a CG rocket immediately flies its full speed along the LOA toward the target. The rocket and aircraft alternate expending FPs as if proportionally moving with a missile (rocket moves first). Shoot-outs are resolved normally if need be. In some cases, the aircraft will run out of FPs early since it fired in mid-flight. On the first turn of flight, the rocket's start speed is its base speed plus that of the aircraft. If the rocket needs additional turns of flight to reach the target, it begins to attenuate speed just like an air to air missile.

\paragraph{CG Rocket Guidance.} As long as the rocket is moving toward the target, the guiding aircraft must be in steep diving flight and must remain on the line of approach. If the aircraft does not meet either of these criteria before the rocket hits, the rocket will miss automatically. Note that guiding aircraft move before other free aircraft.

\paragraph{CG Rocket Attack.} Attacks are resolved normally using the weapon's attack strength. Only the primary target can ever be hit. Against vehicular units, the highest result that can be achieved is a D for weapons of 1000 lbs or less and 2D for larger weapons. For guided rocket attacks, only the aircraft damage, bombsight modifiers are used. Tracking time and rocket attack slant range modifiers do not apply.

%The maximum damage allowed against vehicle units effected by RCG and ACG weapons is D for weapons of less than 1000 lbs and 2D for 1000 lbs or larger weapons.

\section{Laser Guided Weapons}
\label{rule:laser-guided-weapons}

Laser guided weapons have a seeker head that homes in on laser energy (a laser spot) reflected off the target. A laser designator capable aircraft or a FAC unit must designate the ground target before laser guided weapons may be used.

\paragraph{Laser Designators.} Ground FACs and aircraft with either laser designator technology or equipped with laser pods (LP) are allowed to place laser spots on targets. A ground FAC may place a spot on any enemy target in its line of sight within six hexes. Aircraft designators vary in capability depending on type. There are three types:

\begin{itemize}

    \item\itemparagraph{Type-A:} These may place a laser spot anywhere in the aircraft's 180+ arcs if operated by the pilot, or within the 120+ arcs if used by the weapons officer. The maximum range a laser spot may be placed from the aircraft is 18 hexes (counting altitude).

    \item\itemparagraph{Type-B:} These may place a spot anywhere within the aircraft's 90+ arcs out to a range of 24 (counting altitude).

    \item\itemparagraph{Type-C:} These may place a spot anywhere about the aircraft out to a range of 36 hexes (counting altitude) due to telescopic sighting equipment.

\end{itemize}

%Longer ranges or special abilities may be noted for some pods or aircraft.

\paragraph{Laser Spots.} 
\label{rule:laser-spots}
FACs place their spots in the Visual Sighting Phase. Aircraft may place their spots at any time during their movement prior to any laser weapons being released. The spot must be maintained until the weapons impact. The act of designating counts as the aircraft's ground attack, it may make no other attacks except to drop laser guided weapons on its own spot. Laser spots are removed at the end of the game turn unless weapons are still in flight. In this case the spots may be left but the FAC or designating aircraft must maintain them in the next game turn until weapons impact.

\paragraph{Laser Spot Trackers.} Aircraft with LSTs can utilize laser spots for accurate aiming and are thus allowed the die roll modifier shown on the attack table when laser spot marks their target whether they use conventional or laser guided weapons.  

%Aircraft do not require a LST to use a laser guided weapon.

\paragraph{Joint Attacks.} A designator equipped aircraft is allowed to place laser spots for other aircraft to drop bombs on. This is called a joint attack and is declared in the aircraft decision's phase by identifying the participating aircraft.

Order of flight is determined normally but the designator aircraft's order of flight is changed to occur before any of the other attackers who will use his spot. If the designator is already going before, no changes need be made.

The designating aircraft must place the spot at the start of its move, or the instant it is within range to do so, and then maintain it until the end of its move. If the designator aircraft survives its move and does not violate the parameters of its designator, the spot is considered to exist until the end of the game-turn even if the designator aircraft is destroyed later in the same game turn by enemy aircraft.

Once placed, other aircraft are allowed to utilize the spot for attacks. Aircraft designating for joint attacks may themselves drop laser guided weapons on the same spot.

\paragraph{Designator Limitations.} Engaged, stalled and departed aircraft may not designate. Using a designator imposes the following maneuver restrictions on free aircraft for as long as they are maintaining a spot:

\begin{itemize}
    \item\itemparagraph{Type-A:} No maneuvers except slides, no turns of greater than TT rate.

    \item\itemparagraph{Type-B:} No maneuvers except slides, no turns of greater than HT rate.

    \item\itemparagraph{Type-C:} No maneuvers except slides, no turns of greater than BT rate.
\end{itemize}


\paragraph{Laser Guided Bombs.} Laser guided bombs (BG) are aimed and dropped just like BB class weapons including the use of regularly allowed modifiers except:

\begin{itemize}

    \item The maximum damage result allowed against a vehicle unit attacked by a single bomb is D for HE weapons of less than 1,000 lbs and 2D for 1,000 lb. or larger bombs.

    \item If the dropping aircraft is not also the designator and has laser spot tracker technology or a pod allowing it that, an additional $-1$ to the attack roll is allowed.

    \item If no laser spot is available, the weapons may be dropped as regular BB class weapons using the BB attack strengths.

    \item If the laser spot is lost after the weapons are dropped under guidance, they revert to regular BB class weapons but incur the $+3$ modifier for not being aimed.

\end{itemize}

Note: Laser Guided Glide Bombs, move and attack per the glide bomb rules for smart weapons.

\paragraph{Tossing Laser Guided Bombs.} Laser guided bombs may be tossed per the toss bombing rules. Lateral Toss Bombing is not allowed, however offset aimpoint tossing is. The release point requirement is not as stringent with laser bombs. Aiming is still required and tracking time modifiers are not allowed.

\paragraph{Toss Procedure.} Aiming is accomplished by expending HFPs on the line of approach to the target. The aircraft must then expend VFPs in a zoom climb and may release weapons between three and eight hexes away after any VFP is spent. The aircraft must be further away in hexes than it is in altitude levels above the target at time of release.

The tossing aircraft may not be its own laser designator unless equipped with a Type-C designator and a line of sight can be maintained on the target from weapons release to impact. The modifier for laser tossing against a sighted target is $+1$ per two hexes distance away and $+1$ per hex if the target is unsighted.

\paragraph{Laser Guided Rockets.} Laser guided rockets (RG) must be launched from aircraft in level, descending or steep diving flight while wings level (not turning or maneuvering). The aircraft must be on a line of approach to the target and within the minimum and maximum launch ranges of the rockets (counting altitude).

The rockets may be launched after the expenditure of any FP while in the above parameters. Up to two laser guided rockets may be fired in an attack. Both attack the same target. The attacks are resolved separately. A launch roll is required as for other guided weapons.

A laser guided rocket is flown exactly as a CG rocket (including shoot-out procedures) except that the aircraft is free to maneuver after launch as no guidance inputs are required unless it is designating as well. If aiming is not completed, the $+3$ modifier applies and the bombsight is considered manual. The following also apply:

\begin{itemize}

    \item The maximum damage result allowed against a vehicle unit attacked by a single RG is D for weapons of less than 1,000 lb and 2D for 1,000 lb or larger ones.

    \item If the firing aircraft is not also the designator and has laser spot tracker technology or a pod allowing it that, an additional $-1$ to the attack roll is allowed.

    \item RGs may be fired even if no laser spot exists. In such a case, they are treated as unguided rockets using their regular attack value and all the normal unguided air to ground rocket modifiers.

    \item If a laser spot is lost before RGs already in flight reach the target, they revert to being unguided rockets as above but with a $+3$ (no aiming) modifier and the bombsight modifier is manual.

\end{itemize}

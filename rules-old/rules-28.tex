%!TEX root = ./rules-working.tex
%LTeX: enabled=false

\rulechapter{Air-to-Ground Smart (Self-Guiding) Weapons}
\label{rule:smart-weapons}

TV and IR guided glide bombs and TV and IR guided rockets are considered smart weapons as they are self-guiding requiring no outside inputs. Smart weapon codes are BS and RS.  

\paragraph{Allowed Targets.} Smart weapons can only used against well-defined targets. These include the following:

\begin{itemize}
    \item For TV GUIDED: (all uncamouflaged) AAA sites, SAM sites, Artillery sites, vehicle units, ships, locomotives, trains, buildings, hangars, shelters, urban areas, built up areas, POL sites, towers, radar units, and bridges.

    \item For IR GUIDED: (all uncamouflaged) Vehicles, ships, locomotives, radar units, buildings, hangars, shelters, towers, urban and built-up areas.
\end{itemize}

\paragraph{Launch Procedure.} For any smart weapon, the aircraft must be on a LOA, in level flight or steep diving with wings level and within the minimum and maximum launch ranges of the weapons. A launch roll is required. Aiming is also required and if not completed the weapon cannot be launched (smart weapons must be locked-on to the target to self-guide).

Only one smart weapon may be launched per attack, however, if the aircraft has computed or advanced sights, it may make two attacks in the same game-turn against the same target only, provided it completes aiming anew for the second attack. This is a deliberate exception to the one attack per turn limit normally imposed on aircraft.

Note: If a data link pod is in use, only one weapon may be launched per turn period, and the data link cannot be used for additional weapons until the first weapon has attacked or is aborted.

\paragraph{Smart Weapon Attacks.} Smart weapons only use the modifiers for terrain the target is in and damage to the aircraft at time of launch. If a data link pod is in use, then the aircraft's bombsight modifier may also be applied. The maximum result a weapon of under 1,000 lb can achieve against a vehicle unit is D, and for a weapon equal or greater in size is 2D.

\section{Smart Glide Bombs}

\paragraph{Launching Glide Bombs.} Glide bombs must be dropped within the weapon's minimum and maximum ranges (counting altitudes) and from above the minimum release altitude for the horizontal distance in hexes. Minimum release altitude is one level above the target per six or less hexes of horizontal distance.

The maximum and minimum speeds a glide bomb can be launched at is 6.0 and 2.0 respectively. If launched at less than 2.0, the weapon stalls out and is removed from play. If launched at 6.5 or greater, it will pitch up and strike the launch aircraft inflicting a hit with an attack rating of 2 while sending itself out of control (remove from play). The same occurs if jettisoned at speeds greater than 6.0.

Glide bombs may be carried at speeds up to 8.0, however if carried at a faster speed, they become damaged and will automatically fail upon launch.

\paragraph{Glide Bomb Flight.} A glide bomb is given a speed equal to the aircraft's on the turn of launch. On its first turn of flight, it will only fly down the LOA for as many FPs as the aircraft had left to it upon launch. After that it will fly each turn an amount equal to the aircraft's speed upon launch. The weapon may fly forward and/or steep dive, it may not turn or stray off the LOA.

Note: If used against a target that is moving, the weapon may do slide maneuvers as necessary to stay on a LOA.

The glide bomb is not required to lose altitude on the turn of launch, however, it must lose at least one altitude each turn thereafter. Careful planning of where to launch from will prevent a weapon impacting short.

\paragraph{Data Link Pods.} 
\label{rule:data-link-pods}
Some smart weapons may not be used at all unless a data link pod is carried by the aircraft. These weapons are noted as such in their data tables.

\paragraph{Lofted and CG Glide Bomb Attacks.} Certain glide bombs may be used in conjunction with data link pods to perform lofted glide bomb attacks or command guided glide bomb attacks. Such weapons are also noted in the tables. These attacks are performed as follows:

\begin{itemize}

    \item\itemparagraph{Lofted Glide Bomb Attacks.} A lofted glide bomb may be launched from below the minimum release altitude. The aircraft must aim and then zoom climb as in toss bombing attacks. The glide bomb may be released after the expenditure of any VFP. The lofted glide bomb must expend available FPs to climb at least as many additional levels as the aircraft did that game-turn prior to launch (one VFP can gain one or two levels). The rest may be used as HFPs. On following game-turns, the bomb moves normally.

    \item\itemparagraph{Command Guidance Glide Bomb Mode.} Use of this mode allows the launch aircraft or other aircraft equipped with a data link pod to aid the glide bomb in acquiring the target after launch.

    With command mode, a glide bomb may be lofted at unsighted targets using offset aimpoint bombing techniques, or dropped normally at out to 1.5 times its normal maximum range. However, an additional launch roll, over that used to launch it, is required on the second turn of its flight if lofted, or on the first turn in which it enters normal maximum range. This is to see if it acquires a lock-on to the target. In this case the pilot or weapons officer in the data link aircraft is trying to get the weapon a lock-on after launch via the data link. If the roll falls, the weapon is removed from play. If it works, the weapon flies normally.

\end{itemize}

%\paragraph{Wind Corrected Munitions Dispenser.}  The WCMD uses an inertial navigation system (INS) guided munitions dispenser.  It operates like a guided bomb, attacking with a -2 modifier. 

%Inertial Navigation System (INS) Guidance.  INS guidance gives a -2 modification to attacks.

\section{Smart Rockets}

\paragraph{Smart Rocket Specifics.} Smart rockets (RS) must be launched from within the minimum and maximum range parameters. No minimum altitude restrictions apply. Smart rockets move down the LOA and conduct attacks just like command guided rockets except of course, that no guidance inputs from the launch aircraft are required.

%Cruise Rockets/Missiles. Weapons with a parenthesized sustainer value indicate a cruise rocket or missile.  On the first turn of flight, calculate speed normally.  On the second and subsequent turns of sustained flight set the rocket's base speed at the indicated “cruise” speed. It does not alter speed until the sustainer runs out, after which speed attenuates normally.

%A few weapons have a “U” parenthesized as a sustainer value.  This indicates a cruise rocket/ missile with unlimited range in game turns. 

\paragraph{Extended Flight.} If a smart weapon (BS or RS) does not reach a target on the first turn of its flight, it will move in subsequent game-turns according to the flight phase order of movement sequence (see sequence of play in play aids) unless involved in a shoot-out which is conducted normally.

%28.3 – GPS GUIDED WEAPONS

%Global Positioning Satellite (GPS) guidance units use satellite provided signals to guide the weapon to the programed target coordinates.  A GPS guided bomb has a guidance unit that uses a GPS system for attacks on fixed targets only.  It can only be used by aircraft updated to use it.  This includes F-16C/D, B-52H, B-1B, B-2A, F-18E/F, F-18C/D, F-14B/D, F-117A, F-15E, AV-8B, EF-2000, Gripen, Tornado, and F-19, etc.  A GPS bomb attacks with a -4 rating.  The movement rules for smart gliding bombs apply.  Minimum release altitude is one level above the target per six or less hexes of horizontal distance unless otherwise noted.

%GPS Guided Bombs (GBG). GPS guided bombs (GBG) are aimed and dropped just like BB class weapons including the use of regularly allowed modifiers except;

%May be used against fixed targets only; a hex location in other words. Vehicles and other mobile units within the hex are attacked indirectly.  The maximum damage result allowed against a vehicle unit attacked by a single bomb is "D" and only allowed for 1,000 lbs or larger bombs.

%If desired, the weapons may be dropped as a regular BB class weapons using the BB attack strengths.

%The target is a GPS coordinate in a hex; not the actual targeted facility.  Thus, no modifiers apply for camouflage. The target need only be in the \arcplus{120} arc of the aircraft when the weapon is released.  

%No requirement to sight the target.  The bomb can turn up to 120° left or right to establish a glide path to the target.  May be used against unsighted targets. The aircraft can attack any target within its 120° arc within range.

%Targeting data can be uploaded by the aircraft via a TV/IR pod and data link.  

%Some GPS bombs include INS guidance.  The INS can be automatically switched to on a roll of 1 – 8 in the event of GPS failure or jamming, attacking at -3.

%Laser Guided GPS Bomb (LGBG), TV Guided GPS Bomb (TVGBG), and IR Guided GPS Bomb (IRGBG):  These guided bombs use dual mode guidance systems combining a laser guidance kit, TV guidance kit, or an IR guidance kit together with a GPS guide unit into the same weapon. These bombs retain the advantages of each guidance method, guiding to a location and using laser/TV/IR guidance for the final attack.  This enables LGBG, TVGBG, and IRGBG bombs to attack moving targets.  GPS guidance attacks with a -3 modifier, while laser, TV, or IR guidance attacks with a  -4 modifier.

%The bombs move and attack per the glide bomb rules for smart weapons.  Some small bombs can be accommodated on a bomb rack holding four bombs that fits in place of a 2000 lbs bomb. The weapons can be released from any hex within range and the 120° arc of the aircraft.

%Lofted Glide Bomb attack rules apply to LGBG, TVGBG, and IRBGB bombs. Command Guidance Glide Bomb Mode rules only apply to TVGBG bombs.  Lofted GPS guided bombs can engage any target in any arc within range as computed above.

%Aiming. Aiming is not required against a sighted target. The hex is considered the target, or rather the programmed geographic coordinates within the hex, so all that is required is that the aircraft meets the range and facing requirements of the target hex.  

%Any targets other than a fixed location in the hex are attacked as collateral damage under rule \ref{rule:collateral-damage} if using GPS guidance.

%\paragraph{GPS Jamming Systems.} GPS jamming systems may be used to protect ground targets.  If present, on a roll of 1 - 6, a GBG bomb scatters. Dual mode bombs can switch to their laser/TV/IR guidance mode on a roll of 1 – 8.  Otherwise, the bomb scatters. Dual mode bombs targeting a moving target in the hex can complete the attack normally; a scattering check does not occur.  Dual mode GPS/INS guided bombs attack normally at -2 if GPS jammed on a roll of 1 - 9.  On a roll of 10, the bomb scatters.  Failure of guidance also occurs on a roll of 10, scattering the bomb.

\begin{advancedrules}

\section{Air-to-Surface Missiles}
\label{rule:asms}

Air to Surface Missiles (ASMs) are generally large anti-ship weapons which function in a manner similar to smart weapons. They are usually very long-ranged and rely on programmed flight profiles until within a terminal homing range.

\paragraph{ASM Launch.} An aircraft must be in level flight and wings level, not turning or maneuvering at the end of its flight. ASMs are launched in Air to Air Missile Launch Phase. An aircraft launching ASMs may not launch other missile types. Up to two ASMs of identical types may be launched in a turn. A launch roll is required as for other missile launches.

\paragraph{Enroute Guidance Modes.} ASMs may be capable of the following enroute guidance modes. The following are additional launch requirements for each mode:

\begin{itemize}
    \item Inertial Navigation Mode: target coord-inates must be known beforehand and fed into the missile. If the target is radar significant and its position is known (stated in the scenario), one such target may be preprogrammed into the ASM before play. Otherwise, the aircraft must lock the target up on radar and roll a 9 or less to complete targeting.

    \item Autopilot Mode: no targeting is required, missile can simply be launched straight ahead. Targeting before launch as for INAV ASMs is allowed as well. Post launch target acquisition may be through the ASM's own terminal guidance mechanism or sent via data link if the ASM is Midcourse Guidance Capable (MCG).

    \item Terminal Mode: if target within ASM seeker's terminal guidance range, and a weapon lock-on is obtained, the ASM will self-guide like a smart weapon. Terminal guidance methods and their associated rules will be given as necessary in the scenario booklet.

\end{itemize}

\paragraph{Flight Profiles.} When launched, ASM's will utilize either surface skimming, or terminal dive profiles (SS, SS+, or TD).

\begin{itemize}

    \item Surface Skimming missiles must immediately steep dive or level descend after launch to altitude level one or one above the ground. On the next turn they enter terrain following flight. Only SS+ missiles may operate over land. Regular SS missiles crash if they pass over land terrain. SS missiles remain in T-level until reaching their target. SS+ missiles may climb and dive along terrain contours in the same turn and will, if necessary, exit T-level to cross a terrain rise of more than two levels in a single hex. They would reenter T-level in the next game turn.

    \item Terminal Dive missiles are restricted to level flight until within terminal Homing range. Once target lock-on is achieved, the missile's may steep or vertical dive into their target.

\end{itemize}

\paragraph{ASM Speed.} ASMs have a first turn of flight speed equal to their listed cruise speed plus half the launch aircraft's speed. On subsequent turns they fly at cruise speed. This speed never changes except during terminal dives into a target where normal accelerations are allowed.

\paragraph{ASM Course Changes.} An ASM in INAV mode may use turning flight on each of 3 separate game-turns (player's choice) while enroute to affect course changes on the way to its target. An ASM in autopilot may only fly straight ahead until targeting is completed. It is then allowed one game-turn of turning to make a course correction. An ASM that receives a mid-course update may use turning flight on each turn it receives the update in addition to normal turning. In all cases, once in terminal homing phase, the missiles may fly normally to attack their targets.

\paragraph{ASM Attacks.} ASMs attack upon entering the target's hex at zero altitude. Roll to hit as for air to air missiles and modify the roll for the presence of target launched decoys and ECM as given on the ASM modifiers table. A hit allows an air to ground attack using the listed attack strengths.

Note: For more information on ASMs, other naval weapons, and attacking ships, GDW's {\itshape Harpoon} game by Larry Bond is an excellent reference. {\itshape Harpoon} is not directly compatible with Air Powerr though the scales and subject matter are similar.
    
\end{advancedrules}

\trainingnote{You may now play all scenarios at the basic level.}


%!TEX root = ./rules-working.tex
%LTeX: enabled=false

\rulechapter{Aircraft Flight}

\silentlyaddedin{1B}{1B-tables}{
    %!TEX root = ../rules-working.tex
%LTeX: enabled=false

\begin{twocolumntablefloat}
\begin{twocolumntable}

\newcommand{\heading}[1]{\medskip\par\textbf{\MakeUppercase{#1}}\par\smallskip}
\newcommand{\subheading}[1]{\smallskip\par\textbf{#1}\par\smallskip}

\tablecaption{table:aircraft-flight-rules-summary}{Aircraft Flight Rules Summary}
\footnotesize
\begin{tabularx}{\linewidth}{P}
\toprule
\begin{multicols}{2}

\heading{Accel/Decel}
\begin{enumerate}[nosep]
    \item Each 2.0 accel accumulated = \plus{0.5} speed normally.
    \item Each 1.5 accel = \plus{0.5} speed for Rapid Accel aircraft.
    \item If speed ≥ Mach 1, each 3.0 accel = \plus{0.5} speed for normal aircraft and each 2.0 = \plus{0.5} for Rapid Accel aircraft.
    \item Each 2.0 Decel accumulated = \minus{0.5} speed always.
\end{enumerate}

\heading{Level Flight}
\begin{enumerate}[nosep]
    \item All FPs are HFPs. An aircraft may descend one altitude level freely at any point in its move.
\end{enumerate}

\heading{Turning Flight}
\begin{enumerate}[nosep]
    \item Turn Drag decel based on highest turn rate used in game turn, incur it only once per game turn even if aircraft faced more often than once.
    \item \changedin{2A}{2A-sustained}{Extra facings in a game turn constitute sustained turns. 1.0 decel is incurred for each dacing change after the first.}{The second and subsequent facing change in a game turn constitute sustained turns. 1.0 DP are incurred for each 30 degrees of facing change in sustained turns (0.5 DP for LBR and 1.5 DP for HBR aircraft).}
    \item TT, HT, BT, ET turns require start speed of 0.5, 1.0, 1.5, and 2.0 > minimum respectively to perform.
    \item Low Roll Rate aircraft take 1 FP of flight to enter a left or right bank before turning and 2 FPs of flight to reverse bank.
    \item High Roll Rate aircraft may instantly switch from one angle of bank to another; others require 1 FP of flight to reverse.
    \item No attacks of weapon launches allowed during or after an ET turn until a Recovery Period passes.
    \item[--] A recovery period = half of the aircraft's flight (round up) while not ET turning and not doing rolls or prep-moving for them.
\end{enumerate}

\deletedin{2A}{2A-snap}{
\heading{Snap Turning}
\begin{enumerate}[nosep]
    \item Aircraft must be capable of BT turn rate.
    \item One allowed per game-turn; costs one HFP; allows immediate facing change of 30 degrees or of 60 degrees if turn chart = 60 or 90 without moving forward.
    \item One HFP prep required is wings not level or if speed ≥ to High Transonic. If both cases apply, two preps required.
    \item Incur Decel as for BT turn unless aircraft used ET rate.
    \item Unless ET follows a snap turn; the snap counts as a BT turn for purposes of combat and weapon launch modifiers until a recovery period passes.
    \item Risky Snap turns may be tried if aircraft is capable of HT turn but roll for a departure on facing (1 to 4).    
\end{enumerate}
}

\heading{FP Expenditure Restrictions}
\begin{enumerate}[nosep]
    \item If going from level to climbing or diving flight; the first FP expended must be an HFP.
    \item If going from dive to climb or climb to dive; FPs = to half the aircraft's speed (round down) must be expended as HFPs before using VFPs. High Pitch Rate aircraft need only expend FPs = to {\onethird} speed (round down) in this case.
    \item If continuing to climb or dive from previous turn; HFPs and VFPs may be mixed in any order.
\end{enumerate}

\vfill\null\columnbreak

\heading{Speedbrake Usage}
\changedin{2A}{2A-spbr}{
\begin{enumerate}[nosep]
    \item FPs up to amount listed on the ADC may be eliminated.
    \item Eliminated FPs may not be used for any turns or other maneuver/combat/proportional move requirements.
    \item 1.0 decel is incurred for each 0.5 FP eliminated.
\end{enumerate}
}{
\begin{enumerate}[nosep]
    \item DPs up to the maximum listed on the ADC may be incurred.
    \item If the aircraft is supersonic, the maximum is increased by 1 DP.
\end{enumerate}
}

\heading{Climbing Flight}

\subheading{Zoom Climbs}
\begin{enumerate}[nosep]
    \item At least one, but up to {\twothirds} of FPs may be VFPs.
    \item If CCC rate for power setting ≤ 2.0, then each VFP can gain 1 altitude level only.
    \item If CCC rate for power setting > 2, each VFP can gain 1 or 2 altitude levels.
    \itemaddedin{2A}{2A-super-climbs}{If CCC rate for power setting ≥ 6.0, one of the VFPs can gain 1, 2, or 3 altitude levels.}
    \itemdeletedin{2A}{2A-zoom-climbs}{If this is the first turn of climbing flight, 1.0 decel is incurred per level climbed.}
    \itemdeletedin{2A}{2A-zoom-climbs}{If this is the second or subsequent turn of climbing flight, 1.5 decel is incurred per level climbed.}
    \itemaddedin{2A}{2A-zoom-climbs}{1.0 DP per level climbed.}
    \item ET turn rates not allowed in zoom climbs.
\end{enumerate}

\subheading{Sustained Climbs}
\begin{enumerate}[nosep]
    \item Start speed must be at least 1.0 > minimum speed.
    \item If start speed is less than climb speed, then halve CCC value (retain fractions).
    \item If CCC value is < than 1.0, only one VFP may be used in game-turn and it gains only the fractional altitude level.
    \item If CCC value ≥ 1.0 but ≤ 2.0, up to 2/3s of the FPs may be VFPs. The first VFP gains any listed fraction (or 1 if no fractions listed), and the rest gain one altitude level each.
    \item If CCC value is > 2.0, up to 2/3s of the FPs may be VFPs. The first VFP gains 1.0 level \plus{} any fraction and the rest may gain 1.0 or 2.0 altitude levels each.
    \itemaddedin{2A}{2A-super-climbs}{If CCC value ≥ 6.0, one of the VFPs after the first can gain 1, 2, or 3 altitude levels.}
    \item If enough VFPs exist, an aircraft may climb more levels than listed on the CCC. However, the extra levels climbed cause decel as if zoom climbing.
    \item Only EZ turns and Slide maneuvers allowed.
    \item 0.5 decel is incurred for each level up to the CCC limit. Extra levels incur decel as for zoom climbing\addedin{2A}{2A-zoom-climbs}{\ at 1.0 DP per level climbed.}.
\end{enumerate}

\subheading{Vertical Climbs}
\begin{enumerate}[nosep]
    \item Previous game-turn must have involved climbing flight.
    \item Exception; High Pitch Rate aircraft may enter vertical climbs from level flight if speed < 4.0.
    \item On first turn of vertical climb, {\onethird} of FPs must be HFPs. If vertical climb continued, not more than {\onethird} of FPs may be HFPs and up to all may be VFPs.
    \item Each VFP may gain 1.0 or 2.0 altitude levels each.
    \item Each level climbed causes \changedin{2A}{2A-vertical-climbs}{2.0}{1.5} decel points.
    \item No turns or maneuvers except Vertical Rolls allowed.
    \item Diving flight may not follow Vertical climbs.
    \item Exception, High Pitch Rate aircraft may enter Steep Dives or Unloaded Dives the turn after.
    \item Exception, normal aircraft may use a Half-Roll and Dive maneuver to enter Steep Dives after a Vertical Climb.
\end{enumerate}


\end{multicols}
\\
\bottomrule
\end{tabularx}
\end{twocolumntable}
\end{twocolumntablefloat}

\begin{twocolumntablefloat}
\begin{twocolumntable}

\newcommand{\heading}[1]{\medskip\par\textbf{\MakeUppercase{#1}}\par\smallskip}
\newcommand{\subheading}[1]{\smallskip\par\textbf{#1}\par\smallskip}

\tablecaptioncontinued{table:aircraft-flight-rules-summary}{Aircraft Flight Rules Summary}
\footnotesize
\begin{tabularx}{\linewidth}{P}
\toprule
\begin{multicols}{2}

\heading{Diving Flight}

\subheading{Steep Dives}

\begin{enumerate}[nosep]
    \item At least one FP must be and up to 2/3s FPs may be VFPs.
    \item Each VFP may Lose 1.0 or 2.0 altitude levels.
    \itemdeletedin{2A}{2A-steep-dives}{Each level dived gains 0.5 accel on the first turn of Diving.}
    \itemdeletedin{2A}{2A-steep-dives}{If this is the second or subsequent turn of
    continuous Diving, each level dive gains 1.0 accel.}
    \itemaddedin{2A}{2A-steep-dives}{1.0 AP per level.}
\end{enumerate}

\subheading{Unloaded Dives}

\changedin{XX}{XX-unloaded-dives}{
\begin{enumerate}[nosep]
    \item \changedin{2A}{2A-unloaded-dives}{All FPs are HFPs.}{One or two FPs are VFPs. The rest are HFPs.}
    \item \changedin{2A}{2A-unloaded-dives}{At least 1 HFP must be expended with the aircraft “unloaded”. More than 1 and up to all may be expended “unloaded”.}{The first VFP may only be used after half of the HFPs have been expended with the aircraft unloaded. The second VPF may only be used after all of the HFPs have been expended with the aircraft unloaded.}
    \item \changedin{2A}{2A-unloaded-dives}{Each HFP expended while unloaded moves the aircraft forward one hex/hexside and loses it one altitude level.}{The aircraft loses 1 level on each VFP.}
    \item \changedin{2A}{2A-unloaded-dives}{The aircraft gains accel as if Steep Diving.}{The aircraft gains 1.0 AP per level lost.}
    \item The aircraft may not make any attacks, guide weapons or aim while unloaded.
    \item \changedin{2A}{2A-unloaded-dives}{FPs done while unloaded may not be used for turning or prep-moving.}{Unloaded FPs may not be used for turning flight or for preparing or executing any maneuver except a slide.}
    \item All unloaded \changedin{2A}{2A-unloaded-dives}{HFPs}{FPs} done in a single game-turn must be done in one continuous string.
\end{enumerate}
}{
\begin{enumerate}[nosep]
    \item All FPs are HFPs.
    \item At least half of the HFPs (round down) must be expended with the aircraft “unloaded”. All unloaded HFPs must be consecutive.
    \item The aircraft loses on altitude level after half of the HFPs (round down) have expended unloaded. If all of the HFPS are expended unloaded, the aircraft loses another altitude level after the last HFP.
    \item The aircraft gains 1.0 AP per level lost.
    \item The aircraft may not make any attacks, aim, track targets, launch or guide weapons or use radar while unloaded and until it completes a recovery period.
    \item Unloaded FPs may not be used for turning flight or for preparing or executing any maneuver except a slide.
\end{enumerate}
}

\subheading{Vertical Dives}

\begin{enumerate}[nosep]
    \item Previous game turn must have involved diving flight.
    \item Exceptions: a vertical dive may be entered from level flight using a Half Roll and Dive maneuver. If start speed ≤ 4.0, it may also be entered from a zoom or sustained climb by using a Half Roll and Dive maneuver.
    \item On first turn of vertical diving, 1/3 of FPs must be HFPs. If vertical dive continued, no more than 1/3 of FPs may be HFPs and up to all may be VFPs.
    \item Each VFP must lose 2.0 or 3.0 altitude levels.
    \item Each altitude dived gains 1.0 accel.
    \item No turns or maneuvers except vertical rolls allowed.
    \item Climbing flight may never follow vertical dives.
    \item Level flight may follow if A/C's new start speed is 3.0 or less for High Pitch Rate aircraft, or 2.0 or less for others.
    \item[--] If case 8 does not apply, diving flight must follow vertical dive.
    \item When Steep or Unloaded dives follow a vertical dive; at least half an aircraft's FPs (round down) must be expended as VFPs or Unloaded HFPs; except High Pitch Rate aircraft need only expend 1/3 FPs as VFPs or unloaded HFPs.
\end{enumerate}

\heading{Stalled Flight}

\begin{enumerate}[nosep]
    \item Aircraft does not move or change facing.
    \item Altitude lose = start speed (round 0.5 up) + 1.0; increase loss by 1.0 per additional turn of stalled flight.
    \item Aircraft gains accel as it steep diving and by power.
    \item Aircraft may recover to level or diving flight including immediately entering a vertical dive.
\end{enumerate}

\heading{Departed Flight}

\begin{enumerate}[nosep]
    \item Stay in same hex; randomly change facing left or right.
    \item Roll die to find number of facing changes in that direction.
    \item Altitude loss = start speed (round 0.5 up) + 2.0; increase altitude loss by 2.0 per additional turn of departed flight.
    \item Power has no effect, all accel/decel = 0 whole departed.
    \item Recovery is via recovery roll (\minusafter{6} including modifiers).
    \item Upon recovery aircraft must enter diving flight (vertical dives allowed). High Pitch Rate aircraft may recover to level flight.
    \item Upon recovery, start speed reverts to higher of Minimum speed or speed at which departure occurred.
\end{enumerate}

\heading{Aircraft Maneuvers}

\subheading{Slides}

\begin{enumerate}[nosep]
    \item Expend two HFPs to prep for slide. One HFP to execute.
    \item 1 slide allowed if speed ≤ 9.0, two if speed > 9.0 but at least 4 FPs must be expended between execution of first and start of preps for second.
    \item One slide causes no decel; two slides cause 1.0 decel.
\end{enumerate}

\subheading{Lag/Displacement Rolls}

\begin{enumerate}[nosep]
    \item Expend \changedin{2A}{2A-roll-preparatory-fps}{one HFP}{FPs equal to {\onethird} of speed (round down)} to prep for rolls. One HFP to execute.
    \item Shift in direction of roll (see \changedin{1B}{1B-figures}{diagram}{Figures~\ref{figure:displacement-roll-maneuvers} and \ref{figure:lag-roll-maneuvers}}) and optionally face 30 degrees in direction opposite to roll.
    \item A displacement roll from a hexside shifts the aircraft to a hexside as in a slide and not sideways as depicted for the lag roll. Decel for these rolls varies, see ADC.
\end{enumerate}

\subheading{Vertical Rolls}

\begin{enumerate}[nosep]
    \item Aircraft must be in vertical climb or dive and must have just expended a VFP.
    \item Change facing left or right up to 180 degrees.
    \item Decel cost varies; see ADC.
    \item Multiple vertical rolls allowed in a single game turn but each must occur after separate VFP expenditures.
\end{enumerate}

\subheading{Barrel Rolls}

\begin{enumerate}[nosep]
    \item Executed as 2 or more consecutive Lag/Displacement rolls.
    \item If done in level flight, 1 altitude level may be gained or lost upon executing last roll at no additional FP code.
    \itemaddedin{1C}{1C-apj-39-qa}{If done in climbing or diving flight, 1 altitude level may be gained or lost upon executing each roll after the first at no additional FP cost.}
    \item Altitude changes that occur in a diving or climbing B-Roll may be in lieu of, or in conjunction with altitude changes done via VFP expenditure.
    \item Incur 2.0 decel per level gained in climbing Barrel Roll, and gain 0.5 accel per altitude level lost in a Barrel Roll.
\end{enumerate}

\subheading{Half Roll and Dive}

\begin{enumerate}[nosep]
    \item Declare at start of move, perform normal Vertical dive except no vertical rolls allowed until last FP expended and then only if it was a VFP.
    \item Allow vertical dive entry from level flight, ot if speed ≤ 4.0 allows entry from zoom/sustained climbs.
    \item Allows steep dive entry from vertical climbs, with normal turning allowed.
    \item No attacks or weapon launches allowed that turn.
\end{enumerate}

\begin{itemize}[nosep]
    \item For purposes of weapons launch modifiers and gunsights, rolls count as BT turns until recovery period met.
    \item Incur 1.0 extra decel for each roll over one executed in a signle game-turn.
\end{itemize}

\heading{VIFF Maneuvers (VIFF Capable aircraft only)}

\subheading{VIFF Sidestep}

\begin{enumerate}[nosep]
    \item Executed as a slide except no prep-moves required but those imposed by altitude and supersonic speed.
    \item Multiple sidesteps allowed so long as 1 HFP expended in forward flight between execution of each sidestep.
    \item Each costs two HFPs to execute and each causes 2.0 decel.
\end{enumerate}

\subheading{VIFF Assisted Turn}

\begin{enumerate}[nosep]
    \item Reduce listed turn requirements by one (90 is best allowed).
    \item Treat aircraft as High Bleed Rate, incur 2.0 to use.
\end{enumerate}

\subheading{VIFF Vertical Pitch}

\begin{enumerate}[nosep]
    \item Treat as Half Roll and Dive except aircraft may go from vertical climb direct to vertical dive, incur 2.0 decel.
\end{enumerate}

\subheading{VIFF Pop-up}

\begin{enumerate}[nosep]
    \item Allows gain of one Altitude Level from level flight once per turn.
    \item Costs 1 HFP, incurs 2.0 decel, aircraft must be wings level.
\end{enumerate}

\end{multicols}
\\
\bottomrule
\end{tabularx}
\end{twocolumntable}
\end{twocolumntablefloat}

}

An aircraft is flown, in game turns, by moving it horizontally across the game maps and tracking its changes in speed and altitude on an Aircraft Log Sheet.\addedin{1B}{1B-tables}{\ The aircraft flight rules are summarized in Table~\ref{table:aircraft-flight-rules-summary} and detailed in the following chapters.} An aircraft is allowed to move and make changes in altitude by expending Flight Points (FPs). Generally, one flight point moves the aircraft one hex on the map, or up or down one or more levels of altitude. The speed of an aircraft determines how many flight points it has each turn.


\section{The Aircraft Log}
\label{rule:aircraft-log-sheets}

An aircraft log sheet is used to record the starting speed and altitude of an aircraft each turn. Different actions an aircraft may take will affect its speed and altitude from one turn to the next. The different lines on the log provide spaces to record these actions and to calculate resultant changes in speed and altitude. A pad of generic aircraft log sheets is provided with each game. One log should be used for each aircraft in play. A log sheet is divided into 15 columns, enough to record 15 turns of play (about the average length of a game). If you run short you may make copies of the sheets. Where applicable, the various rules sections which follow will provide additional information on using the log sheet.

\paragraph{Set Up Information.} 
When a scenario is set up, each aircraft must be given a start altitude level, and speed. These are noted in lines 1 and 2 of column one of the log sheets. Any scenarios provided in the game will establish these values in their set up instructions. If you create your own scenarios, you will have to decide on these values yourself. Once aircraft and any other required counters are placed on the maps, and the start speeds and altitudes are noted, play may commence.

\paragraph{Using The Log Sheet.} 
As an aircraft is maneuvered, it might change altitude and/or accumulate accel and decel points. The accumulation of accel and/or decel Points may cause changes in the aircraft's speed so spaces are provided on the log to record these points. Spaces are also provided to calculate changes in an aircraft's speed and/or altitude. The final altitude and speed at the end of one game turn becomes the new start speed and altitude for the next game turn.

\section{Flight Points}
\label{rule:flight-points}

The speed of an aircraft is always expressed in terms of Flight Points (FPs), each of which equates to 100 mph of speed. An aircraft with a speed of 5.5 would be traveling at 550 mph. Flight Points are expended to move an aircraft across the game map and/or to change altitude.  An aircraft must expend all whole FPs available to it each game turn. Any unused 0.5 FP remaining can be ignored. An FP may be either a horizontal FP or a vertical FP depending on how it is used.

\paragraph{Horizontal Flight Points.} 
FPs expended to move horizontally across the map are called Horizontal Flight Points (HFPs). One HFP is spent to move an aircraft forward one hex or hexside. It may only fly onto a hexside if it faces parallel to that hexside (see 3.1).

\paragraph{Vertical Flight Points.} 
FPs expended to gain or lose altitude levels are called Vertical Flight Points (VFPs). The amount of altitude levels gained or lost with the expenditure of each VFP varies with the exact type of climb or dive in use.  The amount of FPs that may be VFPs in a game turn also depends on the exact type of climbing or diving flight chosen.

Note: To clarify; both HFPs and VFPs will be available to be expended within a single game turn only when an aircraft elects to climb or dive during its move.

\silentlyaddedin{1B}{1B-tables}{
    \begin{onecolumntable}
\tablecaption{table:fractions}{{\onethird}-{\twothirds} Conversions}
\begin{tabular}{rrr}
\toprule
Base&{\onethird}&{\twothirds}\\
\midrule
1.0&0.5&0.5\\
1.5&0.5&1.0\\
2.0&1.0&1.0\\
2.5&1.0&1.5\\
3.0&1.0&2.0\\
3.5&1.0&2.5\\
4.0&1.0&3.0\\
4.5&1.5&3.0\\
5.0&2.0&3.0\\
5.5&2.0&3.5\\
6.0&2.0&4.0\\
6.5&2.0&4.5\\
7.0&2.0&5.0\\
7.5&2.5&5.0\\
8.0&3.0&5.0\\
8.5&3.0&5.5\\
9.0&3.0&6.0\\
9.5&3.0&6.5\\
10.0&3.0&7.0\\
10.5&3.5&7.0\\
11.0&4.0&7.0\\
11.5&4.0&7.5\\
12.0&4.0&8.0\\
12.5&4.0&8.5\\
13.0&4.0&9.0\\
13.5&4.5&9.0\\
14.0&5.0&9.0\\
14.5&5.0&9.5\\
15.0&5.0&10.0\\
\bottomrule
\end{tabular}
\end{onecolumntable}

}

\addedin{1C}{1C-apj-36-errata}{When an expenditure of 2/3 of an aircraft's speed (or FPs) is required, use the 2/3 entry from \changedin{1B}{1B-tables}{the chart}{Table~\ref{table:fractions}}, not \binarymultiply{2}{1/3}. Drop all fractions read off the chart.}

\addedin{2B}{2B-fractions}{Table~\ref{table:fractions} is used for:
\begin{itemize}
\item Determining the number of HFPs and VFPs required by a given flight type.
\item Engine thrust.
\item SSGT requirements
\item Aiming requirements and modifiers
\end{itemize}
The values in the table follow the rule: round down if the fraction is less than 0.5; round up if the fraction is more than 0.5; and do not round if the fraction is 0.5.

In other cases, multiply by the fraction and round as directed.
}

\addedin{1C}{1C-apj-36-errata}{\paragraph{Speed Requirements Split Between Game Turns.} Some game activities, like ground-attack aiming, are measured in fractions of an aircraft's speed. If such a period extends between two game-turns, the highest speed in either game-turn is used to compute the completion requirements in the second turn.}


\addedin{3A}{3A-fp-stages}{
\paragraph{FP Stages.}

The activities around the expenditure of each FP occur in stages the following order:
\begin{enumerate}
\item Preemption Stage. Any other aircraft threatened by the moving aircraft can declare and execute a preemption.
\item AAA Tracking Stage. If the moving aircraft is within tracking range of any AAA units, any of these may declare that they are tracking the aircraft during the following FP.
\item SSGT Tracking Stage. The moving aircraft may declare that this FP will count towards SSGT of a specified target aircraft.
\item Aiming Stage. The moving aircraft may declare that this FP will count towards aiming requirements of a specified ground unit.
\item Declaration Stage. The moving aircraft can declare or stop a turn or a maneuver. Declaring a turn implicitly sets the bank.
\item Expenditure Stage. The moving aircraft expends an HFP or VFP to change position (i.e., map location and altitude) and possibly facing. It may also simultaneously change position or facing by completing a turn, completing a maneuver, using free descent, or using descent from unloaded HFPs. \addedin{3X}{3X-recovery}{For the purposes of recovery, aiming, and tracking, the FP is considered to have been expended as soon as the movement ends.}
\item Bank Stage. If not turning, the moving aircraft may set its bank according to the rules.
\item Jettison Stores Stage. The moving aircraft may jettison external stores. 
\item Missile Attack Stage. If the moving aircraft has the same position as a missile of which it is a target, the missile attacks.
\item Air-to-Air Attack Stage. The moving aircraft may carry out an air-to-air gun and rocket attack. If the moving aircraft carried out a head-on attack, the target aircraft may simultaneously return fire. 
\item Air-to-Ground and Ground-to-Air Attack Stage. 
\begin{enumerate}
\item[11a.] The moving aircraft may declare an air-to-ground attack. 
\item[11b.] AAA units may declare attacks on the moving aircraft.
\item[11c.] Attacks are resolved simultaneously.
\end{enumerate}
\item Head-On Collision Stage. If the moving aircraft carried out a head-on attack, check for collision.
\item Emergency Egress Stage. Crew may bail out or eject.
\end{enumerate}
}

\section{Types of Flight}

At the start of an aircraft's move, it must commit itself to one of the following three general types of flight: Level, Climbing or Diving. Write the appropriate code in the flight type line of the aircraft log. This represents the aircraft committing its nose to remain level, to move up, or to move down for the game turn. The flight type may not be changed until the next turn. Only one type of flight may be performed each turn and some restrictions may apply when switching from one type to another between turns.

\paragraph{Level Flight.} 
In level flight, all FPs must be expended as HFPs (for example, HHHHHH). The code for Level Flight is "LVL". Level flight is assumed if allowed and not otherwise specified by a player when he begins to move an aircraft.

Example Of Level Flight: An aircraft has a speed of 3.0. The player moves the aircraft through three hexes as shown \changedin{1B}{1B-figures}{below}{in Figure~\ref{figure:level-flight}.}\deletedin{1B}{1B-figures}{\addedin{1C}{1C-apj-23-errata}{\ [The diagram actually shows the left aircraft moving four hexes/hexsides. This is in error.]}} Its turn is over when it has finished moving.

\silentlyaddedin{1B}{1B-figures}{
    \begin{tikzfigure}{0.5\linewidth}

    \drawhexgrid{5}{5.0}  

    \drawaircraftcounter{2.50}{1.25}{120}{F-105}{}
    \drawaircraftcounter{2.00}{2.00}{120}{F-105}{}
    \drawaircraftcounter{1.50}{2.75}{120}{F-105}{}
    \drawaircraftcounter{1.00}{3.50}{120}{F-105}{}

    \drawaircraftcounter{4.00}{1.00}{90}{MiG-21}{}
    \drawaircraftcounter{4.00}{2.00}{90}{MiG-21}{}
    \drawaircraftcounter{4.00}{3.00}{90}{MiG-21}{}
    \drawaircraftcounter{4.00}{4.00}{90}{MiG-21}{}

    \begin{scope}[shift={(165:0.3)},thick,->]
        \miniathex{2.50}{1.25}{\draw (120:0.05) -- (120:0.4);}
        \miniathex{2.00}{2.00}{\draw (120:0.05) -- (120:0.4);}
        \miniathex{1.50}{2.75}{\draw (120:0.05) -- (120:0.4);}
    \end{scope}
    \begin{scope}[shift={(135:0.3)},thick,->]
        \miniathex{4.00}{1.00}{\draw (90:0.1) -- (90:0.5);}
        \miniathex{4.00}{2.00}{\draw (90:0.1) -- (90:0.5);}
        \miniathex{4.00}{3.00}{\draw (90:0.1) -- (90:0.5);}
    \end{scope}

\end{tikzfigure}

}

\paragraph{Climbing Flight.} 
In climbing flight, the aircraft selects a specific type, either a Sustained, Zoom, or Vertical climb, and determines the altitude gain in levels allowed for each VFP expended. A mix of HFPs and VFPs can then be expended subject to the limits outlined in the Climbing rules of section 8.

\paragraph{Diving Flight.} 
In diving flight, the aircraft selects a specific type, either an Unloaded, Steep, or Vertical Dive, and determines the altitude loss in levels allowed for each VFP expended. As with Climbing flight, the Diving Flight rules include limits on how many of each type of FP may be expended in a turn depending on the type of dive but some mix of HFPs and VFPs can be expended.

\paragraph{Expending Mixed FPs.}
HFPs and VFPs may be intermixed and expended in any order so long as any limits for the aircraft's actual climb or dive type chosen are adhered to.

Example Of Altitude Changing Flight: Assume an aircraft with a speed of 7.0 is allowed to expend up to two thirds of its FPs as VFPs. The player could have up to five VFPs and two HFPs but elects only to use three of its FPs as VFPs and the rest as HFPs. He may expend the seven FPs as follows:
HHHHVVV, or VVVHHHH, or HHVVHVH, or in any other order desired.

Note: No matter how the FPs are mixed, the aircraft will only be moved horizontally four hexes as all other FPs will be used to change altitude.


\paragraph{Abnormal Flight.} 
Two other types of flight are possible. These are Stalled and Departed Flight (see rule 6.4).  An aircraft in abnormal flight may not select Level, Climbing, or Diving flight until recovered from the abnormal condition.

\begin{advancedrules}

\section{Half Flight Points}
\label{rule:half-fps}

\notein{1C}{FH in APJ 36 states “If an A/C has a fractional speed and a 0.5 FP carry available, it must use the carry to create a full HFP or VFP.” I don't see what needs changing in the text of the rules.}

Rather than ignoring left over half flight points, they may take into account as explained below.

\paragraph{Half FPs.} 
Any 0.5 VFP or 0.5 HFP not spent during a game turn is carried forward to the next game turn as a generic half FP. This is noted in the 0.5 FP Carry line of the Aircraft Log of the coming game turn. The existence of a carried half FP does not change the aircraft's new start speed for the next turn, but, if the new start speed has a half FP in it, it will marry up with any carried half FP to provide another whole FP.

For example, an aircraft with a start speed of 6.5 having a carried 0.5 FP in the Carry line of the Aircraft Log has 7.0 FPs to expend that game turn. For all applicable game purposes, its speed is still 6.5. If a 0.5 FP carry cannot be mated to a half FP in the start speed, it may be carried forward again (and again) until used. When used, it is gone.

\section{FP Expenditure Restrictions}
\label{rule:changing-flight-type}

To reflect the distance traveled forward while an aircraft is raising or lowering its nose, before any altitude change can occur, use the following restrictions.

\paragraph{After Level Flight.} 
If an aircraft chooses Diving or Climbing flight, and in the previous game turn it used Level flight, the first FP used in the current game turn must be an HFP. The remaining FPs may be mixed normally.

\paragraph{After Climbing or Diving:} 
If an aircraft chooses Diving or Climbing flight, and in the previous game turn it used the opposite (i.e., last turn = climb, this turn = dive), then HFPs equal to at least 1/2 the aircraft's speed (round down) must be expended before any VFPs can be used (representing distance flown while reversing nose attitude). Exception: High Pitch Rate capable aircraft need only expend HFPs equal to 1/3 their speed (round down) before using VFPs.

\paragraph{Same Flight Type.}
If an aircraft continues a climbing or diving type of flight from one turn to the next, then its HFPs and VFPs may be expended in any order.

\section{Formation Flying}

A formation is in effect when one or more aircraft fly and/or operate together as a unit while in close proximity to each other. Formations usually allow for better teamwork.

\paragraph{Formation Types.} Two types of formations are possible: Close and Tactical. Close formations are designated by stacking aircraft in the same hex. Tactical formations exist whenever certain spacing conditions are met.

\paragraph{Formation Size.} Formations can be of the following sizes:

\begin{itemize}
    \item Section (or Element): Two aircraft; one is the leader and the other is the wingman.
    \item Division (or Flight): Three or four aircraft. One is the leader, and the others are wingmen. Alternately, a division may consist of two sections.
\end{itemize}

\paragraph{Formation Leaders.} Each formation must have a designated leader. Leaders are chosen at the start of play or indicated in the scenarios.  If not defined in the scenarios, one division leader is allowed for each four jets in play, and one section leader for each two jets. A division leader doubles as a section leader. During play, formations may split or form according to the leaders available.

\paragraph{Loss of Formation Leaders.} Wingmen always begin play as part of a particular formation. If their original formation leader is lost and no other qualified leader in that formation exists, that formation is dissolved. The former wingmen may join other formations by moving into formation parameters on those leaders. They will not get the initiative benefits but do avoid any penalty for not being in formation.

\subsection{Close Formations}
\label{rule:close-formations}

Up to four aircraft may stack together in the same hex at the same altitude as a close formation. The close formation stack is moved as a single entity when it is the formation leader's time to move. All aircraft in the slack fly exactly as the leader does and must maintain the same speed, altitude and facing as the leader.

\paragraph{Forming Close Formations.} A close formation may be formed prior to the beginning of a scenario (during set up), or in the Aircraft Admin Phase of a game turn whenever two, three, or four friendly aircraft end up in the same position with the same exact facing, speed and altitude.

\paragraph{Splitting Close Formations.} A close formation may split up by declaring the intent to detach airplanes from the stack when it moves. Aircraft which detach are left in the starting hex while the rest of the formation executes its flight. The detached aircraft are then flown.

For example, a four-plane division wishes to split into two sections. Two airplanes (one being a section leader) detach and are left in place while the original division leader and his remaining wingman move. The two detached aircraft may now move elsewhere as a close formation, or split up individually.

A close formation may contain aircraft of differing types or configurations. When an aircraft in the formation is unable to match the leader's moves or speeds, it must detach itself from the formation. Close formations may place restrictions on the activities of wingmen and on the maneuverability of the leader.

\subsection{Tactical Formations}

A tactical formation exists anytime a designated formation leader and his wingmen meet the following parameters:

\begin{itemize}
    \item When they are within six hexes of each other,
    \item and within three altitude levels of each other,
    \item and the wingmen’s' facings are no more than 60{\deg} different from the leader's facing (left or right),
    \item and the leader is not in the wingman's blind arc.
\end{itemize}

Tactical formations can be formed or broken at any time during play by simply flying the aircraft into or out of these established parameters. A tactical formation cannot have more than four aircraft in it. There are no maneuver or combat restrictions in a tactical formation.

\end{advancedrules}
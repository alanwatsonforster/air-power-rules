%!TEX root = ./rules-working.tex
%LTeX: enabled=false


% Use the full page with 1 cm margins.

\usepackage[
  paper=letterpaper,
  nofoot,nomarginpar,
  includehead,
  margin=1cm
]{geometry}

% Use TeX Gyre Schola for text and math.

%\usepackage[T1]{fontenc}
\usepackage{unicode-math}
\setmainfont[
	BoldFont = texgyreschola-bold.otf,
	BoldItalicFont = texgyreschola-bolditalic.otf,
	ItalicFont = texgyreschola-italic.otf
]{texgyreschola-regular.otf}
\setmathfont{texgyreschola-math.otf}

\usepackage{titletoc}
\usepackage{fancyhdr}
\usepackage{enumitem}
\usepackage{hyperref}
\usepackage{graphicx}
\usepackage{xcolor}
\usepackage{tabularx}
\usepackage{tabulary}
\usepackage{booktabs}
\usepackage{multicol}
\usepackage{microtype}
\usepackage{multirow}
\usepackage{pdflscape}
\usepackage{enotez}
\usepackage{comment}

% Basic spacing.

\setlength{\columnsep}{2em}

\setlength{\parindent}{0em}
\setlength{\parskip}{0.8ex plus 0.3ex}
\newcommand{\numberspace}{\phantom{0}}
\setlength{\multicolsep}{0ex}

\newsavebox{\fitwidthbox}
\newenvironment{fitwidth}[1]{%
    \def\fitwidth{#1}%
    \begin{lrbox}{\fitwidthbox}%
}{%
    \end{lrbox}%
    \resizebox{\fitwidth}{!}{\usebox{\fitwidthbox}}%
}

\newsavebox{\fitheightbox}
\newenvironment{fitheight}[1]{%
    \def\fitheight{#1}%
    \begin{lrbox}{\fitheightbox}%
}{%
    \end{lrbox}%
    \resizebox{!}{\fitheight}{\usebox{\fitheightbox}}%
}

\newlength{\wboxwidth}
\newcommand{\wbox}[3][r]{%
  \settowidth{\wboxwidth}{#2}%
  \makebox[\wboxwidth][#1]{#3}%
}

% Hyphenation

\usepackage{hyphenat}
\hyphenation{add-ed}
\hyphenation{sight-ed un-sight-ed}
\hyphenation{Side-wind-er}
\hyphenation{bore-sight}
\hyphenation{ob-struct-ed un-ob-struct-ed}
\hyphenation{crew-mem-ber crew-mem-bers}
\hyphenation{seek-er}

% Headers

\newcommand{\therunningtitle}{}
\newcommand{\runningtitle}[1]{\renewcommand{\therunningtitle}{#1}}

\pagestyle{fancy}
\setlength{\headsep}{1em}
\fancyhead[L]{\textbf{\therunningtitle}}
\fancyhead[C]{}
\fancyhead[R]{\textbf{Page \thepage}}
\fancyfoot[L,C,R]{}
\setlength{\footskip}{4pt}
\renewcommand{\headruleskip}{0.1em}
\addtolength{\headheight}{\headruleskip}

% Line weights

\setlength{\arrayrulewidth}{0.4pt}
\setlength{\columnseprule}{0.4pt}
\renewcommand{\headrulewidth}{0.4pt}

% Lists

\setlist[itemize]{
    align=left,
    labelwidth=1.2em, 
    labelsep=0.3em,
    leftmargin=1.5em,
    nosep,
    parsep=\parskip,
    label=--,
}
\setlist[enumerate]{
    align=left, 
    leftmargin=1.5em,
    labelwidth=1.2em, 
    labelsep=0.3em,
    parsep=\parskip,
}
\setlist[enumerate,2]{
    label=\alph*.
}

\newcommand{\itemparagraph}[1]{{\bfseries #1}}

% Sections

\newlength{\interwordspace}
\setlength{\interwordspace}{\fontdimen2\font plus \fontdimen3\font
minus \fontdimen4\font}

\setcounter{secnumdepth}{2}


\renewcommand{\chaptername}{Rule}

\newcommand{\rulechapter}[1]{
    \typeout{}
    \typeout{==================================================================}
    \chapter{#1}%
    \addtocontents{toc}{\protect\vspace{-1ex}}%
    \thispagestyle{fancy}%
    \typeout{==================================================================}
    \typeout{}
}
\newcommand{\appendixchapter}[1]{
    \typeout{Appendix \thechapter: #1}
    \chapter{#1}%
    \addtocontents{toc}{\protect\vspace{-1ex}}%
    \thispagestyle{fancy}%
}

\makeatletter
\renewcommand\section{%
    \@startsection{section}{1}%
        {\z@}%
        {-3.5ex \@plus -1ex \@minus -.2ex}%
        {2.3ex \@plus.2ex}%
        {\normalfont\Large\bfseries\raggedright}%
}
\renewcommand\paragraph{%
    \@startsection{paragraph}{4}%
        {\z@}%
        {\parskip}%
        {-\interwordspace}%
        {\normalfont\normalsize\bfseries}%
}
\makeatother

\newcommand{\OLDadvancedrules}{
    \vspace{1ex}
    \pagebreak[1]
    \begin{center}
        \bfseries{\Large Advanced Rules}
    \end{center}
}

\newif\ifbasicrulesonly
\basicrulesonlyfalse

\ifbasicrulesonly
    \excludecomment{advancedrules}
\else
    \newenvironment{advancedrules}{
        \vspace{1ex}
        \pagebreak[1]
        \begin{center}
            \bfseries{\Large Advanced Rules}
        \end{center}
    }{
    }
\fi

\newcommand{\trainingnote}[1]{
  \vspace{1ex}
  \begin{center}
    \framebox{\begin{minipage}{0.9\linewidth}\setlength{\parskip}{\smallskipamount}\smallskip #1\par\smallskip\end{minipage}}
  \end{center}
  \vspace{1ex}
}

\newcommand{\relevantadvancedrules}[1]{
\begin{raggedleft} \emph{Relevant advanced rules: #1}\par
\end{raggedleft}
}

%%%%%%%%%%%%%%%%%%%%%%%%%%%%%%%%%%%%%%%%

% Table of Contents

\usepackage[toc]{multitoc}
\renewcommand*{\multicolumntoc}{2}
\setlength{\columnseprule}{\columnseprule}

%%%%%%%%%%%%%%%%%%%%%%%%%%%%%%%%%%%%%%%%

% Style for the standard title page

\let\originaltitle\title
\renewcommand{\title}[1]{\originaltitle{\bfseries\Large\MakeUppercase{#1}}}
\let\originalauthor\author
\renewcommand{\author}[1]{\originalauthor{\itshape\normalsize{#1}}}
\let\originaldate\date
\renewcommand{\date}[1]{\originaldate{\normalsize{#1}}}

% Abbreviations

\newcommand{\AirSup}{\textit{Air Superiority}}
\newcommand{\AirStr}{\textit{Air Strike}}
\newcommand{\DF}{\textit{Desert Falcons}}
\newcommand{\EOTG}{\textit{Eagles of the Gulf}}
\newcommand{\TSOH}{\textit{The Speed of Heat}}
\newcommand{\AirPow}{\textit{Air Power}}
\newcommand{\APJ}{\textit{Air Power Journal}}
\newcommand{\Harpoon}{\textit{Harpoon}}
\newcommand{\HarpoonFour}{\textit{Harpoon$^4$}}

\newcommand{\FP}  {\text{FP}}
\newcommand{\FPs} {\text{FPs}}
\newcommand{\HFP} {\text{HFP}}
\newcommand{\HFPs}{\text{HFPs}}
\newcommand{\VFP} {\text{VFP}}
\newcommand{\VFPs}{\text{VFPs}}
\newcommand{\CC}  {\text{CC}}

\newcommand{\onethird}{$1/3$}
\newcommand{\onehalf}{$1/2$}
\newcommand{\twothirds}{$2/3$}

\renewcommand{\deg}{\ensuremath{^\circ}}

%%%%%%%%%%%%%%%%%%%%%%%%%%%%%%%%%%%%%%%%

% Aids.

% We use this flag to change formatting according to whether we are in
% the main text or the aids.

\newif\ifaids
\aidsfalse

%%%%%%%%%%%%%%%%%%%%%%%%%%%%%%%%%%%%%%%%

% Floats

\newenvironment{onecolumntablefloat}[1][tp]{
    \ifaids
        \begin{table}[tp]
        \centering
    \else
        \silentlychangedin{1C}{1C-tables}{
            \begin{table}[h!]
        }{
            \begin{table}[#1]
        }
    \fi
}{
    \end{table}
}


\newenvironment{onecolumntable}{
    \begin{minipage}{\columnwidth}
    \centering\small
}{
    \end{minipage}
    \vspace{\floatsep}
}

\newenvironment{twocolumntablefloat}[1][tp]{
    \begin{table*}[#1]
    \setlength{\columnseprule}{0pt}
    \centering
}{
    \end{table*}
}

\newenvironment{twocolumntable}{
    \begin{minipage}{\linewidth}
    \centering\small
}{
    \end{minipage}
}

\newenvironment{fullwidthtable}[1][tp]{
    \begin{table*}[#1]
    \centering
}{
    \end{table*}
}

\newenvironment{onecolumnfigure}[1][tbp]{
    \ifaids
        \begin{figure}[#1]
        \centering
        \begin{minipage}{0.5\linewidth}
    \else
        \silentlychangedin{1C}{1C-figures}{
            \begin{figure}[h!]
        }{
            \begin{figure}[#1]
        }        
    \fi
    \centering
}{
    \ifaids
        \end{minipage}
    \fi
    \end{figure}
}

\newenvironment{twocolumnfigure}[1][tp]{
    \begin{figure*}[#1]
    \centering
}{
    \end{figure*}
}

% Align floats in their own page or column to the top.
% https://texfaq.org/FAQ-vertposfp
\makeatletter
\setlength{\@fptop}{5pt}
\setlength{\@fpbot}{0pt plus 1fil}
\setlength{\@fpsep}{5ex minus 0.1ex}
\makeatother

\setlength{\floatsep}{5ex minus 0.1ex}
\setlength{\textfloatsep}{5ex plus 1fil minus 0.1ex}

%%%%%%%%%%%%%%%%%%%%%%%%%%%%%%%%%%%%%%%%

% Table and figure captions. 

% Some tables appear in both the main text
% and the play aids. Others only in the play aids.

\newcommand{\tablecaption}[2]{
    \silentlychangedin{1C}{1C-tables}{
        #2
    }{
        \ifaids
            \textaddedin{1C}{1C-tables}{Table~\ref{#1}: #2}
        \else
            \textaddedin{1C}{1C-tables}{\caption{#2\label{#1}}}
        \fi
    }
    \par
    \medskip
}

\newcommand{\tablecaptioncontinued}[2]{
    \silentlychangedin{1C}{1C-tables}{
        #2 (continued)
        \par
    }{
        \textaddedin{1C}{1C-tables}{Table~\ref{#1} (continued): #2}
        \par
    }
    \medskip
}

\newcommand{\figurecaption}[2]{
    \ifaids
        \par
        \medskip
        \textaddedin{1C}{1C-figures}{Figure~\ref{#1}: #2}
    \else
        \textaddedin{1C}{1C-figures}{\caption{#2\label{#1}}}
    \fi
}

\newcommand{\simpletablecaption}[1]{
    % This is used for the original unnumbered table captions.
    \medskip
    #1\par
}

\newcommand{\simplefigurecaption}[1]{
    % This is used for the original unnumbered figure captions.
    \medskip
    #1\par
}

%%%%%%%%%%%%%%%%%%%%%%%%%%%%%%%%%%%%%%%%

% Tables

\newcommand{\minitable}[3][c]{\begin{tabular}[#1]{@{}#2@{}}#3\end{tabular}}

\newcommand*\vertical{\rotatebox{90}}

\newlength\samewidthlength
\newcommand\samewidth[3][c]{%
  \settowidth{\samewidthlength}{#2}%
  \makebox[\samewidthlength][#1]{#3}%
}

% New tabularx columns

\newcolumntype{P}{X}%
\newcolumntype{L}{>{\raggedright\arraybackslash}X}%
\newcolumntype{R}{>{\raggedleft\arraybackslash}X}%
\newcolumntype{C}{>{\centering\arraybackslash}X}%
\newcolumntype{q}[1]{>{\centering\arraybackslash\hspace{0pt}}p{#1}}

\newenvironment{tablenote}[1]{%
    \tabularx{#1}{L}%
    \vspace{\parskip}%
}{%
    \endtabularx%
}

\newcommand{\tablenotemark}[1]{\mbox{$^{#1}$}}
\newcommand{\tablenotetext}[2]{\tablenotemark{#1}~#2}

\newcommand{\advancedrulemark}{\tablenotemark{*}}
\newcommand{\advancedruletext}{\tablenotetext{*}{Advanced rule.\space}}

\newcommand{\tablespace}{\\[-0.5ex]}

%%%%%%%%%%%%%%%%%%%%%%%%%%%%%%%%%%%%%%%%

\newcommand{\arc}[1]{\mbox{#1{\deg}}}
\newcommand{\arcrange}[2]{\mbox{$#1{\deg}\!#2$}}
\newcommand{\arcborder}[2]{\mbox{$#1{\deg}/#2{\deg}$}}

%%%%%%%%%%%%%%%%%%%%%%%%%%%%%%%%%%%%%%%%

% Figures

\usepackage{tikz}
\usetikzlibrary{calc}
\usetikzlibrary{arrows.meta}

\tikzset{every picture/.append style={font=\scriptsize,>={Stealth[length=1.5mm,inset=0pt]}}}

% Tableau 10 palette
\definecolor{Tblue}{rgb}{0.122,0.467,0.706}
\definecolor{Torange}{rgb}{1.000,0.498,0.055}
\definecolor{Tgreen}{rgb}{0.173,0.627,0.173}
\definecolor{Tred}{rgb}{0.829,0.153,0.157}
\definecolor{Tpurple}{rgb}{0.580,0.404,0.741}
\definecolor{Tbrown}{rgb}{0.549,0.337,0.294}
\definecolor{Tpink}{rgb}{0.890,0.467,0.761}
\definecolor{Tgray}{rgb}{0.498,0.498,0.498}
\definecolor{Tolive}{rgb}{0.737,0.741,0.133}
\definecolor{Tcyan}{rgb}{0.090,0.745,0.812}

\newcommand{\figureboundingboxcolor}{black!10}

\newcommand{\setfiguresize}[5][\figureboundingboxcolor]{
    \draw[#1] (#2,#3) -- (#2,#5) -- (#4,#5) -- (#4,#3) -- cycle;
    \clip (#2,#3) -- (#2,#5) -- (#4,#5) -- (#4,#3) -- cycle;
}

\newcommand{\drawaircraftcounter}[7][0]{
    % The optional first argument is the current rotation of the
    % picture. It is necessary since Tikz rotates text with respect to
    % the original rotation, not the current rotation.
    \begin{athex}{#2}{#3}
        \begin{scope}[rotate=#4-45]
            \draw[fill=white] (-0.3,-0.3) rectangle (0.3,0.3);
            \node [anchor=center, transform shape] {
                \includegraphics[angle=45,width=0.6cm]{plan-views/#5.png}
            };
            \node [anchor=north west,rotate=#4-45+#1] at (-0.35,+0.35) {\tiny #6};
            \node [anchor=south east,rotate=#4-45+#1] at (+0.35,-0.35) {\tiny #7};
        \end{scope}
    \end{athex}
}

\newcommand{\drawgroundunitcounter}[6][0]{
    % The optional first argument is the current rotation of the
    % picture. It is necessary since Tikz rotates text with respect to
    % the original rotation, not the current rotation.
    \begin{athex}{#2}{#3}
        \begin{scope}[rotate=#4-45]
            \draw[fill=white] (-0.3,-0.3) rectangle (0.3,0.3);
            \begin{scope}[rotate=0,shift={(0,-0.05)}]
                \draw (-0.15,-0.10) -- (-0.15,+0.10) -- (+0.15,+0.10) -- (+0.15,-0.10) -- cycle;
                \draw (-0.15,-0.10) -- (+0.15,+0.10);
                \draw (-0.15,+0.10) -- (+0.15,-0.10);
            \end{scope}
            \node [anchor=north west,rotate=#4-45+#1] at (-0.35,+0.35) {\tiny #5};
            \node [anchor=south east,rotate=#4-45+#1] at (+0.35,-0.35) {\tiny #6};
        \end{scope}
    \end{athex}
}

\newcommand{\drawshipcounter}[6][0]{
    % The optional first argument is the current rotation of the
    % picture. It is necessary since Tikz rotates text with respect to
    % the original rotation, not the current rotation.
    \begin{athex}{#2}{#3}
        \begin{scope}[rotate=#4-45]
            \draw[fill=white] (-0.3,-0.3) rectangle (0.3,0.3);
            \begin{scope}[rotate=-45]
                \draw [fill=black!50] (-0.07,-0.15) -- (-0.07,+0.05) -- (0.0, +0.20) -- (+0.07,+0.05) -- (+0.07,-0.15) -- cycle;
            \end{scope}
            \node [anchor=north west,rotate=#4-45+#1] at (-0.35,+0.35) {\tiny #5};
            \node [anchor=south east,rotate=#4-45+#1] at (+0.35,-0.35) {\tiny #6};
        \end{scope}
    \end{athex}
}

\newcommand{\drawcounter}[6][0]{
    % The optional first argument is the current rotation of the
    % picture. It is necessary since Tikz rotates text with respect to
    % the original rotation, not the current rotation.
    \begin{athex}{#2}{#3}
        \begin{scope}[rotate=#4-45]
            \draw[fill=white] (-0.3,-0.3) rectangle (0.3,0.3);
            \node [anchor=north west,rotate=#4-45+#1] at (-0.35,+0.35) {\tiny #5};
            \node [anchor=south east,rotate=#4-45+#1] at (+0.35,-0.35) {\tiny #6};
        \end{scope}
    \end{athex}
}

\newcommand{\drawdashedcounter}[3]{
    \begin{athex}{#1}{#2}
        \begin{scope}[rotate=#3-45]
            \draw[dashed,fill=white] (-0.3,-0.3) rectangle (0.3,0.3);
        \end{scope}
    \end{athex}
}

\newcommand{\drawarrowcounter}[3]{
    \begin{athex}{#1}{#2}
        \begin{scope}[rotate=#3-45]
            \draw[fill=white] (-0.3,-0.3) rectangle (0.3,0.3);
            \draw[very thick, ->] (-0.2,-0.2) -- (0.2,0.2);
        \end{scope}
    \end{athex}
}


% The number is sqrt(0.75) to three decimal places. It is the distance between column centers of a hex grid in which the difference between row centers is 1.0.
\newcommand{\hexxfactor}{0.866}

\newenvironment{athex}[3][]{
    \begin{scope}[shift={(#2*\hexxfactor,#3)},#1]
}{
    \end{scope}
}
\newcommand{\miniathex}[4][]{\begin{athex}[#1]{#2}{#3}#4\end{athex}}

\newcommand{\drawhex}[3][]{
    \begin{athex}{#2}{#3}
        \begin{scope}[xscale=\hexxfactor]
            \draw [color=black!50,#1] (+0.667,+0.000) -- (+0.333,+0.500) -- (-0.333,+0.500) -- (-0.667,+0.000) -- (-0.333,-0.500) -- (+0.333,-0.500) -- cycle;
        \end{scope}
    \end{athex}
}

\newcommand{\drawposition}[1][]{
    \draw [fill,color=black!50,#1] (0,0) circle (0.03);
}

\newcommand{\drawpositions}[3][]{
    \begin{athex}{#2}{#3}
        \miniathex{+0.00}{+0.00}{\drawposition[#1]}
        \miniathex{+0.00}{+0.50}{\drawposition[#1]}
        \miniathex{+0.50}{+0.25}{\drawposition[#1]}
        \miniathex{+0.50}{-0.25}{\drawposition[#1]}
        \miniathex{+0.00}{-0.50}{\drawposition[#1]}
        \miniathex{-0.50}{-0.25}{\drawposition[#1]}
        \miniathex{-0.50}{+0.25}{\drawposition[#1]}
    \end{athex}
}

\newcommand{\drawmegahex}[2]{
    \begin{athex}{#1}{#2}
        \begin{scope}[yscale=5,xscale=5*\hexxfactor,ultra thick,color=black!25]
            \draw (+0.667,+0.000) -- (+0.333,+0.500) -- (-0.333,+0.500) -- (-0.667,+0.000) -- (-0.333,-0.500) -- (+0.333,-0.500) -- cycle;
        \end{scope}
    \end{athex}
}

\newcommand{\cliphexgrid}[4]{
    \clip (#1*\hexxfactor-0.667*\hexxfactor,#2-0.5-0.01) rectangle (#3*\hexxfactor+0.667*\hexxfactor,#4+0.5+0.01);
}

\newcommand{\drawhexgrid}[4]{
    % All columns are the same length.
    \miniathex{#1}{#2}{
        \foreach \x in {1,3,...,#3} {
            \foreach \y in {1,...,#4} {
                \miniathex{-1.0}{-1.0}{\drawhex{\x}{\y}}
            }
        }
        \foreach \x in {2,4,...,#3} {
            \foreach \y in {1,...,#4} {
                \miniathex{-1.0}{-0.5}{\drawhex{\x}{\y}}
            }
        }
    }
}

\newcommand{\drawevenhexgrid}[4]{
    % Odd columns are one hex shorter than even columns.
    \miniathex{#1}{#2}{
        \foreach \x in {1,3,...,#3} {
            \foreach \y in {1,...,#4} {
                \miniathex{-1.0}{-1.0}{\drawhex{\x}{\y}}
            }
        }
        \foreach \x in {2,4,...,#3} {
            \foreach \y in {2,...,#4} {
                \miniathex{-1.0}{-1.5}{\drawhex{\x}{\y}}
            }
        }
    }
}
\newcommand{\drawoddhexgrid}[4]{
    % Even columns are one hex shorter than odd columns.
    \miniathex{#1}{#2}{
        \foreach \x in {1,3,...,#3} {
            \foreach \y in {2,...,#4} {
                \miniathex{-1.0}{-2.0}{\drawhex{\x}{\y}}
            }
        }
        \foreach \x in {2,4,...,#3} {
            \foreach \y in {1,...,#4} {
                \miniathex{-1.0}{-1.5}{\drawhex{\x}{\y}}
            }
        }
    }
}

\newcommand{\drawpositiongrid}[5][]{
    % All columns are the same length.
    \miniathex{#2}{#3}{
        \foreach \x in {1,3,...,#4} {
            \foreach \y in {1,...,#5} {
                \miniathex{-1.0}{-1.0}{\drawpositions[#1]{\x}{\y}}
            }
        }
        \foreach \x in {2,4,...,#4} {
            \foreach \y in {0,2,...,#5} {
                \miniathex{-1.0}{-0.5}{\drawpositions[#1]{\x}{\y}}
            }
        }
    }
}

\newcommand{\drawevenpositiongrid}[5][]{
    % Odd columns are one hex shorter than even columns.
    \miniathex{#2}{#3}{
        \foreach \x in {1,3,...,#4} {
            \foreach \y in {1,...,#5} {
                \miniathex{-1.0}{-1.0}{\drawpositions[#1]{\x}{\y}}
            }
        }
        \foreach \x in {2,4,...,#4} {
            \foreach \y in {2,...,#5} {
                \miniathex{-1.0}{-1.5}{\drawpositions[#1]{\x}{\y}}
            }
        }
    }
}
\newcommand{\drawoddpositiongrid}[5][]{
    % Even columns are one hex shorter than odd columns.
    \miniathex{#2}{#3}{
        \foreach \x in {1,3,...,#4} {
            \foreach \y in {2,...,#5} {
                \miniathex{-1.0}{-2.0}{\drawpositions[#1]{\x}{\y}}
            }
        }
        \foreach \x in {2,4,...,#4} {
            \foreach \y in {1,...,#5} {
                \miniathex{-1.0}{-1.5}{\drawpositions[#1]{\x}{\y}}
            }
        }
    }
}


\newcommand{\drawdotathex}[3][]{
    \miniathex{#2}{#3}{\draw [fill,color=black,#1] circle (0.1);}
}

\newcommand{\drawpositionathex}[3][]{
    \miniathex{#2}{#3}{\draw [fill,color=black,#1] circle (0.02);}
}

\newlength{\standardhexwidth}
\setlength{\standardhexwidth}{8.3mm}


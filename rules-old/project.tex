%!LW recipe=latexmk (xelatex)
%!TEX program = xelatex

\documentclass[10pt]{report}

\input style.tex

\title{Updated Air Power Rules}
\author{%
    Alan Watson Forster\\[2ex]
    in collaboration with\\[2ex]
    Carl Smeaton, Jean-Baptiste Mouillet, and Stephen Loughrey
}
\date{12 January 2026}
\runningtitle{Updated Rules Project}

\setcounter{secnumdepth}{0}
%\setlist[enumerate,1]{label = (\alph*)}

\newcommand{\changetag}[1]{\textbf{Change: #1.}\par}

\begin{document}

\twocolumn
\thispagestyle{empty}
\maketitle
\suppressfloats
\sloppy

\section{Aims}

We aim to produce updated and rewritten rules for J.D.\ Webster's {\AirPow} game system.

\section{Context}

At the time of writing, several sets of rules are available. All of these, except the last, were written by J.D.\ Webster.

\begin{itemize}
    \item The original rules were published in 1987 in {\AirSup} and {\AirStr}.
    \item These were then extended in {\DF} and {\EOTG}.
    \item The first-edition {\AirPow} rules were published in 1992 in {\TSOH}.
    \item One page of errata was provided with {\TSOH}.
    \item Four pages of errata were published in {\APJ} \#23.
    \item Four pages of corrected play-aids were published in {\APJ} \#24.
    \item There were a number of proposals for second-edition {\AirPow} rules outlined mainly in {\APJ} during the 1990s. However, no official second-edition was ever published.
    \item Malcolm Pipes published an unofficial but nevertheless widely accepted set of second-edition {\AirPow} rules in 2022. These are available at the \href{https://airpower.groups.io/g/main}{airpower email group} site. The current text is labeled “v2.4”.
\end{itemize}

\section{Rationale}

The v2.4 rules are widely accepted. Why are another set of rules necessary?
\begin{itemize}
    \item The v2.4 rules do not fully incorporate previous errata, in particular Felix Hack’s errata published with JD Webster's approval in {\APJ} 36, and have some minor errors.
    \item Where the v2.4 rules incorporate errata, it is often inserted literally rather than changing the text to include the sense of the errata. This can make the rules difficult to read.
    \item There is no traceability. Changes have been made, but there is no way to see what changed or why.
    \item There are no internal hyperlinks for cross-references.
    \item Some of the diagrams have not been fixed.
    \item I think it would be helpful to incorporate diagrams and tables into the text (and also have them on the play-aid sheets). With printed rules, this separation reduces costs and to a large degree is more convenient. When viewed on the screen, the considerations are different.
    \item The original text needs copy editing. It could also benefit from being rewritten or perhaps reorganized.
    \item There are few examples or explanations. Indeed, there is little effort to distinguish examples or explanations in the original text.
    \item The text still refers to pilots, crew, and players using masculine pronouns. That was understandable when the first-edition rules were written, but nevertheless needs updating.
\end{itemize}

\section{Process}

So how do we get there from here? We propose to produce a set of intermediate versions of the rules, each with carefully defined and documented minimal changes. The final intermediate version will contain all rules changes. The last step will be to rewrite and reorganized the final intermediate version without (hopefully) changing the meaning of the rules.

\section{Versions}

We use versions labelled with a number and a letter to distinguish my efforts from others. Since the versions are a work in progress, they are also distinguished by date.

There are several versions of the rules and play aids, each based on the previous one:

\begin{itemize}
    \item {\bfseries Version 1A}\nopagebreak

        This corresponds to the first-edition {\AirPow} rules without any inline tables or figures, without any of the tables or figures from the play aids, and without any errata. It corresponds to the original printed text of the rules.

    \item {\bfseries Version 1B}\nopagebreak

        This version adds tables and redrawn figures from the text and play aids of the first-edition {\AirPow} rules. No corrections from errata are applied. The tables and figures are incorporated as floats in the body of the text  and are also compiled in appendix B.

    \item {\bfseries Version 1C}\nopagebreak

        This version incorporates errata taken from authoritative sources, such as the published errata, the “Q\&A” columns by J.D.\ Webster in {\APJ}, other public statements by J.D.\ Webster, and Felix Hack's article in {\APJ} 36.

    \item {\bfseries Version 2A}\nopagebreak

        This version incorporates second-edition changes from {\APJ}, email messages from JD Webster, and Malcolm Pipe's second-edition rules. Not all changes have been adopted since some seem to have issues.

    \item {\bfseries Version 2B}\nopagebreak

        This version has further, minor, non-authoritative changes to clarify and fill apparent omissions in version 2A. The changes are in collaboration with Carl Smeaton, Jean-Baptiste Mouillet, and Stephen Loughrey.

    \item {\bfseries Version 3A}\nopagebreak

        This version has major non-authoritative changes. Again, the changes are in collaboration with Carl Smeaton.

    \item {\bfseries Version 3B}\nopagebreak

        This version is a rewrite. Certain rules have been rewritten for clarity and conciseness. Parts of the text have been reorganized. However, the meaning of the version 3B rules should be the same as the 3A rules.
\end{itemize}

Having multiple versions simplifies reviewing changes. For example, the changes in 1B and 2A come from authoritative sources, but the changes in 2B and 3A are from non-authoritative sources.

The changes in version 3A and 3B are sufficiently significant that it may be necessary to no longer refer to these as versions of the {\AirPow} rules but instead adopt a different name. For the moment, since these versions are only being using for limited private testing, we are delaying resolving this issue.

\section{Which version should you use?}

\begin{itemize}
    \item If you wish to play with second-edition rules with changes only from authoritative sources, I would suggest using version 2A. However, you will miss out on the minor, but non-authoritative, additions in version 2B that cover aspects missing from version 2A.
    \item If you wish to play with second-edition rules and are prepared to accept non-authoritative minor additions, I would suggest using version 2B.
    \item Carl and Alan use version 3A/3B, as we consider it has significant improvements over 2B. However, these changes are neither minor nor authoritative.
\end{itemize}

\section{Products}

We produce two PDF files for each version. One shows the changes with respect to the previous version, with deletions in blue, additions in red, and end-notes that reference the change tag. The aids in understanding the origin of the changes and evolution of the rules. We also produce a clean version that is suitable for actual play.

\section{Rules Changes}

This section describes the rules changes between different versions.

All changes have \emph{change tags}, such as “1C-tsoh-errata”. The first part of the tag refers to the version to which it applies. The second part is a brief descriptive name. All the change tags are listed in this document with fuller descriptions and justifications. These are the primary means of providing traceability.

\subsection{Version 1B}

Version 1B has the following changes:

\begin{itemize}

    \item\changetag{1B-cover}

        The cover features a USN photograph of an A-7A. This appears to have been the inspiration for the line drawing on the cover of the first-edition {\AirPow} rules.

    \item\changetag{1B-credits}

        Additions to the credits.

    \item\changetag{1B-figures}

        All figures have been redrawn.

        Figure 12 (SSGT) now shows a range of 6 hexes. Figure 12 (Genie scatter) now shows the scatter for a Genie centered on a hex and pointing to a hex corner.

        All figures have been incorporated into the body of the rules. Certain figures also continue to appear in the play aids.

        Reference to figures now follow the pattern “Figure 1” rather than “Angle-Off Diagrams”.

    \item\changetag{1B-tables}

        All tables have been incorporated into the body of the rules and also appear in the play aids.

        Reference to tables now follow the pattern “Table 1” rather than “IRM Seeker Head Table”.

\end{itemize}

\subsection{Version 1C}

Version 1B is an update to the text, tables, and figures of version 1B according to errata from authoritative sources. There are changes to the sense of some rules compared to version 1B, but the changes to the text are minimal. It has the following changes:

\begin{itemize}

    \item\changetag{1C-credits}

        Additions to the credits.

    \item\changetag{1C-tsoh-errata}

        All changes in the single sheet of errata included with {\TSOH}, specifically: the replacement of one of the panels Figure 9.2 (Horizontal Arcs); additions of new Tables 19.1 (Internal DDS) with rules for changing PPLs; and 20.1 (Terrain Effects); and a new note to Figure 19.1 (Jamming Cell Formations).

    \item\changetag{1C-apj-22-damage-tables}

        The optional advanced damage tables from APJ 22.

    \item\changetag{1C-original-play-aids}

        All changes to the rules to agree with the original play aids.

    \item\changetag{1C-apj-23-errata}

        All changes in the four sheets of errata published in APJ 23.
        In addition to changes to the text and tables, there are corrections to Figures 1.2 (Game Maps) and 5.1 (Level Flight).

    \item\changetag{1C-apj-24-play-aids}

        All changes in the revised play aids published in APJ 24.

    \item\changetag{1C-apj-36-errata}

        All changes in Felix Hack’s list of errata published in APJ 36, except for the change to sustained turning being assessed per 30 degrees of facing change, which was later clarified as a second-edition change.

    \item\changetag{1C-apj-5-qa}

    \item\changetag{1C-apj-10-qa}

    \item\changetag{1C-apj-12-qa}

    \item\changetag{1C-apj-17-qa}

    \item\changetag{1C-apj-20-qa}

    \item\changetag{1C-apj-21-qa}

    \item\changetag{1C-apj-22-qa}

    \item\changetag{1C-apj-23-qa}

    \item\changetag{1C-apj-25-qa}

    \item\changetag{1C-apj-26-qa}

    \item\changetag{1C-apj-27-qa}

    \item\changetag{1C-apj-28-qa}

    \item\changetag{1C-apj-30-qa}

    \item\changetag{1C-apj-34-qa}

    \item\changetag{1C-apj-35-qa}

    \item\changetag{1C-apj-36-qa}

    \item\changetag{1C-apj-37-qa}

    \item\changetag{1C-apj-38-qa}

    \item\changetag{1C-apj-39-qa}

        Changes and clarifications from JD Webster’s Q\&A articles in {\APJ} 5, 10, 12, 17, 20, 21, 22, 23, 25, 26, 27, 28, 30, 34, 35, 36, 37, 38, and 39.

        Some comments:

        \begin{itemize}
            \item The Q\&A in APJ 23 states that EWRs search in step 7 of the SAM Interaction Phase. The errata in APJ 23 state that EWRs search in the Air Radar Search Phase. The main difference would be that if they search in the SAM Interaction phase, they could be reactivated and search immediately, which seems unlikely. Therefore, the rule from the errata is adopted.
        \end{itemize}

    \item\changetag{1C-tailing}

        A correction to the rule on tailing following \href{https://airpower.groups.io/g/main/message/2847}{an email message} by JD Webster to the airpower group.

    \item\changetag{1C-missile-launch-speed}

        A missile's start speed on its first game turn of flight is based on the launching aircraft's start speed on the game turn of launch. This was confirmed by Tony Valle in \href{https://airpower.groups.io/g/main/message/3104}{an email message} to the airpower group.

    \item\changetag{1A-identifying-at-night}

        Identifying an aircraft at night requires matching its position, facing, and speed. This appears in the v1 sheets, but not in the v1 rules.

\end{itemize}

\subsection{Version 2A}

Version 2A incorporates the following changes. There are changes to the sense of the rules compared to version 1C, but these changes come from authoritative sources. The changes to the text are minimal.

\begin{itemize}

    \item\changetag{2A-credits}

        Additions to the credits.

    \item\changetag{2A-adc}

        The first-edition F-4B/C ADC is replaced with the second-edition F-4F ADC from {\APJ}~44. Also, an appendix gives instructions for converting ADCs from first-edition to second-edition.

    \item\changetag{2A-idle}

        An aircraft that selects idle power no longer has its start speed reduced. Instead, the aircraft incurs DPs for normal power plus the additional DPs listed in the ADC. This change is taken from the Origins rules in {\APJ}~39, 41, 44, and 53 and the v2.4 rules.

    \item\changetag{2A-spbr}

        Similarly, an aircraft that uses speedbrakes no longer loses FPs. Instead, the aircraft incurs DPs up to the value shown in the ADC. If the aircraft is supersonic, the maximum is increased by 1 DP. This change is taken from the Origins rules in {\APJ}~39, 41, 44, and 53 and the v2.4 rules.

    \item\changetag{2A-fp-to-dp}

        The conversion factor from FPs to DPs for the new idle power and speedbrake rules is 2. This change appears in the Origins rules in {\APJ}~41 and the v2.4 rules.

    \item\changetag{2A-supersonic-flame-out}

        An aircraft that selects idle or normal power at supersonic speeds automatically and immediately suffers a flame-out. This change appears the 1995 GEnie post by JD Webster and the v2.4 rules.

    \item\changetag{2A-cruise}

        The cruise speed in the ADC is for CL configuration and is 0.5 less for 1/2 configuration and 1.0 less for DT configuration. This change was initially suggested by Guy Acala in {\APJ}~24, and appears in the Origins rules in {\APJ}~44 and 53 and the v2.4 rules.

    \item\changetag{2A-snap}

        Aircraft and missiles can no longer execute snap turns. This change appears in the 1995 GEnie post by JD Webster, the Origins rules in {\APJ}~41, and the v2.4 rules.

    \item\changetag{2A-sustained}

        Sustained turning penalties are now assessed per 30 degrees of facing change for second and subsequent facing changes. This change was included by Felix Hack in the errata in {\APJ}~36 but was clarified as applying to 2nd edition rules by JD Webster in the {\APJ} 38 QA. It was included in the Origins rules in {\APJ}~39, 44, and 53 and the v2.4 rules.

        For aircraft with low and high bleed rates, the sustained turning penalties are 0.5 and 1.5 DP. This change was included in the “Props against Jets” article in {\APJ}~32 and the Origins rules {\APJ}~39, 44, and 53 and the v2.4 rules.

    \item\changetag{2A-zoom-climbs}

        The deceleration for zoom climbs (and sustained climbs that gain more than the CC) is now always 1.0 DP per level. See the 1995 GEnie post by JD Webster, the Origins rules {in \APJ}~44 and 53, and the v2.4 rules.

    \item\changetag{2A-vertical-climbs}

        The deceleration for vertical climbs is now always 1.5 DP per level. See the 1995 GEnie post by JD Webster, the Origins rules {in \APJ}~44 and 53, and the v2.4 rules.

    \item\changetag{2A-super-climbs}

        Aircraft with a CC of 6.0 or more in a ZC or SC can use one of their VFPs to climb three levels. This change appeared in the 1995 GEnie post by JD Webster and the v2.4 rules.

    \item\changetag{2A-steep-dives}

        The acceleration for steep dives is now always 1.0 AP per level. This change appeared in the 1995 GEnie post by JD Webster, the Origins rules in {\APJ}~44 and 53, and the v2.4 rules.

    \item\changetag{2A-unloaded-dives}

        The unloaded dives rule is changed to follow the Origins rules in {\APJ}~44 and 53 and the v2.4 rules. We use this in version 2A as we believe the rule in {\APJ}~53 is the latest version from an authoritative source. That said, the rule is not completely satisfactory, and is modified somewhat in version 2B.

    \item\changetag{2A-missile-sighting}

        The sighting rules are changed so that attempts to sight missiles come before attempts to sight aircraft, and the modifiers are changed. This change appeared in the Origins rules in {\APJ}~44 and 53.

        The Origins rules mention that missiles launched from sighted aircraft are no longer sighted, but I can’t find this in the original rules.

    \item\changetag{2A-irsts-c}

        IRSTS-C, from the v2.4 rules, is added.

    \item\changetag{2A-advantage}

        An aircraft in a vertical climb may disadvantage aircraft at the same level or lower. This change appeared in the Origins rules in {\APJ}~39, 41, 44, and 53 and the v2.4 rules.

    \item\changetag{2A-tailing}

        Complete a sentence in the rule on tailing: No more than three aircraft \emph{may be part of a multi-aircraft tailing}. This change appeared in the v2.4 rules.

    \item\changetag{2A-formation-initiative-modifiers}

        Add the initiative modifiers for formations to the initiative modifier table.

    \item\changetag{2A-initiative-veterans-and-formations}

        Clarify that the initiative penalty for not being in a formation does not apply if any of the crew are veterans (see rule 18.4). This is an explicit exception to the comment in rule 18.1 that that pilot quality impacts initiative.

    \item\changetag{2A-roll-preparatory-fps}

        Lag and displacement rolls now require preparatory FPs equal to {\onethird} of the aircraft's speed (rounded down). The preparatory FPs may be HFPs or VFPs. These change appeared in the 1995 GEnie post by JD Webster and the v2.4 rules.

    \item\changetag{2A-missile-launch-modifiers}

        The modifiers for being a combat hero and tactics master are now cumulative (i.e., a pilot who is both gets a \minus{2} modifier). This change appeared in the tables for the v2.4 rules.

    \item\changetag{2A-missile-launch-failure}

        The missile launch rules are changed so that if the launch roll fails by exactly one and is not an unmodified ten, the missile fails to launch but remains on the rail and can be used in the future. This change appeared in the Origins rules in \APJ~39, 31, 44 and 53.

    \item\changetag{2A-missile-flight}

        Air-to-air missiles always move before their target. This change appeared in the Origins rules in \APJ~44 and 53.

    \item\changetag{2A-missile-attacks}

        Missile attacks occur when the missile has the same position and altitude as the target or when it has the same altitude, at least one FP left in its proportional move, and the target is one of the seven positions immediately in front of the missile. This change appeared in the Origins rules in \APJ~44 and 53.

    \item\changetag{2A-missile-speed}

        The missile speed attenuation factors are changed. In all cases, they now increase or stay the same with time, which makes physical sense. These changes appeared in the v2.4 tables.

    \item\changetag{2A-ew-coverage}

        RWR-A now detects air-to-air target illumination, not air-to-air locks. RWR-B can no longer detect VF FCR. These changes appear in the v2.4 tables.

    \item\changetag{2A-random-plotted-fire}

        Added text to the random AAA fire table on determining the range of the plotted fire hex from the target aircraft. This text appears in the rules, but is missing from the original table.

    \item\changetag{2A-helicopters}

        Include the rule on helicopters by Scott Forrest, originally in {\APJ}~14 and also in Malcolm Pipe's v2.4 rules.

    \item\changetag{2A-helicopter-data}

        Include helicopter data by Scott Forrest, originally in {\APJ}~14.

    \item\changetag{2A-helicopter-hh-53c}

        Include the HH-53C helicopter in the Helicopter Data Table, adapting it from the data given in Scenario V-11 of {\TSOH}.

    \item\changetag{2A-helicopter-door-guns}

        Add rules for helicopter door guns, adapting them the rules given in Scenario V-11 of {\TSOH}.

    \item\changetag{2A-maritime-operations}

        Include the chapter on maritime operations from the v2.4 rules.

\end{itemize}

The rule for pylon drag (suggested originally by Mark Bovankovich in {\APJ}~18 and adopted in the Origins rules in {\APJ}~41, 44, and 53) is not incorporated simply because applying it uniformly would require extensive research and modifications to almost all ADCs.

\subsection{Version 2B}

Version 2B contains nonauthoritative minor changes, normally to clarify existing rules. These changes have been developed in collaboration with Carl Smeaton, Jean-Baptiste Mouillet, and Stephen Loughrey.

\begin{itemize}

    \item\changetag{2B-credits}

        Additions to the credits.

    \item\changetag{2B-missile-counters}

        Explicitly state that each missile counter represents a single missile.

    \item\changetag{2B-directions}

        ENE, ESE, WSW, and WNW are used in preference to NE, SE, SW, and NW. The original directions fall evenly between two \degrees{30} facings, whereas the new  ones are unambiguously closer to one.

    \item\changetag{2B-stacking}

        Collisions are only possible if the aircraft are at the same altitude. They are also possible if four or more aircraft are stacked at the same altitude, even if they are in a close formation.

    \item\changetag{2B-collisions}

        Potential collisions from head-on attacks are resolved after the attack. Other collisions are resolved at the end of the flight phase. If an aircraft rolls for a collision for a head-on attack at the end of its move, it does not roll again at the end of the flight phase.

    \item\changetag{2B-range}

        Clarify how to calculate range when counters are on hex sides. The new rule gives the correct result, for example, when two counters are on hex sides on the opposite sides of a hex.

    \item\changetag{2B-closeness}

        Define closeness in terms of range. This is important since it is possible to define closeness in other terms, for example, taking into account half hexes and odd altitude level differences.

    \item\changetag{2B-fractions}

        State explicitly that the fractions table is used for:
        \begin{itemize}
            \item Determining the number of HFPs and VFPs required by a given flight type.
            \item Engine thrust.
            \item SSGT requirements
            \item Aiming requirements and modifiers
        \end{itemize}
        Explain the values in the table (round to nearest unless one half). State that in all other cases one multiplies by the fraction and rounds as directed.

    \item\changetag{2B-fractions-of-half}

        Add appropriate entries to the fractions table for fractions of 0.5. These are sometimes needed to determine the reduced thrust at high altitude.

    \item\changetag{2B-ground-fac-marking}

        The text of the original rules states that ground FACs mark targets in the AAA planning phase. This phase does not exist. The extended sequence of play indicates that this occurs during the visual sighting phase. I have changed the text of the rules to match this.

    \item\changetag{2B-ra-speed-limits}

        RA aircraft at their maximum or dive speed, as appropriate, may only carry forward 1.0 APs.

    \item\changetag{2B-military}

        Aircraft can select 0.0 APs when using military power. This gives no acceleration, but allows them to maintain a steady speed above cruise speed

    \item\changetag{2B-propeller-aircraft}

        Add rules for the HT and FT settings of propeller aircraft. The only difference is the fuel consumption. Note that propeller aircraft never suffer flame outs.

    \item\changetag{2B-raa-fp-carry}

        {RAA aircraft can carry up to 1.0 APs.}

    \item\changetag{2B-when-supersonic}

        For purposes other than turns or maneuvers, an aircraft is supersonic if its speed is greater than or equal to the M1 speed in the altitude band in which it started the game turn.

    \item\changetag{2B-stall-recovery}

        Aircraft recover from a stall to a wings-level attitude.

    \item\changetag{2B-departures}

        Aircraft lose any carried APs or DPs when they recover from departure.

    \item\changetag{2B-relights}

        Relights attempts can happen on any three game turns after a flame-out (not the first three game turns). The count is reset for each flame-out.

    \item\changetag{2B-idle-after-relight}

        A relit engine is considered to be in idle power. This is important if the engine does not have RPR and AB is selected on the next game turn.

    \item\changetag{2B-pssm-maneuvering-departures}

        A PSSM aircraft that is supersonic and is carrying a prohibited turn suffers a maneuvering departure. This follows the \href{https://airpower.groups.io/g/main/message/2519}{consensus} on the Air Power mailing list.

    \item\changetag{2B-low}

        The low altitude band extends down to level 0, to account for aircraft at level 0 in TFF.

    \item\changetag{2B-steep-climb}

        An SC must use at least one VFP.

    \item\changetag{2B-unloaded-dives}

        The rule for unloaded dives in version 2A is unsatisfactory as it does not give unloaded dives any advantage over steep dives in terms of acceleration or horizontal distance. Indeed, a steep dive is better since one VFP can lose two altitude levels, whereas in an unloaded dive, each VFP can only lose one altitude level. For example, consider an aircraft with six FPs. If it uses an unloaded dive, it can have either four HFPs and lose two levels or five HFPs and lose one level. If it uses a steep dive, it can have five HTPs and lose two levels, giving the horizontal distance of the first unloaded option with the acceleration of the second.

        Therefore, I have changed the unloaded dive rule so that all FPs are HFPs, but that if the aircraft loses a level if it unloads for at least half its FPs and loses another if it unloads for all its FPs. These loses are similar to those from free descent, in that they occur during HFPs. With this rule, an aircraft with sox FPs can have six HFPs and lose two levels, which does give an advantage over a steep dive.

    \item\changetag{2B-unloading-and-one-fp}
    
        If an aircraft has only one FP and unloads, it can choose to lose one or two altitude levels. If it loses two, it ends in a steep dive.

    \item\changetag{2B-free-descent}

        After a VC, an aircraft in level flight must expend 1/3 of its FPs (not speed) before taking a free descent. This is for consistency with other similar limits on flight, which are in terms of FPs.

    \item\changetag{2B-half-vfps}

        Eliminate the rules in section 8.4 on half VFPs as they are incompatible with the rules in 5.4.

    \item\changetag{2B-recovery-periods}

        Change the rules for recovery for ET turns to be that the aircraft must not turn at the ET rate and must not prepare for or execute rolling maneuvers. This allows slides and eliminates the potentially confusing reference to being wings level.

    \item\changetag{2B-vertical-attacks}

        If an aircraft has not yet moved, use its flight type from the previous turn for determining the vertical attack modifier.

    \item\changetag{2B-same-hex-or-hex-side}

        Change all references to “the same hex” in gun attacks to “the same hex or hex-side”.

    \item\changetag{2B-head-on-attacks}

        Change the rule for head-on attacks to include all the prerequisites except sighting. (If you are being fired at, you will probably notice.) A head-on attack and any response are resolved simultaneously. If the target has not moved, its flight type from the previous game turn is used to determine vertical limits and same-hex vertical modifiers.

    \item\changetag{2B-prohibited-return-fire}

        Note explicitly that preemption and HRDs prohibit returning fire either to head-on attacks or with defensive guns.

    \item\changetag{2B-angle-off-on-hex-side}

        Change the angle-off diagram for an aircraft on a hex side so that the boundaries between the arcs radiate from the aircraft. The previous version makes sense in the context of the Genie scatter diagram, where one wants the lines to trace valid locations, but not for simply determining geometry.

    \item\changetag{2B-gunsight-modifiers}

        The text in 1A on recovery from gunsight modifiers incorrectly says that they apply \emph{after} a recovery period (“Aircraft {\ldots} need only apply turn rate modifiers for turning done after a recovery period has elapsed”). This is changed to state that they apply \emph{until} the recovery period has been completed.

    \item\changetag{2B-ssgt}

        The SSGT modifier is now explicitly \minus{1} for FPs of at least 1/3 speed and \minus{2} for FPs of at least 2/3 speed.

    \item\changetag{2B-ssgt-figure}

        The lines on the SSGT figure are modified to start at a range of six from the target (in agreement with the rules).

    \item\changetag{2B-rocket-restrictions}

        Apply the errata for gun attacks (“An aircraft \emph{in level flight} may fire at a target in another hex only if it is at the same altitude level or at an adjacent altitude level”) to rocket attacks too.

    \item\changetag{2B-rocket-modifiers}

        Include damage modifiers for rocket attacks. The TSOH sheets say all gun attack modifiers apply to rocket attacks, but the TSOH rules after APJ 23 errata exclude snapshots (obviously) and damage.

    \item\changetag{2B-genie-scatter-figure}

        The figure for the scatter of the AIR-2 Genie is completed with a third panel showing the scatter when the target is at a hex center and facing a hex corner.

    \item\changetag{2B-genie-scatter-counter}

        The figure for the scatter of the AIR-2 Genie  uses an aircraft counter to represent the final position of the Genie. This is confusing. Use an arrow counter instead.

    \item\changetag{2B-cumulative-damage}

        Give an example to show that 4L = \binaryplus{H}{L}.

    \item\changetag{2B-optional-damage-tables}

        The actual damage (and not the cumulative damage) is used to determine the appropriate section of the optional damage table.

    \item\changetag{2B-damage-control-restrictions}

        The only climbing flight allowed during damage control is an SC without AB. The only diving flight allowed during damage control is a SD without AB and at most 2 VFPs

    \item\changetag{2B-unloading-and-damage-control}

        In version 1, damage control was incompatible with UD as the only diving flight allowed was a SD without AB and with at most two VFPs. In version 2, UD is no longer a type of diving flight as unloaded FPs can only be taken in level flight. Therefore, unloading is explicitly prohibited during damage control.

    \item\changetag{2B-crash-secondary-attacks}

        State that there are no secondary attacks associated with a crash. Any such attacks would presumably be at 1:4 odds (since the primary attack is at 1:2 odds and the odds for secondary attacks are normally reduced by a factor of 3) and with a +3 modifier. Such an attack has no chance of causing suppression or damage.

    \item\changetag{2B-sighting-higher}

        Clarify that the rule that four altitude levels is one hex of range applies for sighting higher aircraft or missiles during daylight.

        The TSOH errata are slightly inconsistent on whether the rule that four levels of altitude count as one hex of range when sighting a higher target applies just to aircraft (page 2) or also to missiles (page 3) and whether it only applies in daylight (page 2) or all of the time.

    \item\changetag{2B-sighting-and-other-detections}

        Allow a \minus{1} modifier for any type of detection or lock-on.

        The version 1A rules give a \minus{1} modifier for a RWR detection, but not other types of detections. This is odd, since a RWR is likely to be less precise than the other types of detections.

    \item\changetag{2B-sighting-and-hud}

        Allow a \minus{1} modifier for any type of lock-on with HUD interface technology.

        The version 1A rules give a \minus{1} modifier for an IRSTS lock-on and HUD interface technology, but not for a radar lock-on. This seems to be inconsistent.

    \item\changetag{2B-sighting-and-engaging-missiles}

        For engaging, require that the missile be sighted in the aircraft decisions phase.

        The TSOH rules give requirements for engaging missiles include the missile being sighted (a) by the target when the missile starts its move or (b) by a friendly aircraft when the defender starts its move. However, the decision to engage or not occurs in the aircraft decisions phase, not the flight phase.

    \item\changetag{2B-sighting-and-pre-emption}

        Allow preemption if a third aircraft sights the defender and attacker at the start of the \emph{attacker's} movement.

        The sighting restrictions in the original rules state that a defender can preempt a moving aircraft if both are sighted by a third aircraft at the start of the \emph{defender's} movement. Since preemption occurs during the attacker's movement, this does not make sense.

    \item\changetag{2B-padlocking-ground-units}

        Aircraft can padlock ground units.

    \item\changetag{2B-ground-unit-identification}

        Clarify that aircraft can attempt to identify ground and naval units in the same game turn in which they are  sighted.

    \item\changetag{2B-vas-identification}

        VAS can only be used for identification in the \arcplus{180} arc.

    \item\changetag{2B-vas-and-padlocks}

        A crew member who uses VAS may not padlock, but other crew members can still padlock.

    \item\changetag{2B-irsts-c}

        IRSTS-C may attempt to lock on to any six detected targets, not just the closest. IRSTS-C allows IR Uncage on tracked targets, like IRSTS-B.

    \item\changetag{2B-irsts}

        IRSTS detections are lost if any lock-on succeeds (i.e., IRSTS does not have a capability similar to TWS.) IRSTS detections and lock-ons are maintained provided the target satisfies the arc and range requirements at the start of the radar phase and, for single pilot aircraft, provided the pilot does not use radar.

    \item\changetag{2B-same-hex-advantage}

        State that aircraft at the same hex or hex-side have no effect on each other's advantage.

        The 1C-apj-23-errata added that aircraft “in the same hex” do not have any effect on the other's advantage, and this change adds “hex-side” too.

    \item\changetag{2B-declaring-preemptions}

        Rewrite the rule on preemptions and the procedure to make it clear that an aircraft can preempt a moving aircraft before each FP, including the first.

    \item\changetag{2B-no-double-preemptions}

        Add text to explicitly state that a preempting aircraft may not itself be preempted. This is not a rule change, since a preempting aircraft cannot make gun attacks, but making this statement is clearer.

    \item\changetag{2B-limited-arcs}

        Change the limited arcs to be consistent with each other. See the discussion in the airpower.io group starting on 28 February 2024.

    \item\changetag{2B-maneuver-consecutive}

        Require that FPs for maneuvers be consecutive.

    \item\changetag{2B-maneuver-carry}

        Allow that preparatory FPs for maneuvers to be carried from one turn to the next.

    \item\changetag{2B-additional-HFPs-for-slides}

        Require that additional preparatory FPs for slide maneuvers be HFPs.

    \item\changetag{2B-slides-VFPs}

        Note that use of a VFP implicitly aborts a slide maneuver.

    \item\changetag{2B-roll-hfp-cost}

        State that all displacement and lag rolls cost 1 HFP to execute.

        This is not a change in the rules per se; in the first-edition rules displacement and lag rolls cost 1 HFP to execute. However, the v2.4 rules state that the FP cost is specified on the ADC and the first-edition ADCs of some aircraft, including the F-14, indicate that the cost of 1.5 HFPs. This doesn't fit well into more modern rules that assume that whole FPs are expended. These ADCs should be modified to give a cost of 1.0 HFPs.

    \item\changetag{2B-viff-vertical-pitch}

        Require that when a VIFF vertical pitch is used to enter a vertical dive from a vertical climb, all the HFPs must be used before any VFPs. This parallels the requirement for using a HRD to enter a vertical dive from a zoom or sustained climb.

    \item\changetag{2B-missile-launch-modifiers}

        In the missile launch modifier table, change the launch modifier for a critically damaged aircraft from \plus{2} to \plus{3} to match the damage tables.

    \item\changetag{2B-missile-launch-restrictions}

        Change the restriction on missile launches from “firing guns during its last FP” to “firing guns or rockets after its last FP.”

    \item\changetag{2B-sighting-launched-missiles}

        Note that aircraft may only attempt to sight successfully launched missiles. This is implicit in the statement that failed missiles are removed from play, but it is worth stating explicitly.

    \item\changetag{2B-follow-up-missiles}

        Change how the FPs regained by a follow-on missile are assigned so that they are added to the initial complete proportional segments rather than the initial proportional segments. This is relevant when the missile has a life of only one game turn; we do not want an additional FP added to the incomplete proportional moves with the launch delay.

    \item\changetag{2B-missile-launch-speed}

        When the advanced rule on missile speed is used, the aircraft speed used to determine a missile's initial speed is its end speed on the game turn of launch or, equivalently, its start speed on the game turn after launch.

        This is in contradiction to 1B-missile-launch-speed, but is consistent with the basic rules which determine modifiers to the base speed of the missile according to the start speed of the aircraft on the game turn after launch.

    \item\changetag{2B-slower-missile}

        If the missile is slower than the aircraft, in each proportional move the missile moves 1 FP and the aircraft moves 1 or more FPs. Calculate the number of FPs for the aircraft in the same way as for a missile when the missile is faster.

    \item\changetag{2B-missile-power-modifier}

        If a free aircraft under attack by an IRM selects idle power and fails the roll, it uses the modifier for normal power (i.e., \plus{0}). A free aircraft under attack by an IRM can select normal power and automatically gain the corresponding modifier. If an aircraft was using idle power at the moment it became the target of an IRM and continues to use idle power until the missile attacks, then it gains the \plus{1} modifier for idle power even if it does not engage the missile.

    \item\changetag{2B-uncaged-lock-up}

        The roll to determine whether an uncaged IRM lock-up succeeds is made after declaring missile launches. If all declared launches fail, as usual one additional launch may be attempted.

    \item\changetag{2B-irm-vertical-fov}

        For IRMs, the vertical limits for the limited, \arcplus{180}, and \arcplus{150} arcs are used according to the horizontal field-of-view.

    \item\changetag{2B-irm-envelopes}

        Change the rules on out-of-envelope IRM launches state that type A seekers may be launched at “not large” targets (those with a visibility of less than 10), instead of “large” targets, as this makes more sense from the context.

    \item\changetag{2B-radar-requirements}

        Change the requirement for radar work from not having made “an air-to-air gun attack” to not having made “an air-to-air gun or rocket attack”.

        %\item\changetag{XX-limited-look-down}

        %In the first edition rules, are two incompatible statements on the capabilities of radar with of limited look-down. Section 16.3 states that they can potentially detect aircraft from levels 5 to 10 from above provided the target is closer to the radar than to the ground. Section 16.4 states that they can do this from levels 2 to 10. I have adopted the latter.

        %\item\changetag{XX-declaring-special-radar-modes}

        %I have changed the text that states that boresight and auto-track modes are declared “when an aircraft begins its flight” to “in the aircraft decisions phase” (following the expanded sequence of play).

        %\item\changetag{XX-auto-track-selection}

        %I have added that a visually sighted aircraft may be selected for detection in auto-track mode of a \minusafter{7} (in accordance with the charts). If the roll fails, no detection occurs.

    \item\changetag{2B-multiple-dds}

        If an aircraft has two or more DDS systems (e.g., one integrated and the other in a pod), each can have their own DDS program and any combination of them can be turned on or off each game turn, provided other requirements are met.

    \item\changetag{2B-flight-type-disorientation}

        I have changed the moment at which disorientation is determined for risky flight types from the end of the game turn to the end of flight. This allows disorientation to interfere with missile launches and radar work.

    \item\changetag{2B-disorientation-fbw-modifier}

        The FBW modifier for recovering from departed flight should not apply to disoriented flight.

    \item\changetag{2B-gloc-egress}

        The pilot of a pilot-only aircraft may not eject if they are unconscious. However, any conscious crewmember can eject himself and all other crewmembers, including unconscious ones. A crewmember may not bail out if they are unconscious. These changes are necessary because the existing rules suggest that unconscious crewmembers can only be ejected by conscious crewmembers if the aircraft is going to collide with terrain and because some propeller-driven fighters without ejector seats (including the Yak-9, F-51, F4U, and Sea Fury) can ET and hence cause GLOC. There is no need to consider bailing out of multi-crewed aircraft, since all that can ET are equipped with ejector seats.

    \item\changetag{2B-undamaged-egress}

        GLOC or disoriented flight may cause a crew to attempt to egress an undamaged aircraft. I have treated an undamaged aircraft the same as an aircraft with L or 2L damage.

    \item\changetag{2B-ajms}

        The AJM rating (not the BJM rating) is used for jamming rolls with AJMs.

    \item\changetag{2B-infrared-jammers}

        I have added a rule on infrared jammers. These appear as external stores in {\TSOH}.

    \item\changetag{2B-ewr-eccm}

        EWR ECCM is always 0 in TSOH, but it could be different for more modern radars. I have added a rule that it can be different if specified in scenario notes.


    \item\changetag{2B-aiming-modifiers}

        The aiming modifier is now explicitly \minus{1} for FPs of at least 1/3 speed and \minus{2} for FPs of at least 2/3 speed.


    \item\changetag{2B-rpt}

        RPTs are considered to the RPs for the purposes of air-to-air and air-to-ground attacks.
        
    \item\changetag{2B-heavy-barrage-fire}

        The errata in {\APJ} 36 state that in rule 24.1.3, “if at least one heavy AAA unit contributed to the barrage zone, use an attack rating of 2 for chance hits.” Since heavy units cannot carry out barrage fire and since rule 24.1.3 refers to plotted fire, I have changed this to refer to plotted-fire zones.

    \item\changetag{2B-fuel-tank-weight-and-load}

        The weight of a fuel tank scales linearly with the fraction of fuel it contains. The empty load is only used when the tank is completely empty, otherwise the full load is used.


    \item\changetag{2B-disengaging-and-rockets}
    
        Permit aircraft to disengage if they avoid being the target of rockets, in addition to missiles and guns, for three game turns.
        
        %\item\changetag{XX-merging-cloud-layers}

        %Two dense layers that overlap or are adjacent are merged. A stratus layer that overlaps with a dense layer (but not one that is adjacent) is ignored.

        %\item\changetag{XX-stratus}

        %Aircraft and ground units may not visually sight, use VAS, use IRSTS, use TV/IR Optics, use Night IR sights, launch IRMs (even with radar assist), launch IR SAMs, launch or track OG/LG SAMs, use laser designators, or use laser spots on targets on the opposite side of stratus cloud layers. Also, the requirements on IRM and IR SAM tracking concerning stratus cloud layers apply only at the end of the missile’s proportional move and also apply to OG/LG SAMs.

        %\item\changetag{XX-generating-night-conditions}

        %I have added a simple means to determine randomly if a scenario occurs at night. Stating this explicitly in the scenario notes is probably a more satisfactory approach.

        %\item\changetag{XX-aaa-sun-clutter}

        %AAA units stacked with an FCR or with an integrated W-type FCR are not penalized for firing when in the Sun clutter of their target.

        %\item\changetag{XX-contrails-at-night}

        %Missiles, as well as aircraft, that are contrailing and below the highest cloud layer can also be sighted to a range of 6. In both cases, sighting is automatic.

        %\item\changetag{XX-tvir-optics}

        %Haze does not reduce the sighting range for TV/IR optics or OPs. They may be used during the day (to see through haze) and at night.

        %\item\changetag{XX-radar-bombing}

        %The navigation range (or half the air-to-air search range) is used for the detection range. The attack range (or half the air-to-air tracking range) is used for the tracking range. I have added a procedure for lock-ons. I have added that a radar cannot simultaneously be used in the air-to-ground mode and air-to-air mode, and detections and tracks are lost when it switches from one mode to the other. Furthermore, I apply the same restrictions to the use of air-to-ground radar as air-to-air radar in normal mode (e.g., pilot-only aircraft cannot search if they used an HT or more).

    \item\changetag{2B-helicopter-door-gun-range}

        The air-to-air range of helicopter door guns with a caliber of less than 20 mm is reduced from 2 to 1. The air-to-air range of larger guns and the air-to-ground range of all guns is still 2.

\end{itemize}

\subsection{Version 2R}

Version 2R is an internal version. It contains unauthoritative major changes intended for version 2B but currently under review.

\begin{itemize}

    \item\changetag{2R-fas-rounding}
    
    Explicitly state that the FAS is rounded up after multiplying by all applicable factors. In version 2A and earlier, this is stated for a pair of factors, but not in general.

\end{itemize}

\subsection{Version 2W}

Version 2W is an internal version. It contains unauthoritative major changes intended for version 2B but currently being worked on.

\begin{itemize}

    \item\changetag{2W-expending-fps}

        Clarify an FP is considered to have been expended as soon as the aircraft finishes the movement associated with that FP and before any immediately subsequent turn completions or attacks.

    \item\changetag{2W-fp-stages}

        The various actions that can occur around the expenditure of an FP now have a defined order in terms of “FP Stages”.

    \item\changetag{2W-radar-directed}

        Clarify in the rules for aiming and jinking, “radar-directed” AAA guns are those with add-on or “W” FCRs. That is, those with “R” FCRs are manually aimed.

\end{itemize}

\subsection{Version 3A}

Version 3A contains unauthoritative major changes. Again, these changes have been developed in collaboration with Carl Smeaton.

\begin{itemize}

    \item\changetag{3A-credits}

        Additions to the credits.

    \item\changetag{3A-updating-adcs}

        Procedures are given for identifying and updating ADCs from earlier editions. Example version 1, version2, and version 3 ADCs are shown for the same aircraft.

    \item\changetag{3A-adc}

        The example ADC is replaced with a version 3 ADC.

        Rounding down can provoke confusions and, given all the other mathematical manipulations in the game, seems unnecessary. ADCs need a slight adjustment, as a consequence.

    \item\changetag{3A-same-hex-arcs}

        For gun attacks (and for horizontal arcs in general), the arc at the same location is the same as if the attacker were to be moved backwards one hex.

    \item\changetag{3A-missile-attack-arcs}

        The arc for missile attacks is determined at the moment the missile attack is declared. Border-line cases are resolved by first considering the arc into which the faster element is moving and second by choosing the wider arc, the one closer to the \arc{120} arc. (The previous rule made no mention of considering the faster element and required the arc that favors the defender. The wider arc and the arc that favors the defender are only different for E, I, and M seekers on the \arcborder{150}{180} border and E, I, and BR seekers on the \arcborder{120}{150} border.)

    \item\changetag{3A-missile-launch-envelope-arcs}

        When determining the arc for missile launch envelopes, border-line cases are resolved by first considering the arc into which the faster element is moving and second by choosing the wider arc, the one closer to the \arc{120} arc. This is the same procedure as for missile attacks.

    \item\changetag{3A-irm-launch-arcs}

        When determining the arc for IRM launch requirements, border-line cases are resolved by first considering the arc into which the faster element is moving and second by choosing the wider arc, the one closer to the \arc{120} arc. This is the same procedure as for missile attacks.

    \item\changetag{3A-combined-arcs}

        The advanced rule on vertical limits is replaced by an adaptation of Tony Valle's proposal in APJ 39. The phrase “combined arc” is used, as these limits are no longer simply vertical limits in addition to the horizontal ones, as for sufficiently large flight slopes the limits can become purely vertical.

    \item\changetag{3A-AP-carry-at-speed-limits}

        Aircraft whose speed is equal to their maximum speed or maximum dive speed carry no APs forward. Aircraft that suffer a speed fade-back or whose speed it reduced to their dive speed carry neither APs not DPs forward.

    \item\changetag{3A-speedbrakes}

        DPs from speedbrakes must be chosen at the start of flight, just like APs from engine thrust.

    \item\changetag{3A-turn-drag}

        Each turn now gives DPs when it is declared. Sustained turns and the additional drag for more than one turn per game turn are gone.

        The turn drag given by ADCs for earlier versions of these rules can be updated as follows: take half of the old turn drag and add 0.5 for normal aircraft, 0.25 for aircraft with the low bleed rate (LBR) property, or 0.75 for aircraft with the high bleed rate (HBR) property. This is equivalent to averaging the old turn drag over two turns.

        Subsonic EZ turns still have zero drag, since at most speeds an aircraft can only execute at most one per game turn and so would not accrue the old penalty for sustained turns.

        We also average the old DPs per game turn for supersonic turns over two turns.

    \item\changetag{3A-multiple-turns}

        Note that multiple turns acquire turn drag for each facing change of \degrees{30} and that additional FPs can be expended before the first facing change.

    \item\changetag{3A-slide-drag}

        Each slide gives no DPs.  The additional drag for more than one slide per game turn is gone.

    \item\changetag{3A-slide-banking}

        Aircraft have to be appropriately banked for a slide, is banked during the slide, and can come out banked in the sense of the slide or wings-level.

    \item\changetag{3A-multiple-slides}

        An aircraft my declare more than one slide in a game turn, but must spend at least 4 FPs between executing a slide and declaring the next slide. There is no speed limit on this.

        If sufficient preparatory FPs are carried in, at least 9 FPs are required to execute two slides: 1 FP to execute the first; 4 FP of delay; 3 FP of supersonic preparatory FPs; 1 FP to execute the second. If no preparatory FPs are carried in, at least 12 FPs are required.

        In the original rules, only two slides were allowed. With this new rule, if the aircraft is extremely fast, it can potentially execute three. If sufficient preparatory FPs are carried in, at least 17 FPs are required to execute three slides: 1 FP to execute the first; 4 FP of delay; 3 FP of supersonic preparatory FPs; 1 FP to execute the second; 4 FP of delay; 3 FP of supersonic preparatory FPs; 1 FP to execute the third. If no preparatory FPs are carried in, at least 20 FPs are required.

        The roll drag given by ADCs for earlier versions of these rules can be updated as follows: take the old roll drag and add 0.5. This is equivalent to averaging the old roll drag over two rolls.

        We also average the old DPs per game turn for supersonic rolls over two rolls.

    \item\changetag{3A-vertical-attacks}

        The \plus{1} same-hex vertical attack modifier also applies if the attacker is in level flight and the target is climbing or diving.

    \item\changetag{3A-head-on-attacks}

        A non-moving target may declare a head-on attack even if the moving aircraft does not.

    \item\changetag{3A-searcher}

        The searcher is designated by the player and need not be the closest aircraft. The \plus{2} modifier applies if the target is in the restricted arc of the searcher, not all participating aircraft.

    \item\changetag{3A-multi-crew-searchers}

        A searcher may use the crew quality modifiers for any crew member who is conscious and not disoriented (i.e., it may use the modifiers for the highest quality crew member). Provided at least two crew members are conscious and not disoriented, it may also use the multi-crew modifier.

    \item\changetag{3A-missile-targets-declare-flight}

        The target of a missile declares its flight type, power settings, and speedbrake settings before the missile's first proportional move.

    \item\changetag{3A-fps-for-proportional-moves}

        The proportional moves of a missile and its target are based on the number of FPs each has and not on their speeds.

    \item\changetag{3A-follow-on-missiles}

        Follow-on missiles now skip whole proportional moves rather than FPs.

    \item\changetag{3A-missile-fp-stages}

        The various actions that can occur around the expenditure of an FP by a missile now have a defined order in terms of “Missile FP Stages”.

    \item\changetag{3A-missile-tracking}

        The missile tracking requirements (on arcs and range to the target) must be satisfied after each of the missile's proportional moves. They do not have to be satisfied after each of the aircraft's proportional moves or at the end of each game turn.

    \item\changetag{3A-missile-attacks}

        The missile attack rules are changed slightly.

        A missile attacks in a missile attack stage associated with either one of its FPs or one of its targets.

        A missile can attack either by being at the same hex or hex-side and altitude as the target or by having the target in one of the six positions around the position in front of it, having an FP remaining, and having the target either at the same altitude or, if it can execute a free descent, one altitude below.

        If a missile can attack in its next FP by moving to the position and altitude of the target, it must do so. This means that if it can attack with an HD, it cannot use H and then D. This prevents missiles from drawing out moves to gain more time to turn. It also replaced the previous rule that if the target was immediately in front of the missile, it would declare an attack but does so in a way that allows a missile to complete a turn.

    \item\changetag{3A-continuing-to-engage}

        If an aircraft engaged a missile in the previous game turn, it may continue to do so on the current game turn even if the missile has become unsighted.

    \item\changetag{3A-irm-ground-clutter}

        The IRM attack modifier for ground clutter applies if the target is in the LO altitude band and the missile's flight slope since launch or climbing is less than \minus{1}.

    \item\changetag{3A-combat-hero-squadrons}

        The \minus{1} modifier to initiative when a combat hero is lost now applies to all aircraft in their squadron, not just their formation.

    \item\changetag{3A-ship-facing}

        Ships must face adjacent hexes. This is because ships can move and can be the target of attacks by aircraft, but if they are allowed to face hex-sides, they can move onto hex-sides, and there are no rules for lines-of-approach for targets on hex-sides.

    \item\changetag{3A-ship-movement}

        The ship movement rules from v2.4 are replaced with ones based on those in the \emph{Harpoon\superscript{4}} game rules.

        % WORKING HERE

        % \begin{itemize}

        % \item
        % The position modifier is replaced by a single background modifier. If the target is in stratus or is a smoker, an additional modifier is applied. The “dark uncamouflaged” paint scheme and “water” background are introduced.

        % The modifiers for light are the same as the original modifiers for uncamouflaged. The modifiers for “against land” and “against sea” are the same as the original modifiers for lower except that dark over sea is \plus{1} (the same as camouflaged over land), dark over land is \plus{0}, and camouflaged over sea is \minus{1}.

        % \item
        % If the searcher or target is in haze, the search range is considered to be doubled.

        % \item

        % \item
        % A \plus{1} modifier applies if the target is below the highest dense cloud layer.

        % \end{itemize}

\end{itemize}

The following are known deficiencies of version 3A:
\begin{itemize}
    \item UD rules.
    \item High AoA maneuvers.
    \item Radar rules.
    \item Radar-guided missiles.
    \item Weather.
    \item Electronic warfare.
\end{itemize}

\subsection{Version 3R}

Version 3R is an internal version. It contains unauthoritative major changes intended for version 3A but currently under review.

\begin{itemize}

    \item\changetag{3R-aborting-damage-control}

        An aircraft that wishes to abort damage control must declare this before an FP and prior to an opponent having to declare whether they will preempt. This simulates the warning given when an aircraft switches from passive flying (damage control) to aggressive flying (pulling lead for an attack).

\end{itemize}

\subsection{Version 3W}

Version 3W is an internal version. It contains unauthoritative major changes intended for version 3A but currently being worked on.

\begin{itemize}

    \item\changetag{3W-recovery}

        For the purposes of recovery (but also for aiming and tracking), an FP is considered to have been expended as soon as the aircraft finishes the movement associated with that FP.

    \item\changetag{3W-load-rounding}

        Load totals are no longer rounded down.

    \item\changetag{3W-roll-drag}

        Each displacement, lag, or vertical roll now gives the DPs in the ADC when it is declared. The additional drag for more than one roll per game turn is gone.

\end{itemize}

\subsection{Version 3B}

Version 3B is a rewrite of version 3A and is incomplete.

\end{document}

\section{Language Changes}

%\subsection{Copy Editing}
%
%I will apply the standard rules for capitalization, punctuation, and compound adjectives (e.g., “air-to-air attack”). I %will use the Oxford comma.

\subsection{Uniformity}

I use “AP” and “DP” to refer to acceleration and deceleration points. This follows the uniform use of FP, HFP, and VPF.

\subsection{New Terminology}

I have adopted the following terminology:

\begin{itemize}
    \item Individually sighted. This means sighted or friendly, within sighting range of a specific aircraft, and not in the blind arc of that aircraft.
    \item Neutral. This replaces the non-advantaged category.
    \item Superior and Inferior Aircraft. These replace positions of advantage and disadvantage.
    \item Advantage Categories. This replaces position of advantage categories.
    \item Threatening and threatened. These are used to describe aircraft involved in preemptions.
\end{itemize}

\subsection{Replacing Ambiguous Terms}

\begin{itemize}
    \item \itemparagraph{Turn.} The original rules use “turn” to refer to both a game turn and turning flight. I use “game turn” to refer to a game turn and “turn” to refer to turning flight.

    \item \itemparagraph{Within.} The original rules use “within” regarding ranges and other quantities. This is ambiguous.

        For example, an aircraft must be “within four hexes” to conduct a rocket attack. Is this inclusive (i.e., “at a range of no more than four hexes” or \binaryrelation{{range}{≤}{4}) or exclusive (i.e., “at a range of less than four hexes” or \binaryrelation{\textrm{range}}{<}{4})? In the case of rocket attacks, the rocketry table resolves the ambiguity, which shows hit rolls for ranges of 1, 2, 3, and 4 hexes. Thus, here, within is inclusive. It is also apparently inclusive when used in the context of look-down limitations, which are expressed as “within four altitude levels” and “within 5 to 10 levels”.

            In the original rules, within is used in the context of:
            \begin{itemize}
                \item Aircraft horizontal and vertical separations in tactical formations.
                \item AP limits in military and afterburner power.
                \item The maximum range for rocket attacks.
                \item The maximum range for SSGT.
                \item The maximum range for visual sighting.
                \item When considering distances to determine if a line of sight is blocked by terrain.
                \item The maximum horizontal range to gain advantage.
                \item The relative facing of a tailed and tailing aircraft.
                \item The maximum range for defensive preemptions.
                \item The minimum and maximum ranges for missile launch envelopes.
                \item The maximum radar range.
                \item The altitude ranges for look-down limitations.
                \item The altitude difference for ground clutter for BRMs, RHMs, and AHMs.
                \item The range from the target at which an AIM-26A can be detonated.
                \item The altitude difference limits for jamming cell formations.
                \item Sun clutter.
                \item Parachute flares.
            \end{itemize}
            In all of these cases, I believe the use is inclusive.

    \end{itemize}

    %\subsection{Gender-Neutral Language}
    %
    %Many of the terms used in the rules are already gender-neutral, and no change is required to these. However, we need to make the following substitutions:
    %\begin{itemize}
    %    \item “they/them/their” for “he/him/his”
    %    \item “crewmember” for “crewman”
    %    \item “winger” for “wingman” (cf. section “leader”)
    %\end{itemize}

    % \section{Process}

    % I typeset the rules using LaTeX on Overleaf and track the source text using GitHub.

    % The source text for version 1A was generated by taking Malcolm Pipe’s second-edition text, manually comparing it to the first-edition rules, and making any changes necessary to obtain agreement. (I believe permission to do this is implicit in his statement, “Edit as desired after download.” on 2021-02-05 to the airpower.io group.)

    % I use custom LaTeX commands to introduce changes into the source text. These changes are tagged, and the tags described in this document. This gives traceability.

    % \section{Design}

    % % https://danmackinlay.name/notebook/latex_fonts.html#unicode-math
    % % https://danmackinlay.name/notebook/latex_fonts.html#unicode-math-fonts
    % % sudo tlmgr update --self
    % % sudo tlmgr install unicode-math fontspec lualatex-math l3kernel l3packages l3experimental
    % % sudo tlmgr install tex-gyre tex-gyre-math

    % {\AirSup} and {\AirStr} were published in GDW’s house style, with the main text is set in \href{https://en.wikipedia.org/wiki/Univers}{Univers} and the page having two columns with a rule between the columns and below the header. {\TSOH} followed this style to a large degree, but appears to have been set in \href{https://en.wikipedia.org/wiki/Helvetica}{Helvetica}

    % I want the rules to be readable and beautiful, but I also wish to preserve some of the design heritage of the earlier rules.

    % Therefore, I have maintained a layout with a two-column design, with a rule separating the columns and under the header, adapted the header from the GDW editions, and adapted the section format from the first-edition rules.

    % However, I have replaced the san serif font with a serif font designed for readability. After considering several options, I have chosen the \href{https://www.gust.org.pl/projects/e-foundry/tex-gyre/schola}{Schola} font, based on the URW Century Schoolbook font and adapted for LaTeX by the GUST foundry as part of the TeX Gyre project. Schola has a version for math, which is useful for text like “\binaryrelation{1 ≤ \CC}{≤}{2}”.

    % Of the other TeX Gyre fonts, the Palatino version would also be suitable, but Palatino is heavily used in {\itshape Birds of Prey}, and I wish to maintain a clear distinction.

    % \section{Rewriting and Reorganization}

    % This section collects my ideas for rewriting and reorganization.

    % \begin{itemize}
    %     \item Move the rules on formations to their own section.
    %     \item Move the modifications to the rules for aircraft with properties (e.g., LRR, HPR) to their own appendix.
    %     \item Move the rule on loss of thrust with altitude to the section on speed.
    %     \item  I think the following sections in particular could benefit from rewriting:
    %     \begin{enumerate}
    %         \item Recovery periods.
    %         \item Visual sighting.
    %         \item Electronic warfare.
    %     \end{enumerate}

    % \end{itemize}

    % \section{Outstanding Issues}

    % \subsection{Rule Interpretations}

    % \begin{itemize}

    % \item If a free aircraft under attack by an IRM selects idle power and fails the roll, what modifier does it use? The one for the previous power setting or the one for normal power? The consensus in the group is that it uses the modifier for normal power.

    % \item If a free aircraft under attack by an IRM selects normal power, does it automatically gain the corresponding modifier? The consensus in the group is that it can.

    % \item “If an aircraft's speed includes a fraction; say it is 4.5 instead of 5.0, round it up to simplify determining proportions but use its actual speed when moving. This would give the same result as in the above paragraph (5 into 17). The difference comes in the execution of the segments. The moves would be 4 for 1, 4 for 1, 3 for 1, 3 for 1, and then 3 for the aircraft's half FP.”

    % What the heck does “3 for the aircraft's half FP” mean?

    % If the aircraft has a speed of 4.5, then it has either 4 or 5 FPs, depending on whether it has 0.5 FP carry. If the aircraft has 5 FPs, then I can understand the above as meaning the missile moves 4, 4, 3, 3, and 3 FPs before each aircraft FP. If the aircraft has 4 FPs, then does it mean the missile moves 4, 4, 3, and 3 FPs before each aircraft FP and then 3 FPs after the last aircraft FP?

    % This issue would not occur if the proportional moves were calculated in terms of available FPs instead of speed. If the aircraft has 4 FPs and the missile 17 FPs, then the proportional moves would be 5, 4, 4, and 4 FPs. If the aircraft has 5 FPs and the missile 17 FPs, then the proportional moves would be 4, 4, 3, 3, and 3 FPs.

    % \item It seems to be an anomaly in that the missile has to be at the same altitude and one of the positions shown in the diagrams in order to attack, since in most circumstances a missile can use free descent. However, a missile with a turn rate of less than BT/2 cannot use free descent when it is transitioning from climbing to diving flight.

    % It might make sense to say a missile attacks in one of four cases:
    % \begin{enumerate}
    %     \item The target moves to the same position (location and altitude) as the missile.
    %     \item The missile moves to the same position (location and altitude) as the target.
    %     \item The missile moves to one of the locations shown in the diagrams, has the same altitude as the target, and has at least one FP left in its proportional movement.
    %     \item The missile moves to one of the locations shown in the diagrams, is one altitude level above the target, has at least one FP left in its proportional movement, and can use free descent.
    % \end{enumerate}
    % The last case can be generalized to: can use the remaining FPs in its proportional move to move forward one hex and reach the same altitude as the target (using any allowed combination of climbs, dives, and free descents).

    % \end{itemize}

    % \subsection{Rule Changes}

    % \begin{itemize}
    % \item In Malcolm Pipe's second-edition text, VFPs can be used as preparatory FPs for rolls but not for slides. This follows JD Webster's Genie post in 1995.

    % \item The rules state that a second slide may be “performed” if the aircraft speed if 9.5 or more provided 4 FPs are expended between the two slides.

    % An aircraft with a speed of 9.5 or more is supersonic and requires at least three preparatory FPs for each slide, so \emph{declaring and completing} two slide maneuvers requires at least 12 FPs (4 FPs for the first, 4 FPs waiting, and 4 FPs for the second). This requires a speed of at least 11.5 (allowing for 0.5 FP carry).

    % I note that in the original {\AirSup} rules, supersonic aircraft are not required to expend an additional preparatory FP and so require only 10 FPs (3 FPs for the first slide, 4 FPs waiting, and 3 FPs for the second). This requires a speed of at least 9.5 (allowing for 0.5 FP carry). Thus, the limit does not seem to have been updated in the change from {\AirSup} to {\AirPow}.

    % The above ignores maneuver carry, which is not explicitly mentioned in the rules but seems to be widely adopted. If we ignore the explicit speed requirement and simply obey the waiting period of 4 FPs, then a first slide can be \emph{declared and completed}, and a second slide can be \emph{declared} in 8 FPs (3 FPs for the first slide, 4 waiting, and 1 FP to declare the first slide). Furthermore, if sufficient preparatory FPs are carried in, the first slide can be \emph{completed} and the second \emph{declared} in only 6 FPs (1 FP to execute the slide, 4 FPs waiting, and 1 FP to begin the second slide).

    % \item Now that preparatory FPs for displacement and lag rolls can be VFPs, does it make sense to have the additional complication of the rule for climbing and diving barrel rolls? I suspect not.

    % \item Are the two DPs for VIFF-assisted turns in addition to the normal turn cost?

    % \item In the v2.4 rules, RWR-C+/D+ gives a +1 modifier to missile attacks and towed jammers give a +1 or +2 modifier to BRM/RHM/AHM attacks.

    % \end{itemize}

    % \end{document}

%!TEX root = ./rules-working.tex
%LTeX: enabled=false

\rulechapter{Radar Guided Missiles}
\label{rule:radar-guided-missiles}

There are three kinds of radar guided missiles in the game: BEAM-RIDERs (BRMs), RADAR-HOMERs (RHMs), and ACTIVE-HOMERS (AHMs). Each will be discussed separately in the following sections.

\section{Beam Riding Missiles}
\label{rule:beam-riding-missiles}

Beam riding was the earliest missile guidance method adopted for air warfare. The firer simply pointed a compact radar beam at the target, then fired a missile which flew along the beam until it hit something. While more effective than shot-gunning clusters of unguided rockets at lumbering bombers, beam riding missiles had serious limitations. Fighters could easily evade them by maneuvering out of the firer's radar beam. A sudden course change by the firer could even yank the beam away from the missile causing it to become unguided. The early radars used for BR guidance were susceptible to ECM jamming. Nevertheless, BRMs provided aircraft with the first all-weather guided weapons and certainly increased the odds against a heavy bomber.

\paragraph{BRM Launch Prerequisites.}\label{rule:brm-launch-requirements} In order to launch a BR type missile, the firer must:

\begin{itemize}

    \item have a target in his limited radar arc (even if his normal radar arc is different).

    \item have a lock-on to the target unless SNAP-FIRING per advanced rule 17.6.

\end{itemize}

\paragraph{BRM Guidance Requirements.} To successfully guide a BRM, the firing aircraft must ILLUMINATE the target (Rule 17.4) and maintain a lock-on until the missile hits, misses or is removed from play.

\paragraph{BRM Tracking Requirements.} A BRM is removed from play at the end of any proportional move and/or game turn in which:

\begin{itemize}

    \item it ends its move further away in terms of range from the target than when it started.

    \item it ends its move outside the guiding aircraft's limited radar arc.

    \item it ends its move with the target outside the missile's own 180+ arc.

    \item the firer fails to maintain lock-on and illumination.

\end{itemize}

\section{Semi-Active Radar Homing Missiles}
\label{rule:semi-active-radar-homing-missiles}

Radar Homing (RH) was the next method developed for guiding missiles. Engineers soon figured out that by putting a radar receiver in the missile's nose, it could detect radar energy bounced off the target by an illuminating radar beam. This allowed the missile to guide itself to the target giving it greater maneuverability as it was no longer constrained to staying within a guidance beam. So effective is radar homing that it remains the primary guidance method in use today.

\paragraph{RHM Launch Prerequisites.} In order to launch an RHM, the firer must:

\begin{itemize}

    \item have a target inside his normal radar arc, but not past his 150+ arc.

    \item have a lock-on to the target, unless SNAP-FIRING per advanced rule 17.6.

    \item have the target between the missile's minimum firing range and no further than three times the launch aircraft's tracking strength in range.

\end{itemize}

\paragraph{RHM Guidance Requirements.} To successfully guide an RHM, the firer must ILLUMINATE the target and maintain a lock-on until the missile hits, misses, or is removed from play.

\paragraph{RHM Tracking Requirements.} An RH missile is removed from play at the end of any proportional move and/or game-turn in which:

\begin{itemize}

    \item it ends its move further from the target than when it started.

    \item it ends its move with the target outside the missile's own 150+ arcs.

    \item the firer fails to maintain lock-on and Illumination.

\end{itemize}

\section{Active Homing Missiles}
\label{rule:active-radar-homing-missiles}

With the advent of miniaturized electronics, it became possible to build radars small enough to fit inside some missiles. With its own radar, the missile can theoretically guide itself to a target without any help from the firing aircraft. In reality, active homing is only possible at short ranges due to the small radar antenna in the missile. To utilize longer ranges, an AH missile must be guided like a RH one, or through mid-course guidance updates, until it reaches active homing range.

\paragraph{AHM Launch Prerequisites.} As for RHMs except the target may be further than three or four  times the tracking strength in range away. That launch restriction does not apply to AHMs.

\paragraph{AHM Guidance Requirements.} Different requirements apply depending on whether the missile has mid-course guidance capability (as indicated on the MDT under “MCG”) or not.

\begin{itemize}

    \item Normal AHM: As for RHMs until active homing range is reached.

    \item Mid-Course Guidance AHM: Only a lock-on needs to be maintained to the target. The firer does not have to illuminate the target.  

\end{itemize}

\paragraph{AHM Tracking Requirements.}

\begin{itemize}

    \item Normal AHM: As for RH missiles until active homing range is reached.

    \item MCG Capable AHM: An MCG missile need only keep the firer in its 90 degree or less arc (in order to receive guidance signals in its rear antenna). The radar vertical limits table does not apply in this case. While in mid-course guidance, the AHM also does not have to keep the target in its 150+ arc.

\end{itemize}

\paragraph{Terminal Active Homing Phase.} At the instant an AHM moves into active homing range and the target is in its 150+ arc, its own radar takes over. The firer is freed from the requirement to maintain lock-on and/or illuminate the target.  When active, an AHM must end each proportional move and/or game turn no further away than when it started and with the target in its 150+ arc, otherwise it is removed from play.

\paragraph{Multi-Target Track Technology Effects.} If the firing aircraft has multi-target track capability and is using MCG capable AH missiles, it may:

\begin{itemize}

    \item simultaneously guide MCG missiles at as many targets as it has lock-ons with.

    \item launch MCG missiles at different targets in the same game-turn. Still, no more than two missiles per turn may be launched as per normal rules.

\end{itemize}


\section{Target Illumination}
\label{rule:target-illumination}

Illuminating (or “painting”) refers to directing a high-power radar beam at a target.

\paragraph{Illumination Procedure.} To illuminate, an aircraft must have a locked-on target in its radar arc. The act of illuminating is declared in the Aircraft Decisions Phase. Once declared, illumination is automatic and is maintained as long as the illuminator keeps the target in its regular radar arc (check at the end of each proportional move and/or game turn).

\paragraph{Limitations.} An aircraft may only illuminate one target at a time. Only the aircraft which fired the missiles can illuminate a target for those same missiles. Even though only two missiles may be launched per turn, any number may be guided by an illuminating aircraft, thus additional missiles may be launched in later turns even if the first two have not yet reached the target.

\paragraph{Order Of Flight Effect.} An illuminating aircraft has its order of flight modified so that it flies immediately after its target, even if it would have normally moved at another point in time, alternating proportional segments of flight with the target and guided missiles. The missiles and target move first, then the illuminator. This simulates the disadvantage of flying predictably while illuminating a target.

Note: Illuminating may be used to confuse or mislead an enemy. Anytime a missile is fired, even if a heat seeking type, you may declare illumination and act as if you are guiding a radar missile.

\section{Missile Shoot-Outs}
\label{rule:missile-shoot-outs}

It often happens that opposing aircraft will fire and guide radar missiles at each other in the same game turn. In this case, a shoot-out occurs with both aircraft moving proportionally along with their missiles. Whoever’s missile arrives first, attacks first unless some of each side's missiles can arrive in the same proportional move; in this case they all attack simultaneously.

\paragraph{Shoot-out Procedure.} When a shoot-out situation occurs, each player secretly notes on paper whether he will engage his opponent's missiles or not; both reveal their choice in the Aircraft Decisions Phase. Whoever engages loses his lock-on and cannot illuminate and thus has his missiles removed from play. The other resolves missile guidance and attack normally. If both engage, all missiles are removed, however both aircraft will still have to move in the engaged phase. If neither engaged, then a shoot-out occurs normally with both aircraft moving proportionally along with their missiles as above.

\section{BRM, RHM, AHM Countermeasures}

\paragraph{Chaff Decoys.} Radar guided missiles are vulnerable to chaff just like IRMs are vulnerable to flares. Chaff may be dispensed manually or through a DDS program as described in Chapter 15 under IRM Countermeasures. Also see Chapter 19: Electronic Warfare.

Note: Expendable mini-jammers (introduced in the 1980s), function like chaff in the game but are superior against some missile types. See Chapter 19 for additional discussion.

\paragraph{Ground Clutter.}\label{rule:rgm-ground-clutter} Ground clutter affects radar guided missiles that attack targets close to the ground. If the target aircraft is within five altitude levels of the ground or less, and the illuminating or tracking aircraft which is guiding the missile is higher than the target, or if an active AH missile dives to attack a target (loses 2 or more levels in its move), a ground clutter modifier is applied to the to hit die roll.

The modifier is determined first, by subtracting the target's altitude above the ground from six. Then subtract the missile's ECCM rating from the remainder. If the result is still positive, that is the die roll modifier.


\trainingnote{You are now ready to play Training Scenario Four and all air combat only scenarios.}

\begin{advancedrules}

\section{Snap-Firing Missiles}
\label{rule:snap-firing-missiles}

Snap-Firing allows BR, RH, and AH missiles to be launched even when a lock-on is not held against the target. Snap-firing is allowed if:

\begin{itemize}
    \item the firer is using boresight or auto-track radar modes; and
    \item the target is in the firer's limited arc regardless of its normal radar arc.
\end{itemize}

Snap-fired missiles roll for launch with a $+3$ modifier as there is a possibility that they will be out of position to receive guidance signals when, and if, a lock-on is achieved to the target. If, in the immediately ensuing Air Radar phase, a lock-on is not achieved, the missiles are removed from play.

\section{Radar Missile Out of Envelope Shots}
\label{rule:radar-out-of-envelope-launches}

You may reduce the listed minimum range envelope of radar guided missiles by half (round up; i.e. half of 5 is 3) by accepting a $+3$ modifier to the launch roll.

You may extend the maximum range envelope of RHM missiles to 4 times the radar's tracking strength by accepting a $+3$ modifier to the launch roll.

\section{AIM-26A Nuclear Falcon}

The AIM-26A is a radar guided missile which incorporated a nuclear warhead.

\paragraph{Launch and Flight.} In all respects, treat the AIM-26A as a regular RHM for launch and flight purposes.

\paragraph{Nuclear Attack.} The AIM-26A has a nuclear blast zone exactly like the AIR-2 Genie. The AIM-26A detonates upon entering the target's position or upon the guiding player's command when within 3 hexes of the target. The blast zone affects the target aircraft and any others that have already moved in the turn of attack immediately. Other aircraft moving later in the turn, are attacked if they remain within the blast zone.

\end{advancedrules}

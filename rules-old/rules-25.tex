%!TEX root = ./rules-working.tex

%LTeX: enabled=false



\rulechapter{Surface to Air Missiles}

\label{rule:sams}

Surface to Air Missile (SAM) units consist of one or more self-contained launch vehicles, or an infantry fire team with man portable SAMs, or a group of launchers clustered about a target tracking radar (TTR).

\paragraph{SAM Types.} SAMs may be IR guided, radar guided, optically guided, or laser guided. Some SAM unit counter specifics are described in the sections pertaining to the different SAM types and an example is given in the unit identification chart of how to read the information from a counter. The rules below are applicable to all types of SAMs.

\paragraph{SAM Missile Data Tables.} There are several tables detailing the characteristics of specific SAM types. The tables are organized and read in a manner similar to the air to air missile data tables. The Boost and Minimum Altitude entries are unique to SAMs however.

\paragraph{SAM Missile Flight.} SAMs fly and attack aircraft using the air to air missile flight rules. The following sections cover the minor differences between air to air and SAM missiles.

\begin{itemize}

    %!TEX root = ../rules-working.tex
%LTeX: enabled=false
\addedin{1C}{1C-tables}{
\begin{onecolumntablefloat}
\begin{onecolumntable}
\tablecaption{table:sam-launch-modifiers}{SAM Launch Modifiers}

\begin{tabularx}{0.9\linewidth}{lL}
\toprule
\multicolumn{2}{c}{IR SAMs}\\
\midrule
\plus{}     &Lesser of flare PPL or misile flare vulnerability number if DDS program is in effect.\\
\plus{3}    &If fired into Sun clutter.\\
\midrule
\multicolumn{2}{c}{BR, CG, CW, TVM, OG, and LG SAMs}\\
\midrule
\multicolumn{2}{c}{No launch modifiers apply.}\\
\bottomrule
\end{tabularx}
\end{onecolumntable}
\end{onecolumntablefloat}
}


    \item\itemparagraph{Launches.} SAMs are launched in the SAM Interaction Phase and not the air to air missile launch phase. A launch roll is used as for air to air missiles\addedin{1B}{1B-tables}{ with modifiers according to Table \ref{table:sam-launch-modifiers}}. When successfully launched, missiles are placed in the launch unit's hex with any initial facing desired.

    \item\itemparagraph{Boosted Flight.} On the game turn of launch only, SAMs have a booster phase during which the missile is accelerating and/or being gathered under guidance control. The missile is not armed nor can it began maneuvering until a number of FPs have been expended in boosted flight equal to the amount listed under “boost” on the SAM missile data table.

    While in the booster phase, the missile may only expend FPs to move forward and/or gain altitude. Aircraft may not be attacked by missiles in boosted flight. The missile is armed immediately upon completing boosted flight and may then commence flying normally. They may also opt to turn immediately once they are armed as for air to air missiles.

    \item\itemparagraph{Speed.} A SAM missile's start speed is always equal to its listed speed on the first turn of launch and on each turn of sustainer flight. SAMs do not use the speed attenuation tables and factors until the first game turn of unpowered flight. This is different than for air to air missiles which must take into account the varying aircraft launch speeds. 
    
    %On each turn of sustainer-powered flight the SAM’s speed attenuation factor is 1.0, but SAMs are still subject to speed loss from turns, maneuvers, climbs, etc.

\end{itemize}

\paragraph{SAM Interaction Phase.} The expanded sequence of play outlines the order of actions to be taken in the SAM interaction phase. The sequence must be followed exactly to allow for the proper representation of electronic warfare effects and SAM unit capabilities.

\paragraph{SAM Unit Characteristics And Damage Effects.} SAM units may have one or more of the following characteristics:

\begin{itemize}
    \item\itemparagraph{Quick Reaction Capability.} There are two different times in which lock-ons may occur in the SAM Interaction phase. SAM units with a “Quick-Reaction” capability may attempt lock-ons in both. Other SAMs may only attempt lock­ons in the second. As the first lock-on attempts will occur before SAM launch attempts; it may be possible for quick reaction SAMs to acquire and launch at a target in the same game turn.

    %ote: All man portable SAMs are considered quick reaction capable.

    \item\itemparagraph{Volley Capability.} Each SAM unit is given a volley number as indicated on the counter. This is the maximum number of missiles that may be launched in a single game-turn. It is also the maximum number of missile's that may be guided at any one time by the unit. IR and CW guided missiles are excepted; see below. When more than one SAM is launched in a single turn at the same target from the same unit, the “follow-on” missile rules apply.

    \item\itemparagraph{Ready Missiles.} Each SAM unit is given a ready missile number as indicated on the counter. This is the total number of missiles it has ready to launch at the start of each game. During play, keep note of the number of missiles expended by the unit. Each missile that is launched or fails to launch is counted as a ready missile expended. When all have been fired, the unit is considered out of missiles and may no longer fire on aircraft though it remains in play as a target.

    Note: Infantry SAM units have as many ready missiles as they have launchers as stated in the scenario. If not stated, the default value is always two missile launchers.

    \item\itemparagraph{SAM Reload Capability.} All Infantry SAMs, and SAM units indicated as having an auto-reload capability may attempt to replenish expended ready missiles during play. SAM units which do not launch or guide missiles during a game turn may perform reloading. They are allowed to replenish expended missiles as follows:
    
    \begin{enumerate}
        \item[a)] Infantry SAMs: Roll one die in the admin phase of the turn for each empty launcher. A result of 3 or less means the launcher is reloaded.

        \item[b)] Auto-Reload Capable Units: Up to two expended ready missiles may be replaced automatically from the unit's supply of reloads on each turn of reloading.
    \end{enumerate}

    The scenario will usually list the number of reload missiles available to a unit. If not, the default value is two reloads per ready missile originally available.

    \item\itemparagraph{Multi-Target Engagement Ability.} If the volley number of the unit is underlined, it is multi-target capable and is allowed to attempt and obtain lock-ons to as many different targets as its volley number. It may also launch and guide missiles simultaneously at as many different targets as it has lock-ons to with one or more missiles being assigned to each target as they are launched. The volley number is still the limit on how many missiles may be in the air at once.   

\end{itemize}

\paragraph{Damage Effects On SAM Units.} SAM units may not launch or guide missiles while suppressed. Each “D” hit on a SAM unit destroys half of any remaining ready missiles and reloads (round up) if it does not also destroy the unit. Radar SAM units may be disabled if an ARM hit destroys their radar (see Rule \ref{rule:radar-guided-sams}).

%\paragraph{SAM Minimum Altitude Capability.}  SAMs may not be launched or guided at targets within less than the listed number of levels to the ground.  If the SAM data table lists a “T”, then the SAM can be fired at targets in TFF.  If there is a “+N” modifier next to the “T”, then that modifier applies to any missile attack die rolls to targets in TFF.

\section{Early Warning Radars and SAM Target Passdowns}

%!TEX root = ../rules-working.tex
%LTeX: enabled=false
\addedin{1C}{1C-tables}{
\begin{onecolumntablefloat}
\begin{onecolumntable}
\tablecaption{table:ewr-pass-down-modifiers}{EWR Pass-Down Modifiers}

\begin{tabularx}{1.0\linewidth}{LL}
\toprule
Modifier&Condition\\
\midrule
\minus{2}   &Target IFF is on.\\
CH or MJ effectiveness     
            &Target DDS program on.\\
\binaryminus{AJM rating}{EWR ECCM}
            &Target AJM is on.\\
\binaryminus{BJM rating}{EWR ECCM}
            &BJM noise in effect.\\
\bottomrule
\end{tabularx}
\begin{tablenote}{0.9\linewidth}
    If multiple BJMs are in play, consider only the most effective one against the radar.
\end{tablenote}
\end{onecolumntable}
\end{onecolumntablefloat}
}



    \paragraph{Early Warning Radar (EWR).} EWRs assist SAM units in achieving lock-ons. If an EWR is part of an Integrated Air Defense System (IADS), it may provide radar passdowns to units in the IADS. If EWR is integral to the SAM unit, it may provide radar passdowns to its own Target Tracking Radar (TTR). The scenario defines the radar passdown chain of command. A scenario may indicate that a CCU counter may also be required to conduct pass-downs.

\paragraph{Passdown Procedure.} An EWR may attempt one passdown to each unit in its IADS per game turn. A single target may be passed down to one or more different SAM units or different targets may be passed to different units.

EWRs perform a passdown by announcing the EWR detected target and the unit it is being passed to. Roll the ten-sided die and apply any applicable modifiers\addedin{1B}{1B-tables}{ from Table \ref{table:ewr-pass-down-modifiers}}. A result of 7 or less indicates success. A successful passdown provides a \minus{3} modifier to the SAM unit's lock-on attempt.

\paragraph{EWR Target Detection.} EWRs detect aircraft using the same procedure as aircraft radars except they use the EWR line of the Radar Table. Any aircraft within the line of sight of an operating EWR (not jammed, suppressed or shut down) is an eligible target as EWRs have a 360° sweep.  

% EWRs perform radar searches in the Air Radar Search Phase.  

% ECCM is “0”.

\paragraph{Radar Horizon.} Aircraft in TFF may not be detected or locked onto by radars until within 20 hexes of an “MTI” (Moving Target Indicator) capable EWR or “T” level capable SAM unit having a line of sight to them. The detection range is increased to 60 hexes if radar is elevated (on higher terrain than the aircraft or on a ship’s mast or high tower).

\section{IR Guided SAMs}

\paragraph{Missile Data Tables.} There are separate tables for man portable and vehicle mounted IR SAM units. Unlike for Air to Air IR missiles, IR SAMs are not given launch envelopes. Instead they are given a maximum lock-on range.

The Seeker head type is used exactly as for air to air missiles. IR SAMs must meet the same angle-off launch restrictions and tracking requirements that apply to IRMs with the same seeker head.

\paragraph{Target Acquisition And Launch.} All IR SAM units are quick reaction types. In order to launch missiles in the SAM Interaction Phase, they must have previously acquired seeker Optical lock-ons to the target.

\paragraph{Seeker Lock-on Procedure.} 
%!TEX root = ../rules-working.tex
%LTeX: enabled=false
\addedin{1C}{1C-tables}{
\begin{onecolumntablefloat}
\begin{onecolumntable}
\tablecaption{table:sam-optical-lock-on-modifiers}{IR, OG, or LG SAM Unit Lock-On Modifiers}

\begin{tabularx}{1.0\linewidth}{LL}
\toprule
Modifier&Condition\\
\midrule
\plus{2}    &Associated CCU inoperative.\\
\plus{3}    &Unit in target's Sun clutter.\\
\bottomrule
\end{tabularx}
\end{onecolumntable}
\end{onecolumntablefloat}
}


If a target is visually sighted and within maximum lock on range (two altitude levels difference equals a hex of range), roll the die during either or both lock-on times of the SAM interaction phase\addedin{1B}{1B-tables}{ and apply approriate modifiers from Table \ref{table:sam-optical-lock-on-modifiers}}.

One lock-on attempt is allowed per launcher or launch vehicle in the unit. Each may only maintain one lock-on at a time. A result equal to or less than the optical lock-on number given on the counter means target acquisition (lock-on) has occurred. A lock-on is retained until the fine of sight is broken or the aircraft is further away than the maximum lock-on range.

\paragraph{Command and Control Units.} The presence or absence of a command and control unit (CCU) working with the SAM unit may affect the die roll for target acquisition. IR SAM units with a CCU within command range (always 4 hexes or less) may also benefit from Early Warning Radar Information passdowns. See scenario specifics.

\section{Target Tracking Radars and Radar Guided SAMs}
\label{rule:radar-guided-sams}

Radar Guided Surface-to-Air Missiles consist of those with the following guidance methods: Beam Riding (BR), Command Guidance (CG), Continuous Wave Radar Homing (CW), and Track-Via-Missile data link (TVM).

\paragraph{Target Tracking Radars (TTRs).} All radar guided SAM units utilize TTRs to lock-on to targets and guide their missiles. The presence of a TTR with the unit is indicated by the inclusion of a radar frequency code on the front of the counter and a radar lock-on number on the back of the counter across from any optical lock-on numbers.

\paragraph{TTR Lock-Ons.} 
%!TEX root = ../rules-working.tex
%LTeX: enabled=false
\addedin{1C}{1C-tables}{
\begin{onecolumntablefloat}
\begin{onecolumntable}
\tablecaption{table:sam-radar-lock-on-modifiers}{SAM Unit Radar Lock-On Modifiers}

\begin{tabularx}{1.0\linewidth}{lL}
\toprule
Modifier&Condition\\
\midrule
\minus{2}   &Target IFF is on.\\
\plus{2}    &CCU is inoperative.\\
\minus{3}   &Pass-down received.\\
\binaryminus{AJM rating}{TTR ECCM}
            &Target AJM is on.\\
\binaryminus{BJM rating}{TTR ECCM}
            &BJM noise in effect.\\
CH or MJ effectiveness    
            &Target DDS program on.\\
\bottomrule
\end{tabularx}
\begin{tablenote}{1.0\linewidth}
    If multiple BJMs are in play, consider only the most effective one against the radar.
\end{tablenote}
\end{onecolumntable}
\end{onecolumntablefloat}
}


TTR lock-on attempts are allowed at the appropriate lock-on times of the SAM interaction phase, assuming the radar is not stand-off jammed or suppressed by previous attacks. There must be a line of sight to the target (meaning it is not terrain masked per Rule \ref{rule:line-of-sight}) although it need not be sighted, and the target must be within tracking range of the TTR.

Roll the die and modify the roll as required\addedin{1B}{1B-tables}{ from Table \ref{table:sam-radar-lock-on-modifiers}} for the presence of active and barrage jamming taking into account the SAM unit's missile's ECCM rating. A result less than or equal to the lock-on number on the counter means the target has been locked-on to.               

\paragraph{EW Jamming Effects.} TTRs blinded by BJM stand-off attacks may not attempt lock-ons. Aircraft with AJMs and aircraft in the protected arcs of BJMs used in the noise mode, get to use the jamming number as a modifier to the die roll of any lock-on attempts made against them. The modifier may be reduced to zero by the ECCM capability of the SAM unit TTR.

\paragraph{HOJ Capability.} Any radar guided SAM unit with HOME ON JAM capable missiles may opt to launch them in the HOJ Mode at aircraft conducting BJM noise or stand-off jamming attacks in the following cases:

\begin{itemize}

    \item If blinded by a stand-off attack, the unit may launch at the aircraft which did the stand-off attack.

    \item If a regular lock-on attempt failed against an aircraft using a BJM in either mode, HOJ missiles may be fired at it anyway.

\end{itemize}

Note: Home on Jam capable missiles launched under normal guidance at a barrage jamming target may switch to the HOJ Mode if their lock is broken by any means in mid-flight.

Whenever a target of a HOJ missile elects to stop using its BJM (announced in the SAM Interaction Phase), the missile will be removed from play unless a lock-on by the original launch unit is obtained by the end of the SAM Interaction Phase so that normal guidance can be resumed.

\paragraph{Tracking Requirements.} Radar SAMs must meet the following criteria at the end of each proportional move and at the end of the game-turn for the missile to continue tracking its target. If the criteria cannot be met, the missile is removed from play (loses guidance and self-destructs).

\begin{itemize}
    \item\itemparagraph{Beam Riders (BR).} Must keep the target within the missile's 180 degree arc, and be no further than two hexes away from any hex which is touched by a straight line extending from the center of the SAM unit hex to the center of the target's hex/hexside.

    \item\itemparagraph{Command Guided (CG).} Keep the target within the missile's 120+ arc.

    \item\itemparagraph{Continuous Wave Guided (CW).} Keep the target within the missile's 150+ arc.

    \item\itemparagraph{Track-via-Missile (TVM).} As for command guided.

    \item\itemparagraph{HOJ Mode.} As for Beam Rider.
\end{itemize}

Additionally, the TTR must maintain a lock-on (HOJ SAMs excepted) and the missile must end each game turn no further from the target than when it started.

\paragraph{Breaking TTR Lock-Ons.} 
%!TEX root = ../rules-working.tex
%LTeX: enabled=false
\addedin{1B}{1B-tables}{
\begin{onecolumntablefloat}
\begin{onecolumntable}
\tablecaption{table:sam-break-lock}{SAM Break Lock Die Roll.}

\begin{tabularx}{1.0\linewidth}{llL}
\toprule
SAM Type        &ECM Type       &Break-Lock Die Roll\\
\midrule
BR/CG           &DJM-A/B/C/D    &\binaryminus{DJM rating}{TTR ECCM}\\
CW              &DJM-C/D        &\binaryminus{DJM rating}{TTR ECCM}\\
TVM             &DJM-D          &\binaryminus{DJM rating}{TTR ECCM}\\
BR/CG/CW/TVM    &CH             &CH effectiveness\\
BR/CG/CW/TVM    &MJ             &MJ effectiveness\\
BR/CG           &AJM            &\binaryminus{1/2 AJM rating}{TTR ECCM}\\
CW/TVM          &AJM            &---\\
OG/LG           &FL             &1 if any FL PPL is in effect\\
\bottomrule
\end{tabularx}
\begin{tablenote}{1.0\linewidth}
    Round up all half values. Some DJM-A/B and some AJM pods may be noted as effective against CW and TTR SAMs, see EP tables.
\end{tablenote}
\end{onecolumntable}
\end{onecolumntablefloat}
}

TTR lock-ons can be broken as follows during the SAM interaction Phase:

\begin{itemize}
    \item When target has a CHAFF PPL present and a break lock roll succeeds against BR, CG, CW, and TVM SAM units.

    \item When target has a MINI-JAMMER PPL present and a break lock die roll succeeds against BR, CG, CW, and TVM SAM units.

    \item When target is equipped with a DJM working in the SAM unit TTR's frequency and a break lock die roll succeeds (BR, CG, CW, and TVM SAM units as noted). 

    %\item When target is equipped with an AJM working in the SAM unit TTR's frequency and a break lock die roll succeeds (BR and CG SAM units only). 

    \item When the SAM TTR is jammed by a stand-off Jamming attack.
    
    \item If the SAM player opts to break lock, or shuts down the radar (see rule \ref{rule:arms}).

\end{itemize}

\addedin{1B}{1B-tables}{The break-lock rolls are given in Table \ref{table:sam-break-lock}.}

%Towed decoy adds +1 to ECM rating to break TTR lock-ons if DJM/AJM, internal or podded, is ECM effective against the radar type as specified above.

TTR lock-ons are broken in the Flight Phase when:
\begin{itemize}

    \item The SAM unit suffers a combat result of suppressed or greater or is still suppressed from a previous attack.

    \item The SAM player opts to break lock.

    \item The target successfully terrain masks.

\end{itemize}

However, the following consideration is given to a missile beginning the instant terrain masking occurs:

\begin{itemize}
    \item 
the missile may no longer climb, dive or maneuver, but only fly forward in its proportional move. It is not removed from play. If the aircraft reappears from masking during or at the end of its current flight, the original launching SAM unit is immediately allowed a reacquisition die roll, conducted as a normal lock-­on roll. If successful, the missile may resume normal flight. If not, the missile is then removed from play.
\end{itemize}

\paragraph{Dirty Tricks Department.} If desired, CG and TVM SAMs may be launched unguided (without a lock-on). The missile when launched is assigned a climb ratio in the form of levels climbed to hexes entered (2 to 1, 3 to 1 etc.); meaning it will expend FPs as necessary to meet the ratio. The missile may not do anything else until brought under guidance in a later game turn. It may not be brought under guidance in the same turn it is launched unguided, however it is not removed from play until after 5 game turns of being unguided or its TOF is up,  whichever is less.

To be brought under guidance, the original launch unit must obtain a lock-on to a target and the missile must meet the tracking requirements. If a lock-on is obtained with the missile out of tracking parameters, the missile is removed from play. Missiles may not be launched guided and then revert to unguided modes. When a SAM is launched unguided, no launch signals are received by aircraft RWRs, hence no launch warning.

\section{Optical and Laser Guided SAMs}

Optical and/or laser guided SAMs are those with OG or LG guidance codes. Some radar guided SAMs have optical back­up modes.

\paragraph{OG/LG Target Acquisition.} Like IR SAMs, OG and LG SAMs must have an optical lock-on to the target. To obtain an optical lock-on, the target must be sighted and within the maximum lock-on range of the missile. For man portable OG missiles, the max lock-on range is listed, for others which use TV and sometimes IR optics systems, the lock-on range is six times the aircraft's visibility number.

The die is rolled as for IR SAM lock-ons\changedin{1B}{1B-tables}{ except the lock­on modifiers will include}{ with approriate modifiers from Table \ref{table:sam-optical-lock-on-modifiers} and with} the V.A.S.\ range modifiers on the sighting tables when attempting locks with non-Infantry OG/LG type SAMs.

\paragraph{Guidance Requirements.} Optical and laser guided SAMs are basically short ranged line of sight weapons. Most OG SAMs are tracked via a flare in the rear of the missile. Therefore, they must stay in the line of sight of the optical target tracker. Laser guided SAMs ride a laser beam pointed at the target, targeting reflected laser energy off the aircraft. Both are therefore treated as having to meet the same tracking requirements as BR type radar guided SAMs.

\paragraph{OG Guidance Limits.} A SAM unit may never keep more than one OG missile in the air at a time. Therefore, OG missiles may not be volleyed unless stated otherwise in a unit briefing or scenario.

\paragraph{OG and LG Advantages.} OG/LG SAMs cannot be jammed and are not very susceptible to decoys. A target aircraft will receive no RWR indications of OG/LG SAM meaning it must be sighted to be engaged.

% unless the RWR is a Type C+ or D+.  RWR C+ and D+ equipped aircraft can detect OG and LG SAMs identically as IRM SAMs and try to evade.

\paragraph{Dual Guidance Modes.} Many radar guided SAMs have an optical backup capability. These dual mode SAMs are allowed to obtain simultaneous optical and radar lock-ons, but only against the same target. The optical lock can be obtained on earlier or later game turns.

If both locks exist at launch, radar guidance always takes priority and must be used until forced into the backup optical mode. If no radar lock is achieved, or if broken prior to launch (due to jamming, shut downs or other reasons) and an optical lock exists, the SAM may be launched and guided optically. If a radar lock is lost during the missile's flight and an optical lock already exists, it may revert to OG mode, otherwise it is removed from play. If more than one missile was under guidance, all but one are removed from play due to the OG guidance limits. Once in the OG backup mode the SAM unit may not revert back to the radar mode until the optically guided missile's flight is concluded for any reason.
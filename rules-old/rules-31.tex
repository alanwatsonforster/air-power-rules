%!TEX root = ./rules-working.tex
%LTeX: enabled=false

\addedin{2A}{2A-helicopters}{

\rulechapter{Helicopters}

\paragraph{Movement.}

Helicopters always move first in the order of flight.  If more than one side has helicopters, then normal initiative rules apply to determine the order of movement.

During a turn, a helicopter may do any of the following:  
\begin{itemize}
\item Stay in place and change facing by any amount.
\item Move forward one hex and change facing by up to 60°.
\item Change facing by up to 60° then move forward one hex.
\item Hover (do nothing).
\end{itemize}

Helicopters denoted as having High Speed/Agility on the Helicopter Data Table may execute any of the following HSA movement options:
\begin{itemize}
\item Make a high-speed dash by moving two hexes and not turning.
\item Change facing by up to 180° and move forward by one hex.
\item Move forward one hex and change facing by up to 180°.
\end{itemize}

HSA capable helicopters performing HSA moves into or through urban, woods, or built up terrain while in T-level flight may collide with the ground or obstacles (e.g., high tension power wires, tall trees, cell towers, etc.).  Any helo entering such built-up terrain while executing an HSA move must roll a die. On a roll of 1, a collision occurs and the helicopter crashes.

\paragraph{Helicopter Altitude.}

Helicopters may fly either at T-level or at altitude 1 level above the ground.   They may descend to T-level at the start of the move or exit T-level and climb to altitude 1 level above the ground at the start of their move.  The change in altitude may be declared at the start of the turn and cost nothing to perform.  

Helicopters at T-level may be attacked as ground targets or as air targets.  If attacked as a ground target, then a S, D, 2D, or K result gets treated as a L, H, C, or Kill as aircraft damage results respectively.  Ground defense strength and sighting range is listed on Helicopter Data Table.

\paragraph{Helicopter Combat.}

Helicopters may execute one air-to-ground attack or one air-to-air gun attack against another helicopter upon completing its move.

Helicopters not executing a gun attack may execute an air-to-air attack using air-to-air missiles if armed with them.  If equipped with an air-to-air missile, the helicopter may launch a missile during the missile launch phase.

Helicopters may respond to air-to-air gun attacks if their gun is asterisked on the Helicopter Data Table indicating an air-to-air ability subject to the two shots per turn limit as for aircraft.

Helicopters attacked as a ground target are considered soft targets and vulnerability modifiers are ignored.

Helicopters attacked as ground targets receive the terrain modifiers as if a ground target.

Helicopters exposed to barrage fire are hit on a roll of 1 or 2 instead of 1.

\addedin{2A}{2A-helicopter-door-guns}{
\paragraph{Helicopter Door Guns.}

Some transport helicopters have pintle-mounted guns firing out of their side or rear doors. These are indicated in the Helicopter Data Table by arcs of LSD (left side door), RSD (right side door), and RD (rear door).

A side-door gun can fire at targets in the helicopter's corresponding \arc{60}/\arc{90}/\arc{120}/\arc{150} arcs and a rear-door gun can fire at targets in the helicopter's \arcminus{60} arcs.

Each door gun may fire independently. 

A door gun may conduct one air-to-ground strafing attack on any ground target in its arc within two hexes of range. Aiming is not required and there are no modifiers. 

Alternatively, a door gun may return fire on any aircraft carrying out a gunnery attack \changedin{2B}{2B-helicopter-door-gun-range}{within its arc}{within its arc and air-to-air range}.\addedin{2B}{2B-helicopter-door-gun-range}{ The air-to-air range of door guns is 1 for a caliber of less than 20 mm and 2 otherwise.} The hit roll is \minusafter{1} and there are no modifiers. 

}

\paragraph{Helicopter Defilade.}

Helicopters in T-level flight may use buildings and woods as cover.  Any helicopters which begin movement already in T-level and in urban, woods, or built up area hex and which chooses not to move or change facing may declare itself in defilade at the end of the turn.

Helicopters in defilade may not move or change facing in subsequent turns unless they declare exiting defilade to T-level or level 1 at the start of their move. 

Helicopters in defilade may not conduct visual searches against, or attack enemy ground units unless equipped with Mast Mounted Sights (MMS).

MMS-equipped helicopters attacking from defilade are considered at T-level after the attack.

Helicopters in defilade are considered camouflaged ground units for purposes of being sighted from the air and may not be attacked by SAM units or AAA units until they reveal themselves by exiting defilade or making an attack from defilade.

Helicopters may not enter defilade in a hex containing enemy helicopters or ground units or any hex adjacent hexes containing any enemy ground units or enemy helicopters. 

}
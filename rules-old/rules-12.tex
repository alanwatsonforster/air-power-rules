%!TEX root = ./rules-working.tex
%LTeX: enabled=false

\rulechapter{Order of Flight}
\label{rule:order-of-flight}

This chapter covers the procedures which determine the order in which aircraft move during the flight phase. In play, aircraft are moved one at a time based on which has a higher initiative and/or position of advantage.

\section{Initiative}
\label{rule:initiative}


At the start of each game-turn, each side rolls the die to establish a base Initiative number of from 1 to 10. Each aircraft takes this base number and modifies it for any of the applicable reasons given below. The modified number becomes the individual aircraft's initiative and is noted on the initiative line of the aircraft log.

Within a given category of advantage (as explained below) the aircraft with the lowest initiative number will move first followed by the next lowest numbered aircraft and so on. Whenever multiple aircraft have the same initiative number after modification, each again rolls a die. No modifiers apply and the lower roll moves first.

\paragraph{Modifiers To Initiative.} Initiative die rolls are modified by the factors listed below\addedin{1D}{1D-table}{\ and in Table~\ref{table:initiative-modifiers}}. All modifiers are cumulative.

\addedin{1C}{1C-tables}{
    \begin{onecolumntable}

\tablecaption{table:initiative-modifiers}{Initiative Modifiers}

\begin{tabular}{ll}
\hline
\multicolumn{2}{c}{Training Standard}\\
\hline
Excellent               &$+2$\\
Good                    &$+1$\\
Average                 &$+0$\\
Limited                 &$-1$\\
Poor                    &$-2$\\
\hline
\multicolumn{2}{c}{Pilot}\\
\hline
Veteran                 &$+1$\\
Regular                 &$+0$\\
Novice                  &$-1$\\
Green                   &$-2$\\
Sierra Hotel            &$+1$\\
Tactics master          &$+1$\\
Combat hero             &$+1$\\
Excellent confidence    &$+1$\\
Poor confidence         &$-1$\\
\hline
\multicolumn{2}{c}{Kills}\\
\hline
Side with first kill    &$+1$\\
Side with most kills    &$+1$\\
\hline
\end{tabular}

\end{onecolumntable}

}

\begin{itemize}

    \item\itemparagraph{National Training Standard.} The level of training for the pilot provides a modifier to the initiative roll. The scenario will indicate the national training standard.

    \item\itemparagraph{First Kill.} The side which achieves the first aircraft kill in the scenario receives a modifier of $+1$ beginning on the next game turn.

    \item\itemparagraph{Most Kills.} The side having the most kills at any point in the scenario receives a modifier of $+1$.

    \item\itemparagraph{Crew Quality.} See Chapter 18 for the crew quality effects on Initiative.
\end{itemize}




\section{Positions of Advantage}
\label{rule:positions-of-advantage}

An aircraft with an “advantage” over an enemy aircraft is better positioned to maneuver against, react to, and/or attack that enemy. Being advantage or not depends primarily on the relative positions of opposing aircraft. This is reflected in the game by allowing aircraft which are positioned to the rear of others to move after them, thus allowing them the advantage of seeing their opponent's move first.

\paragraph{Positions Of Advantage Categories.} Each turn, aircraft will fall into one of the following categories:

\begin{itemize}

    \item\itemparagraph{Departed.} An aircraft in departed flight.

    \item\itemparagraph{Stalled.} An aircraft in stalled flight.

    \item\itemparagraph{Engaged.} An aircraft (not installed or departed flight) which is actively defending itself against missiles.

    \item\itemparagraph{Disadvantaged.} A spotted aircraft in the 150{\deg} or 180{\deg} angle-off arc of an enemy that is advantaged over it.

    \item\itemparagraph{Nonadvantaged.} A spotted aircraft that is neither advantaged nor disadvantaged. This category includes an aircraft which has an advantage over another aircraft but is also disadvantaged by the same or a different aircraft.

    \item\itemparagraph{Advantaged.} An aircraft which has a spotted enemy aircraft in its 150{\deg} or 180{\deg} angle off arc within 9 hexes and not more than 6 altitude levels above or 9 altitude levels below it. \addedin{1B}{1B-apj-36-errata}{An aircraft may be more than 9 hexes range away from an aircraft it is advantaged over, the 9 hex limit applies only to horizontal range.}

    \item\itemparagraph{Unspotted.} An aircraft not visually spotted by any enemy aircraft.

    \item\itemparagraph{Undetected.} An aircraft not detected by radar or visually spotted.

\end{itemize}

This list is read in order, and an aircraft is categorized by the first situation in which it fits. Any aircraft not departed, stalled, or engaged is termed a “free” aircraft. Only a free aircraft can be advantaged over non-free aircraft. Non-free aircraft cannot be advantaged over any aircraft.

\changedin{2A}{2A-advantage}{\addedin{1B}{1B-apj-23-errata}{Aircraft in vertical climbs or vertical dives may not disadvantage aircraft that are lower or higher than them, respectively.}}{An aircraft in a vertical dive may not disadvantage higher aircraft, but may disadvantage aircraft at the same or lower altitude. An aircraft in a vertical climb may disadvantage aircraft at lower, the same, or higher altitude.
}

\addedin{1B}{1B-apj-23-errata}{Aircraft in the same \changedin{2B}{2B-same-hex-advantage}{hex}{hex or hex-size}, regardless of relative altitudes, have no effect on each other advantage-wise unless one is tailing another.}

\paragraph{Order Of Flight.} Each turn, aircraft will move sequentially during the Flight Phase by category. Categories are executed in the order shown above in the categories list (for example, all departed aircraft move first, then all stalled aircraft, etc.). Initiative is used to determine the order of movement of aircraft within each category. Missiles move when their target moves.

\paragraph{Exceptions.} The following three exceptions apply to the order of movement:

\begin{enumerate}

    \item\itemparagraph{Illuminating Aircraft.} An aircraft performing radar illumination for a radar guided missile must move at the same time as the missile's target regardless of its original category. This may cause a rearrangement of the order of movement to resolve missile shoot-outs when opposing aircraft target and illuminate each other.

    \item\itemparagraph{Tailing Aircraft.} Any aircraft “Tailing” another, moves immediately after the “Tailee” does as explained in  12.3.

    \item\itemparagraph{Preempting Aircraft.} Aircraft which have not yet moved in a turn and which are threatened by an aircraft currently moving may attempt to evade the attacker by Defensively Preempting it as explained in 12.4.
    
\end{enumerate}



\section{Tailing Enemy Aircraft}
\label{rule:tailing-enemy-aircraft}
\label{rule:tailing}

A Free aircraft ending its flight stacked in the same position as an enemy aircraft which has already moved that turn may declare that it is tailing the enemy provided:

\begin{itemize}

    \item The tailing aircraft's facing is within 60{\deg} of the tailee's, and
    
    \item The tailing aircraft's start speed \addedin{1B}{1B-tailing}{for the next game turn }is not more than 1.0 greater than the tailee's.

\end{itemize}

\paragraph{Advantages of Tailing.} An aircraft electing to “tail” an enemy will not collide with it. Tailing negates collisions. An aircraft tailing another will always move after the tailee thus avoiding an overshoot, which could occur otherwise if it were not tailing and ended up with a lower initiative number on the following turn.

\paragraph{Limits on Tailing.} No more than one friendly aircraft may ever tail a given enemy, but the friendly aircraft could in turn be tailed by another enemy which moves later that turn. No more than three aircraft\addedin{2A}{2A-tailing}{ may be part of a multi-aircraft tailing}.  Multiple tailings in a hex/hexside may occur as long as each pursuer meets this criteria.

\paragraph{Effects On Positions Of Advantage.} The tailing aircraft is considered advantaged over the tailee but is not allowed to disadvantage any other enemy aircraft since it is concentrating on the pursuit. The tailing aircraft moves immediately after the pursued aircraft does regardless of normal initiative numbers. Other aircraft may consider or ignore a tailing aircraft for purposes of determining order of flight depending on what would be more advantageous to them.


\section{Defensive Preemptions}
\label{rule:defensive-preemptions}

Due to the rule that allows gunfire during movement, it often happens that aircraft with a higher Initiative, which are waiting to move, get attacked by those supposedly at a disadvantage which are moving first. This rule allows aircraft with the higher Initiative, to react to such threats by preempting the movement of those enemy aircraft.

\paragraph{When Can You Preempt?} An aircraft may pre-empt the normal order of flight once per game-turn by moving before it normally would. This is allowed only when a sighted enemy aircraft, which has a lower Initiative or which is in a lower position of advantage category, is moving or about to move, and is threatening it with gunfire.

To be considered threatening, the moving or about to move enemy aircraft must have the friendly aircraft in its 150{\deg} to 180{\deg} angle off arc and be within six hexes of range (2 altitude levels = 1 hex of range).

\paragraph{Procedure.} If you think you will be preempting an enemy, you should alert the player controlling that aircraft so that he can pause momentarily \changedin{2B}{2B-declaring-preemptions}{between FP expenditures}{before expending each FP, including the first,} to allow you time to announce a preemption. \deletedin{2B}{2B-declaring-preemptions}{The option to preempt may be taken before the enemy aircraft expends its first FP, or between the each of its FPs if the enemy is already moving.}

To avoid confusion, the threatening aircraft should first expend an FP and then the defensive player should announce simply yes or no. If yes, the preemption is executed immediately. If no, the threatening aircraft may then conduct any possible gun attacks and move an additional FP. This process is repeated until a preemption occurs or the enemy completes its move.

\paragraph{Effect On Movement.} When an aircraft elects to preempt, the enemy aircraft's movement is temporarily halted, and the preempting aircraft now expends half its FPs (rounded up) in flight. Once that is done, the enemy aircraft completes its flight making any possible attacks. When the enemy finishes, the preemptor then completes his flight and both are done moving for the game turn. The preemptor may not preempt again that turn even if attacked by another aircraft. \addedin{2B}{2B-no-double-preemptions}{A preempting aircraft may not itself be preempted.}

\paragraph{Restrictions.} An aircraft that does a defensive preemption may not:  

\begin{itemize}
    \item make any attacks or launch weapons and,
    \item may not do any radar work except to use the "Boresight" or "Auto-Track" modes.
\end{itemize}	

\trainingnote{
You are now ready to play all guns only Air Combat Scenarios. For more fun, also read the Special Maneuvers of Chapter 13 which allow you more options in how to maneuver your fighters. Use the Sequence of Play but Ignore the AAA, SAM, and Ground Unit Interaction Phases.
}

\begin{advancedrules}

\section{Formations and Order of Flight}

\paragraph{Initiative.} All aircraft in close formations use the leader's initiative in place of their own. Wingmen aircraft in tactical formations whose initiative ends up being less than their leader's may add one to their initiative (reflecting teamwork and radio calls). Regular or less quality aircrew who are not in a formation of some sort must subtract one from their Initiative.

\paragraph{Order of Flight.} All aircraft in a Close Formation move with and at the same time as their leader. Their leader's order of flight is determined normally. Aircraft in Tactical Formations move individually with their order of flight determined normally.

\end{advancedrules}

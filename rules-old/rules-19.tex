%LTeX: enabled=false

\rulechapter{Electronic Warfare}
\label{rule:electronic-warfare}

This chapter details the procedures for electronic warfare, including jamming and deception. This entire Chapter is an \emph{ADVANCED RULE.}

\begin{advancedrules}

Because modern aircraft are highly dependent on electronic equipment such as radar, they are extremely susceptible to deceptive or disruptive signals.  Electronic warfare makes use of such signals to mislead or blind enemy electronics.

\paragraph{ECM Pods.} Aircraft may mount ECM pods on any station capable of carrying EP stores. ECM pods can provide an aircraft with DDS capability and Jammer capability if it does not already have any or improve upon what it has.  

\paragraph{Internal ECM.} ECM gear may be internally installed in the aircraft and is noted in the ECM section of the ADC. ECM gear is rated in its effectiveness by a letter codes of A or higher. An A indicates an early first-generation system; other letters indicate increasing levels of sophistication and capability. The ECM section of the ADC indicates what Internal ECM is normally carried by an aircraft. The effects of each type are detailed below.

\section{Identification Friend or Foe (IFF)}
\label{rule:iff}

An aircraft's IFF, when turned on, allows other friendly radar equipped aircraft and ground units to automatically identify it as a friendly. Also, when the IFF is on, all friendly radars automatically detect the aircraft when it is in their detection range and arc since IFF acts like a transponder.

This characteristic also helps the enemy. Any enemy radar searching or attempting lock-on against an aircraft equipped with operating IFF receives a modifier of minus 2 on the detect and the lock-on die rolls.

Because of the above, IFF is normally turned off when heading into enemy territory, and turned on when returning to friendly territory or when engaged in a multiple aircraft battle. Friendly radar equipped units will not make mistaken attacks against an aircraft with an operating IFF. If IFF is off, the scenario will state what actions each side might take against unidentified friendly aircraft.

\addedin{1B}{1B-apj-35-qa}{IFF status is declared during the aircraft decisions phase.}

\section{Decoy Dispenser Systems (DDS)}
\label{rule:dds}

An aircraft equipped with a DDS may dispense expendable decoys automatically or manually. A decoy (as used in the game) actually represents a cluster of 2 to 4 actual decoys being dispensed.

\addedin{1C}{1C-tables}{
    \begin{table*}
\centering\small
\addedin{1B}{1B:tsoh-errata}{
\caption{Internal DDS}
\medskip
\begin{tabular}{lccp{32em}}
\hline
Type&Capacity&Allowed Decoys&Load Options\\
\hline
\multicolumn{4}{c}{U.S./NATO/European}\\
\hline
A&16&CH, FL&16 of any one, 8 of each, 12 of one $+$ 4 of the other.\\
B&16&CH, FL, JM&16 of any one, 8 each of any two, 12 of one $+$ 4 of another, 5 each of all three, 8 of one $+$ 4 each of the other two.\\
C&32&CH, FL, JM&10 each of all three, 32 of any one, 16 each of any two, 61 or one $+$ 8 each of the other two, 24 of one $+$ 8 of another, 24 of one $+$ 4 each of the other two.\\
D&32&CH, FL, JM&32 decoys total, any desired mix allowed.\\
\hline
\multicolumn{4}{c}{Soviet/Warsaw Pact}\\
\hline
A&10&FL&10 Flares only\\
B&20&CH, FL&10 of each, 20 of any one.\\
C&40&CH, FL&20 or each, 40 of any one, 30 of one $+$ 10 of the other.\\
D&NA&NA&NA\\
\hline
\end{tabular}

}
\end{table*}
}

\paragraph{Internal DDS.} Internal DDS installations are identified on the ADC. Four types of internal DDS are available: A, B, C, D. \changedin{1C}{1C-tables}{The internal DDS Table}{Table~\ref{table:ew-internal-dds}} shows the standard loads of decoys available for an internal DDS but typically they are capable of holding between sixteen and twenty clusters of decoys.

\paragraph{External DDS.} External DDS pods are identified \changedin{1C}{1C-tables}{on the External Pod Table}{in Table~\ref{table:stores-EP}}, which shows the possible loads of decoys available.

\paragraph{Types of Decoys.} Three types of disposable decoys for DDS are available: flares (FL), chaff (CH), and jammers (MJ).

\paragraph{DDS Programs.} An aircraft may use automatic programs which provide continuous protection by dispensing decoys throughout a game turn. A program is declared as ON or OFF during the Aircraft Decisions Phase. When on, the program will provide a LEVEL of protection of from 1 to 6 depending on its design. This is the aircraft's PPL (Program Protection Level). A PPL is in effect from the time the program is turned on until it is changed, turned off, or available decoys are exhausted. A PPL number represents both the level of protection and the number of decoys dispensed in a game-turn when the program is on.

\paragraph{DDS Program Design.} A DDS program design is noted on paper in the following format (in PPL numbers): 
\begin{center}
    Chaff/Flare/Mini-Jammer. 
\end{center}
The decoy program need not be symmetric; meaning the PPL for each type of decoy carried need not be the same. For example, 4/2/3 is a DDS program calling for 4 Chaff, 2 Flares, and 3 Mini-Jammers to be dispensed each turn providing a 4/2/3 level of protection.

When the decoys in the DDS run low and the remaining quantity of decoys is less than the PPL number called for, the PPL is reduced to equal the quantity of decoys remaining.

\addedin{1C}{1C-tables}{
    \begin{table*}
\centering
\caption{Decoy PPL Effectiveness}
\medskip
\begin{tabular}{*{13}{c}}
\hline
\multicolumn{2}{c}{\minitable{c}{DDS\\Program}}&
\multicolumn{5}{c}{\minitable{c}{EWR Passdown Modifier and\\TTR Lock-on Modifier}}&
\multicolumn{2}{c}{\minitable{c}{TTR\\Break-Lock No.}}&
\multicolumn{2}{c}{\minitable{c}{Air Radar Search\\and Lock-on Modifier}}&
\multicolumn{2}{c}{\minitable{c}{Air Radar\\Break-lock No.}}\\
Chaff&Mini-jam&
\multicolumn{5}{c}{Radar Frequency}&
\multicolumn{2}{c}{SAM Type}&
\multicolumn{2}{c}{Type}&
\multicolumn{2}{c}{Type}\\
PPL \#&PPL \#&
LF&MF&HF&VF&MW&
BR/CG&CW/TWM&
Lim./180&150/120&
Lim./180&150/120\\
\hline
1&---&1&1&---&---&---&---&---&1&---&1&---\\
2&1&2&1&1&---&---&1&---&1&1&1&1\\
3&---&2&2&1&1&1&2&1&2&1&2&1\\
4&2&3&2&1&1&1&3&1&2&2&2&2\\
5&3&2&2&2&2&1&3&2&3&2&3&2\\
6&4&3&3&2&2&2&4&2&4&3&4&3\\
---&5&4&3&3&2&2&5&3&4&3&5&3\\
---&6&4&4&3&3&2&5&4&4&4&6&4\\
\hline
\end{tabular}
\end{table*}
}

\paragraph{Decoy Effects.} \changedin{1C}{1C-tables}{The various EW tables detail}{Tables~\ref{table:ew-decoy-ppl-effectiveness} gives} the specific effects or die roll modifiers PPLs produce. The following is a summary:

\begin{itemize}

    \item Chaff PPL may break BR and CG TTR lock-ons, may cause modifiers to RHM, AHM, and CG/CW SAM attacks, may break air to air radar lock-ons, or may spoof AAA radars.

    \item Flare PPL may cause modifiers to IRM and IR SAM launch attempts, may cause modifiers to IRM and IR SAM attacks, may break OG/LG SAM lock-ons.

    \item Mini-Jammer PPL may break CG, CW and TVM type TTR lock-ons, may cause modifiers to RHM, AHM, CG, CW and TVM SAM attacks, may break air to air radar lock-ons.

\end{itemize}

\paragraph{Sighting Effects.} If the target used a DDS program with flares in the previous game turn, apply the PPL as a modifier to the sighting roll (see rule \ref{rule:sighting-aircraft-and-missiles}).


\section{Radar Warning Receivers (RWR)}
\label{rule:rwr}

\addedin{1C}{1C-tables}{
    \begin{table*}
\centering

\caption{RWR and Internal DJM/AJM Detectable and Jammable Frequencies}
\medskip
\begin{tabular}{*{15}{c}}
\hline
\multirow{3}{*}{\minitable{c}{RWR or\\Jammer\\Type}}\\
&
\multicolumn{3}{c}{EWR}&&
\multicolumn{4}{c}{FCR}&&
\multicolumn{5}{c}{TTR}\\
\cline{2-4}
\cline{6-9}
\cline{11-15}
&
LF&MF&HF&&
LF&HF&VF&MW&&
LF&MF&HF&VF&MW\\
\hline
A&X&---&---&&X&X&---&---&&X&X&X&---&---\\
B&X&X&X&&X&X&X&---&&X&X&X&---&---\\
C&X&X&X&&X&X&X&X&&X&X&X&X&---\\
D&X&X&X&&X&X&X&X&&X&X&X&X&X\\
\hline
\tablemedskip
\tablenotes{15}{0.7\linewidth}{
Notes
\begin{enumerate}
    \item “X” indicates that radar operating in that frequency is detectable to RWR and vulnerable to DJMs and AJMs.
    \item DJM and AJM pods have their frequency capabilities listed in the external stores tables under EPs.
    \item Internal DJMs A and B cannot break CW or TVM lock-ons. Internal DJM C cannot break TVM lock-ons.
\end{enumerate}
}
\end{tabular}

\medskip

\caption{RWR Also Detects}
\medskip
\begin{tabular}{c*{8}{c}}
\hline
\multirow{3}{*}{\minitable{c}{RWR or\\Jammer\\Type}}\\
&
\multicolumn{4}{c}{SAM Launches}&
\multirow{2}{*}{\minitable{c}{Air\\Search}}&
\multirow{2}{*}{\minitable{c}{Air\\Lock}}&
\multirow{2}{*}{\minitable{c}{Tgt.\\Ilum.}}&
\multirow{2}{*}{\minitable{c}{AHM\\Active}}\\
\cline{2-5}
&
BR&CG&CW&TVM\\
\hline
A&X&X&---&---&---&X&---&---\\
B&X&X&---&---&---&X&X&---\\
C&X&X&X&---&---&X&X&X\\
D&X&X&X&X&X&X&X&X\\
\hline
\end{tabular}



\end{table*}
}
Radar Warning Receivers are designed to alert aircraft to hostile radar emissions and missile guidance signals. An aircraft's RWR, if any, is listed in the ECM section of the ADC. As for DDS, RWRs are classified as A, B, C, and D systems.

\paragraph{RWR Capabilities.} \changedin{1D}{1D-table}{The RWR Table}{Table~\ref{table:ew-coverage}} lists ground unit radars (according to the frequency they use) and air radars (by modes) and indicates if that frequency or mode is detectable to each type of RWR.

\paragraph{RWR Benefits Against Radar Equipped Units.} All RWRs have the ability to indicate the relative direction of incoming radar strobes. Thus, if an aircraft with an RWR is searched for and/or locked onto by a detectable radar equipped enemy aircraft or ground unit, that aircraft receives a $-1$ to any visual sighting rolls made against those radar equipped units.

\paragraph{RWR Benefits Against Missiles.} If a SAM TTR, air radar illumination, or AHM active radar is detectable by the RWR, any missiles directed by those radars may be defensively engaged without being visually spotted.

\paragraph{RWR Benefits Against AAA Guns.} If a AAA FCR (fire control radar) is detectable by the RWR, the aircraft may manually deploy 1 or 2 Chaff decoys (if so equipped) each time the FCR-equipped AAA unit fires at it. Roll one die, on a roll equal to or less than the number of chaff dropped, the FCR is “spoofed” and may not add in its hit modifier. Deploying chaff cancels any air-to-ground aiming the aircraft may have accomplished up to that point (the pilot was distracted).

\paragraph{Increased PPL Effectiveness.} Due to the sophisticated logic interfaces of late 1980s and 1990s EW systems, DDS-C or DDS-D, when used with a RWR-C or RWR-D have increased PPL effectiveness as follows:

\begin{itemize}
    \item When a PPL is in effect against radar systems or missiles which are detectable to RWR-C or RWR-D, the selected PPL level is increased by $+1$ unless it was 0 to being with. Decoy use is not increased, just the effective PPL number.
\end{itemize}

\paragraph{RWR Special Capabilities.} RWR-A and RWR-B only indicate the general type of radar and frequency being used against an aircraft (i.e., CG SAM TTR, low freq.). RWR-C and RWR-D tells the crew the exact threat being employed against them (e.g., SA-11 SAM missile lock-on and launch). RWR-C and RWR-D can detect launches of CG and CW SAMs even when launched under OG.

\paragraph{RWRs Versus Hidden Units.} If hidden initial ground forces are used, a player using radars must reveal to the player with the RWR-equipped aircraft the angle-off arc in which the radar energy was detected.

If equipped with RWR-D, the radar player must also reveal the megahex in which the radar is located. The aircraft may then attempt to determine the exact hex on a die roll of 3 or less. If successful, the hidden unit is revealed and placed on the game map. If not found initially, the aircraft may (if the radar continues to operate) roll again in the SAM interaction phase of subsequent game-turns to locate the radar. A cumulative modifier of $-3$ per game-turn after the turn of initial detection applies.

\addedin{2B}{2B-sighting-detections}{
\paragraph{Sighting Effects.} A RWR detection counts for the purposes of the $-1$ modifier to the sighting roll for having a detection of the target. This modifier is applied only once regardless of how many different types of detections or lock-ons the searcher has for the target.

\paragraph{Advantage Category Effects.} A RWR detection \emph{does not} count as a detection for the purposes of determining advantage category.
}

\section{Radar Jamming}
\label{rule:radar-jamming}

Aircraft may carry internal jammers or carry jamming pods to degrade the capabilities of enemy radars. There are three types of jammers: Barrage Jammers, Active Jammers, and Deceptive Jammers. Some ECM pods have multiple jammer capabilities and if so, all characteristics may be used simultaneously. The exact mode and frequency jamming capabilities of ECM pods is listed in the external stores tables. 

\addedin{1B}{1B-apj-27-qa}{Unless otherwise stated in the ADC, aircraft with internal BJMs and dedicated crew to run them (electronic warfare officers) have no frequency limits.}

\subsection{Barrage Jammers (BJMs)}
\label{rule:home-on-jam-seeker}

\addedin{1C}{1C-tables}{
    \begin{table}
\centering
\caption{BJM Stand-Off Jamming}
\medskip
\begin{tabular}{cccp{8em}}
\hline
\multirow{2}{*}{\minitable{c}{BJM\\Type}}&
\multicolumn{2}{c}{Allowed Stand-Off Attacks}&
\multirow{2}{*}{\minitable{c}{Angle-Off\\Coverage}}\\
&Pilot Only&Multi-Crew&\\
\hline
A&1&1&$180+$ and $30-$\\
B&2&2&$150+$ and $60-$\\
C&2&3&As B, or into any 3 adjacent arcs.\\
D&2&4&As B, or into any angle-off arcs.\\
\hline
\tablemedskip
\tablenotes{4}{0.9\linewidth}{
Jamming Success Die Rol = BJM No.\ $-$ Radar ECCM.

Note: Noise Jamming Arcs = as for A, B, C above. Treat a BJM D as a C when noise jamming.
}
\end{tabular}
\end{table}

\begin{table}
\centering
\caption{BJM Programming Flexibility}
\medskip
\begin{tabular}{cp{20em}}
\hline
Type&Programming Options\\
\hline
A&Pick Frequencies and Mode before play.\\
B&Pilot only: as “A”. Multi-crew: may pick Frequencies and Mode during Aircraft Decisions Phase of game-turn.\\
C&Pilot Only: as for Multi-crew above. Multi-crew: Same as above.\\
D&Pilot Only and Multi-crew: may change Frequencies and Mode at start of SAM Interaction Phase.\\
\hline
\end{tabular}

\end{table}

}

A Barrage Jammer floods enemy radar scopes with continuous noise to render them useless. This is the earliest type of jamming; it is countered by the use of home-on-jam missiles.

A barrage jammer has two modes: Noise and Stand-Off. The ability of an aircraft to switch between selected modes and jamming frequencies varies with the type of jammer and crew (pilot only or multi-crew) \changedin{1C}{1C-tables}{as detailed in the EW charts and EP tables}{is shown in Table~\ref{table:ew-bjm-flexibility}}. 

\begin{itemize}

    \item \itemparagraph{Noise Mode.} The aircraft sends jamming signals into its allowed angle-off arcs. Any air or ground radar operating in the jammed frequency, located in the jammed arcs, and searching or attempting lock-ons against friendly aircraft also in those arcs (including the jammer) will be degraded.

    Subtract the ECCM rating of the jammed radar from the barrage jammer rating and apply any positive results as a die roll modifier for any search, passdowns, and/or lock-on attempts by the radar.

    \item \itemparagraph{Stand-Off Mode.} Instead of flooding an arc with noise, the BJM can be focused on particular enemy radars in a concentrated jamming attack. A successfully jammed radar is blind and may not search, lock-on, or guide radar missiles. Stand-off jamming does not otherwise protect other aircraft. \changedin{1C}{1C-tables}{The EW tables}{Table~\ref{table:ew-bjm-capabilities}} indicate how many stand-off attacks can be made by a single jammer and the die roll required for success. A radar's ECCM acts as a modifier against the jamming roll.
    
\end{itemize}

\addedin{1B}{1B-apj-39-qa}{BJMs do not break locks once achieved.}

\addedin{1B}{1B-apj-39-qa}{All internal BJMs may affect both air search and lock-on die roll attempts.}

Barrage jammers making noise or doing stand-off attacks are vulnerable to HOJ (home on jam) missiles which may be fired at them without the usual necessary lock-on and guidance signals since the missile flies up the jamming beam to the aircraft.

\addedin{1B}{1B-apj-37-qa}{The frequency limits of internal BJMs are the same as those of external pods.}

\subsection{Active Jammers (AJMs)}

These self-protection jammers confuse enemy radars by copying their signals and sending back false and/or additional misplaced radar echoes. This makes it harder for the enemy radar to find and lock-on to the real aircraft among false blips. Most active jammers only function in response to a radar pulse and do not provide a continuous beam for home on jam missiles to guide on. An AJM protects only the aircraft equipped with it, only from radars operating in the aircraft's protected arcs, and only against radars operating in a jammable frequency. If an aircraft has both internal and podded AJMs, only the most effective AJM in the given frequency is used.

Subtract the ECCM rating of the jammed radar from the \changedin{2B}{2B-ajms}{barrage}{active} jammer rating and apply is as a die roll modifier for any search, passdowns, and/or lock-on attempts by the radar.

\addedin{1B}{1B-apj-39-qa}{All type A and B internal AJMs affect only air radar lock on attempt rolls. All type C and D internal AJMs affect both search and lock-on attempt rolls.}

\subsection{Deceptive Jammers (DJMs)}
\label{rule:deceptive-jammers}

A Deceptive Jammer breaks radar lock-ons by shifting the radar beam off the intended target through sophisticated manipulation of the radar's signals. False timing of the radar returns gives the lock-on beam a perceived angular error and when it shifts to recenter the target, it actually shifts off the target, losing its lock-on.

When an air radar or TTR lock-on is achieved against an aircraft, Deceptive Jammers come into play to break the lock. When a radar controlled AAA gun fires at an aircraft, a DJM may act to spoof the FCR adding adverse to hit modifiers.

Refer to \changedin{1C}{1C-tables}{the RWR/DJM coverage table}{Table~\ref{table:ew-coverage}} to see if a DJM is effective against particular radars. If so, take the DJM's numerical self-protection rating and subtract the radar's ECCM rating from it.  If the result is positive, this is the number or less that must be rolled on one die to break the lock-on. DJMs also provide modifiers to radar guided missile attacks as detailed in \changedin{1C}{1C-tables}{the EW tables}{Table~\ref{table:missile-attack-modifiers}}. If an aircraft has both podded and internal DJMs, only the most effective one in the given frequency is used.

\addedin{1B}{1B-apj-39-qa}{All internal DJM types may attempt to break any air radar lock-ons.}

\section{Jamming Cell Formations}

\paragraph{Jamming Cells.} Two, three or four aircraft with identical BJMs, AJMs, or DJMs may fly in a “Jamming Cell” formation. \changedin{1B}{1B-apj-23-errata}{Any time aircraft with identical jammers are positioned as illustrated in the jamming cell diagram, the effective rating of their jammers is increased as indicated in the diagrams}{To be in a valid Jamming Cell formation, aircraft must be in adjacent hexes (as depicted \changedin{1C}{1C-figures}{below}{in Figure~\ref{figure:jamming-cell-formations}}), with no more than 30 degrees difference in facing and within 1 altitude level of an adjacent aircraft. The maximum size of a jamming cell is 4 aircraft. Each aircraft within those parameters has its ECM jammer rating increased by 1.}

\notein{1B}{AWF: the TSOH errata states, “All aircraft in a jamming cell must be $+$ or $-$ one altitude level for increased jammer effectiveness.” This is included in the errata above.}    

\newcommand{\drawjammingcellhex}[2]{
    \drawhex[black!10]{#1}{#2}
}

\newcommand{\drawjammingcellaircraft}[3]{
    \drawaircraftcounter{#1}{#2}{#3}{F-105}{}
}

\newcommand{\drawjammingcellhexandaircraft}[3]{
    \drawjammingcellhex{#1}{#2}
    \drawjammingcellaircraft{#1}{#2}{#3}
}

\begin{tikzfigure}{\linewidth}
    \drawhexgrid{12}{10}

    \drawjammingcellhexandaircraft{1.0}{1.5}{90}
    \drawjammingcellhexandaircraft{2.0}{1.0}{90}
    \drawjammingcellhexandaircraft{2.0}{2.0}{90}
    \drawjammingcellhexandaircraft{3.0}{1.5}{90}

    \drawjammingcellhexandaircraft{5.0}{1.5}{120}
    \drawjammingcellhexandaircraft{6.0}{2.0}{120}
    \drawjammingcellhexandaircraft{7.0}{2.5}{120}
    \drawjammingcellhexandaircraft{8.0}{3.0}{120}
    
    \drawjammingcellhexandaircraft{8.0}{1.0}{90}
    \drawjammingcellhexandaircraft{9.0}{1.5}{90}
    \drawjammingcellhexandaircraft{10.0}{1.0}{90}
    \drawjammingcellhexandaircraft{11.0}{1.5}{90}
    
    \drawjammingcellhexandaircraft{1.0}{4.5}{120}
    \drawjammingcellhexandaircraft{1.0}{5.5}{120}
    \drawjammingcellhexandaircraft{2.0}{5.0}{120}
    
    \drawjammingcellhexandaircraft{4.0}{5.0}{90}
    \drawjammingcellhexandaircraft{5.0}{5.5}{90}
    \drawjammingcellhexandaircraft{6.0}{5.0}{90}
    
    \drawjammingcellhexandaircraft{8.0}{5.0}{90}
    \drawjammingcellhexandaircraft{9.0}{5.5}{90}
    \drawjammingcellhexandaircraft{10.0}{6.0}{90}
    
    \drawjammingcellhexandaircraft{1.0}{7.5}{120}
    \drawjammingcellhexandaircraft{2.0}{8.0}{120}
    
    \drawjammingcellhexandaircraft{4.0}{8.0}{90}
    \drawjammingcellhexandaircraft{5.0}{8.5}{90}
    
    \drawjammingcellhex{7.0}{8.5}
    \drawjammingcellhex{8.0}{9.0}
    \drawjammingcellaircraft{6.5}{8.25}{120}
    \drawjammingcellaircraft{8.5}{9.25}{120}
    
    \drawjammingcellhexandaircraft{10.0}{8.0}{120}
    \drawjammingcellhexandaircraft{11.0}{7.5}{120}

    \draw (0.75,3.00) node [anchor=west, fill=white, draw=black] {Four-Aircraft Cells};
    \draw (0.75,6.50) node [anchor=west, fill=white, draw=black] {Three-Aircraft Cells};
    \draw (0.75,9.00) node [anchor=west, fill=white, draw=black] {Two-Aircraft Cells};

\end{tikzfigure}


\addedin{2B}{2B-infrared-jammers}{
\section{Infrared Jammers}
\label{rule:infrared-jammers}

Infrared jammers (IRJM) emit pulsing infrared radiation in an attempt to confuse infrared-homing seekers on IRMs and IR SAMs.

In the game, infrared jammers give the carrying aircraft a modifier to the attack roll of IRMs and IR SAMs. The modifier depends on the arc of the attack.

Integrated infrared jammers are noted on the ADC. Infrared jammer pods are listed in Table~\ref{table:stores-EP-IRJM}, and may be mounted on any weapon station able to carry an ECM pod (EP).
}



\end{advancedrules}

\section{Crew Quality}

This chapter details the effects of crew quality on aircraft operations. This entire chapter is an \emph{ADVANCED RULE}.

\subsection{Quality Levels}

\paragraph{Crew Quality.} Pilots and other crewmembers are classified into one of four quality levels based on experience and training. These are:

\begin{itemize}
    \item Green: A poor pilot/crewman due to lack of ability and/or incomplete training.
    \item Novice: A new pilot/crewman fresh out of normal training, a poor caliber regular, or a green starting to improve.
    \item Regular: An experienced pilot/crewman with good training who may or may not have seen combat, or an above average novice.
    \item Veteran: A well trained professional, possibly combat experienced pilot/crewman with superior skills (an older and wiser regular).
\end{itemize}

Crew quality, and any applicable attributes or characteristics are given in most scenarios for general scenarios, or historical scenarios not providing aircrew information, use the Crew Generation Tables given in the play aids charts.

\paragraph{Aircrew Modifiers.} The quality of a pilot/crewman in an aircraft may affect the die rolls for initiative, sighting, radar use, weapon launches, attacks, departed flight and recovery from departed flight. These actions and their associated die roll modifiers are summarized in the Aircrew Modifiers Table and other tables. In multi-crew aircraft, only the modifiers for the crewman that would logically be affecting or performing an action are used. The following are some guidelines:

\begin{itemize}
\item Pilots affect:
\begin{itemize}
    \item In flight: Initiative, departures, and recoveries.
    \item In combat: Gun attacks, visual bombing or rocket attacks, \changedin{1C}{JDW in APJ 22}{and IRM launches}{IRM launches, and the use of auto-track and boresight radar modes}.
\end{itemize}

\item Crewmen affect:
\begin{itemize}
    \item In combat: Guided weapon launches and attacks, radar bombing, and radar guided missile launches.
    \item For radar work:  Radar searches and lock-ons.
\end{itemize}

\item Both pilots and crewmen affect:
\begin{itemize}
    \item Sighting attempts as described in Chapter 11.
\end{itemize}
\end{itemize}

\subsection{Aircrew Flight Restriction}

Green and novice pilots are restricted in performing certain flight actions as follows:

\paragraph{Green Pilots.} A Green pilot is extremely inexperienced and may not:
\begin{itemize}
    \item perform ET (Emergency Turns) or Snap Turns.
    \item fly at Terrain level.
    \item use VIFF maneuvers or use VTOL flight.
    \item engage attacking missiles.
    \item attempt Vertical Reverse maneuvers.
    \item use High Pitch Rate capabilities of an aircraft. 
\end{itemize}

A Green pilot risks disorientation if he:
\begin{itemize}
    \item performs a rolling maneuver.
    \item performs Vertical Climbs or Vertical Dives.
\end{itemize}

A Green \changedin{1C}{JDW in APJ \#34}{pilot}{crewmember} receives a minus 2 die roll modifier when checking for GLOC.

\paragraph{Novice Pilots.} A Novice pilot may not:

\begin{itemize}
    \item attempt Vertical Reverse maneuvers.
    \item use High Pitch Rate capabilities of an aircraft.
\end{itemize}

A Novice pilot risks disorientation if he performs a Vertical rolling maneuver.

A Novice pilot receives minus 1 die roll modifier when checking for GLOC.

Regular and veteran aircrew are not restricted.

Note: Always check for disorientation immediately after a risky maneuver is performed, and/or at the end of the game turn if a risky flight type was attempted. Disorientation and its affects are described in chapter 30.

\paragraph{Crew Quality and Damage Control.} Green pilots may not normally do damage control. Novice pilots must spend two consecutive game turns applying damage control to stop progressive damage\addedin{1C}{JDW in APJ 39}{, and so there is a risk of progressive damage on the first of these two turns}.

In multi-crew aircraft, a regular or veteran crewman can compensate for a green pilot, allowing damage control to be done as if by novices. Likewise, they can compensate for a novice pilot allowing damage control to be done normally.

\subsection{Aircrew Attributes and Special Charateristics}

Attributes of aircrew that can affect play are eyesight, fitness, and confidence. If not given in the scenarios, attributes are rolled for on the Crew Attributes Table.

\begin{itemize}
    \item Eyesight affects visual sighting die rolls.
    \item Fitness affects GLOC and Post-Egress Fate rolls.
    \item Confidence affects initiative, departure \changedin{1C}{JDW in APJ 39}{recovery}{entry}, and disorientation die rolls.
\end{itemize}

\paragraph{Special Pilot/Crew Characteristics.} Pilots and crew may have some of the following characteristics which benefit them in play:

\begin{itemize}

    \item COMBAT HERO. This represents an ace or a highly decorated pilot/crewman who has been distinguished in combat. Combat Heroes get beneficial modifiers to the die rolls for combat and Initiative due to their proven skills.

    If a combat hero is leading a formation, all other non-hero crews in his formation have their initiative die roll increased by one.  If shot down, a combat hero is worth more points to the other side. Also, anytime a combat hero is shot down, all non-hero crews in his formation (whether he was leading or not) immediately have their initiative rolls reduced by one.

    \item TACTICS MASTERS. This is indicative of aircrew that have attended special schools such as the USAF and USN fighter weapons schools (Top Gun for example) or who have been members of the highly trained adversary squadrons. It also represents those rare gifted aircrew from any country that successfully grasp all the essentials of air combat. In the Warsaw Pact air forces, veterans who have achieved the rating of “Sniper Pilots” would be similarly skilled.
    
    \item SIERRA HOTEL (Shit-Hot) Pilots. These Individuals have the highest levels of confidence and skills in flying due to pure natural ability and/or relentless determined practice. Alternately called “Top-Guns”, “Super-sticks”, “Honchos”, etc. these pilots get a special benefit of having their position of advantage raised one level for purposes of determining order of movement. That is, if they were disadvantaged, they would be considered non-advantaged and so on. An advantaged S.H. pilot is not increased to an unspotted one but would move after all other advantaged aircraft. \addedin{1B}{JDW in TSOH errata}{The benefit of having their position of advantage level increased is optional and should be declared in the aircraft decisions phase of the turn.}

\end{itemize}

\subsection{Formation Leader Considerations}

\paragraph{Pilot Quality.} A section or division leader must be of Regular or better quality. Aircraft with at least one veteran in the crew do not suffer the initiative penalty for not being in formation. Green pilots must always begin a general scenario game as a member of a close formation.

\subsection{Campaigns and Crew Experience}

Players may wish to create a campaign wherein a group of pilots and crew are created using the Generation Tables. Their combat careers are then tracked game to game. These aircrew would have the opportunity to increase in quality based on accumulated experience.

\paragraph{Aircrew Quality Improvement.} Pilots and crewmen may improve in quality or gain special characteristics after participating in a number of combat missions and/or gaining air to air kills and then successfully rolling the die for improvement. The improvement is rolled for at the end of each game after the minimum required amount of experience is garnered. If an aircrew does not improve after one game, he may roll again after the next and so on (some people take longer to absorb the lessons of combat).

To be considered a “combat” mission, the aircrew in question must have been engaged in offensive and/or defensive actions against opposing forces. “Milk-runs”, or attacks against undefended targets do not count as a combat mission.

\paragraph{Minimum Requirements For Improvement:}

\begin{itemize}

    \item \itemparagraph{Green to Novice:} After three combat missions or the gaining of one or more air to air kills: on a roll of 8 or less.

    \item \itemparagraph{Novice to Regular:} After five combat missions as a Novice or the gaining of one or more air to air kills as a Novice: on a roll of 6 or less.

    \item \itemparagraph{Regular to Veteran:} After five combat missions as a Regular or the gaining of one or more air to air kills as a Regular: on a roll of 4 or less.

    \item \itemparagraph{Combat Hero (ACE):} After gaining five or more air to air kills: on a roll of 9 or less.

    \item \itemparagraph{Combat Hero (Decorated):} Upon rolling a 2 or less after any single game in which the players collectively feel the crew in question performed in such an extraordinary manner as to be deserving of medals. This is vague I know, but it usually is in real life too.

    \item \itemparagraph{Tactics Master:} Upon improving to Regular or Veteran quality and rolling a two or less. This is a one-time roll at each stage and if missed after veteran status, it is never achieved.

    \item \itemparagraph{Sierra Hotel:} Upon improving in quality to any level and rolling a one on the die. As above, this is a one-time roll at each level.
    
\end{itemize}

\paragraph{Attributes.} Eyesight and Fitness never change during the course of a campaign, however, confidence can go up or down as follows:
\begin{itemize}

    \item Confidence increases one level each time the aircrew improves in quality or gains an air to air kill.

    \item Confidence decreases one level each time the aircrew is shot down or their aircraft is damaged to the crippled state. 

\end{itemize}

\changedin{1C}{FH in APJ 36}{The maximum is excellent confidence and the minimum is poor confidence.}{Confidence cannot be improved above excellent or below poor. “Excess” changes after each mission are lost.}

\paragraph{Victory Points For Aircrew Losses.} In campaign games V.P.s are awarded to the opposing side for capturing or killing aircrew. An aircrew loss occurs when an aircraft is shot down or destroyed and the crew does not successfully bail out or eject, or if the Post-Egress Fate is to be captured. The Aircrew V.P.s Table indicates values for lost aircrew.

\subsection{Ejections and Bailouts}

In a campaign it is important to know if aircrew survive unfortunate incidents like being shot down and what happens to them after the shootdown.

\paragraph{Egressing Doomed Aircraft.} Pilots and crew will automatically attempt to eject or, if not ejection seat equipped, bail out from destroyed aircraft the instant the kill occurs. They may also elect to abandon undamaged or damaged aircraft at any point in the game-turn (unless GLOC'd) during the aircraft's movement by simply declaring it. Once declared and after any proportional moves and/or attacks by pursuing missiles are resolved, the egress attempt is rolled for. Only one attempt per game-turn is allowed.

\paragraph{Egress Procedure.} Roll one die for each pilot or crewman ejecting/bailing out and consult the Egress Success Table. If the result, after applying any required modifiers is less than or equal to the number given, the aircrew successfully eject/bailout.  If the attempt fails in an undamaged or damaged aircraft, the aircrew may, if possible, still try to fly the aircraft home.

If an egress attempt falls in a destroyed aircraft, the aircrew is killed.

\paragraph{Ball Out Restrictions.} Bailing out of an aircraft is only allowed if the aircraft is or was at a speed of four or less, and if bailing out of a destroyed aircraft, only if it was four or more levels above the ground.

\paragraph{Post-Egress Fate.} Due to the short time frame of most campaign scenarios or games, the fate of ejected or bailed out crew must be determined in order to see if they can be returned to combat. Once aircrew successfully egress, roll the die at the end of the game to see what their fate is on the Post-Egress Fate Table. Apply any required modifiers and read the result.

An aircrew will end up either MIA (missing in action) or as a POW (prisoner of war) or be RESCUED. An MIA aircrew is lost forever (drowned at sea or died on the ground or died in prison). A POW will be repatriated alive after the war ends but is out of the campaign game. Rescued aircrew may be able to re-enter the campaign. Roll one die, the result is the number of campaign days, that aircrew will miss due to injury or rescue delays. After missing the required number of days, the aircrew can be put back on the roster flying missions.

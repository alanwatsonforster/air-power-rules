\section{Heat Seeking Missiles}

This chapter details the operations of infra-red heat-seeking missiles (weapon code IRM). A heat-seeking missile homes on the infra-red heat emissions of a target aircraft.

\subsection{IRM Launch Prerequisites}

To launch IRMs, the firing aircraft must:

\begin{enumerate}
    \item Have a sighted target in the missile seeker's field of view\addedin{1C}{1C-tables}{\ according to Table~\ref{table:irm-seeker-field-of-view}} and be within the launch angle off limits of the missile's seeker head \changedin{1C}{1C-tables}{(see below)}{according to Table~\ref{table:irm-seeker-limits}}.
    \item Obtain a seeker lock-up while within the minimum and maximum ranges given on the MDT under the missile's launch envelope for the type of shot (front, side, or rear).
    \item Not have violated the missile launch restrictions of rule 14.2.
\end{enumerate}

\paragraph{Seeker Head Field Of View (FOV).} IRMs can only lock-up targets in their FOV\addedin{1C}{1C-tables}{\ according to Table~\ref{table:irm-seeker-field-of-view}}. IR seeker heads normally have an FOV equal to a limited radar arc (see \changedin{1C}{1C-figures}{limited radar arc diagram on play aid reference sheet}{Figure~\ref{figure:limited-arcs}}). If turning left or right or banked left or right, the FOV can optionally be considered equal to the firer’s 180° left or right arc as appropriate. \addedin{1B}{1B-apj-23-errata}{The option to treat a limited arc FOV as a left of right 180 arc for missiles mounted on aircraft in left or right banks or turns is also applicable to all other uses of limited arcs in the rules (radar, BRM, SAM, ARM or otherwise).}

\paragraph{IR Uncage Technology FOV.} Some aircraft have the ability to uncage a missile's IR seeker head allowing it to swivel freely to acquire targets over a wider area. An uncaged IR seeker always has an FOV equal to a 180° radar arc. The act of uncaging missiles is declared in the Aircraft Decisions Phase. The technology section of the ADC indicates whether an aircraft has “IR Uncage” ability or not. Only “I”, “M”, or “A” type seekers can be uncaged.

\addedin{1C}{1C-tables}{
    %!TEX root = ../rules-working.tex
%LTeX: enabled=false

\begin{onecolumntablefloat}
\begin{onecolumntable}
\tablecaption{table:irm-seeker-field-of-view}{IR Seeker Field of View Limits for Launch}
\medskip
\begin{tabularx}{\linewidth}{X}
\toprule
\begin{enumerate}
    \item Regular FOV = As Limited radar arc
    \item Uncaged FOV = 180+ angle-off arcs
    \item Uncaged FOV with Helmet sight = 150+ angle-off arcs
    \item Uncaged FOV with radar assist = lesser of 150+ or radar arc
    \item Uncaged FOV with VAS assist (M, A only) = 180+ arcs
    \item IRSTS Assisted FOV = Same as IRSTS system
    \medskip
    \item[--] If target one of several in unassisted Uncaged FOV a roll of \minusafter{8} is required for seeker lock-on; \minusafter{9} with helmet sights. Modifier of \plus{1} to roll for each aircraft the seeker must look past.
\end{enumerate}
\\
\bottomrule
\end{tabularx}
\end{onecolumntable}
\end{onecolumntablefloat}
}
\begin{onecolumntable}
\tablecaption{table:irm-seeker-limits}{IRM Seeker Limits}
\begin{tabularx}{\linewidth}{clP}
\toprule
Seeker  &Type       &Angle-Off Limits\\
\midrule
E       &Early      &Target's 60 degree arc or less if target used A/B power; target's 30 degree arc or less otherwise.\\
I       &Improved   &Target's 60 degree arc or less at any target power setting.\\
M       &Modern	    &Target's 120 degree arc or less if it used A/B power; target's 90 degree arc or less otherwise.\\
A       &Advanced   &Any of target's angle-off arcs at any target power setting.\\
\bottomrule
\end{tabularx}
\end{onecolumntable}


\paragraph{Seeker Head Launch Angle-Off Limits.} IR seeker heads cannot lock-up a target unless the firer is within the listed target angle-off arcs as given \changedin{1C}{1C-tables}{below by seeker type}{by Table~\ref{table:irm-seeker-limits}}.

\addedin{1B}{1B-apj-23-errata}{IRMs of any type may be launched from any angle-off arc about propellor driven and helicopter targets. If launch or attack modifiers apply to these targets for low heat signature, it will be stated in the scenarios.}

\paragraph{Seeker Head Lock-Ups.} Anytime there is only one target in a seeker's FOV, it is automatically locked-up.

If more than one aircraft, including friendlies, are in a LIMITED FOV, the closest aircraft is automatically locked-up. If several aircraft are equally close, randomly determine by die roll which was actually locked-up AFTER the missile is launched.

If more than one aircraft, including friendlies, are in an UNCAGED FOV, the firing player chooses the intended target and rolls a die. If the target is the nearest aircraft, the lock-up succeeds on a roll of 8 or less. If other aircraft in the FOV are equally near, or if the target is not the closest, then the die roll is modified by a cumulative $+1$ for every closer or equally near aircraft. If successful, missiles may be fired normally. If the lock-up attempt fails, missiles may not be launched.

\addedin{1b}{1B-apj-27-qa and APJ 37 QA}{A separate roll is made for each missile fired after it is successfully launched.}

\paragraph{IRM Tracking Requirements.} Once launched, all IRMs have an uncaged FOV. At the end of every game turn, and at the end of each proportional move for the missile, the target must still be in the missile's FOV otherwise the missile loses its lock-up and becomes unguided. At the instant a missile becomes unguided it is removed from play.

\subsection{IRM Countermeasures}

\paragraph{Flare Decoys.} If an aircraft is equipped with an internal DDS, or is carrying a DDS pod, then it can be equipped with flare decoys to be used against attacking missiles. Flares may be dispensed either through automatic programs as described in rule 19, or manually if the aircraft engages an attacking missile (rule 14.6).

\paragraph{Manual Flare Procedure.} When a missile declares an attack, but before it rolls to hit, an aircraft with a DDS system may declare using flares and “pop” one or two flare clusters. For each flare popped, roll one die; if the result of either roll is equal to or less than the missile's flare vulnerability rating, the missile is decoyed and removed from play. When a flare is popped, it is used up even if it is the second of two popped and the first decoyed the missile. When the aircraft has expended all of its flares, it can no longer decoy IRMs.

\paragraph{Flare Program Procedure.} If a DDS program is in use and includes flares, then a die roll modifier equal to the lesser of the Program Protection Level (PPL) of the flares or the flare vulnerability rating of the missile is used as a positive modifier to the attack die roll. Note: A flare PPL also provides a modifier to IRM launch rolls equal to the program's PPL (see rule 19) or the missile's flare vulnerability rating whichever is less.

\paragraph{Ground Clutter.} Ground Clutter interferes with an IR missile's ability to track targets. Add 2 to the launch roll if the firing aircraft is in the LO or ML altitude band and fires IR missiles at a “lower” target. A lower target by definition, is one more than one altitude level lower than the firer for each two hexes of horizontal range it is away. Example: a target 7 hexes away is lower if it is more than 3 altitude levels below the firer.

If a missile dives in its proportional move (meaning it loses 2 or more altitude levels by any means) to attack target in the LO altitude band, add 2 to the hit die roll. If a missile attacks a target in “T” level flight (rule 20) add 1 to the hit roll. This can be cumulative with the above modifiers.

\trainingnote{
You may now play all guns and heat seeking missile only combat scenarios! Ignore the AAA, SAM and Ground Unit Interaction Phases of the SOP.
}

\advancedrules

\subsection{Realistic Seeker Head Vertical Field of View Limits}

Rather than just allowing aircraft that climbed or dived to launch at targets an unlimited distance above or below respectively, a more realistic set of seeker head limits can be simulated by using the Radar Vertical Limits Table (see 16.5) to define the vertical FOV limits for caged and uncaged seekers as follows:

\begin{itemize}

    \item Use the limited radar arc Vertical Limits for a caged seeker head.

    \item Use the 180 degree radar arc Vertical Limits Tables described for uncaged IRMs.

\end{itemize}

\subsection{Helmet Mounted Sights}

\paragraph{Helmet Mounted Sights (HMS) Technology.} Modern aircraft can be equipped with helmet mounted sights (see scenario notes or the technology section of the ADC). HMS allows a pilot firing uncaged IR missiles to attempt lock­up against any one sighted enemy aircraft in the firer's 150 to 180 degree arcs (essentially expanding the uncaged missile's FOV). The lock-up succeeds on an unmodified roll of 9 or less regardless of any closer or equally near aircraft. If the lock-up attempt falls, missiles may not be launched.

\subsection{IRM Seeker Lock-Up Assistance Methods}

\paragraph{Radar Assist.} Uncaged IR missiles may be slaved to an aircraft's radar (declare in the Aircraft Decisions Phase). If the firing aircraft currently has a radar lock-on (see rule 16) to the intended target, the missiles may be automatically locked-up to it without rolling and regardless of how many aircraft are currently in the missile's FOV. IR missiles may be fired at night against otherwise unsighted targets using radar assist. This is an exception to the rule requiring targets to be sighted.

\paragraph{VAS Assist.} Type “M” and “A” seeker head equipped missiles, if uncaged, may be slaved to the VAS system and automatically lock-up a VAS spotted target as above in radar assist. Declare in the Aircraft Decisions Phase.

\paragraph{IRSTS Assist.} Any IR missile may be slaved to an IRSTS system (declare in the Aircraft Decisions Phase) and may automatically lock-up any target the IRSTS system is locked onto as above in radar assist. A Type B IRSTS lock-on allows missiles to be fired at targets in the firer's 180 arcs even if the missile normally would not have an FOV that wide (i.e., non-uncaged missiles). This, along with HMS technology, are the only exceptions to the missile FOV requirements.

\subsection{Expanded and Reduced IRM Envelopes}

\paragraph{Expanded/Reduced Envelopes.} The listed missile envelopes are for fighter sized targets at normal or military power. Larger or hotter targets may be acquired from greater distances, while targets at idle power may be more difficult to acquire. The following rules reflect this:

\begin{itemize}

    \item Any target with a visibility number of 10 or more, or any target using AB power, or which has a fuel usage number greater than 5 for its chosen power setting increases an IRM's existing lock-on envelope by 50\% (round up).

    \item Any target using idle power reduces an IRM's lock-on envelope to 2/3d's normal amount (round up). Exception: for a large (Vis 10+) target in idle, use the normal missile envelope.

\end{itemize}

\paragraph{Out Of Envelope IRM Launches.} An IRM may be launched inside its minimum range with a launch roll modifier of +3, except range 0 launches are not allowed and range 1 launches for IRMs that do not instantly arm automatically fail.

Type “A” seekers may still be launched at extended range, as defined above, at large targets or those which are not in AB power or which do not meet the fuel use parameters given above by accepting the +3 out of envelope launch roll modifier. If the target is at idle power, extended range out of envelope shots are those over 2/3rds the listed range up to the original listed range.

Note: With these rules it is possible to lock-up and launch at targets beyond the missile's flight range capabilities resulting in wasted shots.
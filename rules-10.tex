\rulechapter{Aircraft Damage}

\x{
This chapter details the procedures for determining the damage results of successful hits against aircraft.
}{
This rule describes the procedures for determining the damage inflicted by successful hits against aircraft and the effects of that damage.
}

\section{Aircraft Damage Resolution}
\label{rule:aircraft-damage-resolution}

\x{

When a gun, rocket, or missile hit is achieved on an aircraft, determine the damage inflicted and apply it to the aircraft.

\addedin{1C}{1C-tables}{
    \begin{onecolumntablefloat}[t!]
\begin{onecolumntable}
\tablecaption{table:aircraft-damage-effects}{Aircraft Damage Effects}
\x{
\begin{tabularx}{\linewidth}{cX}
\toprule
\minitable{c}{Damage}&
Effect\\
\midrule
---&Superficial Damage; no adverse effects.\\
L&Light Damage; no ET turns allowed; lose High Pitch Rate; aircraft becomes Low Roll Rate\\
2L&Light Damage; as L plus no BT turns allowed, $+1$ to all preparatory move requirements.\\
H&Heavy Damage; as 2L plus Mil and A/B power halved, CCC halved, no roll maneuvers allowed, no supersonic flight allowed. Roll once for Systems loss.\\
C&Crippled; as H plus lose A/B power, no HT turns, aircraft smokes, lose all technology. Roll again for Systems loss.\\
K&Aircraft Killed (shot down), remove from play.\\
\bottomrule
\end{tabularx}
\begin{tablenote}{\linewidth}
Note: if end speed > High Transonic when “H” or “C” damaged, roll twice for prog.\ damage even if Damage Control done.
\end{tablenote}
}{
\begin{tabularx}{\linewidth}{cX}
\toprule
Damage&
\multicolumn{1}{c}{Effects}\\
\midrule
\addlinespace
L&The aircraft can no longer turn at the ET rate, loses HPR, and becomes LRR.\\
\addlinespace
&The aircraft suffers a $+1$ modifier to air-to-air gun and rocket attacks and to weapon launches.\\
\addlinespace
2L&As L, and additionally the aircraft can no longer turn at the BT rate and requires an additional preparatory FP before all maneuvers.\\
\addlinespace
&The aircraft suffers a $+1$ modifier to air-to-air gun and rocket attacks and to weapon launches.\\
\addlinespace
H&As 2L, and additionally the aircraft receives only half of the normal APs from military and afterburner, has its CC halved, and will potentially suffer damage if it uses supersonic speeds.\\
\addlinespace
&The aircraft must check for a system loss.\\
\addlinespace
&The aircraft suffers a $+2$ modifier to air-to-air gun and rocket attacks and to weapon launches.\\
\addlinespace
C&As H, and additionally the aircraft loses afterburner power, can no longer turn at the HT rate, smokes, and loses all technology.\\
\addlinespace
&The aircraft must check for a system loss.\\
\addlinespace
&The aircraft suffers a $+3$ modifier to air-to-air gun and rocket attacks and to weapon launches.\\
\addlinespace
K&The aircraft is destroyed and removed from play.\\
\addlinespace
&The crew may attempt emergency egress (rule~\ref{rule:emergency-egress}).\\
\addlinespace
\bottomrule
\end{tabularx}

}
\end{onecolumntable}
\end{onecolumntablefloat}
}

\paragraph{Levels of Damage.} A hit can potentially produce one of five levels of damage.

\begin{itemize}

    \item\itemparagraph{Superficial Damage.} The hit has no affect.

    \item\itemparagraph{Light Damage.} The aircraft experiences some performance loss.
    
    \item\itemparagraph{Heavy Damage.} The aircraft experiences major performance loss and/or systems damage.

    \item\itemparagraph{Crippled.} The aircraft is heavily damaged and possibly combat ineffective.
    
    \item\itemparagraph{Killed.} The aircraft is destroyed. Remove the aircraft counter from play.

\end{itemize}

A result of no affect, or aircraft kill requires no further action. Light, Heavy, and Crippled damage produce the appropriate performance and combat restrictions \changedin{1C}{1C-tables}{listed on the Aircraft Damage Table}{Table~\ref{table:aircraft-damage-effects}}.

\paragraph{Procedure.} When a gun or missile hit is achieved on an aircraft, consult \changedin{1C}{1C-tables}{the Damage Table}{Table~\ref{table:aircraft-damage}}. Roll the die once and apply any required modifiers. Cross index the modified roll with the attacking gun or rocket's Air To Air rating or missile's listed Attack Rating for the type of hit achieved (Proximity or Direct). The result gives the level of damage achieved.

\addedin{1C}{1C-tables}{
    \begin{TABLE}

\TABLECAPTION{table:aircraft-damage}{Aircraft Damage}

\begin{tabularx}{0.8\linewidth}{X*{10}{@{ }c@{ }}}
\hline
\multirow{2}{*}{\minitable{c}{Die\\Roll}}&
\multicolumn{10}{c}{Weapon Attack Rating}\\
&1&2&3&4&5&6&7&8&9&10\\
\hline
$0-$&K&K&K&K&K&K&K&K&K&K\\
1&C&C&K&K&K&K&K&K&K&K\\
2&H&H&C&K&K&K&K&K&K&K\\
3&L&H&H&C&K&K&K&K&K&K\\
4&L&L&H&H&C&C&K&K&K&K\\
5&L&L&2L&H&C&C&C&K&K&K\\
6&L&L&L&H&H&C&C&C&K&K\\
7&---&L&L&L&H&H&C&C&C&K\\
8&---&---&L&L&2L&H&H&H&C&C\\
9&---&---&---&L&L&2L&H&H&H&C\\
$10+$&---&---&---&---&L&L&2L&H&H&H\\
\hline
&\phantom{2L}&\phantom{2L}&\phantom{2L}&\phantom{2L}&\phantom{2L}&\phantom{2L}&\phantom{2L}&\phantom{2L}&\phantom{2L}&\phantom{2L}\\[-3ex]
\end{tabularx}

\medskip

\begin{tablenote}{0.8\linewidth}
Damage Modifiers
\medskip

\begin{itemize}
    \item Shift one column right if aircraft already L or more damaged.
    \item Shift one column left if hit was from gun snap shot.
    \item $-2$ to die roll if air to air rocket hit or direct hit from missile.
    \item Plus or Minus Aircraft Vulnerability as listed on target ADC.
\end{itemize}

\end{tablenote}

\end{TABLE}
    \begin{onecolumntablefloat}[t]
\begin{onecolumntable}
\tablecaption{table:system-loss}{Systems Loss}
\begin{tabularx}{\linewidth}{cX}
\toprule
\multirow{2}{*}{\minitable{c}{Die\\Roll}}&
\multicolumn{1}{c}{Critical System Lost}\\
&\\
\midrule
1&Cockpit: Pilot Killed, remove aircraft from play.\\
2&Cockpit: Crewman killed. Lose multi-crew bonuses and lose radar and weapon technology. Bomb system = manual.\\
3&One engine permanently flamed out\\
4,5&Radar disabled. Lose all radar functions.\\
6,7&ECM disabled. Lose all ECM functions.\\
8&Weapons System disabled, aircraft may no longer attack. Jettison stores.\\
9&Internal guns and any gunpods disabled.\\
10&Technology disabled, lose all technology.\\
\bottomrule
\end{tabularx}
\end{onecolumntable}
\end{onecolumntablefloat}
}
\paragraph{Previous Damage Effect.} \changedin{1B}{1B-apj-36-errata}{If the aircraft was previously damage, the hitting weapon's attack rating is increased one.}{The damage column is shifted once to the right if the aircraft has sustained L or worse damage, regardless of the prior or current cause of damage.}

\paragraph{Damage Table Modifiers.} The damage table die roll is modified for the following:

\begin{itemize}

    \item\itemparagraph{A/C Vulnerability.} The target's Vulnerability shown on the ADC is used directly as a modifier to the die roll.
    
    \item\itemparagraph{Direct Missile Hit.} If the damage roll was prompted by a Direct Missile Hit, apply a modifier of $-2$. This does not apply to Proximity Missile Hits.
    
    \item\itemparagraph{Rocket Attack.} If the damage roll is the result of a rocket attack, apply a modifier of $-2$ just as for Direct Missile Hits.
    
\end{itemize}


\section{Cumulative Damage Effects}

\addedin{1C}{1C-tables}{
    \begin{onecolumntable}

\tablecaption{table:cummulative-damage}{Cumulative Hits Effects}

\begin{tabular}{l@{ }c@{ }l}
\hline
Three L&=&H Damage\\
Two H&=&C Damage\\
Two C&=&K Aircraft Killed\\
C + H&=&K Aircraft Killed\\
\hline
\end{tabular}

\end{onecolumntable}
}

\paragraph{Cumulative Damage.} At the end of a scenario, an aircraft's damage level for purposes of victory points equals the highest level of damage it has sustained. When an aircraft is repeatedly hit, it can accumulate several hits of one or more types. These multiple hits are cumulative and sufficient hits of a lesser type will equal a worse level of damage as follows\addedin{1C}{1C-tables}{\ and as shown in Table~\ref{table:cummulative-damage}}:

\begin{itemize}
    \item 3L = H. Three cumulative L hits equal an H hit. The aircraft's level of damage is now at least H.
    \item 2H = C. Two cumulative H hits equal an C hit. The aircraft's level of damage is now at least C.
    \item C + H = K. Any H or C hit, or cumulative hits eqating to an H or C hit, on an already Cripped aircraft destroy it.
\end{itemize}

For example, if an aircraft was hit twice before and had H and L hits. Its damage level is considered H. On this turn, it is hit again and receives a 2L. Since 3L = H, the aircraft damage becomes 2H. And since 2H = C, the aircraft damage level becomes C. It is now Crippled.

}{


\paragraph{Damage Procedure.} When a gun, rocket, or missile attack hits an aircraft, determine the weapons attack rating and apply any required modifiers. For missiles, the attack rating will depend on whether a direct or proximity hit was achieved. Roll the die and apply any required modifiers. Consult Table~\ref{table:aircraft-damage} and cross-index the modified roll with the modified attack rating. If the result is a dash, no damage is inflicted. Otherwise, the result gives the damage level suffered by the target. Note this damage on the target's log sheet.

\paragraph{Attack Rating Modifiers.}
The weapon attack rating is modified as follows:
\begin{itemize}
\item If the target already has L or worse damage, apply a $+1$ modifier.
\item If the hit was from a gun snap shot, apply a $-1$ modifier.
\end{itemize}

\paragraph{Damage Die Roll Modifiers.} The damage die roll is modified as follows:

\begin{itemize}

    \item Apply the target's vulnerability, shown on its ADC, as a modifier.
    
    \item For direct missile hits, apply a modifier of $-2$. Proximity hits do not receive this modifier.    
    
    \item For rocket hits, apply a modifier of $-2$ just as for direct missile hits.
    
\end{itemize}

\paragraph{Levels of Damage.} A single hit can potentially produce one of five damage levels:

\begin{itemize}

    \item\itemparagraph{L or 2L} Light damage. The aircraft suffers minor performance loss.

    \item\itemparagraph{H.} Heavy damage. The aircraft suffers major performance loss and possibly a system loss.

    \item\itemparagraph{C.} Crippled. The aircraft suffers severe performance loss, possibly a system loss, and is likely ineffective for further combat.
    
    \item\itemparagraph{K.} Killed. The aircraft is destroyed. Remove the aircraft from play.

\end{itemize}

\begin{TABLE}

\TABLECAPTION{table:aircraft-damage}{Aircraft Damage}

\begin{tabularx}{0.8\linewidth}{X*{10}{@{ }c@{ }}}
\hline
\multirow{2}{*}{\minitable{c}{Die\\Roll}}&
\multicolumn{10}{c}{Weapon Attack Rating}\\
&1&2&3&4&5&6&7&8&9&10\\
\hline
$0-$&K&K&K&K&K&K&K&K&K&K\\
1&C&C&K&K&K&K&K&K&K&K\\
2&H&H&C&K&K&K&K&K&K&K\\
3&L&H&H&C&K&K&K&K&K&K\\
4&L&L&H&H&C&C&K&K&K&K\\
5&L&L&2L&H&C&C&C&K&K&K\\
6&L&L&L&H&H&C&C&C&K&K\\
7&---&L&L&L&H&H&C&C&C&K\\
8&---&---&L&L&2L&H&H&H&C&C\\
9&---&---&---&L&L&2L&H&H&H&C\\
$10+$&---&---&---&---&L&L&2L&H&H&H\\
\hline
&\phantom{2L}&\phantom{2L}&\phantom{2L}&\phantom{2L}&\phantom{2L}&\phantom{2L}&\phantom{2L}&\phantom{2L}&\phantom{2L}&\phantom{2L}\\[-3ex]
\end{tabularx}

\medskip

\begin{tablenote}{0.8\linewidth}
Damage Modifiers
\medskip

\begin{itemize}
    \item Shift one column right if aircraft already L or more damaged.
    \item Shift one column left if hit was from gun snap shot.
    \item $-2$ to die roll if air to air rocket hit or direct hit from missile.
    \item Plus or Minus Aircraft Vulnerability as listed on target ADC.
\end{itemize}

\end{tablenote}

\end{TABLE}

\paragraph{Cumulative Damage.} 
If an aircraft has only been damaged once, that damage is used directly. However, when an aircraft is damaged more than once,
damage from each hit can combine to create more severe damage. The rules for combining damage are given here and in Table~\ref{table:cummulative-damage}.
\begin{eqnarray*}
3\textrm{L}&=&\textrm{H}\\
2\textrm{H}&=&\textrm{C}\\
2\textrm{C}&=&\textrm{K}\\
\textrm{C} + \textrm{H}&=&\textrm{K}
\end{eqnarray*}
If two hits do not combine according to these rules, both are noted, but only the more severe damage is considered for damage effects.

\begin{onecolumntable}

\tablecaption{table:cummulative-damage}{Cumulative Hits Effects}

\begin{tabular}{l@{ }c@{ }l}
\hline
Three L&=&H Damage\\
Two H&=&C Damage\\
Two C&=&K Aircraft Killed\\
C + H&=&K Aircraft Killed\\
\hline
\end{tabular}

\end{onecolumntable}

For example, consider an aircraft that is hit once and suffers an H result. Its damage level is H. If it is hit a second time and suffers an L result, its damage becomes $\textrm{H}+\textrm{L}$ (since $\textrm{H}+\textrm{L}$ does not appear in the rules above), but for damage effects its damage level is still considered to be H (since H is the more severe than L). If it is hit a third time and suffers a 2L result, its damage becomes C (since by the rules above $2\textrm{L} + \textrm{L} = 3\textrm{L} = \textrm{H}$ and $\textrm{H} + \textrm{H} = 2\textrm{H} = \textrm{C}$).


\paragraph{Damage Effects.}
Damage degrades an aircraft's flight and combat performance. The precise effect depends on the highest damage level suffered, either directly or cumulatively, according to Table~\ref{table:aircraft-damage-effects}.

\begin{onecolumntablefloat}[t!]
\begin{onecolumntable}
\tablecaption{table:aircraft-damage-effects}{Aircraft Damage Effects}
\x{
\begin{tabularx}{\linewidth}{cX}
\toprule
\minitable{c}{Damage}&
Effect\\
\midrule
---&Superficial Damage; no adverse effects.\\
L&Light Damage; no ET turns allowed; lose High Pitch Rate; aircraft becomes Low Roll Rate\\
2L&Light Damage; as L plus no BT turns allowed, $+1$ to all preparatory move requirements.\\
H&Heavy Damage; as 2L plus Mil and A/B power halved, CCC halved, no roll maneuvers allowed, no supersonic flight allowed. Roll once for Systems loss.\\
C&Crippled; as H plus lose A/B power, no HT turns, aircraft smokes, lose all technology. Roll again for Systems loss.\\
K&Aircraft Killed (shot down), remove from play.\\
\bottomrule
\end{tabularx}
\begin{tablenote}{\linewidth}
Note: if end speed > High Transonic when “H” or “C” damaged, roll twice for prog.\ damage even if Damage Control done.
\end{tablenote}
}{
\begin{tabularx}{\linewidth}{cX}
\toprule
Damage&
\multicolumn{1}{c}{Effects}\\
\midrule
\addlinespace
L&The aircraft can no longer turn at the ET rate, loses HPR, and becomes LRR.\\
\addlinespace
&The aircraft suffers a $+1$ modifier to air-to-air gun and rocket attacks and to weapon launches.\\
\addlinespace
2L&As L, and additionally the aircraft can no longer turn at the BT rate and requires an additional preparatory FP before all maneuvers.\\
\addlinespace
&The aircraft suffers a $+1$ modifier to air-to-air gun and rocket attacks and to weapon launches.\\
\addlinespace
H&As 2L, and additionally the aircraft receives only half of the normal APs from military and afterburner, has its CC halved, and will potentially suffer damage if it uses supersonic speeds.\\
\addlinespace
&The aircraft must check for a system loss.\\
\addlinespace
&The aircraft suffers a $+2$ modifier to air-to-air gun and rocket attacks and to weapon launches.\\
\addlinespace
C&As H, and additionally the aircraft loses afterburner power, can no longer turn at the HT rate, smokes, and loses all technology.\\
\addlinespace
&The aircraft must check for a system loss.\\
\addlinespace
&The aircraft suffers a $+3$ modifier to air-to-air gun and rocket attacks and to weapon launches.\\
\addlinespace
K&The aircraft is destroyed and removed from play.\\
\addlinespace
&The crew may attempt emergency egress (rule~\ref{rule:emergency-egress}).\\
\addlinespace
\bottomrule
\end{tabularx}

}
\end{onecolumntable}
\end{onecolumntablefloat}

\paragraph{System Losses.}
When an aircraft's damage level reaches H or C, it may suffer a system loss. Roll one die, consult Table~\ref{table:system-loss}, and apply the corresponding result. Check once when the damage level reaches H, once again if the damage level increases from H to C, and twice if the damage level reaches C directly.

\begin{onecolumntablefloat}[t]
\begin{onecolumntable}
\tablecaption{table:system-loss}{Systems Loss}
\begin{tabularx}{\linewidth}{cX}
\toprule
\multirow{2}{*}{\minitable{c}{Die\\Roll}}&
\multicolumn{1}{c}{Critical System Lost}\\
&\\
\midrule
1&Cockpit: Pilot Killed, remove aircraft from play.\\
2&Cockpit: Crewman killed. Lose multi-crew bonuses and lose radar and weapon technology. Bomb system = manual.\\
3&One engine permanently flamed out\\
4,5&Radar disabled. Lose all radar functions.\\
6,7&ECM disabled. Lose all ECM functions.\\
8&Weapons System disabled, aircraft may no longer attack. Jettison stores.\\
9&Internal guns and any gunpods disabled.\\
10&Technology disabled, lose all technology.\\
\bottomrule
\end{tabularx}
\end{onecolumntable}
\end{onecolumntablefloat}

}

\trainingnote{
\centering
\x{
You are now ready to play Training Scenario Three.

The Sequence of play is still not required.
}{
You are now ready to play training scenario 3.

The sequence of play is still not required, and you may ignore it.
}
}

\begin{advancedrules}

\section{Progressive Damage}
\label{rule:progressive-damage}

\addedin{1C}{1C-tables}{
    \begin{TABLE}

\TABLECAPTION{table:progressive-damage}{Progressive Damage}
\medskip
\begin{tabular}{ccc}
\hline
\minitable{c}{Current\\Damage}&
\minitable{c}{Die Roll\\or less}&
\minitable{c}{Increased\\Damage}\\
\hline
L or 2L&2&H\\
H&3&C\\
C&4&K\\
\hline
\end{tabular}

\end{TABLE}
}

An aircraft's damage can worsen or spread if not contained by Damage Control (for example, leaking hydraulic fluid could burst into flame, or a damaged engine could fall).

\paragraph{Procedure.} Roll the die once for each damaged aircraft during the Aircraft Admin Phase of every game-turn following the one in which it was hit. Do not check for progressive damage on any game-turn in which a damaged aircraft is hit again if the hit resulted in a higher damage level.

Consult \changedin{1C}{1C-tables}{the progressive damage table}{Table~\ref{table:progressive-damage}}, if the die roll is less than or equal to the number given next to the aircraft's current damage level, then the aircraft's damage increases to the next higher level.

\paragraph{Damage Control.} Performing damage control involves the crew activating back-up electrical, fuel transfer, and hydraulic systems and/or shutting down damaged ones and/or fighting fires. Damage Control essentially stabilizes the aircraft's current damage level so that it will not progress. Aircraft which perform Damage Control negate progressive damage and do not check for it until they suffer a new hit.

\paragraph{Damage Control Procedure.} Declare the intent to perform damage control at the start of the aircraft's move. Fly the aircraft normally within the following restrictions:

\begin{itemize}
    \item No attacks or weapons launches are allowed.
    \item External stores may be jettisoned
    \item No radar work may be performed.
    \item No maneuvers except Slides may be performed.
    \item No turns above EZ may be performed.
    \item Only non-A/B power sustained climbs allowed.
    \item Only non-A/B power steep dives using up to 2 VFPs may be performed.
    \item Free descent is allowed.
    \item Terrain Following flight is not allowed.
\end{itemize}

Once an aircraft completes its flight without violating the above restrictions, damage control is automatically completed. An aircraft damaged before moving may not perform damage control in the game-turn it was hit. It may perform damage control on any subsequent turn. Previous damage control is nullified when an aircraft is hit anew.

\section{Optional Aircraft Damage Tables}

\addedin{1C}{1C-tables}{
    \addedin{1B}{1B-apj-22-damage-tables}{

\begin{twocolumntable}
\x{

\tablecaption{table:optional-damage}{Optional Advanced Aircraft Damage}
\small
\begin{tabularx}{\linewidth}{lcX}
\toprule
\multirow{2}{*}{\minitable{c}{Die\\Roll}}&
\multirow{2}{*}{\minitable{c}{Hit\\Location}}&
Damage Effects to Aircraft and Notes\\
\\
\midrule
\multicolumn{3}{c}{Light Damage}\\
\midrule
&&$+1$ to all attacks, $+1$ to weapons launches.\\
1&Airframe&No supersonic speeds allowed.$^*$\\
2&Airframe&No BT or ET turn rate allowed.\\
3&Controls&Aircraft becomes Low roll rate and loses any high pitch rate capability.\\
4&Controls&Add one to prep-move requirements for all maneuvers.\\
5&Fuel&Lose 1 point of fuel per game turn, double bingo fuel amount. White Vapor emitted.\\
6&Avionics&Radar knocked out and lose all technology.\\
7&Avionics&Bombsight degraded to manual, gunsight = TT+1, HT+2, BT+3.\\
8&Weapons&Internal guns/gun pods disabled.\\
9&Engines&Afterburner disabled. No A/B power available.\\
10&Engines&Thrust loss; reduce Mil and A/B power numbers by 0.5. White Vapor emitted.\\
\midrule
\multicolumn{3}{c}{Heavy Damage}\\
\midrule
&&+2 to all attacks, +2 to weapon launches.\\
1,2&Airframe&No supersonic flight.$^*$ No BT or ET turns. Low roll rate. No high pitch rate. Add two to prep-move requirements for all maneuvers.\\
3&Controls&No rolling maneuvers allowed. Low roll rate. No high pitch rate. No ET turn rate.\\
4&Controls&Throttle jammed, no power changes. No HT, BT, or ET turn rate. No high pitch rate.\\
5&Fuel&Lose 2 points of fuel per turn. Bingo fuel tripled. Roll one die each turn after flight. On 1, a FUEL FIRE occurs (see description below). Aircraft becomes Smoker.\\
6,7&Avionics&Radar knocked out, all ECM knocked out, and lose all technology\\
8&Weapons&Guns disabled, no RHM/AHM/ARM/ASM missile or BS/RS weapon launches allowed.\\
9,10&Engines&Mil and A/B power halved. Roll for flame out whenever power setting changed. On 1, a Flame-Out occurs in one good engine. Aircraft becomes a smoker.\\
\midrule
\multicolumn{3}{c}{Critical Damage}\\
\midrule
&&$+3$ to all attacks, $+3$ to weapon launches. Roll once on the H table and once below.\\
1&Cockpit&Roll die again: 1, 2 = pilot killed (aircraft lost). 3, 4 = Crewman Killed (lose multi-crew functions). +5 = crew okay. Regardless, treat aircraft as having both an Airframe and Avionics hit as described immediately below.\\
2,3&Airframe&
No supersonic flight.$^*$ Low roll rate. No high pitch. No rolling maneuvers. EZ turns only. Aircraft must jettison external stores.\\
4,5,6&Avionics&Lose radar, all ECM, and all technology. No external pods work. Bombsight degrades to manual and gunsight = TT+1, HT+2, BT+3.\\
7&Fuel Fire**&Aircraft becomes Smoker. Lose 3 die rolls of fuel/turn. Jettison external stores. Roll die after each turn after flight. On 1 or 2 aircraft explodes, crew killed. On 9 or 10, fire goes out permanently but fuel leak still there.\\
8,9,10&Engines&A/B power lost. Mil power halved. Aircraft becomes Smoker. Roll for flame-outs twice per engine (1 or 2 = F.O. in this case). On subsequent turns, roll for flame-out whenever power seting is changed (1 = F.O.).\\
\bottomrule
\end{tabularx}
\begin{tablenote}{\linewidth}
\begin{itemize}
    \item The specifics of the damage rolled are not revealed to the enemy, but are recorded on paper. Visible signs of damage such as smoke or vapor, as indicated on the tables, must be told to the opponent.
    \item Jettison of Stores: Only the results calling for jettison of ordnance require an aircraft to do so. If this is the case, the aircraft must jettison enough stores to become CL configured. 
    \item If table indicates damage to a system or performance capability that the aircraft does not have, roll again. If the second roll also results in non-applicable damage, the hit is still recorded for the die roll penalties on combat and for possible progressive damage.
   \item[$^*$] If supersonic speeds exceeded in game turn, roll for progressive damage twice even if Damage Control was done.
   \item[$^{**}$] A Fuel Fire and risk of explosion can only be stopped by shutting down all engines. Declare this when doing damage control. Treat aircraft as flamed-out. When engines relit, A/B is disabled. Roll one die, on 1 to 4 fire resumes permanently; otherwise it stays out.
\end{itemize}
\end{tablenote}
\end{twocolumntable}

}{

\tablecaption{table:optional-damage}{Optional Aircraft Damage}
\small
\begin{tabularx}{\linewidth}{ccX}
\toprule
\multirow{2}{*}{\minitable{c}{Die\\Roll}}&
\multirow{2}{*}{\minitable{c}{Hit\\Location}}&
\multicolumn{1}{c}{\multirow{2}{*}{Effects}}
\\
\\
\midrule
\multicolumn{3}{c}{L or 2L Damage}\\
\midrule
&&$+1$ to all attacks and weapons launches.
\\
\cmidrule{3-3}
1&Airframe&
The aircraft must check twice for progressive damage at supersonic speeds.
\\
2&Airframe&
The aircraft may not use the BT or ET turn rates.
\\
3&Controls&
The aircraft becomes LRR and loses any HPR capability.
\\
4&Controls&
The aircraft requires one additional preparatory FP before all maneuvers.
\\
5&Fuel&
The aircraft loses 1 fuel point per game turn. 
Its bingo fuel level is doubled.
It emits a white vapor.
\\
6&Avionics&
The aircraft's radar and technology are disabled.
\\
7&Avionics&
The aircraft's bombsight is degraded to manual, and its gunsight to TT +1, HT +2, and BT +3.
\\
8&Weapons&
The aircraft can no longer use internal guns or gun pods.
\\
9&Engines&
The aircraft can no longer use afterburner power.
\\
10&Engines&
The aircraft's thrust at military and afterburner power is reduced by 0.5 AP.
It emits a white vapor.
\\
\midrule
\multicolumn{3}{c}{H Damage}\\
\midrule
&&+2 to all attacks and weapon launches.\\
\cmidrule{3-3}
1 or 2&Airframe&
The aircraft must check twice for progressive damage at supersonic speeds.
It may not use the BT or ET turn rates.
It becomes LRR and loses any HPR capability.
It requires two additional preparatory FP before all maneuvers.
\\
3&Controls&
The aircraft may not perform rolling maneuvers. 
It becomes LRR and loses any HPR capability.
It may not use the ET turn rate.
\\
4&Controls&
The aircraft's throttle jams. 
It may not change its power setting.
It may not use the HT, BT, or ET turn rates.
It loses any HPR capability.
\\
5&Fuel&
The aircraft loses 2 fuel points per game turn. 
Its bingo fuel level is tripled. 
It becomes a smoker.
The aircraft must check each game turn after flight for a fuel fire. On a die roll of $1-$, a fire starts.
\\
6 or 7&Avionics&
The aircraft's radar, ECM, and technology are disabled.
\\
8&Weapons&
The aircraft can no longer use internal guns or gun pods or launch RHM/AHM/ARM/ASM missiles or BS/RS weapons.
\\
9 or 10&Engines&
The aircraft's thrust at military and afterburner power is reduced to one-half the normal values.
It must check for a flame-out each time it changes its power setting. On a $1-$, one good engine flames out.
The aircraft becomes a smoker.
\\
\midrule
\multicolumn{3}{c}{C Damage}\\
\midrule
&&Roll once on the H table and once on the C table.\\
&&$+3$ to all attacks and weapon launches.\\
\cmidrule{3-3}
1&Cockpit&
Roll the die again: 1 or 2 = pilot killed (equivalent to K). 3 or 4 = crewmember killed (lose multi-crew functions). $5+$ = crew unhurt. 
Treat the aircraft as having both the airframe and avionics results described immediately below.
\\
2 or 3&Airframe&
The aircraft must check twice for progressive damage at supersonic speeds.
It may only use the EZ turn rate.
It becomes LRR and loses any HPR capability.
It may not perform rolling maneuvers. 
It must jettison stores.
\\
4 to 6&Avionics&
The aircraft's radar, ECM, technology, and all external pods are disabled.
Its bombsight is degraded to manual, and its gunsight to TT +1, HT +2, and BT +3.
\\
7&Fuel&
The aircraft suffers a fuel fire.
It loses 3 fuel points per game turn. 
%Its bingo fuel level is quadrupled. 
It becomes a smoker.
It must jettison stores.
%Roll die after each turn after flight. On 1 or 2 aircraft explodes, crew killed. On 9 or 10, the fire goes out permanently, but the fuel leak is still there.
\\
8 to 10&Engines&
The aircraft loses afterburner power, and its thrust at military power is reduced to one-half the normal value.
It becomes a smoker.
It immediately checks twice per engine for a flame-out on a $2-$. On subsequent turns, if it changes its power setting, each engine flames out on a $1-$.
\\
\bottomrule
\end{tabularx}

}

\end{twocolumntable}

}

}

The optional Damage Tables may be substituted for the generic damage level restrictions. These tables determine specific effects of L, H, and C level hits. For each hit consult \changedin{1C}{1C-tables}{the appropriate Damage Table}{Table~\ref{table:optional-damage}} and roll the die with no modifiers. \deletedin{1C}{1C-tables}{(Optional damage tables are in play aids sheets.)}

\paragraph{Optional Damage Table Results.} Only the indicated results from the die roll are applied to a damaged aircraft. In some cases, an aircraft may lose systems and not performance, or lose performance and not systems, or it may lose both systems and performance. The specifics of the damage rolled are not revealed to the enemy; they are recorded on paper. Visible signs of damage such as smoke or fire indicated on the tables are told to the opponent.

\paragraph{Progressive Damage} The progressive damage process is checked normally. If it occurs, roll on the tables as if a hit at the new damage level had occurred.

\paragraph{Hollow Aircraft} If the table indicates damage to a system or performance capability the target aircraft does not have, roll again. If the second roll also results in non-applicable damage, record the hit for progressive damage purposes and for die roll penalties for combat. The aircraft is otherwise not affected.

\paragraph{Jettison of Stores.} Only the results calling for jettison of ordnance require an aircraft to do so. It must jettison enough stores to become CL configured.

\paragraph{Combat Ability.} As long as an aircraft has functional combat systems it may attack air and land targets regardless of damage level. The damage modifiers do apply to (and degrade) attack rolls.

\section{Aircraft Crash Sites}

\silentlyaddedin{1C}{1C-figures}{
    \begin{figure}[tbp]
\centering
\begin{tikzfigure}{1.0\linewidth}

    \drawhexgrid{13}{6}

    \newcommand{\drawcrashhex}[3]{
        \miniathex{#1}{#2}{
            \drawhex[black!10]{0}{0}
            \draw node {#3};
        }
    }

    \begin{athex}{2.00}{1.00}
        \drawcrashhex{+0}{+1.0}{1}
        \drawcrashhex{-1}{+1.5}{2}
        \drawcrashhex{+0}{+2.0}{3}
        \drawcrashhex{+1}{+1.5}{4}
        \drawcrashhex{-1}{+2.5}{5}
        \drawcrashhex{+0}{+3.0}{6}
        \drawcrashhex{+1}{+2.5}{7}
        \drawcrashhex{-1}{+3.5}{8}
        \drawcrashhex{+0}{+4.0}{9}
        \drawcrashhex{+1}{+3.5}{10}
        \drawaircraftcounter{0.00}{0.00}{90}{F-4}{}
    \end{athex}

    \begin{athex}{5.00}{1.50}
        \drawcrashhex{+0}{+1.0}{1}
        \drawcrashhex{+1}{+0.5}{2}
        \drawcrashhex{+0}{+2.0}{3}
        \drawcrashhex{+1}{+1.5}{4}
        \drawcrashhex{+2}{+1.0}{5}
        \drawcrashhex{+1}{+2.5}{6}
        \drawcrashhex{+2}{+2.0}{7}
        \drawcrashhex{+1}{+3.5}{8}
        \drawcrashhex{+2}{+3.0}{9}
        \drawcrashhex{+3}{+2.5}{10}
        \drawaircraftcounter{0.00}{0.00}{60}{F-4}{}
    \end{athex}

    \begin{athex}{9.50}{1.25}
        \drawcrashhex{-0.5}{+1.25}{1}
        \drawcrashhex{+0.5}{+0.75}{2}
        \drawcrashhex{+1.5}{+0.25}{3}
        \drawcrashhex{+0.5}{+1.75}{4}
        \drawcrashhex{+1.5}{+1.25}{5}
        \drawcrashhex{+0.5}{+2.75}{6}
        \drawcrashhex{+1.5}{+2.25}{7}
        \drawcrashhex{+2.5}{+1.75}{8}
        \drawcrashhex{+1.5}{+3.25}{9}
        \drawcrashhex{+2.5}{+2.75}{10}
        \drawaircraftcounter{0.00}{0.00}{60}{F-4}{}
    \end{athex}    
    
\end{tikzfigure}
\caption{Crash Sites}
\label{figure:crash-sites}
\end{figure}

}

An aircraft may impact the ground as a result of a variety of mishaps. Whenever an aircraft is shot down, abandoned, or crashes because of stalled, departed, terrain collision, GLOC/Disoriented/Fatal error flight, it must impact somewhere.

\paragraph{Crash Site Determination.} The crash site is determined by the nature of the crash. For stalled, departed, terrain collision, GLOC/Disoriented/Fatal error, GLOC crashes, the crash site is the aircraft's present hex or hexside. For shot down or abandoned aircraft, the crash site is determined by \changedin{1C}{1C-figures}{the Crash Site Location Diagram}{Figure~\ref{figure:crash-sites}}. Consult the \changedin{1C}{CSLD in the play aids}{Figure} for the exact location based on the aircraft's facing and position at the moment of the craft.

\paragraph{Crash Site Damage.} If a crash site hex contains ground units, buildings, or target type terrain, damage may be inflicted on them by the explosion and crash. One ground unit, building, POL marker, or one of the target terrains printed in the hex is immediately attacked at 1 - 2 odds (no modifiers). Determine which is attacked randomly.

\end{advancedrules}

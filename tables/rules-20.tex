\chapter
{Ground Terrain and Terrain Following Flight}

This chapter details the effects of terrain features on aircraft operations. As an aircraft approaches the ground, it must take into account terrain elevation and terrain features and how they may interfere with flight and combat.

\section
{Ground Terrain}

\paragraph{Terrain Elevation.} Terrain occurs in, and is mapped in, levels. Each level represents an increase of 1000 feet of altitude (one altitude level). Each terrain level is defined by a solid contour line and a solid color printed over the entire area. The lightest color is the lowest level of terrain; terrain increases in level as the terrain color darkens.

An aircraft crossing a contour line or diving or descending to an altitude equal or less than that of the terrain in the hex it is in impacts the ground and is destroyed killing its crew. Exception; an aircraft may fly at ground level in Terrain Following flight.

Ridge lines or hills within a terrain level which rise less than a full altitude level are shown by a dashed contour line. Ridgelines and hills are obstacles to Terrain Following Flight.

\paragraph{Ground Level Attitude.} The lowest level of altitude is Ground Level. On an unfeatured game map, ground level is altitude level 0. If terrain elevation is present on the map, ground level becomes equal to the elevation of the terrain in the hex.

\paragraph{Terrain Features.} A map key is provided for all maps which have features. Terrain features, contours, and elevations apply exactly where they are printed on the maps. Terrain altitude only affects an aircraft if its flight path takes it across a contour line or terrain feature.

For the purpose of applying this rule, an aircraft in a hex is located in the exact center of the hex and an aircraft on a hexside is at the midpoint of the hexside. If aircraft are flying in a Close formation, they are evenly spaced about the center points.

\addedin{1C}{1C-tables}{
    \begin{table}
\centering
\addedin{1B}{1B:tsoh-errata}{
\caption{Terrain Effects}
\medskip
\begin{tabular}{lccp{20em}}
\hline
Hex Type&SR&Def.\ Str.&Combat Effects\\
\hline
Clear&NA&NA&None\\
Forest&NA&NA&All units are considered camouflaged. $+2$ to Strafe attacks, $+1$ to all other attacks.\\
River&NA&NA&None\\
Rail Line&24&6 soft&None\\
Road&24&8 soft&Units on a Road are treated as if in Clear terrain.\\
Trail&12&6 soft&Units on a Trail defend using other terrain in the hex, except vehicles are not considered camouflaged.\\
Runway&36&10 hard&None\\
Major Bridge&48&18 hard&None\\
Minor Bridge&36&12 soft&None\\
City&48&5 soft&All units except heavy AAA, SAMs and Radars are consider camouflaged. $+2$ to all attacks.\\
Town/Village&36&3 soft&Infantry units only are considered camouflaged. $+1$ to all attacks.\\
\hline
\end{tabular}

}
\end{table}
}

\paragraph{Terrain Effects\protect\deletedin{1C}{1C-tables}{ Chart}.} \changedin{1C}{1C-tables}{The Terrain Effects Chart}{Table~\ref{table:terrain-effects}} details the specific types of terrain and their effects on units on the ground. Terrain may affect the degree of camouflage of units, and may affect defensive abilities.

\section
{Terrain Following Flight}

An aircraft flying near the ground may enter Terrain Following Fight (TFF). TFF represents flying at an altitude of less than 500 feet above the ground (usually 75 to 200 feet depending on the size and speed of the aircraft).

Any aircraft may use TFF in daylight conditions. An aircraft with Terrain Following technology may use TFF at night or in Adverse Weather.

\paragraph{Entering and Exiting TFF.} Aircraft may enter or exit TFF only once per game turn. To enter TFF, an aircraft must have begun the game turn in Level Flight and no more than one altitude level above the ground. At any point in the turn (while still in Level Flight), it may announce Entering TFF and descend to terrain level (at no cost in VFPs or Decel points).

To exit TFF, an aircraft must declare Exiting TFF. The act of exiting may be declared at any point in the aircraft's flight and costs no FPs. The aircraft rises to the level above Terrain level. An aircraft may remain in TFF from turn to turn. It may enter and exit TFF within a single turn but the opposite is not true.  If it started the turn in TFF and then exited TFF, it may not return to TFF until the following game turn. \addedin{1B}{1B-apj-23-errata}{An aircraft that starts the game turn in TFF that wishes or needs to defensively engage missiles may do so normally, it is considered to exit TFF in the aircraft decision phase in this case.}

\paragraph{Contour Following.} Although TFF is a form of level flight, an aircraft can change altitude while following rising and falling terrain contours.  Changing altitude with rising terrain is a form of climbing and diving flight while in T-level flight, and like actual climbs and dives, only one or the other may be made.  

\paragraph{Restrictions.} Contour Following is conducted as follows:

\begin{itemize}

    \item If the terrain drops 1 level in a hex, the aircraft drops with the terrain and receives 0.5 accel points. An aircraft may do this any number of times in a game turn.

    \item If the terrain drops 2+ levels in a hex, the aircraft must exit TFF. It may use diving flight to stay within 1 level of the ground.
    
    \item If the terrain rises 1 level in a hex, the aircraft rises with terrain and receives 1 decel point. An aircraft may do this any number of times in a game turn.

    \item If terrain rises 2+ levels in a single hex, the aircraft must avoid the obstacle, or exit TFF before entering this hex. If it intends to cross the rise it must climb to one level above the terrain before entering the hex.

    \item If the terrain contains a ridgeline, or is a built-up area, the aircraft must exit TFF prior to entering the terrain hex.

    An aircraft attempting to cross a ridge, or built-up area while in TFF crashes. An aircraft following falling contours which attempts to cross a rising contour in the same game turn crashes. An aircraft entering a terrain hex that rises more than one level while in T-level flight crashes.

\end{itemize}

\paragraph{TFF Restrictions.}  An aircraft in TFF may not:

\begin{itemize}
    \item Perform any rolling maneuvers.
    \item Declare or perform damage control.
    \item Use ET turn rates.
    \item Engage missiles.
    \item Laser designate targets.
    \item Use radar (if pilot only).
    \item Guide RG weapons.
    \item Padlock enemy aircraft in the sighting phase.
    \item Sight ground units or targets more than 12 hexes away unless they are on visible terrain higher than the TFF aircraft.
    \item Launch missiles.
\end{itemize}

\paragraph{TFF Benefits.} An aircraft in TFF receives the following benefits:

\begin{itemize}
    \item Attacking air to air missiles apply a modifier of $+1$ to the To Hit roll (in addition to modifiers for ground clutter).
    \item A SAM with a minimum altitude ability of greater than T may not track or be guided at an aircraft in TFF. Some T-capable SAMs have die roll modifiers when engaging TFF aircraft.
    \item A ground unit cannot see TFF aircraft more than 12 hexes away unless they are on higher terrain.
    \item Regular Early Warning Radar cannot detect or track TFF aircraft.
    \item MTI Early Warning Radar can detect a TFF aircraft within 20 hexes (if at the same altitude) or within 60 hexes (if the radar is at a higher altitude, on higher terrain or on a tower or ship's mast).
\end{itemize}

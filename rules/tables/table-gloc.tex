\begin{onecolumntable}

\x{

\tablecaption{table:gloc-avoidance}{G-Induced Loss of Consciousness}
\begin{tabularx}{1.0\linewidth}{Xl}
\toprule
\multicolumn{2}{c}{Crewmember}\\
\midrule
Non-pilot crewmember &$-1$\\
Excellent fitness    &$+1$\\
Poor fitness         &$-1$\\
\midrule
\multicolumn{2}{c}{Aircraft}\\
\midrule
\tablerowdeletedin{2A}{2A-snap}{\textdeletedin{2A}{2A-snap}{Used snap-turn this phase}   &\textdeletedin{2A}{2A-snap}{$-1$}}
Has canted seat (e.g., F-16)    &$+1$\\
2nd or subsequent GLOC die roll in GLOC cycle (cumulative)&$-1$\\
\bottomrule
\end{tabularx}
\begin{tablenote}{1.0\linewidth}
\begin{itemize}
    \item Check for GLOC if aircraft turned at ET rate while in the LO, ML, or MH altitude bands.
    \item Roll one die after each facing at the the ET rate for each crewmember. A “1” or less indicates he has GLOC'd.
    \item Cycle lasts until no BT/ET turns used in a game-turn.
\end{itemize}
\end{tablenote}

}{

\tablecaption{table:gloc-avoidance}{Avoiding GLOC}
\begin{tabularx}{1.0\linewidth}{Xl}
\toprule
\multicolumn{2}{c}{Crewmember}\\
\midrule
Non-pilot crewmember &$-1$\\
Excellent fitness    &$+1$\\
Poor fitness         &$-1$\\
\midrule
\multicolumn{2}{c}{Aircraft}\\
\midrule
Has canted seat (e.g., F-16)    &$+1$\\
Each check after the first in a cycle (cumulative)&$-1$\\
\bottomrule
\end{tabularx}
\begin{tablenote}{1.0\linewidth}
\begin{itemize}
    \item Check for GLOC after each facing change at the the ET rate in the LO, ML, or MH altitude bands.
    \item Roll one die for each crewmember. On a $1-$, the crewmember suffers GLOC.
    \item A GLOC cycle lasts until the aircraft does not use BT or ET turns in a game turn.
\end{itemize}
\end{tablenote}

}

\end{onecolumntable}

\begin{onecolumntable}

\x{
\tablecaption{table:gloc-recovery}{Recovery from GLOC}
\begin{tabularx}{\linewidth}{X}
\toprule
\begin{itemize}
    \item Automatic during admin phase of 2d game turn following the one in which GLOC occured.
    \item Early recovery possible in admin phase of game turn of GLOC occurence and in the admin phase of the turn following if crewmember has excellent fitness or is in a multi-crew aircraft where other member not GLOC'd. Die roll 4 or less equals early recovery.
\end{itemize}
\\
\bottomrule
\end{tabularx}
}{
\tablecaption{table:gloc-recovery}{Recovering from GLOC}
\begin{tabularx}{\linewidth}{X}
\toprule
\begin{itemize}
    \item If a crewmember has excellent fitness or is in a multi-crew aircraft with another member who is not suffering from GLOC, they may check for an early recovery in the aircraft administrative phases of the game turn in which GLOC occurred and the following game turn. Early recovery requires a roll of $4-$ with no modifiers.
    \item All crewmembers recover automatically in the aircraft administrative phase of the second game turn following the one in which GLOC occurred.
\end{itemize}
\\
\bottomrule
\end{tabularx}
}
\end{onecolumntable}

\begin{onecolumntable}

\x{
\tablecaption{table:gloc-flight}{GLOC/Disoriented Flight}
\small
\begin{tabularx}{\linewidth}{lX}
\toprule
Die&Aircraft Random Movement\\
Roll&(Based on Current Flight Type)\\
\midrule
\multicolumn{2}{c}{Level Flight}\\
\midrule
1   &Stay level, no turns.\\
2   &Stay level, TT turn.\\
3   &Stay level, HT turn.\\
4   &Descend one level, TT turn as above.\\
5   &Descend one level, HT turn as above.\\
6   &Maximum sustained climb, EZ turn.\\
7   &Maximum zoom climb, TT turn.\\
8   &Maximum zoom climb, HT turn.\\
9   &Maximum steep dive, HT turn.\\
10  &Half roll and dive, minimum vertical dive, random vertical rolls.\\
\midrule
\multicolumn{2}{c}{Climbing Flight}\\
\midrule
1   &Maximum sustained climb, EZ turn.\\
2   &Maximum zoom climb, HT turn.\\
3   &Maximum zoom climb, no turns.\\
4   &Maximum zoom climb, TT turn.\\
5   &Minimum vertical climb, no vertical rolls.\\
6   &Maximum vertical climb, random vertical rolls.\\
7   &Level flight, TT turns.\\
8   &Level flight, HT turns.\\
9   &Half roll and dive, minimum steep dive.\\
10  &Half roll and dive, maximum steep dive.\\
\midrule
\multicolumn{2}{c}{Diving Flight}\\
\midrule
1   &Level flight if able or meet steep dive requirements while exiting vertical dive.\\
2   &As above plus TT turns.\\
3   &As 1 above plus HT turns.\\
4   &Minimum steep dive, no turns.\\
5   &Minimum steep dive, TT turns.\\
6   &Minimum steep dive, HT turns.\\
7   &Maximum steep dive, TT turns.\\
8   &Maximum steep dive, HT turns.\\
9   &Minimum vertical dive, random vertical rolls.\\
10  &Maximum vertical dive, random vertical rolls.\\
\bottomrule
\end{tabularx}
\begin{tablenote}{\linewidth}
{\centering Directions\par\smallskip}

\begin{itemize}
    \item Expend all remaining FPs via directions above, it is allowed to switch between climbs and dives in mid-moves if required. Randomly determine direction of turns. Random vertical rolls occur on last VFP only, roll for direction and number of facings.
    \item For climbs and dives, use maximum allowed VFPs. A maximum climb/dive means each VFP gains max possible levels. Minimum means each gains least amount possible.
\end{itemize}
\end{tablenote}
}{
\tablecaption{table:gloc-flight}{GLOC or Disoriented Flight}
\small
\begin{tabularx}{\linewidth}{lX}
\toprule
Die&Aircraft Random Movement\\
Roll&\\
\midrule
\multicolumn{2}{c}{Aircraft in Level Flight}\\
\midrule
1   &Level flight with no turns.\\
2   &Level flight with a TT turn.\\
3   &Level flight with a HT turn.\\
4   &Level flight with free descent and a TT turn.\\
5   &Level flight with free descent and a HT turn.\\
6   &Maximum sustained climb with an EZ turn.\\
7   &Maximum zoom climb with a TT turn.\\
8   &Maximum zoom climb with a HT turn.\\
9   &Maximum steep dive with a HT turn.\\
10  &Half roll and dive, followed by a minimum vertical dive with random vertical roll.\\
\midrule
\multicolumn{2}{c}{Aircraft in Climbing Flight}\\
\midrule
1   &Maximum sustained climb with an EZ turn.\\
2   &Maximum zoom climb with a HT turn.\\
3   &Maximum zoom climb with no turns.\\
4   &Maximum zoom climb with a TT turn.\\
5   &Minimum vertical climb with no vertical rolls.\\
6   &Maximum vertical climb with a random vertical roll.\\
7   &Level flight with a TT turn.\\
8   &Level flight with a HT turn.\\
9   &Half roll and dive, followed by a minimum steep dive.\\
10  &Half roll and dive, followed by a maximum steep dive.\\
\midrule
\multicolumn{2}{c}{Aircraft in Diving Flight}\\
\midrule
1   &Level flight, if the aircraft can, or a steep dive meeting the requirements for exiting a vertical dive.\\
2   &1 with TT turns.\\
3   &1 with HT turns.\\
4   &Minimum steep dive with no turns.\\
5   &Minimum steep dive with a TT turn.\\
6   &Minimum steep dive with a HT turn.\\
7   &Maximum steep dive with a TT turn.\\
8   &Maximum steep dive with a HT turn.\\
9   &Minimum vertical dive with a random vertical roll.\\
10  &Maximum vertical dive with a random vertical roll.\\
\bottomrule
\end{tabularx}
\begin{tablenote}{\linewidth}
\begin{itemize}
    \item Expend all remaining FPs via directions above.
    \item Aircraft are allowed to switch between climbs and dives in mid-moves if required. 
    \item Turns are continuous. Roll for their direction. 
    \item Random vertical rolls occur on last VFP only. Roll for their direction and the number of facing changes.
    \item For climbs and dives, use the maximum number of VFPs permitted. In a maximum climb or dive, each VFP gains or loses the maximum number of levels. In a minimum climb or dive, each VFPs gains or loses the minimum number of levels.
\end{itemize}
\end{tablenote}

}

\end{onecolumntable}

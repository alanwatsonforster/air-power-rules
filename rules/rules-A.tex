\rulechapter{Range and Arcs}

The relative position of one aircraft or missile with respect to another is crucial for many aspects of air combat. 

This rule describes how the relative position is determined. The relative position has three components: the range, the horizontal arc, and the vertical arc. 

% including air-to-air gun and rocket attacks (rule~\ref{rule:air-to-air-gun-combat}), air-to-air missile launches, tracking, and attacks  (rule~\ref{rule:air-to-air-missiles}), visual sighting (rule~\ref{rule:sighting-aircraft-and-missiles}), advantage (rule~\ref{rule:initiative}), and radar detection, tracking, and illumination (rule~\ref{rule:air-to-air-radar}).

\AY[3A-range]{

\section{Range}
\label{rule:range}

Many situations described in these rules depend on the \emph{range} from one element to another. For example, it is more difficult to sight or identify more distant aircraft and radars often have a limited range.

Other than in the exceptional cases noted subsequently, the range in hexes from one element to another is determined using the following procedure.

\paragraph{Range Procedure.}
The range has two components, the horizontal and vertical ranges. They are determined and combined as follows:

\begin{onecolumnfigure}[tb]

\begin{tikzfigure}{3.333\standardhexwidth}

    \drawhexgrid{0}{0}{2}{2}  

    \begin{athex}{0.50}{0.25}
        \drawdotathex[black!25]{+0.00}{0.50}
        \drawdotathex[black!25]{+0.00}{1.00}
        \drawdotathex[black!25]{+0.50}{0.25}
        \drawdotathex{+0.50}{0.75}
        \drawdotathex[black!25]{+0.50}{1.25}
        \drawdotathex[black!25]{+1.00}{0.50}
        \drawdotathex[black!25]{+1.00}{1.00}
    \end{athex}

    %\begin{athex}{3.50}{0.25}
    %    \drawdotathex[black!25]{+0.00}{0.50}
    %    \drawdotathex[black!25]{+0.00}{1.00}
    %    \drawdotathex[black!25]{+0.50}{0.25}
    %    \drawdotathex{+0.50}{0.75}
    %    \drawdotathex[black!25]{+0.50}{1.25}
    %    \drawdotathex[black!25]{+1.00}{0.50}
    %    \drawdotathex[black!25]{+1.00}{1.00}
    %\end{athex}
    
\end{tikzfigure}

\figurecaption{figure:range}{Range. The map locations shown in gray are one-half of a hex from the map location shown in black.}

\end{onecolumnfigure}


\begin{itemize}
\item
The \emph{horizontal range} is the length of the shortest route between the map locations, measured in hexes and counting only whole hexes.

If both elements are on hex centers, then the procedure is straightforward and probably familiar. However, if one or both are on hex sides, it may be less familiar. Figure~\ref{figure:range} shows map locations that are half a hex and one hex apart, including both hex centers and hex sides. These diagrams can be understood by noting that the distance from each map location to the surrounding six map locations is half a hex.

Figure~\ref{figure:range-example} gives examples of horizontal ranges.

\item
The \emph{vertical range} is the difference in altitude levels divided by two and rounded down. That is, each two whole altitudes levels of difference counts as one hex of vertical range.
\item
The \emph{range} is the sum of the horizontal and the vertical ranges.
\end{itemize}

The ranges determined using this method are symmetric, in the sense that the range from a first element to a second is the same as the range from the second element to the first. This is not necessarily the case with some of the exceptional ranges defined below.

\paragraph{Range Exceptions.}
The normal range procedure is \emph{not} used in the following cases:
\begin{itemize}

\item
The range for air-to-air gun attacks is determined according to rule~\ref{rule:air-to-air-gun-combat}.

\item
When attempting to search for a higher aircraft in daylight, the vertical range is the difference in altitude levels divided by \emph{four} and rounded down (see rule~\ref{rule:sighting-aircraft-and-missiles}).

\item
When considering positions of advantage, the horizontal range and vertical altitude difference are considered separately (see rule~\ref{rule:positions-of-advantage}).

\item
When determining if a radar target is in ground clutter, the horizontal range and vertical altitude difference are considered separately (see rule~\ref{rule:radar-ground-clutter}).

\end{itemize}

\paragraph{Closeness.}
\label{rule:closeness}

Range is used to determine relative closeness. If two aircraft have different ranges from a reference element, the one with the smaller range is considered to be closer to the reference element. If two aircraft have the same range, they are considered to be equally close. (That is, horizontal differences of half a hex and vertical differences of an odd number of altitude levels are ignored when determining closeness.)

For example, assuming all the aircraft in Figure~\ref{figure:range-example} have the same altitude, then aircraft C and D are equally close to aircraft A, since both are at a range of 2 hexes. Furthermore, assuming that aircraft A and C are at the same altitude and aircraft D is one altitude level below or above, then aircraft C and D are still equally close to aircraft A since both are still at a range of 2 hexes.

}


\AY[3A-angle-off]{

\section{Horizontal Arcs}
\label{rule:horizontal-arcs}
\label{rule:angle-off}

Other situations described in these rules depend on the angle at which one element is seen from another. For example, radars are typically restricted to scanning over forward cone, the aircraft fuselage and wings can prevent the crew from sighting to the rear and below, early infrared-homing missiles could only track targets from behind, and it is easier to compensate gunfire for deflection when firing from the rear or front of the target than from the beam.

The angle can have two components. The horizontal angle is described by the rules in in this section, and the vertical angle by the rules in the following section.

\paragraph{Reference and Distant Elements.} 

For clarity, we will define the relative position of a \emph{distant element}  with respect to a \emph{reference element}. These roles depend on the context:

\begin{itemize}

\item In air-to-air gun, rocket, and missile attacks (rules~\ref{rule:air-to-air-gun-combat}, \ref{rule:air-to-air-rocket-combat}, and \ref{rule:missile-attacks}), the target aircraft is the reference element, and the attacking aircraft or missile is the distant element.

\item For visual sighting (rule~\ref{rule:sighting-aircraft-and-missiles}), the sighting aircraft is the reference element, and the aircraft or missile being sighted is the distant element.

\item For advantage (rule~\ref{rule:initiative}), the aircraft potentially in a superior position is the reference element, and the aircraft potentially in an inferior position is the distant element.

\item For envelopes and IRM target aspect launch requirements (rules~\ref{rule:missile-launches} and \ref{rule:irm-launch-requirements}), the target aircraft is the reference element, and the launching aircraft is the distant element.

\item When determining the field of view of sensors, the aircraft or missile sensing is the reference element, and the aircraft being sensed is the distant element. Sensors include:
\begin{itemize} 
\item VAS (rule~\ref{rule:vas}),
\item IRSTS (rule~\ref{rule:irsts}),
\item missile seekers (rule~\ref{rule:air-to-air-missiles}), and
\item air-to-air radar (rule~\ref{rule:air-to-air-radar}).
\end{itemize}


\item When determining the field of effect of jammers, the aircraft jamming is the reference element, and the radar being jammed is the distant element.

\item When determining the field of laser designators (rule~\ref{rule:laser-guided-weapons}), the designating aircraft is the reference aircraft, and the target is the distant element.

\end{itemize}


%!TEX root = ../rules-working.tex
%LTeX: enabled=false

\silentlychangedin{1B}{1B-figures}{

\begin{twocolumnfigure}
\includegraphics[width=0.3\linewidth]{figures/figure-angle-off-facing-hex-side.pdf}
\includegraphics[width=0.3\linewidth]{figures/figure-angle-off-facing-hex-corner.pdf}
\includegraphics[width=0.3\linewidth]{figures/figure-angle-off-on-hex-side.pdf}
\end{twocolumnfigure}

}{

\begin{twocolumnfigure}[tbp]

\begin{minipage}[t]{0.33\linewidth}
\begin{fitwidth}{\linewidth}
\begin{tikzpicture}
    \tiny
    %setfiguresize{-4.1}{-5*\hexxfactor-0.1}{+4.1}{+5*\hexxfactor+0.1}
    %\setfiguresize{-4.67*\hexxfactor-0.1}{-4.5-0.1}{+4.67*\hexxfactor+0.1}{+4.5+0.1}
    \setfiguresize{-5.1}{-5.1}{+5.1}{+5.1}
    \begin{scope}
       \drawevenhexgrid{-5}{-4.5}{11}{10}
        \begin{scope}[dashed,very thick,->]
            \draw (0,0) --   (0:10);
            \draw (0,0) --  (30:10);
            \draw (0,0) --  (60:10);
            \draw (0,0) --  (90:10);
            \draw (0,0) -- (120:10);
            \draw (0,0) -- (150:10);
            \draw (0,0) -- (180:10);
            \draw (0,0) -- (210:10);
            \draw (0,0) -- (240:10);
            \draw (0,0) -- (270:10);
            \draw (0,0) -- (300:10);
            \draw (0,0) -- (330:10);
        \end{scope}
        \miniathex{+0.0}{+2.0}{\draw node [rotate=90, anchor=south] {\arcline{180} line};}
        \miniathex{+0.0}{-2.0}{\draw node [rotate=90, anchor=south] {\arcline{0} line};}
        \miniathex{+1.0}{-3.5}{\draw node {\minitable{c}{Right\\\arc{30}}};}
        \miniathex{+3.0}{-2.5}{\draw node {\minitable{c}{Right\\\arc{60}}};}
        \miniathex{+4.0}{-1.0}{\draw node {\minitable{c}{Right\\\arc{90}}};}
        \miniathex{+4.0}{+1.0}{\draw node {\minitable{c}{Right\\\arc{120}}};}
        \miniathex{+3.0}{+2.5}{\draw node {\minitable{c}{Right\\\arc{150}}};}
        \miniathex{+1.0}{+3.5}{\draw node {\minitable{c}{Right\\\arc{180}}};}
        \miniathex{-1.0}{+3.5}{\draw node {\minitable{c}{Left\\\arc{180}}};}
        \miniathex{-3.0}{+2.5}{\draw node {\minitable{c}{Left\\\arc{150}}};}
        \miniathex{-4.0}{+1.0}{\draw node {\minitable{c}{Left\\\arc{120}}};}
        \miniathex{-4.0}{-1.0}{\draw node {\minitable{c}{Left\\\arc{90}}};}
        \miniathex{-3.0}{-2.5}{\draw node {\minitable{c}{Left\\\arc{60}}};}
        \miniathex{-1.0}{-3.5}{\draw node {\minitable{c}{Left\\\arc{30}}};}

        \ifaids
            \drawaircraftcounter{0.00}{0.00}{90}{MiG-21}{}{}
        \else
            \drawaircraftcounter{0.00}{+0.00}{90}{MiG-21}{A}{1}
            \drawaircraftcounter{2.00}{-1.00}{150}{F-4}{B}{1}
            \drawaircraftcounter{3.00}{-0.50}{150}{F-4}{C}{1}
        \fi
    \end{scope}
\end{tikzpicture}
\end{fitwidth}
\ifaids\else
\par\bigskip

\begin{minipage}{0.8\linewidth}
A1 is the reference aircraft.

B1 is in the right \arc{60} arc.

C1 is in the right \arc{90} arc.

\end{minipage}
\fi
\end{minipage}
\hfil
\begin{minipage}[t]{0.33\linewidth}
\begin{fitwidth}{\linewidth}
\begin{tikzpicture}
    \tiny
    \setfiguresize{-5.1}{-5.1}{+5.1}{+5.1}
    \begin{scope}[rotate=90]
        \drawevenhexgrid{-5}{-4.5}{11}{10}
        \silentlychangedin{1C}{1C-apj-23-errata}{
            \node at (0,0) [draw,fill=white] {Diagram Incorrect in Original};
        }{
            \begin{scope}[dashed,very thick,->]
                \draw (0,0) --   (0:10);
                \draw (0,0) --  (30:10);
                \draw (0,0) --  (60:10);
                \draw (0,0) --  (90:10);
                \draw (0,0) -- (120:10);
                \draw (0,0) -- (150:10);
                \draw (0,0) -- (180:10);
                \draw (0,0) -- (210:10);
                \draw (0,0) -- (240:10);
                \draw (0,0) -- (270:10);
                \draw (0,0) -- (300:10);
                \draw (0,0) -- (330:10);
            \end{scope}
            \miniathex{+2.0}{+0.0}{\draw node [rotate=90, anchor=south] {\arcline{180} line};}
            \miniathex{-2.0}{-0.0}{\draw node [rotate=90, anchor=south] {\arcline{0} line};}
            \miniathex{+4.0}{+1.0}{\draw node {\minitable{c}{Left\\\arc{180}}};}
            \miniathex{+3.0}{+2.5}{\draw node {\minitable{c}{Left\\\arc{150}}};}
            \miniathex{+1.0}{+3.5}{\draw node {\minitable{c}{Left\\\arc{120}}};}
            \miniathex{-1.0}{+3.5}{\draw node {\minitable{c}{Left\\\arc{90}}};}
            \miniathex{-3.0}{+2.5}{\draw node {\minitable{c}{Left\\\arc{60}}};}
            \miniathex{-4.0}{+1.0}{\draw node {\minitable{c}{Left\\\arc{30}}};}
            \miniathex{-4.0}{-1.0}{\draw node {\minitable{c}{Right\\\arc{30}}};}
            \miniathex{-3.0}{-2.5}{\draw node {\minitable{c}{Right\\\arc{60}}};}
            \miniathex{-1.0}{-3.5}{\draw node {\minitable{c}{Right\\\arc{90}}};}
            \miniathex{+1.0}{-3.5}{\draw node {\minitable{c}{Right\\\arc{120}}};}
            \miniathex{+3.0}{-2.5}{\draw node {\minitable{c}{Right\\\arc{150}}};}
            \miniathex{+4.0}{-1.0}{\draw node {\minitable{c}{Right\\\arc{180}}};}
            \ifaids
                \drawaircraftcounter[90]{+0.00}{+0.00}{60}{F-105}{}{}
            \else
                \drawaircraftcounter[90]{+0.00}{+0.00}{0}{F-105}{A}{2}
                \drawaircraftcounter[90]{+1.00}{-0.50}{210}{MiG-21}{B}{2}
                \drawaircraftcounter[90]{-3.00}{-0.00}{0}{F-105}{C}{2}
            \fi
        }
    \end{scope}
\end{tikzpicture}
\end{fitwidth}
\ifaids\else
\par\bigskip
\begin{minipage}{0.8\linewidth}
A2 is the reference aircraft.

B2 is in the right \arc{150} arc.

C2 is on the \arcline{0} line.

\end{minipage}
\fi
\end{minipage}
\hfil
\begin{minipage}[t]{0.33\linewidth}
\begin{fitwidth}{\linewidth}
\begin{tikzpicture}
    \tiny
    \setfiguresize{-5.1}{-5.1}{+5.1}{+5.1}
    \begin{scope}[rotate=90]
        \drawoddhexgrid{-5}{-4.0}{11}{10}
        \silentlydeletedin{2B}{2B-angle-off-on-hex-side}{
            \begin{scope}[dashed,very thick,->]
                \miniathex{+0.333}{+0.000}{\draw (0,0) -- (300:10);}
                \miniathex{+1.000}{+0.000}{\draw (0,0) -- (330:10);}
                \draw (0,0) --  (0:10);
                \miniathex{+1.000}{+0.000}{\draw (0,0) --  (30:10);}
                \miniathex{+0.333}{+0.000}{\draw (0,0) --  (60:10);}
                \draw (0,0) -- (90:10);
                \miniathex{-0.333}{+0.000}{\draw (0,0) -- (120:10);}
                \miniathex{-1.000}{+0.000}{\draw (0,0) -- (150:10);}
                \draw (0,0) -- (180:10);
                \miniathex{-1.000}{+0.000}{\draw (0,0) -- (210:10);}
                \miniathex{-0.333}{+0.000}{\draw (0,0) -- (240:10);}
                \draw (0,0) -- (270:10);
            \end{scope}
        }
        \silentlyaddedin{2B}{2B-angle-off-on-hex-side}{
            \begin{scope}[dashed,very thick,->]
                \draw (0,0) --   (0:10);
                \draw (0,0) --  (30:10);
                \draw (0,0) --  (60:10);
                \draw (0,0) --  (90:10);
                \draw (0,0) -- (120:10);
                \draw (0,0) -- (150:10);
                \draw (0,0) -- (180:10);
                \draw (0,0) -- (210:10);
                \draw (0,0) -- (240:10);
                \draw (0,0) -- (270:10);
                \draw (0,0) -- (300:10);
                \draw (0,0) -- (330:10);
            \end{scope}        
        }
        \miniathex{+2.0}{+0.0}{\draw node [rotate=90, anchor=south] {\arcline{180} line};}
        \miniathex{-2.0}{-0.0}{\draw node [rotate=90, anchor=south] {\arcline{0} line};}
        \miniathex{+4.0}{+1.0}{\draw node {\minitable{c}{Left\\\arc{180}}};}
        \miniathex{+3.0}{+2.5}{\draw node {\minitable{c}{Left\\\arc{150}}};}
        \miniathex{+1.0}{+3.5}{\draw node {\minitable{c}{Left\\\arc{120}}};}
        \miniathex{-1.0}{+3.5}{\draw node {\minitable{c}{Left\\\arc{90}}};}
        \miniathex{-3.0}{+2.5}{\draw node {\minitable{c}{Left\\\arc{60}}};}
        \miniathex{-4.0}{+1.0}{\draw node {\minitable{c}{Left\\\arc{30}}};}
        \miniathex{-4.0}{-1.0}{\draw node {\minitable{c}{Right\\\arc{30}}};}
        \miniathex{-3.0}{-2.5}{\draw node {\minitable{c}{Right\\\arc{60}}};}
        \miniathex{-1.0}{-3.5}{\draw node {\minitable{c}{Right\\\arc{90}}};}
        \miniathex{+1.0}{-3.5}{\draw node {\minitable{c}{Right\\\arc{120}}};}
        \miniathex{+3.0}{-2.5}{\draw node {\minitable{c}{Right\\\arc{150}}};}
        \miniathex{+4.0}{-1.0}{\draw node {\minitable{c}{Right\\\arc{180}}};}
        \ifaids
            \drawaircraftcounter[90]{+0.0}{+0.00}{0}{MiG-21}{}{}
        \else
            \drawaircraftcounter[90]{+0.00}{+0.00}{0}{MiG-21}{A}{3}
            \drawaircraftcounter[90]{+0.00}{+1.50}{300}{F-4}{B}{3}
            \drawaircraftcounter[90]{-1.00}{+0.00}{30}{F-4}{C}{3}
        \fi
    \end{scope}
\end{tikzpicture}
\end{fitwidth}

\notein{2B}{New angle-off figure for 2B-angle-off-on-hex-side.}

\ifaids\else

\par\bigskip

\begin{minipage}{0.8\linewidth}
A3 is the reference aircraft.

B3 is in the left \arc{90} arc.

C3 is in the right \arc{30} arc (if it was facing A it would be on the \arcline{0} line).

\end{minipage}

\fi
\end{minipage}

\figurecaption{figure:angle-off}{Horizontal arcs.}

\end{twocolumnfigure}
}




%!TEX root = ../rules-working.tex
%LTeX: enabled=false

\begin{twocolumntablefloat}

\begin{twocolumntable}

\tablecaption{table:angle-off-borderlines}{Horizontal Arc Borderline Summary.}
\small
\begin{tabularx}{0.7\linewidth}{lLl}
\toprule
Context&First&Second\\
\midrule
Sensors&Included in the field.&---\\
Jammers&Included in the field.&---\\
Laser Designators&Included in the field.&---\\
Gun and Rocket Attacks&Move the faster element forward.&Use the narrower arc.\\
Blind and Restricted Arcs&Move the faster element forward.&Use the rearward arc.\\
Superior Position&Move the faster element forward.&Use the rearward arc.\\
Missile Launch Envelopes&Move the faster element forward.&Use the wider arc.\\
IRM Launch Target Aspect&Move the faster element forward.&Use the wider arc.\\
Missile Attacks&Move the faster element forward.&Use the wider arc.\\
\bottomrule
\end{tabularx}


\end{twocolumntable}

\vspace{\floatsep}

\begin{twocolumntable}

\tablecaption{table:angle-off-adjacent}{Wider, Narrower, and Rearward Horizontal Arcs.}

\small
\begin{tabularx}{0.7\linewidth}{CCCC}
\toprule
Borderline&Narrower Arc&Wider Arc&Rearward Arc\\
\midrule
\arcline{0}                                &\phantom{0}\arc{30}&\phantom{0}\arc{30}&\phantom{0}\arc{30}\\
\phantom{0}\arc{30} and \arc{60}\phantom{0}&\phantom{0}\arc{30}&\phantom{0}\arc{60}&\phantom{0}\arc{30}\\
\phantom{0}\arc{60} and \arc{90}\phantom{0}&\phantom{0}\arc{60}&\phantom{0}\arc{90}&\phantom{0}\arc{60}\\
\phantom{0}\arc{90} and \arc{120}\phantom{}&\phantom{0}\arc{90}&\phantom{}\arc{120}&\phantom{0}\arc{90}\\
\phantom{}\arc{120} and \arc{150}\phantom{}&\phantom{}\arc{150}&\phantom{}\arc{120}&\phantom{}\arc{120}\\
\phantom{}\arc{150} and \arc{180}\phantom{}&\phantom{}\arc{180}&\phantom{}\arc{150}&\phantom{}\arc{150}\\
\arcline{180}                              &\phantom{}\arc{180}&\phantom{}\arc{180}&\phantom{}\arc{180}\\
\bottomrule
\end{tabularx}

\end{twocolumntable}


\end{twocolumntablefloat}


\paragraph{Angle-Off.} 

The relative horizontal position of the distant element with respect to the reference element is quantified by the horizontal angle between two imaginary lines, one extending behind the tail of the reference element and the other between the reference element and the distant element. This angle is known as the \emph{angle off the tail} or simply the \emph{angle-off}. 

If the distant element is directly behind the reference element, the angle-off is \arc{0}; if it is directly on the beam, the angle-off is \arc{90}; and if it is directly in front, the angle-off is \arc{180}.

\paragraph{Angle-Off Arcs.} 

Angle-off is grouped into \emph{arcs} covering \arc{30} to the left or right of the target. Figure~\ref{figure:angle-off} shows these arcs. The reference element is at the center, and the arcs radiate from it. The appropriate figure is used according to whether the reference element is in a hex facing a hex corner, in a hex facing a hex side, or on a hex side.

\paragraph{Borderline Procedure.} 

The distant element will often unambiguously be in an angle-off arc. However, sometimes it will be located on the borderline between two arcs. In these cases, use the disambiguation procedure described below and summarized in Tables \ref{table:angle-off-borderlines} and \ref{table:angle-off-adjacent}.

In the context of the fields of sensors, jammers, and laser designators, if either or both of the adjacent arcs are within the field, the borderline is also considered to be within the field. (In other words, the fields include the borderlines.)

In other contexts, if the distant element is on the border between the two \arc{0} arcs or the two \arc{180} arcs, it is considered to be in the \arc{0} arc or \arc{180} arc. This rule does not specify whether it falls in the left or right arcs, but this is never relevant to the game rules. 

% TODO: How do we disambiguate missile envelopes and IRM launch requirements? Included would work for the latter, but not for the former.

For the other borderlines, the procedure for determining the arc is more complex. Consider the speeds of the reference element and distant element at the start of the game turn:
\begin{itemize}
\item If the reference element is slower, the distant element is in the arc it would move into if it moved forward.
\item If the reference element is faster, the distant element is in the arc it would move into if the reference element moved forward.
\item If neither is faster or if the distant element remains on the borderline after one of the previous two cases (i.e., it is facing directly towards or away from the reference element), the resolution again depends on the context:
\begin{itemize}
\item For gun and rocket attacks (rule~\ref{rule:air-to-air-gun-combat}), the distant element (attacking aircraft) is considered to be in the narrower arc.
\item For missile attacks (rule~\ref{rule:missile-attacks}), the distant element (attacking missile) is considered to be in the wider arc.
\item For determining blind and restricted arcs (rule \ref{rule:sighting-aircraft-and-missiles}), the distant element (element being sighted) is considered to be in the rearward arc.
\item For determining advantage purposes (rule~\ref{rule:initiative}), the distant element (element potentially disadvantaged) is considered to be in the rearward arc.
\end{itemize}
\end{itemize}

The narrower arc is the arc further from the \arc{120} arc. The wider arc is the arc closer to the \arc{120} arc. The rearward arc is the arc closer to the \arc{30} arc.

Ground elements are always considered to be slower than any aircraft.

\paragraph{Same-Location Procedure}
If the reference and distant element are at the same map location (i.e., the same hex or hex side), the angle-off arc is the same as if the distant element (i.e., the non-reference element) were moved backward one hex.

Moving the distant element backward would always leave it on a borderline facing the reference element, requiring the use of the disambiguation procedure for borderlines described above.

\paragraph{Angle-Off Lines.} The imaginary lines extending behind and ahead of the reference element are the \arc{0} and \arc{180} lines, respectively. Although not formally arcs, these lines are, at times, treated as such. To count as being on one of these lines, the distant element must be on the borderline and facing directly along it towards the reference element. If they are on the line but not facing directly along it towards the target, they are considered in either the \arc{30} or \arc{180} arc but not on the \arc{0} or \arc{180} lines.

\paragraph{Angle-Off Examples.}

Consider Figure~\ref{figure:angle-off}, in which the aircraft A1, A2, and A3 are the reference elements and aircraft B1, B2,  B3, C1, C2, and C3 are distant elements. 

Aircraft C2 is unambiguously on the \arc{0} line, and C1 is unambiguously in the right \arc{90} arc, but the other attackers are on borderlines between two arcs. Aircraft C3 cannot be on the \arc{0} line as it is not facing the target and so is in the \arc{30} arc.

\begin{itemize}

\item
In the context of the field sensors, jammers, or laser designators, distant elements on borderlines are considered to be within the field if either or both adjacent arcs are within the field. B1 would be within the field only if either the right \arc{60} arc, the right \arc{90} arc, or both were within the field. B2 would be within the field only if either the right \arc{150} arc, the right \arc{180} arc, or both were within the field. B3 would be within the field only if either the left \arc{90} arc, the left \arc{120} arc, or both were within the field.

\item In other contexts, we need to consider the relative speeds of the reference and distant elements:

\begin{itemize}
\item
If the reference elements were slower than the distant elements, then B2 and B3 would be considered to be in the arcs into which they move if they moved forward: B2 would be in the right \arc{150} arc, and B3 would be in the left \arc{120} arc.

\item
If the reference elements were faster than the distant elements, then B2 and B3 would be considered to be in the arcs into which they move if the reference element moved forward: B2 would be in the right \arc{150} arc and B3 would be in the left \arc{90} arc.

\item
If the reference element and the distant aircraft had the same speeds, then the context would determine whether B2 and B3 are considered to be in the narrower, wider, or rearward arc. The same procedure would be used for B1 regardless of the relative speeds since B1 is facing along a borderline.

\begin{itemize}

\item For gun and rocket attacks, the distant elements are in the narrower arc, so B1 would be in the right \arc{90} arc, B2 in the right \arc{180} arc, and B3 in the right \arc{90} arc.

\item For missile attacks (for which the distant elements would be missiles), the distant elements are in the wider arc, so B1 would be in the right \arc{120} arc, B2 in the right \arc{150} arc, and B3 in the right \arc{120} arc.

\item For visual sighting and advantage, the distant elements are in the rearward arc, so B1 would be in the right \arc{90} arc, B2 in the right \arc{150} arc, and B3 in the right \arc{120} arc.

\end{itemize}

\end{itemize}

\end{itemize}


\paragraph{Ranges of Angle-Off Arcs.}
\label{rule:ranges-of-angle-off-arcs}

Ranges of angle-off arcs are used for restricted or blind regions for sighting, sensor fields of view, and determining advantage. If a range is given as an arc followed by a plus sign, it consists of those arcs and all more forward arcs. If a range is given as an arc followed by a minus sign, it consists of those arcs and all more rearward arcs. Ranges specified in this way consist of both the left and right arcs.

For example, a blind arc given as $\arcrange{60}{-}$ consists the \arc{30} and \arc{60} arcs and a radar search field given as $\arcrange{150}{+}$ consists of the \arc{150} and \arc{180} arcs. 

\paragraph{Lower Angle-Off Arcs.}
\label{rule:lower-angle-off-arcs}

Restricted or blind regions for sighting are sometimes given as \arc{30}L, \arc{60}L, or \arc{180}L. These refer to the \arcrange{30}{-}, \arcrange{60}{-}, or \arcrange{180}{+} arc ranges but only apply to distant elements at lower altitudes.

For example, a restricted arc given as \arc{180}{L} refers to the \arc{180} arcs but only for distant elements at lower altitudes and might apply to an aircraft whose nose obscures the view from the cockpit forward and downwards.

}

\AY[3A-limited-arcs]{

\paragraph{Limited Arcs.}
\label{rule:limited-arcs}

\begin{twocolumnfigure}[tbp]

% These are arcs derived directly from the TSOH play aids.

\newcommand{\drawlimitedarcA}[1][]{   
    \draw [yscale=\hexxfactor,#1]
        (-1.600,20.000) --
        (-1.600,10.000) --
        (-1.100, 9.000) --
        (-1.100, 5.000) --
        (-0.600, 4.000) --
        (-0.600, 2.000) --
        (-0.000, 0.333) -- 
        (+0.600, 2.000) --
        (+0.600, 4.000) --
        (+1.100, 5.000) --
        (+1.100, 9.000) --
        (+1.600,10.000) --
        (+1.600,20.000);   
    \draw[yscale=\hexxfactor, <->, transform shape]
        (-1.6,10.5) -- 
        (0,10.5) node [anchor=south] {\minitable{c}{maximum\\width}} --
        (+1.6,10.5);
}
\newcommand{\drawlimitedarcB}[1][]{  
    \draw [yscale=\hexxfactor,#1]
        (-1.6,20.000) --
        (-1.6,11.000) --
        (-1.1,10.000) --
        (-1.1, 6.000) --
        (-0.6, 5.000) --
        (-0.6, 2.000) --
        (+0.0, 0.333) -- 
        (+0.6, 2.000) --
        (+0.6, 5.000) --
        (+1.1, 6.000) --
        (+1.1,10.000) --
        (+1.6,11.000) --
        (+1.6,20.000);  
    \draw[yscale=\hexxfactor, <->, transform shape]
        (-1.6,11.5) -- 
        (0,11.5) node [anchor=south] {\minitable{c}{maximum\\width}} --
        (+1.6,11.5);
}
\newcommand{\drawlimitedarcC}[1][]{  
    \draw [xscale=\hexxfactor,#1]
        (-1.6,20.000) --
        (-1.6, 9.250) --
        (-1.1, 8.500) --
        (-1.1, 5.000) --
        (-0.6, 4.250) --
        (-0.6, 1.250) --
        (-0.0, 0.333) -- 
        (+0.6, 1.250) --
        (+0.6, 4.250) --
        (+1.1, 5.000) --
        (+1.1, 8.500) --
        (+1.6, 9.250) --
        (+1.6,20.000);   
    \draw[xscale=\hexxfactor, <->, transform shape]
        (-1.6,9.75) -- 
        (0,9.75) node [anchor=south] {\minitable{c}{maximum\\width}} --
        (+1.6,9.75);
}

% This figure shows them in the same orientation as in the TSOH play aids.
%\begin{tikzfigure}{0.5\linewidth}
%\begin{scope}[rotate=0]
%    \drawdottedhexgrid{15.0}{15.5}
%    
%    \begin{athex}{1.00}{13.00}
%        \begin{scope}[rotate=-90,thick]
%            \drawlimitedarcA
%        \end{scope}
%        \drawaircraftcounter{0.00}{0.00}{90}{F-4}{}{}
%    \end{athex}
%
%    \begin{athex}{1.00}{8.50}
%        \begin{scope}[rotate=-90,thick,]
%            \drawlimitedarcB
%        \end{scope}
%        \drawaircraftcounter{0.00}{0.00}{90}{F-4}{}{}
%    \end{athex}
%
%    \begin{athex}{1.00}{7.50}
%        \begin{scope}[rotate=-120,thick]
%            \drawlimitedarcC
%        \end{scope}
%        \drawaircraftcounter{0.00}{0.00}{60}{F-4}{}{}
%    \end{athex}
%    
%\end{scope}
%\end{tikzfigure}

\begin{tikzfigure}{0.6\linewidth}
\begin{scope}[rotate=0]

    \drawhexgrid{0}{0}{16}{12}
    \drawpositiongrid{0}{0}{20}{12}
    
    \begin{athex}{7.50}{0.25}
        \begin{scope}[rotate=-30,thick]
            \drawlimitedarcA
        \end{scope}
        \drawaircraftcounter{0.00}{0.00}{60}{F-4}{}{}
    \end{athex}

    \begin{athex}{4.00}{2.00}
        \begin{scope}[rotate=-30,thick]
            \drawlimitedarcB
        \end{scope}
        \drawaircraftcounter{0.00}{0.00}{60}{F-4}{}{}
    \end{athex}

    \begin{athex}{2.00}{0.00}
        \begin{scope}[rotate=0,thick]
            \drawlimitedarcC
        \end{scope}
        \drawaircraftcounter{0.00}{0.00}{90}{F-4}{}{}
    \end{athex}
    
\end{scope}
\end{tikzfigure}

\figurecaption{figure:limited-arcs}{Limited Arcs}

\end{twocolumnfigure}



Some sensors and rules make use of a \emph{limited arc} that is even narrower than the \arcrange{180}{+} arc. Figure~\ref{figure:limited-arcs} shows the limited arc. In this figure, only the map locations within the limited arc line are within the limited arc. 

}

\AY[3A-vertical-arcs]{

\section{Combined Arcs}
\label{rule:vertical-arcs}
\label{rule:vertical-limits}


Aircraft and missile sensors have fields that are limited not just horizontally but also vertically. The \emph{combined arcs} described here extend the horizontal arcs above to include appropriate vertical limits.

Combined arcs are used to determine if a target is in the field of:
\begin{itemize} 
\item VAS (rule~\ref{rule:vas}),
\item IRSTS (rule~\ref{rule:irsts}),
\item missile seekers (rule~\ref{rule:air-to-air-missiles}), 
\item air-to-air radar (rule~\ref{rule:air-to-air-radar}), and
\item laser designators (rule~\ref{rule:laser-guided-weapons}).
\end{itemize}
They are also used to determine if the target satisfies the aspect requirements for IRM launches (rule~\ref{rule:irm-launch-requirements}). They are not used elsewhere.

\paragraph{Flight Slope.}

The \emph{flight slope} characterizes the vertical attitude of the reference element just as its facing characterizes is horizontal aspect.

The flight slope is calculated over a number of FPs as the number of altitude levels gained divided by the number of HFPs expended. 

The flight slope will be positive if the element gains altitude, zero if it maintains altitude, and negative if it loses altitude. For purely vertical flight with no HFPs, the flight slope will be positive infinity ($+\infty$) for climbing or negative infinity ($-\infty$) for diving.

For aircraft and missile prior to launch, determine the flight slope by considering all of the FPs expended by the aircraft in the previous flight phase. 

For missiles in flight, if the previous proportional move had two or more FPs, determine the flight slope by considering them all. Otherwise, if the previous proportional more only had one FP and was the first one of the missile’s flight, determine the flight slope from that FP and the last FP of the aircraft before launch. Otherwise, determine the flight slope from the missile’s last two FPs.

For example, if an aircraft executes level flight with four HFPs and no free descent, its flight slope is 0. If it executes the same but with free descent, its flight slope is $-1/4 = -0.25$. If it executes a zoom climb with three HFPs and four VFPs, gaining seven altitude levels, its flight slope is $+7/3 \approx 2.33$. If it executes a vertical dive with no HFPs and six VFPs, losing twelve altitude levels, its flight slope is $-12/0 = -\infty$.

\paragraph{Combined Arcs Procedure.}

%!TEX root = ../rules-working.tex
%LTeX: enabled=false

\begin{twocolumntablefloat}
\begin{twocolumntable}

\tablecaption{table:vertical-limit-factors}{Vertical Limit Factors.}

\footnotesize
\begin{tabularx}{0.9\linewidth}{r@{ FS }lR@{ }c@{ }LR@{ }c@{ }LR@{ }c@{ }LR@{ }c@{ }Lr@{ FS }l}
\toprule
\multicolumn{2}{c}{\minitable{c}{Flight Slope (FS)\\for Forward Arcs}}&
\multicolumn{3}{c}{Limited}&
\multicolumn{3}{c}{\arcplus{180} and \arcminus{30}}&
\multicolumn{3}{c}{\arcplus{150} and \arcminus{60}}&
\multicolumn{3}{c}{\arcplus{120} and \arcminus{90}}&
\multicolumn{2}{c}{\minitable{c}{Flight Slope (FS)\\for Rearward Arcs}}\\
\midrule
           &= \plus{\textinfinity}&
    \multicolumn{3}{c}{\greaterthanorequaltobefore{(\plus{7})}}  &
    \multicolumn{3}{c}{\greaterthanorequaltobefore{(\plus{3})}}  &
    \multicolumn{3}{c}{\greaterthanorequaltobefore{(\plus{2})}}  &
    \multicolumn{3}{c}{\greaterthanorequaltobefore{(\plus{1})}}  &
            &= \minus{\textinfinity}\\
\plus{3} <     &\lessthanbefore{\plus{\textinfinity}}&
    \multicolumn{3}{c}{\greaterthanorequaltobefore{\plus{3.0}}}      &
    \greaterthanorequaltobefore{\plus{2.0}}&or &\greaterthanorequaltobefore{(\plus{7})}            &
    \greaterthanorequaltobefore{\plus{1.0}}&or &\greaterthanorequaltobefore{(\plus{3})}            &
    \greaterthanorequaltobefore{\plus{0.5}}&or &\greaterthanorequaltobefore{(\plus{2})}            &
\minus{3} >     &\greaterthanbefore{\minus{\textinfinity}}\\
\plus{1} <     &\lessthanorequaltobefore{\plus{3}}   &
    \greaterthanorequaltobefore{\plus{1.0}}&and&\lessthanorequaltobefore{\plus{3.0}}            &
    \greaterthanorequaltobefore{\plus{0.5}}&and&\lessthanorequaltobefore{\plus{7.0}}            &
    \multicolumn{3}{c}{\greaterthanorequaltobefore{\plus{0.0}}}      &
    \greaterthanorequaltobefore{\minus{0.5}}&or&\greaterthanorequaltobefore{(\plus{7})}             &
\minus{1 >}     &\greaterthanorequaltobefore{\minus{3}}   \\
0 <      &\lessthanorequaltobefore{\plus{1}}   &
    \greaterthanorequaltobefore{\plus{0.0}}&and&\lessthanorequaltobefore{\plus{1.0}}            &
    \greaterthanorequaltobefore{\minus{0.5}}&and&\lessthanorequaltobefore{\plus{2.0}}            &
    \greaterthanorequaltobefore{\minus{1.0}}&and&\lessthanorequaltobefore{\plus{3.0}}            &
    \greaterthanorequaltobefore{\minus{2.0}}&and&\lessthanorequaltobefore{\plus{7.0}}            &
0 >       &\greaterthanorequaltobefore{\minus{1}}  \\
           &= 0      &
    \greaterthanorequaltobefore{\minus{0.5}}&and&\lessthanorequaltobefore{\plus{0.5}}            &
    \greaterthanorequaltobefore{\minus{1.0}}&and&\lessthanorequaltobefore{\plus{1.0}}            &
    \greaterthanorequaltobefore{\minus{2.0}}&and&\lessthanorequaltobefore{\plus{2.0}}            &
    \greaterthanorequaltobefore{\minus{3.0}}&and&\lessthanorequaltobefore{\plus{3.0}}            &
           &= 0      \\
0 >       &\greaterthanorequaltobefore{\minus{1}}       &
    \lessthanorequaltobefore{\plus{0.0}}&and&\greaterthanorequaltobefore{\minus{1.0}}            &
    \lessthanorequaltobefore{\plus{0.5}}&and&\greaterthanorequaltobefore{\minus{2.0}}            &
    \lessthanorequaltobefore{\plus{1.0}}&and&\greaterthanorequaltobefore{\minus{3.0}}            &
    \lessthanorequaltobefore{\plus{2.0}}&and&\greaterthanorequaltobefore{\minus{7.0}}            &
0 <      &\lessthanorequaltobefore{\plus{1}}   \\
\minus{1 >}     &\greaterthanorequaltobefore{\minus{3}}      &
    \lessthanorequaltobefore{\minus{1.0}}&and&\greaterthanorequaltobefore{\minus{3.0}}            &
    \lessthanorequaltobefore{\minus{0.5}}&and&\greaterthanorequaltobefore{\minus{7.0}}            &
    \multicolumn{3}{c}{\lessthanorequaltobefore{\plus{0.0}}}      &
    \lessthanorequaltobefore{\plus{0.5}}&or &\lessthanorequaltobefore{(\minus{7})}            &
\plus{1} <     &\lessthanorequaltobefore{\plus{3}}   \\
\minus{3 >}&>\minus{\textinfinity}      &
    \multicolumn{3}{c}{\lessthanorequaltobefore{\minus{3.0}}}      &
    \lessthanorequaltobefore{\minus{2.0}}&or &\lessthanorequaltobefore{(\minus{7})}            &
    \lessthanorequaltobefore{\minus{1.0}}&or &\lessthanorequaltobefore{(\minus{3})}            &
    \lessthanorequaltobefore{\minus{0.5}}&or &\lessthanorequaltobefore{(\minus{2})}            &
\plus{3} <     &\lessthanbefore{\plus{\textinfinity}}\\
           &= \minus{\textinfinity}&
    \multicolumn{3}{c}{\lessthanorequaltobefore{(\minus{7})}}  &
    \multicolumn{3}{c}{\lessthanorequaltobefore{(\minus{3})}}  &
    \multicolumn{3}{c}{\lessthanorequaltobefore{(\minus{2})}}  &
    \multicolumn{3}{c}{\lessthanorequaltobefore{(\minus{1})}}  &
            &= \plus{\textinfinity}\\
\bottomrule
\end{tabularx}

\smallskip

\begin{tablenote}{0.9\linewidth}\footnotesize
\begin{itemize}

\item
Flight slope (FS) = altitude levels gained divided by HFPs.

\item Forward arcs are the limited, \arcplus{180}, \arcplus{150}, and \arcplus{120} arcs. Rearward arcs are the \arcminus{30},  \arcminus{60}, and  \arcminus{90} arcs. 

\item
Multiply the vertical limit factors by the horizontal range to determine the allowed altitude difference. After multiplication, lower limits (≥) are rounded up and upper limits (≤) are rounded down. 

\item
Factors without parentheses are mixed and require the distant element to also be within the corresponding horizontal arc. Factors with parentheses are pure and do not require the distant element to be with the corresponding horizontal arc. 

\end{itemize}
\end{tablenote}


\end{twocolumntable}
\end{twocolumntablefloat}


%\begin{itemize}
%
%\item For a pure vertical climb the flight slope is $+\infty$ and Table~X gives a single positive vertical limit factor in parentheses. Multiply it by the horizontal range to the distant element and add the altitude of the reference element to determine the lower altitude limit.
%
%The distant element is within the combined arc if:
%\begin{itemize}
%\item Its altitude is greater than or equal to the lower altitude limit.
%\end{itemize}
%There is no requirement on the horizontal arc; the distant element can be in any horizontal arc since for pure vertical flight the interpretation of facing is ambiguous.
%
%\item For a pure vertical dive the flight slope is $-\infty$ and Table~X gives a single negative vertical limit factor in parentheses. Multiply it by the horizontal range to the distant element and add the altitude of the reference element to determine the upper altitude limit.
%
%The distant element is within the combined arc if:
%\begin{itemize}
%\item Its altitude is less than or equal to the lower altitude limit.
%\end{itemize}
%Again, there is no requirement on the horizontal arc.
%
%\item If Table~X gives two vertical limit factors that are not in parentheses, then the first is the lower limit factor and the second is the upper limit factor. Multiply both of them by the horizontal range to the distant element to determine the lower and upper altitude limits. Round up the lower altitude limit and round down the upper altitude limit. 
%
%The distant element is within the combined arc if:
%\begin{itemize}
%\item It is within the corresponding horizontal arc,
%\item The altitude difference is greater than or equal to the lower altitude limit, and
%\item The altitude difference is less than or equal to the upper altitude limit.
%\end{itemize}
%
%\item If Table~X gives two vertical limit factors and the second is in parentheses, then the first is the normal lower limit factor and the second is the all-round lower limit factor. Multiply both of them by the horizontal range to the distant element to determine the normal lower and all-round lower altitude limits. Round up the normal lower altitude limit.
%
%The distant element is within the combined arc if either:
%\begin{itemize}
%\item It is within the corresponding horizontal arc, and
%\item Its altitude is greater than or equal to the normal lower altitude limit,
%\end{itemize}
%or
%\begin{itemize}
%\item Its altitude is greater than or equal to the all-round lower limit.
%\end{itemize}
%In the second case, there is no requirement on the horizontal arc.
%
%\item If Table~X gives two vertical limit factors and the first is in parentheses, then  the first is the all-round upper limit factor and the second is the normal upper limit factor. Multiply both of them by the horizontal range to the distant element to determine the normal upper and all-round upper altitude limits. Round down the normal upper altitude limit.
%
%The distant element is within the combined arc if either:
%\begin{itemize}
%\item It is within the corresponding horizontal arc, and
%\item Its altitude is less than or equal to the normal upper altitude limit,
%\end{itemize}
%or
%\begin{itemize}
%\item Its altitude is less than or equal to the all-round upper limit.
%\end{itemize}
%In the second case, there is no requirement on the horizontal arc.
%
%
%\end{itemize}

Cross-reference the arc and the flight slope in Table~\ref{table:vertical-limit-factors}. The entries in the table are either one or two \emph{vertical limit factors}. The factors themselves are either in parentheses or not in parentheses.

\begin{itemize}

\item
The factors not in parentheses are \emph{mixed} vertical limit factors. They are multiplied by the horizontal range from the reference to the distant element, rounded up if lower limits and rounded down if upper limits, and then give the \emph{maximum} number of levels that the distant aircraft can be below (negative factor) or above (positive factor) the reference aircraft and still be within the combined arc. 

In these cases, the distant aircraft must also be in the corresponding horizontal arc to be within the combined arc.

\item
The factors in parentheses are \emph{pure} vertical limit factors. They are multiplied by the horizontal range from the reference to the distant element, and then give the \emph{minimum} number of levels that the distant aircraft can be above (positive factor) or below (negative factor) the reference aircraft and still be in the combined arc. 

In these cases, the distant aircraft \emph{need not} be in the corresponding horizontal arc; it can be in \emph{any} horizontal arc.

\end{itemize}

To determine if the distant element is within the field:

\begin{itemize}
\item 
If Table~\ref{table:vertical-limit-factors} gives a single pure vertical limit factor, this will be a lower limit (positive factor and pure vertical climb) or upper limit (negative factor and pure vertical dive). The distant aircraft is within the field if its altitude satisfies the corresponding limit.

\item 
If Table~\ref{table:vertical-limit-factors} gives two mixed factors, the first is for the lower limit and the second for the upper limit. The distant aircraft is within the field if its altitude satisfies \emph{both} corresponding limits.

\item 
If Table~\ref{table:vertical-limit-factors} gives a mixed factor followed by a pure factor, both are lower limits. The distant aircraft is in the combined arc if it satisfies \emph{either} of the corresponding limits.

\item 
If Table~\ref{table:vertical-limit-factors} gives a pure factor followed by a mixed factor, both are upper limits. The distant aircraft is in the combined arc if it satisfies \emph{either} of the corresponding limits.

\end{itemize}

For example, consider an aircraft at an altitude of 10 attempting to detect another at a horizontal range of 5 hexes using radar with an \arcrange{180}{+} arc.
\begin{itemize}
\item
If the aircraft has a flight slope of 0, Table~\ref{table:vertical-limit-factors} gives two mixed factors (without parentheses) of $-1.0$ and $+1.0$. 

Applying them, we see that the target is within the field of the radar if it has an altitude of 5 to 15 inclusive and is within the \arcrange{180}{+} horizontal arc.

\item
If the aircraft has a flight slope of more than 0 but no more than 1 ($0 < \mbox{FS} \le 1$), Table~\ref{table:vertical-limit-factors} gives two mixed factors (without parentheses) of $-0.5$ and $+2.0$. 

Applying them, we see that the target is within the field of the radar if it has an altitude of 8 to 20 inclusive and is within the \arcrange{180}{+} horizontal arc.

\item
If the aircraft has a flight slope of more than 1 but no more than 3 ($1 < \mbox{FS} \le 3$), Table~\ref{table:vertical-limit-factors} gives two mixed factors (without parentheses) of $+0.5$ and $+7.0$. 

Applying them, we see that the target is within the field of the radar if it has an altitude of 13 to 45 inclusive and is within the \arcrange{180}{+} horizontal arc.

\item
If the aircraft has a flight slope of more than 3 but that is still finite ($3 < \mbox{FS} < +\infty$), Table~\ref{table:vertical-limit-factors} gives one mixed factor (without parentheses) of $+2.0$ and one pure factor (with parentheses) of $(+7)$.

Applying them, we see that the target is within the field of the radar if it either has an altitude of 20 or more and is within the \arcrange{180}{+} horizontal arc or if it has an altitude of at least 45 and is at any horizontal position.

\item
If the aircraft in a pure vertical climb ($\mbox{FS} = +\infty$), Table~\ref{table:vertical-limit-factors} gives only one pure factor (with parentheses) of $(+3)$.

Applying it, we see that target is within the field of the radar if it has an altitude of at least 25 and is at any horizontal position.

\end{itemize}
}


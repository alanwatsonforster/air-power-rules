\rulechapter{Visual Sighting}

\CX{
This chapter details the procedures for visual detection and tracking of targets. \addedin{1C}{1C-tables}{The rules are summarized in Table~\ref{table:sighting-rules-summary}}

}{
This rule details the procedures for visual detection and identification of targets.
}

\section{Sighting of Aircraft and Missiles by Aircraft}
\label{rule:sighting-aircraft-and-missiles}

\CX{

\paragraph{When is Sighting Checked?}  Enemy aircraft and missiles in flight are determined to be sighted or not during the Visual Sighting Phase of every game-turn. Sighting probability is based on range and visibility numbers. Visually aimed weapons may only be fired at sighted aircraft targets.
}{
Aircraft and missiles can be visually sighted or unsighted by the opposing aircraft. This is important for many reasons. For example, aircraft may only fire visually aimed weapons, including guns and infrared-homing missiles, at sighted aircraft and may only defensively engage sighted missiles.

\paragraph{Actions in the Visual Sighting Phase.} 
Each side first declares padlocks of currently sighted opposing aircraft and missiles. Any opposing aircraft or missile that is not padlocked is considered to become unsighted. Each side can then attempt to sight unsighted opposing aircraft and missiles. At the end of the visual sighting phase, the only opposing aircraft and missiles that are sighted are those that were padlocked or for which the sighting attempt succeeded. All other opposing aircraft and missiles are unsighted.

\paragraph{Duration of Sighting.}
An aircraft's or missile's status as sighted or unsighted is determined anew in each game turn's visual sighting phase. It then maintains that status until the following game turn's visual sighting phase. A sighted aircraft or missile that enters a blind arc does not immediately cease to be sighted; it remains sighted at least until the following game turn's visual sighting phase.

\paragraph{Sighted or Unsighted by All.}
If an opposing aircraft or missile is unsighted, it is unsighted by all friendly aircraft. If it is sighted, it is sighted by all of the friendly aircraft, even those that have it in their blind arcs. For most purposes, this is sufficient. However, for aircraft carrying out attacks with visually aimed weapons, using defensive preemptions, and defensively engaging missiles, there are additional requirements on individual sighting described below.

}

\CX{

\addedin{1C}{1C-tables}{
    \begin{table*}
\centering
\caption{Relative Range Effects}
\medskip
\begin{tabular}{lcccccccc}
\hline
Visual Range in Hexes&0--3&4--6&7--9&10--12&13--15&16--20&21--30&31+\\
Die Roll Modifier&$-2$&$-1$&$+0$&$+1$&$+2$&$+3$&$+5$&$+8$\\
\hline
\\
\hline
V.A.S. Range in Hexes&0--20&21--30&31--40&41--50&51--60&61--70&71--80&81+\\
Die Roll Modifier&$+0$&$+1$&$+2$&$+3$&$+4$&$+5$&$+6$&$+8$\\
\hline
\tablemedskip
\tablenotes{9}{0.75\linewidth}{\addedin{1B}{1B:apj-23-errata}{When looking for higher targets in daytime, count each 4 levels difference of altitude as 1 hex of range.}}
\end{tabular}
\end{table*}

\begin{table*}
\centering
\caption{Paint Scheme / Position / Weather Effects}
\medskip
\begin{tabular}{lcccccccc}
\hline
Target Aircraft Position&Lower&Level&Higher&In Haze&In Stratus&Silh,\ by Cloud\\
\hline
Silver          &$-2$&$-1$&$-1$&$+1$&$+3$&$-1$\\
Uncamouflaged   &$-1$&$+0$&$+0$&$+1$&$+3$&$-1$\\
Camouflaged     &$+1$&$+0$&$-1$&$+0$&$+3$&$-2$\\
Low Vis.\ Grey  &$+0$&$+1$&$+1$&$+2$&$+3$&$-1$\\
Aircraft Smoking&$-1$&$-2$&$-2$&NA&NA&NA\\
\hline
\end{tabular}
\end{table*}

\begin{table}
\centering
\caption{Addtitional Modifiers}
\medskip
\begin{tabular}{p{18em}l}
\hline
Number of aircraft searching:\\
1 or 2&$+0$\\
3 or 4&$-1$\\
5 or 8&$-2$\\
9+    &$-3$\\
\changedin{2A}{2A:missile-sighting}{Tgt.\ is just launched Missile or SAM$^*$}{Target is a missile with a normal boost or sustainer motor burning}&$-3$\\
\tablerowaddedin{2A}{2A:missile-sighting}{\addedin{2A}{2A:missile-sighting}{Target is a missile with a smokeless boost or sustainer motor burning}&\addedin{2A}{2A:missile-sighting}{$-1$}}
\tablerowaddedin{2A}{2A:missile-sighting}{\addedin{2A}{2A:missile-sighting}{Target is a missile just launched from a spotted aircraft}&\addedin{2A}{2A:missile-sighting}{$-2$}}
\changedin{2A}{2A:missile-sighting}{Tgt.\ is aircraft which just launched Missile$^*$}{Target just launched a missile that was spotted}&\changedin{2A}{2A:missile-sighting}{$-3$}{$-2$}\\
Tgt.\ is in all searcher's Restricted Arcs&$+2$\\
Searcher is Looking out of Stratus&$+2$\\
Searcher is Veteran and Tactics Master&$-2$\\
Searcher is Veteran or Tactics Master&$-1$\\
Multi-crew aircraft Searching&$-1$\\
\x{AWF as the only reference to the HUD modifier in the rules is in the context of an IRSTS lock-on}{\changedin{1B}{1B:apj-25-qa and 1B:apj-34-qa}{Hud Interface Technology used}{Searcher has HUD Interface Technology}}{Searcher has IRSTS lock-on and HUD interface technology}&$-1$\\
Searcher has RWR indications&$-1$\\
Has Poor Eyes&$+1$\\
Has Good Eyes&$-1$\\
is Novice&$+1$\\
is Green&$+2$\\
Target using DDS Flare program&$+$PPL No.\\
\hline
\tablemedskip
\tablenotes{2}{0.9\linewidth}{
\deletedin{2A}{2A:missile-sighting}{$^*$ does not apply to smokeless missiles.}

NOTE: All modifiers are cumulative. Disoriented or GLOC aircrew may not search.
}
\end{tabular}
\end{table}

\begin{table}
\centering
\caption{Sighting Rules Summary}
\medskip
\begin{tabular}{p{12em}p{12em}}
\hline
\multicolumn{2}{c}{Maximum Sighting Ranges}\\
\hline
by Eyeball&$4 \times \mbox{aircraft Vis.\ No.}$\\
by V.A.S. &$6 \times \mbox{aircraft Vis.\ No.}$\\
by V.A.S.\ with radar assist&$10 \times \mbox{aircraft Vis.\ No.}$\\
at Night&2 hexes\\
at Night in A/B&6 hexes\\
\hline
\multicolumn{2}{c}{Target I.D. Ranges}\\
\hline
by Eyeball&$2 \times \mbox{aircraft Vis.\ No.}$\\
by V.A.S. &$4 \times \mbox{aircraft Vis.\ No.}$\\
at Night&Same position, facing and speed required.\\
With Tgt.\ I.D. radar technology available&2d turn of Lock\\
\hline
\multicolumn{2}{c}{Padlocking (PL)}\\
\hline
\multicolumn{2}{p{24em}}{
One PL allowed per aircraft.\newline
Two PLs if multi-crew aircraft.\newline
No PLs allowed into blind arcs.\newline
No PLs by Novices or Greens.\newline
1 extra PL per Vet.\ or Tac.\ Mstr.\newline
No PLs if in Target Sun Arc.
}\\
\hline
\end{tabular}
\end{table}


}

\paragraph{Visibility Number.} All aircraft, and all missiles in flight have visibility numbers (presented on the ADC for aircraft, and on the missile charts for missiles). The visibility number is used to determine the maximum range for sighting. The Maximum Visual sighting range for a target is 4 times its visibility number.

\paragraph{Sighting Eligibility.} Any aircraft or missile which is not currently sighted, and is within visibility range, and is not in the blind arc of the searcher, is eligible to be sighted. Count each two levels of altitude as one hex of range.

\paragraph{Visual Sighting Procedure.} In the Visual Sighting Procedure Phase, the friendly aircraft which is closest to\addedin{1B}{1B-apj-23-errata and 1B-apj-35-qa}{, and not blind to or blocked by clouds,} an unsighted or unpadlocked enemy aircraft or missile is used as the reference aircraft for sighting attempts against it. If more than one friendly aircraft is the same distance from an enemy, the searching player may designate which of the friendly aircraft will be the searcher. \addedin{1B}{1B-apj-23-errata and 1B-apj-35-qa}{All aircraft on a side within sighting range \addedin{1B}{1B-apj-25-qa}{of and not blind to an enemy }may participate in the sighting attempt of that enemy even if they were used to padlock other targets in the same sighting phase.}

All ranges and applicable sighting die roll modifiers are determined from the relative positions of the reference aircraft and the targets it is attempting to sight. Unless padlocked by friendly aircraft, previously sighted enemy aircraft or missiles do not remain sighted. Sighting must be successfully accomplished anew to keep them in sight (padlocking previously sighted targets, makes sighting anew automatic).

Declare any Padlocks first. Roll the die for each unsighted or unpadlocked eligible target.\addedin{2A}{2A-missile-sighting}{ Check first for unsighted missiles and then for unsighted aircraft.} If the result is less than or equal to the target's visibility number, it is considered sighted for the game-turn. \changedin{1B}{1B-apj-36-errata}{A sighted target is considered sighted to all friendly aircraft.}{All enemy aircraft sighted by an aircraft on one side are sighted by all aircraft on that side, even if nominally out of spotting range or in a blind arc. Note, however, the individual restrictions in this rule on gun/missile attacks, Defensive Preemption, and Engaging Missiles.}

\paragraph{Padlocking Enemy Targets.} Each pilot only crewed friendly aircraft may padlock one previously sighted enemy target not in its blind arc. A padlocked target remains sighted and does not have to be rolled for. Each multi-crew aircraft may padlock two previously sighted targets instead of one. Crew quality (see Chapter 18) may also affect how many padlocks are possible.

Padlocking is an administrative procedure used only in the Visual Sighting Phase to keep enemy threats in sight. An aircraft which padlocks an enemy is in no way restricted and may move and attack any sighted enemy aircraft or engage sighted missiles whether they were padlocked by it or not. \addedin{1B}{1B-apj-36-errata}{Padlocking does not automatically carry over between aircraft during the game-turn or at the next Sighting Phase.}

\paragraph{Blind and Restricted Arcs.} Each ADC indicates its aircraft's blind and restricted arcs (if any). An L on the entry indicates that the aircraft is blinded or restricted only to targets at a lower altitude. An aircraft is always blind to enemy aircraft in the same hex at lower altitudes unless the target is padlocked. An aircraft may sight aircraft in the same hex at higher altitudes (through the top of the Canopy). It is permissible to padlock same hex lower altitude aircraft but not to search for them (if you know about them, you would roll slightly to keep them in sight). An aircraft may not be used as the reference for sighting against targets to which it is blind.

\addedin{1B}{1B-apj-23-qa}{A sighted or padlocked aircraft entering blind arcs is not considered to be unsighted until the start of the next game turn.}

}{

\paragraph{Sighting Range.} The range for sighting is determined normally according to  rule~\ref{rule:range}.

When attempting to sight a higher aircraft or missile in daylight, the vertical range is the difference in altitude levels divided by \emph{four}, rather than two, and rounded down. This is because it is easier to see an aircraft silhouetted against the sky.

\AY[3A-sighting-weather]{If the searcher or the target or both are in haze (see advanced rule \ref{rule:haze}), the sighting range is considered to be twice the actual range. This reduces both the maximum range to which an aircraft or missile can be sighted and gives more adverse modifiers.}

\paragraph{Maximum Sighting Range.} An aircraft or a missile in flight has a visibility number, given for aircraft by the ADC and for missiles by Table~\ref{table:missile-data}. The visibility number determines the maximum range for sighting. During daylight and clear weather, the maximum visual sighting range to an aircraft or missile is four times its visibility number. Table \ref{table:maximum-sighting-range} summarizes the maximum sighting range in other conditions (see advanced rule \ref{rule:night-and-weather}).

\paragraph{Blind and Restricted Arcs.} A crewmember’s view may be obstructed by their aircraft’s fuselage or wings. Each aircraft’s ADC indicates its blind and restricted \AY[3A-horizontal-arcs]{horizontal} arcs (if any). An L after the arc indicates that the aircraft is blinded or restricted only to lower altitude in that arc (e.g., by a long nose). 

An aircraft cannot padlock or attempt to sight targets in its blind arcs. An aircraft can padlock and attempt to sight targets in its restricted arcs, but receives an adverse modifier to the sighting die roll.

\AY[3A-horizontal-arcs]{
When determining the horizontal arc, the aircraft attempting sighting is the reference element and the target aircraft or missile being sighted is the distant element. Resolve borderline cases by first moving the faster element forward. If the faster element remains on the borderline, use the rearward arc.
}

An aircraft also cannot attempt to sight targets at the same map location (i.e., the same hex or hex side) but lower altitude, but it can padlock them (as the padlocking aircraft can roll slightly to keep the target in view). An aircraft may sight or padlock targets at the same map location but higher altitude.

\paragraph{Padlocking Procedure.}

Padlocking allows aircraft to maintain sight of enemy aircraft and missiles.

If advanced rule~\ref{rule:crew-ability} on crew ability is not being used, a pilot can padlock one target, and an observer (or equivalent crewmember) of a multi-crew aircraft can padlock another. If it is being used, the crew quality and characteristics determine how many targets they can padlock: a novice or green crewmember cannot padlock any, a regular can padlock one, and a veteran or tactics master can padlock two (see Table~\ref{table:padlocks}). 

Crewmembers who are disoriented or suffering from GLOC cannot padlock.

For example, a multi-crew aircraft with two veteran crewmembers would be able to declare four padlocks (two for the pilot and two for the observer), but a multi-crew aircraft with a veteran pilot and green observer would only be able to declare two (two for the pilot and none for the observer).

To be eligible for padlocking, a target must have been sighted on the previous game turn, must be within maximum visual sighting range of the padlocking aircraft, and must not be in its blind arc. 

While it is normally not possible to sight lower targets at the same map location (i.e., the same hex or hex side), it is possible to padlock them.

\addedin{1C}{1C-tables}{
    \begin{table*}
\centering
\caption{Relative Range Effects}
\medskip
\begin{tabular}{lcccccccc}
\hline
Visual Range in Hexes&0--3&4--6&7--9&10--12&13--15&16--20&21--30&31+\\
Die Roll Modifier&$-2$&$-1$&$+0$&$+1$&$+2$&$+3$&$+5$&$+8$\\
\hline
\\
\hline
V.A.S. Range in Hexes&0--20&21--30&31--40&41--50&51--60&61--70&71--80&81+\\
Die Roll Modifier&$+0$&$+1$&$+2$&$+3$&$+4$&$+5$&$+6$&$+8$\\
\hline
\tablemedskip
\tablenotes{9}{0.75\linewidth}{\addedin{1B}{1B:apj-23-errata}{When looking for higher targets in daytime, count each 4 levels difference of altitude as 1 hex of range.}}
\end{tabular}
\end{table*}

\begin{table*}
\centering
\caption{Paint Scheme / Position / Weather Effects}
\medskip
\begin{tabular}{lcccccccc}
\hline
Target Aircraft Position&Lower&Level&Higher&In Haze&In Stratus&Silh,\ by Cloud\\
\hline
Silver          &$-2$&$-1$&$-1$&$+1$&$+3$&$-1$\\
Uncamouflaged   &$-1$&$+0$&$+0$&$+1$&$+3$&$-1$\\
Camouflaged     &$+1$&$+0$&$-1$&$+0$&$+3$&$-2$\\
Low Vis.\ Grey  &$+0$&$+1$&$+1$&$+2$&$+3$&$-1$\\
Aircraft Smoking&$-1$&$-2$&$-2$&NA&NA&NA\\
\hline
\end{tabular}
\end{table*}

\begin{table}
\centering
\caption{Addtitional Modifiers}
\medskip
\begin{tabular}{p{18em}l}
\hline
Number of aircraft searching:\\
1 or 2&$+0$\\
3 or 4&$-1$\\
5 or 8&$-2$\\
9+    &$-3$\\
\changedin{2A}{2A:missile-sighting}{Tgt.\ is just launched Missile or SAM$^*$}{Target is a missile with a normal boost or sustainer motor burning}&$-3$\\
\tablerowaddedin{2A}{2A:missile-sighting}{\addedin{2A}{2A:missile-sighting}{Target is a missile with a smokeless boost or sustainer motor burning}&\addedin{2A}{2A:missile-sighting}{$-1$}}
\tablerowaddedin{2A}{2A:missile-sighting}{\addedin{2A}{2A:missile-sighting}{Target is a missile just launched from a spotted aircraft}&\addedin{2A}{2A:missile-sighting}{$-2$}}
\changedin{2A}{2A:missile-sighting}{Tgt.\ is aircraft which just launched Missile$^*$}{Target just launched a missile that was spotted}&\changedin{2A}{2A:missile-sighting}{$-3$}{$-2$}\\
Tgt.\ is in all searcher's Restricted Arcs&$+2$\\
Searcher is Looking out of Stratus&$+2$\\
Searcher is Veteran and Tactics Master&$-2$\\
Searcher is Veteran or Tactics Master&$-1$\\
Multi-crew aircraft Searching&$-1$\\
\x{AWF as the only reference to the HUD modifier in the rules is in the context of an IRSTS lock-on}{\changedin{1B}{1B:apj-25-qa and 1B:apj-34-qa}{Hud Interface Technology used}{Searcher has HUD Interface Technology}}{Searcher has IRSTS lock-on and HUD interface technology}&$-1$\\
Searcher has RWR indications&$-1$\\
Has Poor Eyes&$+1$\\
Has Good Eyes&$-1$\\
is Novice&$+1$\\
is Green&$+2$\\
Target using DDS Flare program&$+$PPL No.\\
\hline
\tablemedskip
\tablenotes{2}{0.9\linewidth}{
\deletedin{2A}{2A:missile-sighting}{$^*$ does not apply to smokeless missiles.}

NOTE: All modifiers are cumulative. Disoriented or GLOC aircrew may not search.
}
\end{tabular}
\end{table}

\begin{table}
\centering
\caption{Sighting Rules Summary}
\medskip
\begin{tabular}{p{12em}p{12em}}
\hline
\multicolumn{2}{c}{Maximum Sighting Ranges}\\
\hline
by Eyeball&$4 \times \mbox{aircraft Vis.\ No.}$\\
by V.A.S. &$6 \times \mbox{aircraft Vis.\ No.}$\\
by V.A.S.\ with radar assist&$10 \times \mbox{aircraft Vis.\ No.}$\\
at Night&2 hexes\\
at Night in A/B&6 hexes\\
\hline
\multicolumn{2}{c}{Target I.D. Ranges}\\
\hline
by Eyeball&$2 \times \mbox{aircraft Vis.\ No.}$\\
by V.A.S. &$4 \times \mbox{aircraft Vis.\ No.}$\\
at Night&Same position, facing and speed required.\\
With Tgt.\ I.D. radar technology available&2d turn of Lock\\
\hline
\multicolumn{2}{c}{Padlocking (PL)}\\
\hline
\multicolumn{2}{p{24em}}{
One PL allowed per aircraft.\newline
Two PLs if multi-crew aircraft.\newline
No PLs allowed into blind arcs.\newline
No PLs by Novices or Greens.\newline
1 extra PL per Vet.\ or Tac.\ Mstr.\newline
No PLs if in Target Sun Arc.
}\\
\hline
\end{tabular}
\end{table}


}

\paragraph{Visual Sighting Procedure.}

After padlocking, each side attempts to sight first unsighted opposing missiles and then unsighted opposing aircraft.

For each unsighted opposing target, identify which friendly aircraft can participate in the sighting attempt. To participate, the aircraft must have at least one crewmember who is not disoriented or suffering from GLOC, must be within maximum visual sighting range, must not have the target in its blind arc, and must not have its line of sight blocked by terrain. 

If no friendly aircraft can participate, the unsighted target remains unsighted.

\CY[3A-sighting]{Otherwise, determine which of the participating aircraft is the searcher. This will normally be the participating friendly aircraft closest to the target, but if two or more are equally close, their players may designate which is the searcher.}{Otherwise, the player may designate which of the participating aircraft is the searcher. This will often be the participating friendly aircraft closest to the target, but sometimes it is worthwhile to designate a further aircraft with more advantageous modifiers.} Roll a die and apply appropriate modifiers. The target is sighted if the modified result is less than or equal to the target's visibility number.


}

%!TEX root = ../rules-working.tex
%LTeX: enabled=false


\begin{twocolumntablefloat}
\begin{twocolumntable}
\tablecaption{table:sighting-range-modifiers}{Relative Range Effects.}
\begin{tabularx}{0.75\linewidth}{lCCCCCCCC}
\toprule
Visual Range in Hexes&0--3&4--6&7--9&10--12&13--15&16--20&21--30&\plusafter{31}\\
Die Roll Modifier&\minus{2}&\minus{1}&\plus{0}&\plus{1}&\plus{2}&\plus{3}&\plus{5}&\plus{8}\\
\midrule
\\
\midrule
V.A.S. Range in Hexes&0--20&21--30&31--40&41--50&51--60&61--70&71--80&\plusafter{81}\\
Die Roll Modifier&\plus{0}&\plus{1}&\plus{2}&\plus{3}&\plus{4}&\plus{5}&\plus{6}&\plus{8}\\
\bottomrule
\end{tabularx}
\begin{tablenote}{0.75\linewidth}
\addedin{1C}{1C-apj-23-errata}{When looking for higher targets in daytime, count each 4 levels difference of altitude as 1 hex of range.}
\end{tablenote}
\end{twocolumntable}
\end{twocolumntablefloat}

\begin{twocolumntablefloat}
\begin{twocolumntable}
\tablecaption{table:sighting-position-modifiers}{Paint Scheme / Position / Weather Effects.}
\begin{tabularx}{0.75\linewidth}{lCCCCCC}
\toprule
Target Aircraft Position&Lower&Level&Higher&In Haze&In Stratus&Silh.\ by Cloud\\
\midrule
Silver          &\minus{2}&\minus{1}&\minus{1}&\plus{1}&\plus{3}&\minus{1}\\
Uncamouflaged   &\minus{1}&\plus{0}&\plus{0}&\plus{1}&\plus{3}&\minus{1}\\
Camouflaged     &\plus{1}&\plus{0}&\minus{1}&\plus{0}&\plus{3}&\minus{2}\\
Low Vis.\ Gray  &\plus{0}&\plus{1}&\plus{1}&\plus{2}&\plus{3}&\minus{1}\\
Aircraft Smoking&\minus{1}&\minus{2}&\minus{2}&NA&NA&NA\\
\bottomrule
\end{tabularx}
\end{twocolumntable}
\end{twocolumntablefloat}

\begin{onecolumntablefloat}
\begin{onecolumntable}
\tablecaption{table:sighting-additional-modifiers}{Additional Modifiers.}
\begin{tabularx}{\linewidth}{Xl}
\toprule
Number of aircraft searching:\\
\quad 1 or 2        &\plus{0}\\
\quad 3 or 4        &\minus{1}\\
\quad 5 to 8        &\minus{2}\\
\quad \plusafter{9} &\minus{3}\\
\textchangedin{2A}{2A-missile-sighting}{Tgt.\ is just launched Missile or SAM\asteriskmark}{Target is a missile with a normal boost or sustainer motor burning}&\minus{3}\\
\tablerowaddedin{2A}{2A-missile-sighting}{\textaddedin{2A}{2A-missile-sighting}{Target is a missile with a smokeless boost or sustainer motor burning}&\textaddedin{2A}{2A-missile-sighting}{\minus{1}}}
\tablerowaddedin{2A}{2A-missile-sighting}{\textaddedin{2A}{2A-missile-sighting}{Target is a missile just launched from a spotted aircraft}&\textaddedin{2A}{2A-missile-sighting}{\minus{2}}}
\textchangedin{2A}{2A-missile-sighting}{Tgt.\ is aircraft which just launched Missile\asteriskmark}{Target just launched a missile that was spotted}&\changedin{2A}{2A-missile-sighting}{\minus{3}}{\minus{2}}\\
\tablerowdeletedin{3A}{3A-searcher}{Tgt.\ is in all searcher's Restricted Arcs&\plus{2}}
\tablerowaddedin{3A}{3A-searcher}{Target in searcher's restricted arcs&\plus{2}}
Searcher is Looking out of Stratus&\plus{2}\\
Searcher is Veteran and Tactics Master&\minus{2}\\
Searcher is Veteran or Tactics Master&\minus{1}\\
Multi-crew aircraft Searching&\minus{1}\\
\changedin{2B}{2B-sighting-and-hud}{\changedin{1C}{1C-apj-25-qa/1C-apj-34-qa}{Hud Interface Technology used}{Searcher has HUD Interface Technology}}{Searcher has radar or IRSTS lock-on and HUD interface technology}&\minus{1}\\
Searcher has \changedin{2B}{2B-sighting-and-other-detections}{RWR indications}{detection or lock-on from radar, IRSTS, or RWR}&\minus{1}\\
Has Poor Eyes&\plus{1}\\
Has Good Eyes&\minus{1}\\
is Novice&\plus{1}\\
is Green&\plus{2}\\
Target using DDS Flare program&\changedin{1C}{1C-apj-36-errata}{\plus{PPL No.}}{\minus{PPL No.}}\\
\bottomrule
\end{tabularx}
\begin{tablenote}{\linewidth}
\deletedin{2A}{2A-missile-sighting}{\asteriskmark~does not apply to smokeless missiles.}

NOTE: All modifiers are cumulative. Disoriented or GLOC aircrew may not search.
\end{tablenote}
\end{onecolumntable}
\end{onecolumntablefloat}



\CX{
\paragraph{Modifiers.} The die roll for sighting may be modified by a variety of circumstances. The modifiers available are shown on the \changedin{1C}{1C-tables}{Sighting Charts}{Tables~\ref{table:sighting-range-modifiers}, \ref{table:sighting-position-modifiers}, and \ref{table:sighting-additional-modifiers}} and include:

\begin{itemize}
    \item Number of searching aircraft and crew quality.
    \item Range and relative altitude to target aircraft.
    \item Paint Scheme of target aircraft.
    \item Whether lock-ons exist to target.
    \item Whether target is smoking, \changedin{1B}{1B-apj-35-qa}{ejecting flares, or}{was ejecting flares on the previous game turn, or is} in various weather conditions.
\end{itemize}

\addedin{1B}{1B-apj-20-qa and 1B-apj-34-qa}{Eyesight modifiers, position modifiers, and any modifiers defined by the term “searcher” apply only to the reference aircraft. For the other modifiers, if any of the aircraft in the search group quality to use the modifier, it is used. The restricted arcs modifier applies only if all participating aircraft in the search would be affected by it.}

\notein{1B}{ISSUE: AWF: APJ 25 QA says that multicrew, HUD, and RWR only apply if the reference aircraft of the search has them.” This may partially contradict the answer in APJ 34, since the HUD modifier does not have the term “searcher”.}

\changedin{1E}{AWF}{
\addedin{1B}{1B-apj-23-errata}{
\paragraph{Look-Up and Smoker Effects} When visually searching for aircraft that are higher, count each four levels of altitude difference as a hex of range instead of two (it is easier to see aircraft outlined against the sky.) Some aircraft are noted on their power charts as being smokers at military power. When Mil power is used the smoker modifier for sighting attempts against them applies. Some results on the Optional damage tables cause aircraft to become smokers regardless of power setting. The modifier applies to them as well.}
}{
\paragraph{Look-Up.} When visually searching\addedin{2B}{2B-sighting-higher}{ in daylight} for aircraft\addedin{2B}{2B-sighting-higher}{ or missiles} that are higher, count each four levels of altitude difference as one hex of range instead of two. (It is easier to see aircraft outlined against the sky.) 
\paragraph{Smokers.} Some aircraft are noted on their power charts as being smokers at military power. When military power is used the smoker modifier for sighting attempts against them applies. Some results on the optional damage tables cause aircraft to become smokers regardless of power setting. The modifier applies to them as well.
}
}{

\paragraph{Die Roll Modifiers.} The modifiers for the sighting die roll are described here and summarized in Tables~\ref{table:sighting-range-modifiers}, \ref{table:sighting-position-modifiers}, and \ref{table:sighting-additional-modifiers}:

\begin{itemize}
    \item \itemparagraph{Range.} Apply the appropriate modifier from Table \ref{table:sighting-range-modifiers}. 
    
    \AY[3A-sighting-weather]{If the searcher or the target or both are in haze (see advanced rule \ref{rule:haze}), the sighting range is considered to be twice the actual range. For example, if the searcher is in haze and the range is 6, use the modifier for a range of 12.}
    
    \IDY[3A-sighting]{\itemparagraph{Target Position.} Apply the appropriate modifiers from Table~\ref{table:sighting-position-modifiers} for altitude of the target aircraft relative to the searcher, whether it is in haze or stratus, and whether it is silhouetted against clouds as seen by the searcher. These modifiers depend on the paint scheme of the target and do not apply to missile.}
    
    \IAY[3A-sighting]{\itemparagraph{Target Background.} Apply the appropriate modifier from Table~\ref{table:sighting-position-modifiers} for the background against which the target aircraft is seen:
    \begin{itemize}
    \item\itemparagraph{Against Clouds.} This modifier takes priority over the others and applies if the target is in stratus clouds or between the searcher and a cloud layer (see advanced rule \ref{rule:clouds}). 
    \item\itemparagraph{Against Sky.} This modifier applies if the target is above the searcher.
    \item\itemparagraph{Level.} This modifier applies if it has the same altitude as the searcher.
    \item\itemparagraph{Against Water.} This modifier applies if the target is below the searcher and on a map sheet that is completely water hexes.
    \item\itemparagraph{Against Land.} This modifier applies if the target is below the searcher and on any map sheet that is not completely water hexes.
    \end{itemize}
    These modifiers depend on the paint scheme of the target aircraft:
    \begin{itemize}
    \item\itemparagraph{Silver.} This corresponds aircraft in a natural-metal finish that produces glints.
    \item\itemparagraph{Light Uncamouflaged.} This corresponds to aircraft with an uncamouflaged paint scheme in light colors (for example, USN aircraft in Light Gull Grey and Ivory White in the 1960s).
    \item\itemparagraph{Dark Uncamouflaged.} This corresponds to aircraft with an uncamouflaged paint scheme in dark gray or blue (for example, USN aircraft in Glossy Sea Blue in the early 1950s).
    \item\itemparagraph{Camouflaged.} This corresponds to aircraft with an camouflaged paint scheme appropriate for the zone.
    \item\itemparagraph{Low-Visibility Gray.} This corresponds to aircraft with a low-visibility gray paint scheme.
    \end{itemize}
    If an aircraft has an inappropriate camouflaged paint scheme for the zone (for example, a desert camouflage scheme in a temperate zone), it should use the light or dark modifiers when seen against land. The scenario notes will indicate the paint schemes of the aircraft.
    
    The target background modifiers do not apply when sighting missiles.
    }

    
    \IAY[3A-sighting]{\itemparagraph{Target in Stratus.} If the target is in stratus clouds (see advanced rule \ref{rule:stratus-clouds}), apply a $+3$ modifier.}
    \IAY[3A-sighting]{\itemparagraph{Target below Clouds.} If the target is below the highest cloud layer (see advanced rule \ref{rule:clouds}), apply a $+1$ modifier.}
    \item \itemparagraph{Target is Smoker.} 
    \CY[3A-sighting]{If the target is a smoker, apply the modifiers from Table~\ref{table:sighting-position-modifiers} for altitude of the target relative to the searcher.}{If the target is a smoker and lower than the searcher, apply a $-1$ modifier. If it is a smoker and level with or higher than the searcher, apply a $-2$ modifier.} Some aircraft are noted on their power charts as being smokers at military power. When they use military power, this modifier applies to them. Some results on the optional damage tables cause aircraft to become smokers regardless of their power setting. The modifier applies to them, too.
    \item \itemparagraph{Number of Participants.} Apply the modifier from \ref{table:sighting-additional-modifiers} for the number of aircraft participating in the sighting attempt. 
    \item \itemparagraph{Restricted Arc.} If the target is in the restricted arcs of \CY[3A-sighting]{\emph{all} of the participants}{the searcher}, apply a $+2$ modifier.
    \item \itemparagraph{Missile Motor.} If the target is a missile whose boost or sustainer motor burned in the previous game turn, apply a $-3$ modifier for a normal motor and a $-1$ modifier for a smokeless motor.
    \item \itemparagraph{Missile Launch.} If the target is a missile launched from a sighted aircraft in the previous game turn or an aircraft that launched a sighted missile in the previous game turn, apply a $-2$ modifier. (For these modifiers, it is important to attempt to sight missiles before aircraft.)
    \item \itemparagraph{Target using DDS Flares.} If the target used a DDS program with flares (see rule~\ref{rule:dds}) in the previous game turn, apply the PPL as a modifier.
    \item \itemparagraph{Searcher is Multi-Crew.} If the searcher is a multi-crew aircraft, apply a $-1$ modifier. A searcher with one crewmember who is disoriented or suffering from GLOC does not gain this modifier.
    \item \itemparagraph{Searcher in Stratus.} If the searcher is in stratus clouds (see advanced rule \ref{rule:stratus-clouds}), apply a $+2$ modifier.
    \item \itemparagraph{Searcher Ability.} If advanced rule \ref{rule:crew-ability} on crew ability is being used, apply these modifiers:
    \begin{itemize}
    \item If the searcher is a veteran, apply a $-1$ modifier. 
    \item If the searcher is a novice, apply a $+1$ modifier. 
    \item If the searcher is green, apply a $+2$ modifier. 
    \item If the searcher has poor eyesight, apply a $+1$ modifier.
    \item If the searcher has excellent eyesight, apply a $-1$ modifier.
    \item If the searcher is a tactics master, apply a $-1$ modifier.
    \end{itemize}
    \IDY[3A-sighting-detection]{\itemparagraph{Searcher has RWR Indications.} If the searcher has RWR indications, apply a modifier of $-1$.}
    \IAY[3A-sighting-detection]{\itemparagraph{Searcher has Detection.} If the searcher has a radar, IRSTS, or RWR detection (or lock-on) of the target, apply a $-1$ modifier. This modifier is applied only once regardless of how many different types of detections the searcher has of the target.}
    \IDY[3A-sighting-hud]{\itemparagraph{Searcher has IRSTS Lock-on and HUD Interface.} If the searcher has an IRSTS lock-on to the target and HUD interface technology, apply a $-1$ modifier.}
    \IAY[3A-sighting-hud]{\itemparagraph{Searcher has Lock-On and HUD Interface.} If the searcher has a radar or IRSTS lock-on for the target and HUD interface technology, apply an additional $-1$ modifier. Again, this modifier is applied only once regardless of how many different types of lock-ons the searcher has for the target.}

\end{itemize}

}

\CX{
\paragraph{Gun and IR Missile Attack Restrictions.} A target aircraft may not be attacked by guns or fired on by IR missiles, unless the attacker has visual sight of the target at the start of his movement\addedin{1B}{1B-apj-36-errata}{, i.e., the target must be spotted by his side, and the target can be neither beyond the attacker's visual sighting range nor in the attacker's blind arc.}. However, if an enemy aircraft moved first and was lost from sight due to entering haze or passing through a stratus layer, it may be followed and attacked.
}{

\paragraph{Individual Sighting.} 
\label{rule:individual-sighting}
For certain purposes, a target must be individually sighted by an aircraft. This requires that:
\begin{itemize}
    \item The target is sighted or is a friendly aircraft,
    \item The range to the target from the aircraft checking for individual sighting is within the maximum visual sighting range, and
    \item The target is not in the blind arc of the aircraft checking for individual sighting.
\end{itemize}

\paragraph{Sighting and Visually Aimed Weapons.} 
A target aircraft may not be attacked with guns (see rule~\ref{rule:air-to-air-gun-combat}) or fired on by IRMs (see rule~\ref{rule:infrared-heat-seeking-missiles}) unless it is individually sighted by the attacker at the start of the attacker's movement. However, if the target moved first and was lost from sight due to entering haze or passing through a stratus layer (see advanced rule \ref{rule:haze}), it may be followed and attacked.

One exception to this restriction is that IRMs may be fired upon targets that have been locked-on with radar or an IRSTS or sighted with a VAS (see advanced rule~\ref{rule:irm-seeker-lock-up-assistance-methods} for the complete requirements). 
}

\CX{
\paragraph{Sighting and Defensive Preemption (see Chapter 12).} A moving enemy aircraft may not be defensively preempted against unless:

\begin{itemize}
    \item it is sighted and not in the blind arc of the defender when the attacker begins moving, or
    \item the defender has just been fired on by its guns, or
    \item the attacker is sighted, and both attacker and defender are within spotting range of (and not in the blind or restricted arc of) another friendly aircraft at the start of the \changedin{2B}{2B-sighting-and-pre-emption}{defender's}{attacker's} movement.
\end{itemize}
}{
\paragraph{Sighting and Defensive Preemptions.}
A threatened aircraft may not preempt (see rule~\ref{rule:defensive-preemptions}) a threatening aircraft unless:
\begin{itemize}
     \item At the start of the threatening aircraft’s move, it is individually sighted by the threatened aircraft,
     \item At the start of the threatening aircraft’s move, it and the threatened aircraft are individually sighted by an aircraft friendly to the threatened aircraft and neither are in that aircraft’s restricted arc, or
     \item The threatening aircraft has just carried out a gun attack on the threatened aircraft.
\end{itemize}
}

\CX{
\paragraph{Sighting \& Defensively Engaging Missiles (see Chapter 14).} An enemy missile may not be defensively engaged unless;

\begin{itemize}
    \item it is sighted and not in the blind arc of the defender \changedin{2B}{2B-sighting-and-engaging-missiles}{when the missile starts its move}{at the start of the aircraft decisions phase}, or
    \item it is sighted, and both defender and missile are within spotting range of (and not in the blind or restricted arc of) another friendly aircraft \changedin{2B}{2B-sighting-and-engaging-missiles}{at the start of the defender's movement}{at the start of the aircraft decisions phase}.
    \item RWR indications allow the engagement of an unsighted missile.
\end{itemize}
}{
\paragraph{Sighting and Defensively Engaging Missiles.} 
A target aircraft may not defensively engage a missile (see rule~\ref{rule:engaging-missiles}) unless:
\begin{itemize}
    \item In the aircraft decisions phase, the missile is individually by the target aircraft, 
    \item In the aircraft decisions phase, the missile and the target aircraft are individually sighted by an aircraft friendly to the target aircraft and neither are in that aircraft’s restricted arc, or
    \item The target aircraft has RWR indications of a missile attack (see rule~\ref{rule:rwr}).
\end{itemize}
}

\CX{
\addedin{1B}{1B-apj-36-errata}{\paragraph{Initial Spotting.} When a scenario provides information on initial spotting, this applies to 'game-turn 0'. The normal spotting rules should be followed starting with game-turn 1, taking into account possible padlocks due to previous spotting.}
}{
\paragraph{Initial Sighting.} If a scenario notes that an aircraft is initially sighted, the aircraft can be padlocked on the first game turn. Other than this, the normal visual sighting rules should be followed.
}

\section{Sighting Ground and Naval Units}
\label{rule:sighting-ground-and-naval-units}

[Text on air-to-ground combat not yet incorporated.]

\iffalse

Each ground or naval unit (or target terrain type) has a sighting number on its counter. The sighting number is the range (in terms of hexes, count two altitude levels as one hex) at which it becomes automatically sighted to an aircraft. Any aircraft which starts a game-turn within that range may see and attack the sighted ground units and/or targets.

\paragraph{Ground Sighting Procedure.} During the visual sighting phase, aircraft may either participate in searching for enemy aircraft and missiles or they may sight ground targets.

If they choose to sight ground units, two adjacent angle-off arcs about the aircraft may be looked into (except aircraft may not look into their blind arcs). Ground units and targets within their printed sighting range to the searching aircraft are automatically spotted unless;

\begin{itemize}
    \item The LINE OF SIGHT is blocked by terrain.
    \item The targets or ground units are camouflaged.
\end{itemize}

Note: Sighting ground targets (or FAC marks) is an individual affair. Only the aircraft looking for them, can see them. Aircraft looking for ground units do not normally participate in the sighting of aircraft or missiles. Exception, see incidental sighting rule below.

Aircraft may only aim at and attack sighted ground units, or FAC (Forward Air Controller) supplied laser marks, or smoke marks.

\addedin{1B}{1B-apj-22-qa}{\paragraph{Padlocking}. If an aircraft chooses to padlock sighted ground units in the following turns, they remain sighted to it. If the aircraft elects to sight aircraft or missiles, or it cannot padlock the ground units, they become unsighted and must be sighted anew later.}

\notein{1B}{ISSUE: AWF: The previous change is the first mention of padlocking ground units. How many ground units can be padlocked by an aircraft? I presume the same number as the number of aircraft that it can padlock. Can an aircraft padlock both ground units and aircraft or missiles? I presume not, and it must choose to padlock or search for one class or the other. What are the circumstances under which an aircraft “cannot” padlock ground units? If it padlocks or searches for aircraft or missiles? If the line of sight is blocked? If the unit exceeds the sighting range?}

\paragraph{Incidental Sighting Of Aircraft and Missiles.} When aircraft are searching for ground targets, they may participate in the sighting of aircraft and missiles located at the same or lower altitude in the two arcs they are seeking ground units in. If applicable, they may be the designated reference aircraft.

\paragraph{Line Of Sight.} In order for aircraft to see ground targets, or for ground units to see aircraft, a line of sight must exist which is unblocked by terrain. The line of sight is the straight line from the aircraft's position to the center of the target's hex.

An unblocked line of sight exists between aircraft and ground units if:

\begin{itemize}
    \item There is no intervening contour line or terrain higher than both the aircraft and target.
    \item There is an intervening contour line and/or terrain feature higher than the target but not the aircraft, which is closer to the aircraft, and the aircraft is at least two altitude levels above the terrain.
    \item There is an intervening contour line and/or terrain feature higher than the target but not the aircraft, which is closer to the target or midway between the aircraft and target, and the aircraft is at least four altitude levels higher than the terrain.
\end{itemize}

There exists a blocked line of sight between the aircraft and target if:

\begin{itemize}

    \item A ground unit is at an altitude below that of an intervening contour and/or terrain feature and the aircraft is at or below that contour or terrain's altitude.

    \item A ground unit is at the same altitude as an intervening contour and/or terrain feature and the aircraft is lower unless the contour or terrain is part of a downward slope of a hill the unit is on the side of.

    \item A ground unit is higher than an intervening contour or terrain feature which is not defining the slope of the rising terrain or hill the unit is on, and the aircraft is lower by the same amount or is in T-level flight (Chapter 20) and within two hexes of the intervening contour or terrain.

    \item The aircraft and ground unit are at equal altitudes and the line of sight crosses a woods, urban, or built up area hex feature, or a ridgeline also at the same altitude.

\end{itemize}

Note: Aircraft deliberately flying so that terrain is blocking a line of sight between them and a ground unit, are using “Terrain Masking”.

\paragraph{Effects of Camouflage.} Ground units may be camouflaged as a result of being in certain terrain types, or if designated as camouflaged in the scenario instructions. The sighting range of a camouflaged ground unit is half its normal range. Also, sighting is no longer automatic. When sighting a camouflaged unit, roll the die. On a 5 or less it is spotted. A \changedin{1E}{AWF}{minus 2}{$-2$} modifier applies to the roll if the target is marked by a FAC with smoke, or by laser if the sighting aircraft has laser spot tracker technology. Any crew quality or eyesight modifiers also apply unless VAS is used.

\paragraph{Effects of Activity by Camouflaged Units.}  Any camouflaged AAA or SAM units which fired or launched missiles on the previous game turn may be sighted out to normal range with a die roll as above.

A camouflaged SAM unit that launched a missile in the SAM Interaction Phase of the current game turn whose basic missile visibility number is 7 or more is sighted automatically if in sighting range and the just launched missile is sighted.

When a camouflaged AAA or SAM unit fires or launches a missile, it becomes “detected” for the rest of the game turn and for the following game turn. Whether sighted or not, a detected unit can be attacked as its general position is known. If sighted, it may be attacked normally, if detected but not also sighted, it may be attacked but is treated as a secondary target. Aiming is done normally against a detected target but the bombsight and tracking time modifiers are disallowed. Smart or guided weapons may not be used against detected only targets.

\fi

\begin{advancedrules}

\CX{
\section{Visual Augmentation Systems (VAS)}
\label{rule:vas}

Aircraft can be augmented by television and optical equipment to enhance the aircraft crew's ability to sight targets at longer than normal ranges.

\paragraph{Independent Search.} An aircraft with a VAS may conduct a visual search independent of the sighting process or its radar. It is allowed to do the following:

\begin{itemize}

    \item attempt to sight one eligible target already detected by radar (but not locked on) in the searcher's 180{\deg} angle off arcs and within 10 times the target's visibility number, or

    \item visually search in the limited radar arc within 6 times the target's visibility number for one eligible target (whether previously detected or not).

\end{itemize}

In either case, roll the die once for each eligible target. Modify the roll as appropriate. If the modified roll is less than or equal to the eligible target's visibility number, it is sighted. If the roll is greater, the target remains unsighted.

Ignore multi-crew and HUD interface modifiers. The VAS range modifiers are used in place of the eyesight modifiers.

\paragraph{Radar/VAS Interface.} An aircraft equipped with VAS may (upon establishing a radar lock) automatically sight one of the locked targets in the next Visual Sighting Phase if it is:

\begin{itemize}
    \item in the searcher's 180{\deg} angle off arc and,
    \item within 10 times the target's visibility number.
\end{itemize}

\paragraph{VAS Considerations.} Enemy aircraft sighted by VAS systems are sighted only to the VAS equipped aircraft that specifically saw them and may not be attacked by other aircraft unless also visually sighted by the regular procedure. VAS systems may be used to accomplish target identification. An aircraft that elects to use VAS during the Visual Sighting Phase may not padlock.

}{

\section{Visual Augmentation Systems}
\label{rule:vas}

A visual augmentation system (VAS) uses television and optical equipment to enhance the aircraft crew’s ability to sight targets at longer than normal ranges.

% ISSUE: If an aircraft conducts a VAS search, can it also conduct a normal search? It cannot padlock. Does "independent" here mean "in additon to participating in normal searches" or "instead of participating in normal searches"?

\paragraph{VAS Sighting Procedure.} An aircraft with a VAS may conduct a visual search independent of the sighting process or its radar use. It is allowed to do one of the following:

\begin{itemize}

    \item attempt to sight a target already detected by radar in the searcher’s \arcrange{180}{+} \CY[3A-combined-arcs]{angle-off arcs}{combined arcs (see rule \ref{rule:combined-arcs})} and at a range of no more than 10 times the target’s visibility number, or

    \item attempt to sight a target in the searcher’s limited \CY[3A-combined-arcs]{arc}{combined arcs (see rule \ref{rule:combined-arcs})} and at a range of no more than 6 times the target’s visibility number (whether previously detected or not).

\end{itemize}

In the first case, if the searcher has a radar lock on the target, sighting automatically succeeds. Otherwise, roll a die and apply appropriate modifiers. The VAS range modifiers are used instead of the visual sighting range modifiers, and eyesight, multi-crew, and HUD interface modifiers are ignored. The target is sighted if the modified result is less than or equal to the target’s visibility number.

\AY[3A-sighting]{If the searcher or the target or both are in haze (see advanced rule \ref{rule:haze}), the sighting range is considered to be twice the actual range.}

\paragraph{VAS Restrictions.}
An aircraft that elects to use VAS during the visual sighting phase may not padlock. An aircraft sighted by a VAS system is sighted only to the VAS-equipped aircraft that sighted it and may only be attacked by other aircraft if it is also visually sighted normally.

\paragraph{VAS Considerations.}   VAS systems may be used to accomplish target identification (see rule~\ref{rule:limited-intelligence}).

}

\CX{
\section{Infrared Search and Track Systems (IRSTS)}
\label{rule:irsts}

An Infrared Search and Track System is capable of passively detecting and tracking aircraft. IRSTS operation is not detectable by any aircraft's ECM or RWR gear. IRSTS can be used to sight aircraft at night and to target IR missiles in a manner similar to IR Uncage technology.

\paragraph{IRSTS Types.} \changedin{2A}{2A-irsts-c}{Two types of IRSTS systems exist: A (Early) and B (Advanced)}{Three types of IRSTS systems exist: A (Early), B (Modern), and C (Advanced)}. IRSTS is used in the Radar Search and Lock-On Phase. A single pilot aircraft may use radar or IRSTS in this phase, a multi-crew aircraft may do both. \addedin{1B}{1B-apj-25-qa}{Unlike radar, there are no maneuver-related restrictions on the use of IRSTS.}

\paragraph{Detection.} IRSTS detects heat based on fuel consumption: Detection range equals four times the fuel used number (based on the Fuel column of the Power Chart for the current power setting)\addedin{2A}{2A-irsts-c}{ for IRSTS-A and B and six times the fuel used number for IRSTS-C}. The range is doubled if the IRSTS is in the target's 30{\deg} arc or less. The target is detected if it is in detection range.

\begin{itemize}
    \item IRSTS-A operates in the limited radar arc of an aircraft.
    \item IRSTS-B operates in the 180{\deg} arc of an aircraft.
    \itemaddedin{2A}{2A-irsts-c} IRSTS-C operates in the 150{\deg} arc of an aircraft.
\end{itemize}

\paragraph{IRSTS Lock-on.} \changedin{2A}{2A-irsts-c}{One lock-on attempt is allowed}{An IRSTS-A and B can attempt one lock-on} against the nearest detected target or the target with the highest fuel point use (and consequently the largest IR signature); double the value as seen from 30{\deg} arc or less. \addedin{2A}{2A-irsts-c}{An IRSTS-C can attempt to lock-on up to \addedin{2B}{2B-irsts-c}{any }six detected targets.} Roll the die. On a 6 or less for type A or 8 or less for type B\addedin{2A}{2A-irsts-c}{ and C}, the target is locked-up. \addedin{2B}{2B-irsts}{After the lock-on attempts are resolved, if any succeeded, the aircraft loses all other IRSTS detections.}

\addedin{2B}{2B-irsts}{
\paragraph{Maintaining IRSTS Detections and Lock-Ons.} IRSTS detections or lock-ons are maintained provided the target satisfies the arc and range requirements for detection at the start of each subsequent Radar Search and Lock-On Phase and, if the aircraft has a single pilot, provided the pilot does not use radar.
}

\paragraph{Lock-on Sighting Benefits.} In the visual sighting phase, an aircraft with an IRSTS lock-on and having HUD interface receives the HUD modifier. Visual Sighting at night is permitted against a locked-on target out to twice the normal range.

\paragraph{IR Missile Targeting.} An IRM equipped-fighter with an IRSTS lock-on may launch missiles at the target even if they are not visually spotted (handy at night and in haze).  A Type-B \addedin{2B}{2B-irsts-c}{or -C }system lock-on allows IRMs to be launched as if IR uncage technology were available even if the fighter does not normally have that technology.
}{

\section{Infrared Search and Track Systems}
\label{rule:irsts}

An infrared search and track system (IRSTS) uses infrared television and optical equipment to provide the aircraft crew with the ability to detect and track aircraft using their infrared emissions. IRSTS operation is not detectable by any aircraft’s ECM or RWR gear. IRSTS can be used to detect aircraft at night and to target IR missiles in a manner similar to IRM uncage technology.

\paragraph{IRSTS Types.} Three types of IRSTS exist: 
\begin{itemize}
    \item IRSTS-A (early) covers the limited \CY[3A-combined-arcs]{arc}{combined arc (see rule \ref{rule:combined-arcs})} and successfully locks-on on a roll of $6-$.
    \item IRSTS-B (modern) covers the \arcrange{180}{+} \CY[3A-combined-arcs]{arc}{combined arcs (see rule \ref{rule:combined-arcs})} and successfully locks-on on a roll of $8-$.
    \item IRSTS-C (advanced) covers the \arcrange{150}{+} \CY[3A-combined-arcs]{arc}{combined arcs (see rule \ref{rule:combined-arcs})} and successfully locks-on on a roll of $8-$.
    
\end{itemize}

\paragraph{IRSTS Requirements.}  IRSTS is used in the Radar Search and Lock-On Phase. A pilot-only aircraft may use either radar or IRSTS in this phase, and a multi-crew aircraft may do both. Unlike radar, there are no maneuver-related restrictions on the use of IRSTS.

\paragraph{IRSTS Detection.} The target is automatically detected its range is no more than the maximum detection range, which depends on the targets fuel consumption and aspect. For IRSTS-A/B, the maximum detection range is four times the number of fuel points used by the target at its current power setting (according to the fuel column in the power chart of the target’s ADC). For IRSTS-C, the maximum detection range is six times the number of fuel points. The maximum range is doubled if the IRSTS is used in the target’s \arcrange{30}{-} arc. 

\AY[3A-irsts-weather]{IRSTS systems are not affected by haze (see advanced rule \ref{rule:night-and-weather}).}

\paragraph{IRSTS Lock-On.} An IRSTS-A/B may attempt to lock-on either the nearest detected target or the target with the highest fuel point use. For this purpose, the fuel use is doubled if the IRSTS is used in the target’s \arcrange{30}{-} \AY[3A-combined-arcs]{combined} arc. An IRSTS-C may attempt to lock-on up to six of the detected targets, with no restrictions on closeness or fuel usage.

For each target, roll the die. The lock-on succeeds on a roll of $6-$ for an IRSTS-A and $8-$ for an IRSTS-B/C. 

After the lock-on attempts are resolved, if any succeeded, the aircraft loses all other IRSTS detections.

\paragraph{Maintaining IRSTS Detections and Lock-Ons.} IRSTS detections or lock-ons are maintained provided the target satisfies the arc and range requirements for detection at the start of each subsequent Radar Search and Lock-On Phase and, if the aircraft has a single pilot, provided the pilot does not use radar.

\paragraph{IRSTS Lock-on Sighting Benefits.} An IRSTS can provide a \emph{detection} of the target aircraft but does not by itself \emph{sight} the target. Nevertheless, an IRSTS can aid in subsequent sighting attempts. When searching for a locked-on target, an aircraft with a HUD interface receives the HUD modifier and can attempt visual sighting at night to twice the normal range.

\paragraph{IRSTS Assist for IRMs.} An aircraft may launch IRMs at a locked-on target even if it is not sighted. An IRSTS-B/C allows IRMs to be launched at locked-on targets as if IRM uncage technology were available, even if the fighter does not have that technology.

}

\section{Limited Intelligence}
\label{rule:limited-intelligence}

\CX{

Some scenarios allow players to start with aircraft types unknown to their opponent or they restrict one side (or both sides) from firing until an unknown aircraft is identified.

\paragraph{Aircraft Identification.} A sighted aircraft is identified:

\begin{itemize}

    \item Automatically when it is within twice its visibility number in hexes to a friendly aircraft.

    \item Automatically by VAS aircraft when it is within four times its visibility number in hexes\addedin{2B}{2B-vas-identification}{ and in the \arcrange{180}{+} arc}.

    \item By VAS when within 10 times its visibility number in hexes\addedin{2B}{2B-vas-identification}{ and in the \arcrange{180}{+} arc} on a die roll of 7 or less.

    \item When it begins its second turn being radar locked by an aircraft with TGT I.D. technology.

\end{itemize}

Note: IFF ECM gear may help determine if an aircraft is friendly but it does not identify them.

\paragraph{Missile Identification.} A missile's type is not revealed until it has been defeated by decoys or until it attacks a target. If a missile misses, the type need not be revealed until the end of the game. At missile launch, the target aircraft's player should be told only that he is the target of a missile. If the missile is AHM and the target aircraft has an RWR capable of detecting an AHM in active mode, he must be told that the missile is an AHM when it goes active. A player's aircraft may use radar lock-ons and target illumination procedures to confuse the enemy player even if he launches only IR missiles.

}{

Some scenarios allow players to employ aircraft types unknown to their opponent or restrict them from firing until an aircraft is identified.

\paragraph{Aircraft Identification.} \label{rule:aircraft-identification} A sighted enemy aircraft is identified:

\begin{itemize}

    \item Automatically, when its range to a friendly aircraft is no more than twice its visibility number.

    \item Automatically, when its range to a VAS-equipped friendly aircraft is no more than 4 times its visibility number, and it is in the \arcrange{180}{+} arc \AY[3A-combined-arcs]{combined} of that aircraft.

    \item On a roll of $7-$, when its range to a VAS-equipped friendly aircraft is no more than 10 times its visibility number, and it is in the \arcrange{180}{+} \AY[3A-combined-arcs]{combined} arc of that aircraft.

    \item Automatically, when it begins its second turn with a radar lock-on from an aircraft with Target ID technology.

\end{itemize}

IFF may help determine if an aircraft is friendly, but does not identify it (see rule~\ref{rule:iff}).

\paragraph{Missile Identification.} 
When an aircraft is the target of a missile, the target's player is given only the following information:

% ISSUE: Is a "type" the specific missile model (e.g., AIM-9L) or the generic seeker type (e.g, IRM)?

\begin{itemize}
    \item Which aircraft is the target, when the missile is launched.
    \item The missile is an active AHM, when an AHM goes active and if the target has a RWR capable of detecting this.
    \item The missile type, if decoys defeat it.
    \item The missile type, if it hits.
\end{itemize}

If the missile attacks and misses, its type need not be revealed until the end of the game.

Aircraft may use radar lock-ons and target illumination to confuse the enemy even if they launch only IRMs.
}

[Text on air-to-ground combat not yet incorporated.]

\iffalse

\paragraph{Ground Unit and Naval Unit Identification.} Ground and naval units, especially vehicles, seen from the air are hard to identify. To reflect this, always set up ground units and Naval units with their generic side up. Their exact type is not revealed unless visually identified or the instant they fire on aircraft or are attacked by aircraft.

Furthermore, generic AAA and SAM site counters (mobile and static types) may be provided which are placed on the map instead of the actual units. Only when a AAA or SAM unit fires for the first time, or the site is attacked, is the actual unit revealed and returned to the map in place of the silt marker. In some scenarios, dummy sites are provided, these are not revealed until attacked or I.D.'d.

\paragraph{I.D. Procedure.} Visual identification of ground and Naval units may occur before a unit is attacked or fires. Aircraft may attempt to visually I.D. targets in the sighting phase after sighting them. Roll once for each sighted unit in an aircraft's searched arc. If the number is less than or equal to 10 minus the range in hexes (counting 2 altitude levels as a hex), the unit is identified and flipped over to its non-generic side. Once a unit is identified through visual sighting or by attack or when it fires, it remains I.D.'d for the rest of the game.

\fi

\CX{
\section{Forward Air Controllers (FACs)}
}{
\section{Forward Air Controllers}
}
\label{rule:sighting-facs}

[Text on air-to-ground combat not yet incorporated.]

\iffalse
Forward Air Controllers identify targets and direct ground attacks for pilots. A FAC may be ground-based or airborne. A FAC marks a ground target with smoke or with a laser spot. An attack on a marked target receives a \changedin{1E}{AWF}{minus 1}{$-1$} die roll modifier.

\paragraph{Ground FAC.} A ground FAC is a vehicle-mounted observer on the ground. A Ground FAC automatically sees any enemy ground units (camouflaged or not) within six hexes provided LOS is not blocked. It sees units in adjacent hexes even if LOS is blocked.

Each Ground FAC may mark one enemy unit (which it can see) during the \changedin{2B}{2B-ground-fac-marking}{AAA Planning Phase}{visual sighting phase} of each turn. If the Ground FAC marks the target with a laser spot, the marking is removed at the end of the game turn. If the Ground FAC marks the target with smoke, the marking is removed at the end of the next game-turn (the marking lasts two turns).

\paragraph{Airborne FAC.} An airborne FAC is in an aircraft flying over the target area. The aircraft can usually carry smoke rocket pods for marking targets, and some may be equipped with laser designators. An Airborne FAC sights ground targets using normal Visual Sighting but is better at sighting camouflaged units.

In order to mark a target, the airborne FAC must initially sight the target just as other aircraft do. Due to special training a FAC may always try to sight camouflaged targets out to their normal sighting range using a modified roll of 5 or less, and at or inside half range, on a modified roll of 8 or less. Once sighted by a FAC, a camouflaged target is considered to remain sighted to that FAC for the rest of the game (he has marked it on his grid-map). A FAC that searches for a “detected” camouflaged unit gets a $-2$ modifier to his die roll.

\paragraph{Laser Spots.} Airborne FACs with laser designators may mark targets with laser spots as described in Chapter 27.

\paragraph{Smoke Spots.} Airborne FACs with smoke rockets may mark targets by attacking them with smoke during their flight. Normal aiming and a release point must be reached by the FAC to declare his shot. All modifiers that apply to a normal rocket attack apply to the marking shot roll.

Roll one die, on a modified 7 or less, the target is successfully marked and a smoke counter is placed on it. The smoke, as above, is removed at the end of the following game-turn. Smoke rocket pods are detailed in the external stores tables. Light spotter plane characteristics are detailed in the accompanying game booklet if applicable. Aircraft may attack laser or smoke marked but unsighted targets after aiming normally except that the bombsight and tracking time modifiers are ignored; the marked target modifier still applies.

\fi
\section{Long-Range Ground Unit Sighting}

[Text on air-to-ground combat not yet incorporated.]

\iffalse
Aircraft with observers on board (scout helos, some FAC aircraft and two seat jets or trainers opting to carry an observer) may double the distance at which ground units may be sighted and identified (binoculars or other optics in use).  Identification is as per Rule 11.5 except treat each two hexes or four altitude levels as one hex of range.

\paragraph{V.A.S. or TV/IR Optics.} Aircraft equipped with VAS or TV/IR Optics technology or pods, may sight and identify ground units and targets out to triple the normal range if sighting into arcs into which the VAS or TV/IR optics can see. In this case treat each 3 hexes or six altitude levels as one hex of range.

Also, aircraft with laser designators type B and C, and having TV/IR optics capability, may do long range sighting and identification in the same arcs as the designators are capable of designating into.
\fi

\section{Hidden Initial Placement of Units}

[Text on air-to-ground combat not yet incorporated.]

\iffalse
Some scenarios allow units to be hidden at the start of play, meaning their hex position is noted on paper and the unit is kept off the game map until an aircraft sights it or it reveals itself by firing on the aircraft.

Hidden but uncamouflaged units are revealed whenever the other player elects to search for ground targets and is in sighting range per 11.2. When an aircraft searches for ground targets, the player with the hidden units must tell the other player if any camouflaged units are in sighting range so that the die can be rolled to determine if they're sighted.

Hidden units which fire on or launch missiles at aircraft are immediately “detected” and must be placed on the map and are eligible to be sighted normally in following game turns.

\fi

\DX{
\section{Formation Effects on Sighting}

\paragraph{Close Formations Restrictions.} Wingmen in close formations may not padlock enemy aircraft, nor be counted for the multiple searching aircraft modifier. They may not be the reference aircraft either. They are concentrating on holding formation on the leader. Aircraft in tactical formations have no sighting restrictions.

\paragraph{Close Formation Effects.} Any sighting attempts against aircraft in a close formation are done treating the close formation as a single entity. The sighting number of the largest aircraft in the formation is used and a \changedin{1E}{AWF}{minus 1}{$-1$} modifier is applied to the sighting roll for each two aircraft in the formation. Success indicates all aircraft in the formation are spotted.
}
\end{advancedrules}

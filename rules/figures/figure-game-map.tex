\silentlychangedin{1C}{1C-figures}{

\begin{onecolumnfigure}[h!]
\includegraphics[width=\linewidth]{figures/figure-game-map.pdf}
\end{onecolumnfigure}

}{

\begin{onecolumnfigure}

\begin{fitwidth}{\linewidth}
\begin{tikzpicture}
    \cliphexgrid{0}{0}{16}{12}
    \drawmegahex{-2}{+6.0}
    \drawmegahex{-2}{+11.0}
    \drawmegahex{-2}{+16.0}
    \drawmegahex{3}{-1.5}
    \drawmegahex{3}{+3.5}
    \drawmegahex{3}{+8.5}
    \drawmegahex{3}{+13.5}
    \drawmegahex{8}{+6.0}
    \drawmegahex{8}{+11.0}
    \drawmegahex{8}{+16.0}
    \drawmegahex{13}{-1.5}
    \drawmegahex{13}{+3.5}
    \drawmegahex{13}{+8.5}
    \drawmegahex{13}{+13.5}
    \drawmegahex{18}{+6.0}
    \drawmegahex{18}{+11.0}
    \drawmegahex{18}{+16.0}
    \drawhexgrid{0}{0}{17}{13}
    \drawaircraftcounter{10}{10}{60}{MiG-21}{B}{}
    \drawaircraftcounter{7}{5.5}{30}{F-4}{A}{}
\end{tikzpicture}
\end{fitwidth}

\figurecaption{figure:game-map}{\protect\CX{Game Maps}{The game maps are printed with a hex grid to clearly define the positions of the game counter and determine the horizontal distances between them. The two aircraft game counters are six hexes apart horizontally. The larger megahexes are five hexes across and can be used to count long distances in fives.
}}

\end{onecolumnfigure}

}

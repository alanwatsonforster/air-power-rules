\rulechapter{In This Game}

This rule introduces the {\AirPow} game system and its components.


\CX{

\section{The Game System}

The {\AirPow} game system allows you to control one or more jet fighters in scenarios reflecting the actual life and death situations faced by fighter pilots in modern air combat. These rules provide the procedures for simulating jet fighter combat on a game board. As players, you must provide the brains, the strategy, and the tactics that will allow your jets to survive combat and win the various game scenarios.

Note: The {\AirPow} system is a derivative of the game system I used in GDW’s {\AirSup} and {\AirStr} games (now out of print), and as a result the data cards and most of the information tables used in the {\AirSup} games are fully compatible with this game system and may be used with it.

}{

\section{Game System}

The {\AirPow} game system allows you to control one or more jet fighters in scenarios reflecting the life-and-death situations fighter pilots faced in air combat historically since 1950 and continue to face in the present day. These rules provide the procedures for simulating jet fighter combat on a game board. As players, you must provide the brains, the strategy, and the tactics that will allow your jets to survive combat and win the various game scenarios.

The {\AirPow} system is a derivative of the game system used in GDW’s {\AirSup} and {\AirStr} games and Clash of Arms’ {\TSOH} game (now all out of print). As a result, the aircraft data cards and most of the weapons data tables from those games can be adapted to this game system and used with it.

}

\section{Game Scale}

\AX{
The scales in the game are as follows:
}

\begin{itemize}
    \item Each game turn represents 12 seconds of real time.
    \item Each map hex covers a distance of {\onethird} \CX{a}{of a} statute mile\AX{\ or 1,760 feet}.
    \item Each altitude level represents 1,000 feet of height.
    \item Each altitude band \addedin{1B}{1B-apj-23-errata}{typically }is 8,000 to 10,000 feet thick.
    \item Each aircraft speed point equals 100 mph of speed.
    \item Each aircraft counter represents a single \CX{jet}{aircraft}.
    \itemaddedin{2B}{2B-missile-counters}{Each missile counter represents a single missile.}
\end{itemize}

\AX{
The length of the game turn fixes the relation in the game between aircraft speed and distance moved. In 12 seconds, a real aircraft moving in level flight at 100~mph will move $1/3$ of a mile (100 miles per hour divided by 3600 seconds per hour multiplied by 12 seconds is $1/3$ of a mile). Similarly, in the game an aircraft moving in level flight at a speed of 1.0 (equivalent to 100~mph) will move 1 hex (equivalent to $1/3$ of a mile).
}

\DY{

\section{Learning the Game System}

\CX{

\paragraph{Read The Basic Rules First.} You should read only the rules that appear before any “Advanced Rules” header within each chapter. Skip the advanced rules and read on into the next chapter. Keep reading until you are instructed to play a Training Scenario. Take a break, then set up the scenario, and play it out. All of the Training Scenarios are designed for solitaire play. When you have finished, return to where you left off and continue reading. Do this until you’ve been exposed to all the basic rules and played all training scenarios.

\paragraph{Read the Advanced Rules If Desired.} Advanced rules are optional in nature and allow you to raise the level of detail and realism contained in the game. Using the advanced rules will increase the complexity of play but rewards you with a more accurate depiction of modern air combat. All, some, or none of the advanced rules may be learned and used as agreed upon by the players. The training scenarios may be replayed with advanced rules for practice.

\paragraph{Be Patient and Have Fun!} Don’t expect to learn all the rules in a single \CX{seating}{sitting}. Take your time and enjoy yourself. After playing several of the training scenarios, you will begin to get the hang of the system. Don’t forget, this is a game. Play it for its entertainment value, and have a good time as you strive to master the intricacies of modern air combat.

}{

\paragraph{Read The Basic Rules First.} You should read only the basic rules before any “Advanced Rules” header within each chapter. Skip the advanced rules and keep reading the basic rules until you are instructed to play a training scenario. Take a break, set up the scenario, and play it out. All of the training scenarios are designed for solitaire play. When you have finished, return to where you left off and continue reading. Do this until you’ve been exposed to all the basic rules and played all the training scenarios.

\paragraph{Read the Advanced Rules If Desired.} Advanced rules are optional and allow you to raise the level of detail and realism in the game. Using the advanced rules will increase the complexity of play but will reward you with a more accurate depiction of modern air combat. All, some, or none of the advanced rules may be used as agreed upon by the players. You may replay the training scenarios with advanced rules for practice.

\paragraph{Be Patient and Have Fun!} Don’t expect to learn all the rules in a single sitting. Take your time and enjoy yourself. After playing several training scenarios, you will begin to get the hang of the system. Don’t forget, this is a game. Play it for its entertainment value, and have a good time as you strive to master the intricacies of modern air combat.

}

}

\section{Game Components}

\CX{

Like most simulation games, those in this system have the following basic types of components:

\paragraph{1. Game Rules.} This set of rules defines how aircraft, missiles, and ground units move, detect enemies, and fight. It is not necessary to memorize the rules. Play aid sheets summarizing the key points of the rules are provided.

Once familiar with the rules, you should be able to play the game referring to the play aid charts alone; returning to the rules booklet only to clarify questions. There are two different levels of rules: Basic, and Advanced. Basic rules are all that are necessary to play any scenario in the game.

\paragraph{2. Game Charts.} Game charts distill large amounts of information into easily used tables. Players use the charts to determine specific capabilities of their aircraft and weapons, and to resolve combat. Charts in this system include: Flight and combat rules summaries, weapons data tables, aircraft data cards, and aircraft logsheets.

\paragraph{3. Game Counters.} The die-cut cardboard counters are the game pieces used to represent the aircraft, missiles, and ground units involved in play. Some of the counters represent information (such as target hits) rather than objects. Usually less than 20 pieces will be required for the play of any scenario. The counters shown \changedin{1C}{1C-figures}{here}{in Figure~\ref{figure:game-counters}} represent a fighter and an air to air missile. Each has a distinctive silhouette and color scheme that lets players recognize them easily.

\paragraph{4. Dice.} Each game in this system includes one ten-sided die as a random number generator. {\AirPow} uses the die to resolve events of chance during play. The die is marked with ten digits: 0 to 9. When rolled, it produces numbers from 0 to 9; the topmost number on the die is the one read. The 0 is always read as a 10. Thus, when the ten-sided die is rolled, it will produce a number from 1 to 10.

Die usage example: if a game chart indicates that a certain missile will hit its target 80\% of the time when it attacks. A player must use the die and roll 8 or less in order to hit. That die roll represents the 80\% probability of hitting. In this case, a miss would occur if a 9 or 10 was rolled.

\paragraph{5. Game Maps.} The game maps are the surface over which the playing pieces will be moved. Though several are provided, sometimes only a few will be used in a scenario. The hexagon grid on each map provides spaces in which players place and move their counters. Like a chessboard, the hexagons help to clearly define where a counter is. Distance between counters (range) is determined by counting the number of hexes between them. \addedin{1B}{1B-apj-23-errata}{Altitude differences may affect range calculations.}

For example, \changedin{1C}{1C-figures}{as}{Figure~\ref{figure:game-map} shows two game counters on a game map. As} each hex represents {\onethird} of a mile of distance, and as the two aircraft counters shown\deletedin{1C}{1C-figures}{ in the diagram below} are 6 hexes apart horizontally, they are a scale two miles apart. 
\deletedin{1C}{1C-figures}{\addedin{1B}{1B-apj-23-errata}{[The two aircraft counters were left off the diagram.]}}
Notice the large shaded hex outlines on the map. These are termed “megahexes”. Each megahex is 5 regular hexes across and allows players to determine long ranges and distances easier by counting in fives. 

}{

Games in this system have the following components:

\paragraph{Game Rules.} These rules define how aircraft, missiles, and ground units move, detect enemies, and fight. It is not necessary to memorize the rules. \DY{There are two different levels of rules: basic and advanced. Basic rules are all that are necessary to play any scenario in the game.}

\paragraph{Game Tables and Figures.} The tables present large amounts of information in a compact form. The figures show spatial relations. Once familiar with the rules, you should be able to play the game by referring to the play-aids alone, returning to the text of the rules only to clarify doubts.

\paragraph{Aircraft Data Cards and Weapon Data Tables.} The aircraft data cards present the flight and combat capabilities of the aircraft in terms of the game rules and are described in more detail in Rule~\ref{rule:aircraft-data-cards}. The weapon data tables give the combat capabilities of air-to-air, air-to-ground, and ground-to-air weapons.

\paragraph{Game Counters.} Some of the die-cut cardboard counters are used to represent elements such as aircraft, missiles,  ground units, and naval units. Other counters represent information such as target hits. Usually, less than 20 counters are required to play a scenario. The game counters shown in Figure~\ref{figure:game-counters} represent a fighter and an air-to-air missile. Each has a distinctive silhouette and color scheme, allowing players to recognize it easily.

\paragraph{Game Maps.} The game maps are the surface over which the counters are moved. Though several are provided, sometimes only a few will be used in a scenario. The hexagon grid (“hex grid”) on the maps provides spaces in which players place and move their game counters. Like a chessboard, the hexagons clearly define the position of a counter.

For example, Figure~\ref{figure:game-map} shows two game counters on a game map. Since each hex represents {\onethird} of a mile and as the two aircraft counters shown are six hexes apart horizontally, they are a scale two miles apart. Notice the large shaded hex outlines on the map. These are termed “megahexes”. Each megahex is five regular hexes across, allowing players to determine long horizontal distances more easily by counting in fives.

\paragraph{Aircraft Log Sheets.} The aircraft log sheets are used to record the speed and altitude of the aircraft and to help determine the changes in these as they maneuvers. A pad of generic aircraft log sheets is provided with each game. One log should be used for each aircraft in play. You may make copies of the sheets. The use of the log sheets is described in more detail in rule~\ref{rule:aircraft-log-sheets}.

\paragraph{Die.} The game system uses a ten-sided die to resolve events of chance during play. The die is marked with the digits 0 to 9. When rolled, the topmost number on the die is the one read, and 0 is always read as a 10. Thus, each die roll produces a number from 1 to 10.

The rules often state that the die roll should be modified under certain circumstances. For example, the rules may state that if a certain condition is fulfilled, the die roll should be modified by $+2$. In this case, if the condition is fulfilled, two should be added to the die roll's result, producing a number from 3 to 12.

The results of die rolls are typically used in one of two ways. If the die roll is to determine success or failure, the die roll result for success is given as a number or less. For example, if the rules state that success occurs on a die roll of $6-$, then success will occur if the modified die roll result is six or less. In the absence of modifiers, this corresponds to a 60\% chance of success. If the die roll is to determine which of several events occur, then a range of die roll results will be given for each event. For example, $1-$ (one or less) for outcome A, 2--7 (two to seven) for outcome B, and $8+$ (eight or more) for outcome C.


}

\changedin{1C}{AWF}{

\begin{FIGURE}[h!]
\includegraphics[width=\linewidth]{figures/figure-game-counters.pdf}
\end{FIGURE}

}{

\begin{FIGURE}[tp]

\includegraphics[width=1.5cm]{figures/counter-f-4.jpg}
\hspace{1.5cm}
\includegraphics[width=1.5cm]{figures/counter-aam.jpg}

\CAPTION{figure:game-counters}{\protect\x{Game Counters}{The game counter on the left represents an F-4 Phantom II and the one on the right an air-to-air missile.}}

\end{FIGURE}

}



\begin{tikzfigure}{\linewidth}
    \drawhexgrid{0}{0}{16}{12}
    \drawmegahex{3}{-1.5}
    \drawmegahex{3}{+3.5}
    \drawmegahex{3}{+8.5}
    \drawmegahex{3}{+13.5}
    \drawmegahex{8}{+6.0}
    \drawmegahex{8}{+11.0}
    \drawmegahex{8}{+16.0}
    \drawmegahex{13}{-1.5}
    \drawmegahex{13}{+3.5}
    \drawmegahex{13}{+8.5}
    \drawmegahex{13}{+13.5}
    \drawhexgrid{0}{0}{16}{+12}
    \drawaircraftcounter{10}{10}{60}{MiG-21}{B}
    \drawaircraftcounter{7}{5.5}{30}{F-4}{A}
\end{tikzfigure}


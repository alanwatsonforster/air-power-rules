\documentclass[10pt]{article}

\input style.tex

\title{Updated Rules Project}
\author{Alan Watson Forster}
\date{24 March 2024}
\runningtitle{Updated Rules Project}

\setcounter{secnumdepth}{0}
%\setlist[enumerate,1]{label = (\alph*)}

\newcommand{\itemtag}[1]{\item \textbf{Change: #1.}\par}

\begin{document}

\twocolumn
\thispagestyle{empty}
\maketitle
\suppressfloats

\section{Aims}

This project aims to produce updated rules for {\AirPow}.

\section{Context}

At the time of writing, there are several sets of rules available:
\begin{itemize}
    \item The original {\AirSup} and {\AirStr} rules from 1987, with extensions in {\DF} and {\EOTG}. I understand that a limited number of copies of {\AirSup} are still available. Nevertheless, these rules are considered obsolete.
    \item The first-edition {\AirPow} rules, with official errata, published in 1992 with {\TSOH}. There are two sets of errata: one page provided with {\TSOH} and four pages published in {\APJ} \#24. Again, I understand that a very limited number of copies are still available.
    \item Several proposals for second-edition rules outlined by J.D.\ Webster mainly in {\APJ} during the 1990s.
    \item The second-edition {\AirPow} rules, edited by Malcolm Pipes and published in 2022 on the \href{https://airpower.groups.io/g/main}{airpower email group} site. The current text is labeled “v2.4”.
\end{itemize}

\section{Rationale}

There is already a set of second-edition rules. Why is the project necessary?
\begin{itemize}
    \item The second-edition rules do not fully incorporate previous errata, in particular Felix Hack’s errata published by JD Webster in {\APJ} 36, and have certain errors.
    \item Where they incorporate errata, it is often inserted literally rather than changing the text to include the sense of the errata.
    \item There is no traceability. Changes have been made, but there is no way to see what changed or why.
    \item There are no internal hyperlinks for cross references.
    \item Many of the diagrams have not been fixed.
    \item I think it would be helpful to incorporate diagrams and tables into the text (and also have them on the play-aid sheets). With printed rules, this separation reduced costs and, to a large degree, was more convenient. When viewed on the screen, these considerations are different.
    \item The text needs copy editing and, in certain places, could benefit from being rewritten or perhaps reorganized.
    \item There are few examples or explanations. Indeed, there is little effort to distinguish examples or explanations in the original text.
    \item The text still refers to pilots, crew, and players using masculine pronouns. That was understandable when the first-edition rules were written, but nevertheless it needs correcting.
\end{itemize}

\section{Versions}

I use “versions” to distinguish my efforts from the other “editions.” There will be several versions of the rules and play aids, each based on the previous one. 

\begin{itemize}
    \item Version 1A will correspond to the original first-edition rules without the official errata.
    \item Version 1B will incorporate the errata from authoritative sources, including the pubished errata, statements by J.D.\ Webster, and Felix Hack’s article in {\APJ} 36.
    \item Version 1C will have corrected and redrawn diagrams and will incorporate these into the body of the text in addition to the play aid sheets.
%    \item Version 1E will have additional minor changes from non-authoritative sources.
    \item Version 2A will incorporate second-edition changes from {\APJ}, email messages from JD Webster, and  Malcolm Pipe’s second-edition rules. Not all changes will be adopted since some seem to have issues.
    \item Version 2B will have additional non-authoritative changes.
    \item Version 2C aims to clarify the rules while maintaining the meaning. Certain rules may be rewritten for clarity and conciseness while preserving their meaning. Parts of the text may be reorganized.
\end{itemize}

Having multiple versions allow us to more easily review changes. For example, the changes between 1A and 1B do change the rules according to authoritative sources, but the changes between 1B and 1C are mainly concerned with improving the figures.

The versions up to and including 2A will include only minimal changes to the text, and these changes will largely be drawn from authoritative sources. 

% I’m pretty sure about the nature of these versions. I’m less sure about 2C, and I am considering designating version 2C as version 3 since it will include significantly rewritten and reorganized text.

\section{Products}

I will produce two PDF files for each version, on showing the changes with respect to the previous version and one clean version.

%(Other versions can be generated from the sources.) Furthermore, there will be two options for version 2C: with comments and examples and without. The latter option will have the advantage of conciseness. 

%There will also be a PDF file showing all of the changes and their origins. This will give traceability.

\section{Rules Changes}

This section describes the rules changes between different versions.

\subsection{Version 1B}

Version 1B is an update to the text of version 1A according to authoritative sources. The are changes to the sense of some rules compared to version 1A, but the changes to the text are minimal. It incorporates the following errata:

\begin{itemize}
    \itemtag{1B-credits} Additions to the credits.
    \itemtag{1B-tsoh-errata} All changes in the single sheet of errata included with TSOH.
    \itemtag{1B-apj-22-damage-tables} The optional advanced damage tables from APJ 22.
    \itemtag{1B-apj-23-errata} All changes in the four sheets of errata published in APJ 23.
    \itemtag{1B-apj-24-play-aids} All changes in the revised play aids published in APJ 24. 
    \itemtag{1B-apj-36-errata} All changes in Felix Hack’s list of errata published in APJ 36, except for the change to sustained turning being assessed per 30 degrees of facing change, which was later clarified as a second-edition change.
    \itemtag{1B-apj-XX-qa} Changes and clarifications from JD Webster’s Q\&A articles in {\APJ} 20, 21, 22, 23, 24, 25, 26, 27, 28, 29, 30, 34, 35, 36, 37, 38, and 39.
    \itemtag{1B-tailing} A correction to the rule on tailing following an email message by JD Webster to the airpower group. 
\end{itemize}

\subsection{Version 1C}

Version 1C incorporates the following changes. There should be no changes to the sense of the rules compared to version 1B.

\begin{itemize}
    \itemtag{1C-credits} Additions to the credits.

    \itemtag{1C-figures} All figures have been redrawn. 
    
    Figures 2 (the game map), 6 (level flight), 11 (angle-off on a hexside) have been corrected according to the comments in corresponding the errata incorporated in 1B. Figure 12 (SSGT) now shows a range of 6 hexes. Figure 12 (Genie scatter) now shows the scatter for a Genie centered on a hex and pointing to a hex corner.
    
    All figures have been incorporated into the body of the rules. Certain figures also continue to appear in the play aids.
    
    Reference to figures now follow the pattern “Figure 1” rather than “Angle-Off Diagrams”.

    \itemtag{1C-tables} All tables have been have been incorporated into the body of the rules and also appear in the play aids. 
    
    Reference to tables now follow the pattern “Table 1” rather than “IRM Seeker Head Table”.

    \itemtag{1C-cover} The rule have a cover featuring a USN photograph of an A-7A. This appears to have been the inspiration for the line drawing on the cover of the {\TSOH} rules.

\end{itemize}

\subsection{Version 2A}

Version 2A incorporates the following changes. There are changes to the sense of the rules compared to version 1C, but the changes to the text are minimal.

\begin{itemize}

    \itemtag{2A-credits} Additions to the credits.

    \itemtag{2A-adc} The first-edition F-4B/C ADC is replaced with the second-edition F-4F ADC from {\APJ}~44.

    \itemtag{2A-idle} An aircraft that selects idle power no longer has its start speed reduced. Instead, the aircraft incurs DPs for normal power plus the additional DPs listed in the ADC. This change is taken from the Origins rules in {\APJ}~39, 41, 44, and 53 and the v2.4 rules. 

    \itemtag{2A-spbr} Similarly, an aircraft that uses speedbrakes no longer loses FPs. Instead, the aircraft incurs DPs up to the value shown in the ADC. If the aircraft is supersonic, the maximum is increased by 1 DP. This change is taken from the Origins rules in {\APJ}~39, 41, 44, and 53 and the v2.4 rules.

    \itemtag{2A-fp-to-dp} The conversion factor from FPs to DPs for the new idle power and speedbrake rules is 2. This change appears in the Origins rules in {\APJ}~41 and the v2.4 rules.

    \itemtag{2A-supersonic-flame-out} An aircraft that selects idle or military power at supersonic speeds automatically and immediately suffers a flame-out. This change appears the 1995 GEnie post by JD Webster and the v2.4 rules.

    \itemtag{2A-cruise} The cruise speed in the ADC is for CL configuration and is 0.5 less for 1/2 configuration and 1.0 less for DT configuration. This change was initially suggested by Guy Acala in {\APJ}~24, and appears in the Origins rules in {\APJ}~44 and 53 and the v2.4 rules.

    \itemtag{2A-snap} Aircraft and missiles can no longer execute snap turns. This change appears in the 1995 GEnie post by JD Webster, the Origins rules in {\APJ}~41, and the v2.4 rules.

    \itemtag{2A-sustained} Sustained turning penalties are now assessed per 30 degrees of facing change for second and subsequent facing changes. This change was included by Felix Hack in the errata in {\APJ}~36 but was clarified as applying to 2nd edition rules by JD Webster in the {\APJ} 38 QA. It was included in the Origins rules in {\APJ}~39, 44, and 53 and the v2.4 rules.

    For aircraft with low and high bleed rates, the sustained turning penalties are 0.5 and 1.5 DP. This change was included in the “Props against Jets” article in {\APJ}~32 and the Origins rules {\APJ}~39, 44, and 53 and the v2.4 rules.

    \itemtag{2A-zoom-climbs} The deceleration for zoom climbs (and sustained climbs that gain more than the CC) is now always 1.0 DP per level. See the 1995 GEnie post by JD Webster, the Origins rules {in \APJ}~44 and 53, and the v2.4 rules.

    \itemtag{2A-vertical-climbs} The deceleration for zoom climbs is now always 1.5 DP per level. See the 1995 GEnie post by JD Webster, the Origins rules {in \APJ}~44 and 53, and the v2.4 rules.

    \itemtag{2A-super-climbs} Aircraft with a CC of 6.0 or more in a ZC or SC can use one of their VFPs to climb three levels. This change appeared in the 1995 GEnie post by JD Webster and the v2.4 rules.

    \itemtag{2A-steep-dives} The acceleration for steep dives is now always 1.0 AP per level. This change appeared in the 1995 GEnie post by JD Webster, the Origins rules in {\APJ}~44 and 53, and the v2.4 rules.

    \itemtag{2A-unloaded-dives} The unloaded dives rule is changed to follow the Origins rules in {\APJ}~44 and 53 and the v2.4 rules.
    
    The proposed wording is not completely clear, and in an attempt to state the rule more clearly, I have added that the VFPs are unloaded, too. 
    
    Two things cause me to pause. First, with this rule, unloaded dives do not give any advantage over steep dives in terms of acceleration of horizontal distance. Indeed, a steep dive is better since one VFP can lose two altitude levels, whereas in an unloaded dive, each VFP can only lose one altitude level. Second, the rule appears to state that unloaded dives are a form of level flight, in which case other rules need modifying (e.g., an aircraft cannot recover from a vertical dive into an unloaded dive, and unloaded dives do not count as diving for barrel rolls). If unloaded dives are a form of level flight, treating them separately from diving flight would probably be better.

    \itemtag{2A-missile-sighting} The sighting rules are changed so that attempts to sight missiles come before attempts to sight aircraft, and the modifiers are changed. This change appeared in the Origins rules in {\APJ}~44 and 53.
    
    The Origins rules mention that missiles launched from sighted aircraft are no longer sighted, but I can’t find this in the original rules.

    \itemtag{2A-advantage} An aircraft in a vertical climb may disadvantage aircraft at the same level or lower. This change appeared in the Origins rules in {\APJ}~39, 41, 44, and 53 and the v2.4 rules.

    \itemtag{2A-roll-preparatory-fps} Lag and displacement rolls now require preparatory FPs equal to {\onethird} of the aircraft's speed (rounded down). The preparatory FPs may be HFPs or VFPs. These change appeared in the 1995 GEnie post by JD Webster and the v2.4 rules.

    \itemtag{2A-missile-launch-modifiers} The modifiers for being a combat hero and tactics master are now cummulative (i.e., a pilot who is both gets a $-2$ modifier). This change appeared in the tables for the v2.4 rules.
    
    \itemtag{2A-missile-launches} The missile launch rules are changed so that if the launch roll fails by exactly one and is not an unmodified ten, the missile fails to launch but remains on the rail and can be used in the future. This change appeared in the Origins rules in \APJ~39, 31, 44 and 53.

    \itemtag{2A-missile-flight} Air-to-air missiles always move before their target. This change appeared in the Origins rules in \APJ~44 and 53.

    \itemtag{2A-missile-attacks} Missile attacks occur when the missile has the same position and altitude as the target or when it has the same altitude, at least one FP left in its proportional move, and the target is one of the seven positions immediately in front of the missile. This change appeared in the Origins rules in \APJ~44 and 53.

    \itemtag{2A-missile-speed} The missile speed attenuation factors are changed. In all cases, they now increase or stay the same with time, which makes physical sense. These changes appeared in the v2.4 tables.

    \itemtag{2A-ew-coverage} RWR-A now detects air-to-air target illumination, not air-to-air locks. RWR-B can no longer detect VF FCR. These changes appear in the v2.4 tables.

\end{itemize}

The rule for pylon drag (suggested originally by Mark Bovankovich in {\APJ}~18 and adopted in the Origins rules in {\APJ}~41, 44, and 53) is not incorporated simply because applying it uniformly would require extensive research and modifications to almost all ADCs.

\subsection{Version 2B}

Version 2B contains unauthoritaive minor changes.

\begin{itemize}
    \itemtag{2B-credits} Additions to the credits.

    \itemtag{2B-directions} ENE, ESE, WSW, and WNW are used in preference to NE, SE, SW, and NW. The original directions fall evenly between two 30{\deg} facings, whereas the new  ones are unambiguously closer to one.

    \itemtag{2B-stacking} Collisions are only possible if the aircraft are at the same altitude. They are also possible if four or more aircraft are stacked at the same altitude, even if they are in a close formation.

    \itemtag{2B-collisions} Potential collisions from head-on attacks are resolved after the attack. Other collisions are resolved at the end of the flight phase.

    \itemtag{2B-range} I have written a rule to clarify how to calculate range when counters are on hex sides. Moving from the hex side to either of the two adjacent hexes or four adjacent hex sides counts as half a hex. Determine the horizontal range including half-hexes, then round down. This gives the correct result, for example, when two counters are on hex sides on the oposite sides of a hex.

    \itemtag{2B-half-fractions} We sometimes need to take $1/3$ or $2/3$ of 0.5 in order to determine the reduced thrust at high altitude. I have added appropriate entries to the fractions table.

    \itemtag{2B-ground-fac-marking} The text of the original rules states that ground FACs mark targets in the AAA planning phase. This phase does not exist. The extended sequence of play indicates that this occurs during the visual sighting phase. I have changed the text of the rules to match this.

    \itemtag{2B-ra-speed-limits} RA aircraft at their maximum or dive speed, as appropriate, may only carry forward 1.0 APs.

    \itemtag{2B-military} Aircraft can select 0.0 APs when using military power. This gives no acceleration, but allows them to maintain a steady speed above cruise speed
    
    \itemtag{2B-stall} Aircraft recover from a stall to a wings-level attitude.

    \itemtag{2B-departures} Aircraft lose any carried APs or DPs when they recover from departure.

    \itemtag{2B-low} The low altitude band extends down to level 0, to account for aircraft at level 0 in TFF.

    \itemtag{2B-steep-climb} An SC must use at least one VFP.

    \itemtag{2B-unloaded-dives} The UD rules in version 2A gives no advantage over an SD. I have changed the rule so that all FPs are HFPs, one altitude level is lost if half of the FPs are spent unloaded, and two if all are spent unloaded. This gives one additional HFP over an otherwise similar SD.    

    \itemtag{2B-free-descent} After a VC, an aircraft in level flight must expend 1/3 of its FPs (not speed) before taking a free descent. This is for consistency with other similar limits on flight, which are in terms of FPs.

    \itemtag{2B-half-vfps} The rules in section 8.4 on half VFPs seem to be incompatible with the rules in 5.4. Perhaps they are left over from when speedbrakes eliminated FPs? I have deleted this rule.

    \itemtag{2B-recovery} I have changed the rules for recovery for ET turns to be that the aircraft must not turn at the ET rate and must not prepare for or execute rolling maneuvers. This allows slides and eliminates the potentially confusing reference to being wings level.

    \itemtag{2B-head-on-attacks} I have changed the rule for head-on attacks to include all of the prerequisites except sighting. (If you are being fired at, you will probably notice.)

    \itemtag{2B-angle-off-on-hex-side} I have changed the angle-off diagram for an aircraft on a hex side so that the boundaries between the arcs radiate from the aircraft. The previous version makes sense in the context of the Genie scatter diagram, where one wants the lines to trace valid locations, but not for simply determining geometry.

    \itemtag{2B-rocket-restrictions} I have applied the errata to the restrictions to rocket attacks on aircraft at different altitudes to those applied to gun attacks (“An aircraft \emph{in level flight} may fire at a target in another hex only if it is at the same altitude level or at an adjacent altitude level”).

    \itemtag{2B-rocket-modifiers} The TSOH sheets say all gun attack modifiers apply to rocket attacks, but the TSOH rules after APJ 23 errata exclude snap shots (obviously) and damage. I am going to include damage, for simplicity.

    \itemtag{2B-sighting-and-engaging-missiles} The TSOH rules give requirements for engaging missiles include the missile being sighted (a) by the target when the missile starts its move or (b) by a friendly aircraft when the defender starts its move. However, the decision to engage or not occurs in the aircraft decisions phase, not the flight phase. I have changed both requirements to be “in the aircraft decisions phase.”

    \itemtag{2B-sighting-and-pre-emption} The sighting restrictions in the original rules state that a defender can pre-empt a moving aircraft if both are sighted by a third aircraft at the start of the \emph{defender's} movement. I have changed this to the start of the \emph{attacker's} movement.

    \itemtag{2B-vas-identification} VAS can only be used for identification in the \arcrange{180}{+} arc.

    \itemtag{2B-same-location-advantage} Change 1B-apj-23-errata added that aircraft “in the same hex” do not have any effect on the other's advantage. I have changed “hex” to “map location” to include hex sides.

    \itemtag{2B-preemptions} I have explicitly added that a preempting aircraft may not itself be preempted. There is no point, since preempting aircraft cannot make gun attacks, but making this statement is clearer.

    \itemtag{2B-limited-arcs} I changed the limited arcs to be consistent with each other. See the discussion in the airpower.io group starting on 28 February 2024.

    \itemtag{2B-maneuver-consecutive} I have added a statement that FPs for maneuvers must be consecutive.

    \itemtag{2B-maneuver-carry} I have added a statement that preparatory FPs for maneuvers can be carried from one turn to the next.

    \itemtag{2B-viff-vertical-pitch} I have added that when a VIFF vertical pitch is used to enter a vertical dive from a vertical climb, all of the HFPs must be used before any VFPs. This parallels the requirement for using a HRD to enter a vertical dive from a zoom or sustained climb.

    \itemtag{2B-missile-launch-modifiers} In the missile launch modifier table, I have changed the launch modifier for a critically damaged aiircraft from $+2$ to $+3$ to match the damage tables.
    
    \itemtag{2B-launch-restrictions} I have changed the restriction on missile launches from “firing guns during its last FP” to “firing guns or rockets after its last FP.”

    \itemtag{2B-follow-up-missiles} I have slightly changed how the FPs regained by a follow-on missile are assigned. I have stated that they are added to the initial complete proportional segments rather than the initial proportional segments. This is relevant when the missile has a life of only one game turn; we do not want an additional FP added to the incomplete proportional moves with the launch delay.

    \itemtag{2B-irm-vertical-fov} I have added that the vertical limits for the limited, \arcrange{180}{+}, and \arcrange{150}{+} arcs are used according to the horizontal field-of-view.

    \itemtag{2B-irm-envelopes} The rules on out-of-envelope IRM launches state that type A seekers may be launched at “large” targets, but I have changed this to “not large” targets (those with a visibility of less than 10), as this makes more sense from the context.

    \item{2B-radar-requirements} I have changed the requirement for radar work from “made an air to air gun attack” to “made an air to air gun or rocket attack”.

\end{itemize}

\section{Language Changes}

%\subsection{Copy Editing}
%
%I will apply the standard rules for capitalization, punctuation, and compound adjectives (e.g., “air-to-air attack”). I %will use the Oxford comma.

\subsection{Uniformity}

I will use “AP” and “DP” to refer to acceleration and deceleration points. This follows the uniform use of FP, HFP, and VPF.

\subsection{New Terminology}

I have adopted the following terminology:

\begin{itemize}
    \item Individually sighted. This means sighted or friendly, within sighting range of a specific aircraft, and not in the blind arc of that aircraft.
    \item Neutral. This replaces the nonadvantaged category.
    \item Superior and Inferior Aircraft. These replace positions of advantage and disadvantage.
    \item Advantage Categories. This replace position of advantage categories.
    \item Threatening and threatened. These are used to describe aircraft involved in preemptions.
\end{itemize}



\subsection{Replacing Ambiguous Terms}

\begin{itemize}
    \item \itemparagraph{Turn.} The original rules use “turn” to refer to both a game turn and turning flight. I will use “game turn” to refer to a game turn and “turn” to refer to turning flight.

    \item \itemparagraph{Within.} The original rules use “within” regarding ranges and other quantities. This is ambiguous. 
    
    For example, an aircraft must be “within four hexes” to conduct a rocket attack. Is this inclusive (i.e., “at a range of no more than four hexes” or $\textrm{range}\le4$) or exclusive (i.e., “at a range of less than four hexes” or $\textrm{range}<4$)? In the case of rocket attacks, the rocketry table resolves the ambiguity, which shows hit rolls for ranges of 1, 2, 3, and 4 hexes. Thus, here, within is inclusive. It is also apparently inclusive when used in the context of look-down limitations, which are expressed as “within four altitude levels” and “within 5 to 10 levels”.

    In the original rules, within is used in the context of:
    \begin{itemize}
    \item Aircraft horizontal and vertical separations in tactical formations.
    \item AP limits in military and afterburner power.
    \item The maximum range for rocket attacks.
    \item The maximum range for SSGT.
    \item The maximum range for visual sighting.
    \item When considering distances to determine if a line of sight is blocked by terrain.
    \item The maximum horizontal range to gain advantage.
    \item The relative facing of a tailed and tailing aircraft.
    \item The maximum range for defensive preemptions.
    \item The minimum and maximum ranges for missile launch envelopes.
    \item The maximum radar range.
    \item The altitude ranges for look-down limitations.
    \item The altitude difference for ground clutter for BRM/RHM/AHM.
    \item The range from the target at which an AIM-26A can be detonated.
    \item The altitude difference limits for jamming cell formations.
    \item Sun clutter.
    \item Parachute flares.
    \end{itemize}
    In all of these cases, I believe the use is inclusive.

\end{itemize}


%\subsection{Gender-Neutral Language}
% 
%Many of the terms used in the rules are already gender-neutral, and no change is required to these. However, we need to make the following substitutions: 
%\begin{itemize}
%    \item “they/them/their” for “he/him/his”
%    \item “crewmember” for “crewman”
%    \item “winger” for “wingman” (cf. section “leader”)
%\end{itemize}

\section{Process}

I will typeset the rules using LaTeX on Overleaf and track the source text using GitHub.

The source text for version 1A will be generated by taking Malcolm Pipe’s second-edition text, manually comparing it to the first-edition rules, and making any changes necessary to obtain agreement. (I believe permission to do this is implicit in his statement, “Edit as desired after download.” on 2021-02-05 to the airpower.io group.)

I will use custom LaTeX commands to introduce changes into the source text. These changes will be tagged and the tags described in this document. This gives traceability.

\section{Design}

{\AirSup} and {\AirStr} were published in GDW’s house style, with the main text is set in \href{https://en.wikipedia.org/wiki/Univers}{Univers} and the page having two columns with a rule between the columns and below the header. {\TSOH} followed this style to a large degree, but appears to have been set in \href{https://en.wikipedia.org/wiki/Helvetica}{Helvetica}

I want the rules to be readable and beautiful, but I also wish to preserve some of the design heritage of the earlier rules.

Therefore, I have maintained a layout with a two-column design, with a rule separating the columns and under the header, adapted the header from the GDW editions, and adapted the section format from the first-edition rules. 

However, I have replaced the san serif font with a serif font designed for readability. After considering several options, I have chosen the \href{https://www.gust.org.pl/projects/e-foundry/tex-gyre/schola}{Schola} font, based on the URW Century Schoolbook font and adapted for LaTeX by the GUST foundry as part of the TeX Gyre project. Schola has a version for math, which is useful for text like “$1 ≤ \CC ≤ 2$”. 

Of the other TeX Gyre fonts, the Palatino version would also be suitable, but Palatino is heavily used in {\itshape Birds of Prey}, and I wish to maintain a clear distinction.

\section{Rewriting and Reorganization}

This section collects my ideas for rewriting and reorganization.

\begin{itemize}
    \item Move the rules on formations to their own section.
    \item Move the modifications to the rules for aircraft with properties (e.g., LRR, HPR) to their own appendix.
    \item Move the rule on loss of thrust with altitude to the section on speed.
    \item  I think the following sections in particular could benefit from rewriting:
    \begin{enumerate}
        \item Recovery periods.
        \item Visual sighting.
        \item Electronic warefare.
    \end{enumerate}
    
\end{itemize}

\section{Outstanding Issues}

\subsection{Rule Interpretations}

\begin{itemize}

\item Does the use of a VFP during the preparation for a slide abort the slide?

\item Can an aircraft add additional preparatory FPs to a maneuver, stretching it out like it can a turn?

\item If a free aircraft under attack by an IRM selects idle power and fails the roll, what modifier does it use? The one for the previous power setting or the one for normal power? The consensus in the group is that it uses the modifier for normal power.

\item If a free aircraft under attack by an IRM selects normal power, does it automatically gain the corresponding modifier? The consensus in the group is that it can.

\item “If an aircraft's speed includes a fraction; say it is 4.5 instead of 5.0, round it up to simplify determining proportions but use its actual speed when moving. This would give the same result as in the above paragraph (5 into 17). The difference comes in the execution of the segments. The moves would be 4 for 1, 4 for 1, 3 for 1, 3 for 1, and then 3 for the aircraft's half FP.”

What the heck does “3 for the aircraft's half FP” mean?

If the aircraft has a speed of 4.5, then it has either 4 or 5 FPs, depending on whether it has 0.5 FP carry. If the aircraft has 5 FPs, than I can understand the above as meaning the missile moves 4, 4, 3, 3, and 3 FPs before each aircraft FP. If the aircraft has 4 FPs, then does it mean the missile moves 4, 4, 3, and 3 FPs before each aircraft FP and then 3 FPs after the last aircraft FP?

This issue would not occur if the proportional moves were calculated in terms of available FPs instead of speed. If the aircraft has 4 FPs and the missile 17 FPs, then the proportional moves would be 5, 4, 4, and 4 FPs. If the aircraft has 5 FPs and the missile 17 FPs, then the proportional moves would be 4, 4, 3, 3, and 3 FPs.

\item It seems to be an anomaly in that the missile has to be at the same altitude and one of the positions shown in the diagrams in order to attack, since in most circumstances a missile can use free descent. However, a missile with a turn rate of less than BT/2 cannot use free descent when it is transitioning from climbing to diving flight. 

It might make sense to say a missile attacks in one of four cases:
\begin{enumerate}
    \item The target moves to the same position (location and altitude) as the missile.
    \item The missile moves to the same position (location and altitude) as the target.
    \item The missile moves to one of the locations shown in the diagrams, has the same altitude as the target, and has at least one FP left in its proportional movement.
    \item The missile moves to one of the locations shown in the diagrams, is one altitude level above the target, has at least one FP left in its proportional movement, and can use free descent.
\end{enumerate}
The last case can be generalized to: can use the remaining FPs in its proportional move to move forward one hex and reach the same altitude as the target (using any allowed combination of climbs, dives, and free descents).

\end{itemize}

\subsection{Rule Changes}

\begin{itemize}
\item In Malcolm Pipe's second-edition text, VFPs can be used as preparatory FPs for rolls but not for slides. This follows JD Webster's Genie post in 1995.

\item The rules state that a second slide may be “performed” if the aircraft speed if 9.5 or more provided 4 FPs are expended between the two slides. 

An aircraft with a speed of 9.5 or more is supersonic and requires at least three preparatory FPs for each slide, so \emph{declaring and completing} two slide maneuvers requires at least 12 FPs (4 FPs for the first, 4 FPs waiting, and 4 FPs for the second). This requires a speed of at least 11.5 (allowing for 0.5 FP carry). 

I note that in the original {\AirSup} rules, supersonic aircraft are not required to expend an additional preparatory FP and so require only 10 FPs (3 FPs for the first slide, 4 FPs waiting, and 3 FPs for the second). This requires a speed of at least 9.5 (allowing for 0.5 FP carry). Thus, the limit does not seem to have been updated in the change from {\AirSup} to {\AirPow}.

The above ignores maneuver carry, which is not explicitly mentioned in the rules but seems to be widely adopted. If we ignore the explicit speed requirement and simply obey the waiting period of 4 FPs, then a first slide can be \emph{declared and completed} and a second slide can be \emph{declared} in 8 FPs (3 FPs for the first slide, 4 waiting, and 1 FP to declare the first slide). Furthermore, if sufficient preparatory FPs are carried in, the first slide can be \emph{completed} and the second \emph{declared} in only 6 FPs (1 FP to execute the slide, 4 FPs waiting, and 1 FP to begin the second slide).

\item Now that preparatory FPs for displacement and lag rolls can be VFPs, does it make sense to have the additional complication of the rule for climbing and diving barrel rolls? I suspect not.

\item Are the two DPs for VIFF-assisted turns in addition to the normal turn cost?

\item In the v2.4 rules, RWR-C+/D+ gives a +1 modifier to missile attacks and towed jammers give a +1 or +2 modifier to BRM/RHM/AHM attacks.

\end{itemize}

\end{document}


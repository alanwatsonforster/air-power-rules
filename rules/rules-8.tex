\rulechapter{Changing Aircraft Altitude}
\label{rule:changing-aircraft-altitude}

\notein{1B}{FH in APJ 36 has: “Rules 8.1, 8.2, \& 8.3: Climb/dive requirements, such as in vertical climbs/dives and in pulling out of vertical dives, are based on speed, not FPs.” I can't see what needs changing in the rules.}

\x{
This chapter details the procedures involved in changing aircraft altitudes.
}{
This rule details the procedures involved in changing aircraft altitudes.
}

\x{
\paragraph{Climbs or Dives Only.} 
An aircraft may either climb or dive in a turn, but it may not do both.  When climbing or diving, a portion of the aircraft's FPs are spent as Vertical Flight Points (VFPs); the rest are used as Horizontal Flight Points (HFPs). As an aircraft changes altitude during flight, mark the number of altitude levels gained or lost on the Aircraft Log. At the end of the aircraft's flight use this record to determine the aircraft's new start altitude for the next turn.
}{
\paragraph{Climbs or Dives Only.} 
An aircraft may either climb or dive in a turn, but it may not do both. 

}

\Dx{
\paragraph{Altitude Structure.} In the game, the atmosphere is divided into 1,000-foot levels. These altitude levels are further grouped into altitude bands, each several levels thick. Aircraft and missile performance vary within each altitude band. An aircraft's altitude at the beginning of a turn is noted on the Start Altitude line of the aircraft log. A missile's altitude at the beginning of a turn is noted on the Missile Notes line of the launching aircraft's aircraft log. An aircraft may not climb higher than its ceiling. The altitude structure is depicted \changedin{1C}{1C-tables}{below:}{in Table~\ref{table:altitude-bands}.}

\begin{onecolumntable}

\tablecaption{table:altitude-bands}{Altitude Bands}

\begin{tabular}{lll}
\hline
ALTITUDE BAND&CODE&LEVELS\\
\hline
Low             &LO &1 to 7\\
Medium Low      &ML &8 to 16\\
Medium High     &MH &17 to 25\\
High            &HI &26 to 35\\
Very High       &VH &36 to 45\\
Extremely High  &EH &46 to 60\\
Ultra High +    &UH &61 and Higher\\
\hline
\end{tabular}

\end{onecolumntable}

}

\paragraph{Vertical Flight Points.} 
\x{
FPs spent to gain or lose altitude levels are called Vertical Flight Points (VFPs). VFPs vary in the amount of altitude increase or decrease provided depending on the specific type of climb or dive used. VFPs are usually available as 1/3 or 2/3 of the total FPs; \changedin{1C}{1C-tables}{the 1/3-2/3 Table}{Table~\ref{table:fractions}} is a quick reference to the accepted proportions of VFPs to FPs available for most aircraft speeds. VFPs are only spent as full FPs. When \changedin{1C}{1C-tables}{the 1/3-2/3 Table}{Table~\ref{table:fractions}} provides half VFPs, they are ignored. \notein{1B}{AWF: APJ 21 QA, APJ 22 QA, and APJ 23 QA have comments on this, but they are superseded by the following change in the APJ 23 errata.}\changedin{1B}{1B-apj-23-errata}{A single VFP will never gain or lose more than two altitude levels.}{A single VFP will normally never gain or lose more than two altitude levels. (\changedin{2A}{2A-super-climbs}{The exception is in vertical dives where up to three levels may be lost per VFP.}{The exceptions are vertical dives, where up to three levels may be lost per VFP, and aircraft with CC values of 6.0 or more, which in zoom and sustained climbs can use one VFP to gain three levels.})}
}{

When climbing or diving, some of the aircraft's FPs are used as Vertical Flight Points (VFPs); the rest are used as Horizontal Flight Points (HFPs).

VFPs are used to gain or lose altitude levels. VFPs are usually available as 1/3 or 2/3 of the total FPs. Table~\ref{table:fractions} gives these fractions for all aircraft speeds. VFPs are only spent as full FPs. When Table~\ref{table:fractions} provides half VFPs, they are ignored.

A single VFP will gain or lose one, two, or three altitude levels. The number of levels depends on the specific type of climb or dive used. If a VFP can gain or lose a range of levels, the player can choose the precise value within that range for each VFP separately.  

\paragraph{Altitude Change Procedure.}
As an aircraft changes altitude during its flight, mark the number of altitude levels gained or lost on the aircraft log. At the end of the aircraft's flight, determine the aircraft's new start altitude for the next game turn.
}

\section{Climbing Flight}
\label{rule:climbing-flight}

\x{
An aircraft in climbing flight selects one of three types of climb to use: Zoom Climb, Sustained Climb, or Vertical Climb. Each type of climb prescribes how many VFPs and HFPs an aircraft must or may have available to it.
}{
An aircraft in climbing flight selects one of three types of climb: zoom, sustained, or vertical. 
}

\Ax{\paragraph{Climb Capability.} The climb capability (CC) quantifies the sustained climb rate of an aircraft. To determine the climb capability, refer to aircraft's ADC and cross-index the aircraft's altitude band, configuration, and power setting in the climb-capability chart.}

\Ax{
\paragraph{Supersonic Climbs.} If advanced rule~\ref{rule:supersonic-speed-effects} for supersonic flight is in use, the CC of an aircraft at supersonic speeds is 2/3 of that listed on the ADC
}



\subsection{Zoom Climbs}
\label{rule:zoom-climbs}

\x{
A zoom climb is a maneuvering climb in which the aircraft gains altitude more from inertia than wing lift. Some wing lift may be involved but most of it is being applied to aircraft maneuvering instead of altitude gain. In a zoom climb, the player is less restricted than in other climbs. Altitude is gained less efficiently though sometimes at a greater rate.

\begin{itemize}

    \item\itemparagraph{ZC Procedure:} Declare climbing flight and note ZC as flight type on the Aircraft Log. At least 1/3 of FPs must be HFPs\changedin{1B}{1B-apj-21-qa and Flight Rules Summary Page 1}{; the remainder may be VFPs.}{. At least one and at most $2/3$ of the FPs may be VFPs.}

    \item\itemparagraph{ZC Altitude Gain:} ZC VFPs produce an altitude gain based on the Climb Capability Chart. Refer to the CCC on the ADC and find the appropriate configuration, power setting, and altitude band. If the climb capability is 2.0 or less, \changedin{2A}{2A-super-climbs}{one}{each} VFP produces a gain of 1 altitude level; if the climb capability is more than 2.0\addedin{2A}{2A-super-climbs}{\ but less than 6.0}, \changedin{2A}{2A-super-climbs}{one}{each} VFP produces a gain of \changedin{1B}{1B-apj-23-errata}{1}{1 or 2} altitude levels\addedin{2A}{2A-super-climbs}{; and if the climb capability is 6.0 or more, one of the VFPs produces a gain of 1, 2, or 3 altitude levels and the rest produce a gain or 1 or 2 altitude levels}.

    \item\itemparagraph{ZC Restrictions:} Aircraft in a Zoom Climb may not use the ET turn rate.

    \item\itemparagraph{ZC Decel Points:} \changedin{2A}{2A-zoom-climb}{In the first turn of \changedin{1B}{1B-apj-36-errata}{a Zoom Climb, the aircraft}{climbing flight, an aircraft in a ZC} receives 1.0 decel point per altitude level gained; on \changedin{1B}{1B-apj-36-errata}{subsequent consecutive turns of ZC, the aircraft}{the second or subsequent consecutive turns of climbing flight, an aircraft in a ZC} receives 1.5 decel points per altitude level gained.}{An aircraft in a ZC receives 1.0 DP per altitude level gained.}

\end{itemize}

}{
A zoom climb (ZC) is a maneuvering climb in which the aircraft gains altitude more from inertia than wing lift. Some wing lift may be involved, but most of it is applied to aircraft maneuvering instead of altitude gain. The aircraft is less restricted in a zoom climb than in other climbs. Altitude is gained less efficiently, though sometimes at a greater rate.

\paragraph{ZC Procedure.} Declare climbing flight and note ZC as flight type on the aircraft log sheet. At least 1/3 of the FPs must be HFPs. 

\paragraph{ZC FPs.} At least one and at most 2/3 of the FPs may be VFPs. ZC VFPs produce an altitude gain based on the aircraft's CC:
\begin{itemize}
    \item If the CC is 2.0 or less, each VFP produces a gain of 1 altitude level.
    \item If the CC is more than 2.0 but less than 6.0, each VFP produces a gain of 1 or 2 altitude levels.
    \item If the CC is 6.0 or more, one of the VFPs produces a gain of 1, 2, or 3 altitude levels and the rest produce a gain or 1 or 2 altitude levels.
\end{itemize}

\paragraph{ZC Restrictions.} An aircraft in a ZC may not use the ET turn rate.

\paragraph{ZC DPs.} An aircraft in a ZC receives 1.0 DPs per altitude level gained
}


\subsection{Sustained Climbs}
\label{rule:sustained-climbs}

\x{

A sustained climb relies on maximum power and full lift and is the most efficient way to climb over several game turns. Sustained climbs do restrict the aircraft's maneuverability.

\begin{itemize}

    \item\itemparagraph{SC Procedure:} Declare climbing flight and note SC as flight type on the Aircraft Log.
    
    \item\itemparagraph{SC Altitude Gain:} SC VFPs produce an altitude gain based on the Climb Capability Chart. Refer to the CCC and find the appropriate configuration, power setting, and altitude band. \changedin{2A}{2A-super-climbs}{Three}{Four} cases may apply:
    \begin{enumerate}
        \item If the climb capability value is a fraction, only 1 VFP is allowed, the rest are HFPs. The VFP gains only the fractional altitude level.
        \item If the \changedin{1B}{1B-apj-21-qa and 1B-apj-23-errata}{value is 1.0 to 1.5}{CCC value is 1.0 to 2.0}, then up to 2/3ds the FPs may be VFPs and the first VFP gains \changedin{1B}{1B-apj-36-errata}{any fraction}{either any fraction or 1 level} and the rest gain 1 level each. \addedin{1B}{1B-apj-38-qa}{(For example, if the aircraft has a climb capability of 1.5, the first VFP used in a game turn gains only the 0.5 level, and any others gain 1.0 level each.)}
        \item If the \changedin{2A}{2A-super-climbs}{\changedin{1B}{1B-apj-21-qa and 1B-apj-23-errata}{value is 2.0 or more}{CCC value is greater than 2.0}}{climb capability is greater than 2.0 but less than 6.0}, then up to 2/3ds the FPs may be VFPs and \changedin{1B}{1B-apj-34-qa and play aids}{each may gain 1 or 2 levels}{the first gains 1 level plus any fraction and the rest gain 1 or 2 levels each}.
        \itemaddedin{2A}{2A-super-climbs}{If the climb capability is 6.0 or more, then up to {\twothirds} of the FPs may be VFPs. The first VFP gains 1 level plus any fraction, one of the others gains 1, 2, or 3 altitude levels, and the rest gain 1 or 2 altitude levels each.}
    \end{enumerate}
    \addedin{2B}{2B-steep-climb}{At least one FP must be a VFP.}

    \item\itemparagraph{SC Prerequisites and Limits:} To use a sustained climb the aircraft must have a start speed at least 1.0 greater than its minimum speed. If the start speed is less than the aircraft's optimum climb speed, the CCC values are halved (retain fractions). Sustained climb decel applies only to an amount of levels gained equal to the CCC value (halved if applicable).

    \item\itemparagraph{SC Excess Altitude Gain:} Aircraft may expend VFPs to climb more levels than listed or normally allowed if sufficient VFPs are available. However, any levels gained beyond the listed or allowed CCC limits incur decel points as if the aircraft were zoom climbing instead of sustained climbing.

    \item\itemparagraph{SC Restrictions:} Aircraft in a sustained climb may only use EZ turn rates \deletedin{2A}{2A-snap}{(snap turning prohibited)} and may only use slide maneuvers.

    \item\itemparagraph{SC Decel Points:} 0.5 Decel points are incurred for each altitude level gained in a sustained climb until the sustained climb limit is reached and then decel is accumulated as if zoom climbing\addedin{2A}{2A-zoom-climbs}{\ at 1.0 DP per altitude level}. \notein{1B}{AWF: the TSOH errata calls for the clarification on decel for fraction gains to be added to the text on sustained climbs. I have added it below to the text of partial altitude gains.}

    \notein{1B}{FH in APJ 36 states: “Excess altitude levels gained in sustained climb cost 1.5 decel per level if \emph{any} kind of climbing flight was performed on the previous turn.” The corresponding change has been made above for ZCs, and so is implied by the phrase "as if zoom climbing" here.}

    \addedin{1B}{1B-apj-36-errata}{For purposes of computing SC decel, 
    on the turn an aircraft attains a whole altitude level by accumulating fractional climbs, its climb capability has any fraction rounded up to the next whole number. Otherwise, its climb capability has any fraction rounded down.}

\end{itemize}

Example of a SC with excess altitude gain: A MiG-21 with a speed of 6.0 in the LO band, CL configured, and at AB power has a CCC value of 4. It may have up to 4 VFPs and chooses to do so. Since the CCC value is greater than 2.0 it may gain two levels per VFP. The player elects to move as follows; H, H, V+2, V+2, V+1, V+1 (moves forward two hexes and uses the four VFPs to gain 6 altitude levels). 0.5 Decel is incurred for each of the first four levels gained (the amount = to CCC value), and 1.0 decel for each of the last two (the amount exceeding CCC value). Total decel for climbing is 4.0.

}{

A sustained climb relies on lift and is the most efficient way to climb over several game turns. Sustained climbs do restrict the aircraft’s maneuverability.

\paragraph{SC Requirements.} To use a SC, an aircraft must have a start speed of at least 1.0 greater than its minimum speed. If the start speed is less than the aircraft’s climb speed according to the ADC, its CC value is halved (retain fractions).

\paragraph{SC Procedure.} Declare climbing flight and note SC as flight type on the aircraft log sheet.

\paragraph{SC FPs.} At least one FP must be a VFP. The number of VFPs available and the altitude gain they produce depends on the aircraft’s CC:

\begin{itemize}
    \item If the CC is less than 1.0, only 1 VFP is allowed, and the rest are HFPs. The VFP gains only the CC value in levels.

    \item If the CC is 1.0 or more and 2.0 or less, up to 2/3 of the FPs may be VFPs.
If the CC is exactly 1.0 or 2.0, each VFP gains 1 level. Otherwise, the first VFP gains the fractional part of the CC, and the rest gain 1 level each. (For example, if the aircraft has a CC of 1.5, the first VFP gains only 0.5 levels, and the others gain 1.0 levels each.) 

    \item If the CC is more than 2.0 but less than 6.0, up to 2/3 of the FPs may be VFPs. The first gains 1 level plus the fractional part of the CC, and the rest gain 1 or 2 levels each. 

    \item If the CC is 6.0 or more,  up to 2/3 of the FPs may be VFPs. The first gains 1 level plus the fractional part of the CC, one of the others gains 1, 2, or 3 altitude levels, and the rest gain 1 or 2 altitude levels each. 

\end{itemize}

\paragraph{SC Restrictions.} An aircraft in a SC may only use EZ turn rates and slide maneuvers.

\paragraph{SC DPs.} An aircraft in an SC receives 0.5 DPs for each altitude level gained up to its CC and 1.0 DPs for each altitude level gained in excess of its CC.

For computing DPs, on the game turn in which an aircraft attains a whole altitude level by accumulating fractional level, its climb capability has any fraction rounded up to the following whole number. Otherwise, its climb capability has any fraction rounded down. 

Example: Consider a MiG-21 with a speed of 6.0 in the LO band, CL configured, and using afterburner power. It has a CC of 4.0. It may use up to 4 VFPs, and chooses to use all of them. Since its CC is more than 2.0, it may gain two levels per VFP. The player elects to move as follows; H, H, V+2, V+2, V+1, V+1 (i.e., move forward two hexes and use the four VFPs to gain six altitude levels). The aircraft receives 0.5 DPs for each of the first four levels gained (up to the CC) and 1.0 DPs for each of the last two (in excess of the CC). It receives a total of 4.0 DPs for climbing.
}

\subsection{Vertical Climbs}
\label{rule:vertical-climbs}

\x{

A vertical climb gains altitude quickly but at great cost in energy; no wing lift is involved. The aircraft is coasting upward on power and inertia.

\begin{itemize}

    \item\itemparagraph{VC Prerequisite:} A Vertical climb may be selected only if the aircraft climbed in the previous game turn. Exception: A High Pitch Rate aircraft (if its current speed is less than 4.0) may declare a Vertical Climb from Level Flight.

    \item\itemparagraph{VC Procedure and Limits:} Declare climbing flight and note VC as flight type on the aircraft log. On the first turn of VC, exactly 1/3 of FPs must be HFPs; the remainder are VFPs. On the second or subsequent turns of a consecutive VC, no more than 1/3 of FPs may be HFPs (and up to all FPs may be VFPs).

    \item\itemparagraph{VC Altitude Gain:} All aircraft may gain 1 or 2 altitude levels per VFP regardless of normal climb ability.

    \item\itemparagraph{VC Restrictions:} No turns or maneuvers except Vertical Rolls are allowed. The aircraft may not dive on the game turn following a vertical climb. Exceptions: A High Pitch Rate aircraft may freely steep dive or use unloaded dives following a vertical climb. A non-High Pitch Rate aircraft may use the Half-Roll Dive and vertical Reverse maneuvers to enter diving flight (see Ch 13) after vertical climbs.

    \item\itemparagraph{VC Decel Points:} The aircraft receives \changedin{2A}{2A-zoom-climbs}{2}{1.5} decel points for each altitude level gained.

\end{itemize}

}{

A vertical climb gains altitude quickly using power and inertia, but at a great cost in energy as no wing lift is used.

\paragraph{VC Requirements.} To use a VC, an aircraft must have climbed in the previous game turn, unless its ADC notes that it has a high pitch rate and its speed is 3.5 or less, in which case it may enter a vertical climb from level flight.

\paragraph{VC Procedure.} Declare climbing flight and note VC as flight type on the aircraft log sheet. 

\paragraph{VC FPs.} On the first game turn of VC, exactly 1/3 of FPs must be HFPs; the remainder are VFPs. On the second or subsequent consecutive game turns of a VC, no more than 1/3 of FPs may be HFPs and up to all FPs may be VFPs. All aircraft may gain 1 or 2 altitude levels per VFP regardless of their CC.

\paragraph{VC Restrictions.} An aircraft in a VC may not perform turns or maneuvers except for vertical rolls. 

\paragraph{VC DPs.} An aircraft in a VC receives 1.5 DPs for each altitude level gained.

\paragraph{VC Recovery.} An aircraft in a VC may not dive on the following game turn, unless:

\begin{itemize}

    \item Its ADC notes that it has a high pitch rate, in which case it may freely use a steep or unloaded dive on the following game turn.

    \item It uses a half-roll and dive (see rule~\ref{rule:half-rolls-and-dives}) or successfully performs a vertical reverse (see rule~\ref{rule:vertical-reverses}) on the following game turn to enter diving flight.
    
\end{itemize}
}


\x{
\subsection{Additional Considerations}

\paragraph{Partial Altitude Gains.} The CCC at times indicates fractional altitude gains. You will have noticed that the climb charts sometimes allow fractional gains in altitude levels. Some aircraft may require more than one turn of climbing flight to gain an altitude level. An aircraft's starting altitude is always the last full altitude level it climbed to. \addedin{1B}{1B-apj-23-errata}{Decel for climbs is only incurred for each full level climbed through. With fractional climbs, only incur decel when enough turns of climbing have passed to gain a full level.}

\paragraph{Altitude Carry.} If an aircraft's total altitude change during flight included a fractional amount, the fraction is carried forward to the next game-turn to be added to any further climbing. Note this on the climb notes line of the log sheet. Fractional gains may be carried and added only as long as the aircraft continues to climb from turn to turn. The moment an aircraft chooses level or diving flight, any fractional climb carry is lost. Climb carry is ignored when determining an aircraft's altitude for spotting, combat or any other purposes.

}{

\subsection{Fractional Altitude Gains}

\paragraph{Fractional Gains.}
Sustained climbs can result in an aircraft gaining a fraction of an altitude level. Such gains are added to any that have been carried into the current game turn. If the sum is one or more, the aircraft gains a full level. This may take several game turns. 

\paragraph{Carrying Fractional Gains.}
Any remaining fractional gain is carried forward to the next game turn. Note it on the altitude carry line of the aircraft log sheet. Fractional gains are carried forward as long as an aircraft continues to use climbing flight. Fractional gains are lost if an aircraft chooses level or diving flight.


\paragraph{DPs for Fractional Gains.}
An aircraft does not receive DPs for fractional level gained. It only receives DPs for each full level gained.

\paragraph{Altitude and Fractional Gains.}
Fractional gains are ignore for all other purposes. A climbing aircraft's altitude is always the last full level gained.

Example. Consider an aircraft with an altitude of 10, a CC of 0.25, and initially no climb carry. If performs a sustained climb over several game turns. Each game turn it receives one VFP that allows it to gain 0.25 levels. It will carry 0.25 levels into the second game turn, 0.50 into the third, and 0.75 into the fourth. In the fourth game turn, the 0.25 levels gained by the VFP will combine with the 0.75 levels carried in to give a gain of one full level. The aircraft will now have an altitude of 11 and will receive 0.5 DPs.

}

\Dx{
\paragraph{Supersonic Climbs.} If the advanced rules for supersonic flight are in use, the CCC value is reduced to 2/3ds that listed on the ADC when aircraft are at supersonic speeds.
}

\section{Diving Flight}
\label{rule:diving-flight}

\x{

An aircraft in diving flight selects one of three types of dive: Steep Dive, Unloaded Dive, or Vertical Dive. Diving flight is handled in a manner similar to climbing flight. \changedin{1B}{1B-apj-21-qa and 1B-apj-36-errata}{For purposes of maintaining dive speeds (rule 6), at least two or more altitude levels must be lost in a game turn through diving flight.}{

For purposes of maintaining dive speeds (rule 6), at least two or more altitude levels must be lost in a game turn through diving flight. Aircraft that dive fewer than 2 altitude levels in a turn do not use dive speed as their maximum speed, and may accordingly be forced to perform a Speed Fadeback.}

}{

An aircraft in diving flight selects one of three types of dive: steep, unloaded, or vertical. 

\paragraph{Dive Speed.} To be limited by its dive speed (rule~\ref{rule:speed-limits}), an aircraft must lose at least two altitude levels per game turn through diving flight. An aircraft that only loses one altitude level in a game turn is limited in speed as if in level flight.

}

\subsection{Steep Dives}
\label{rule:steep-dives}

\x{

A steep dive is a maneuvering dive in which some of the aircraft's acceleration is committed toward maneuvering the aircraft instead of speeding up. A steep dive is the least restrictive type of dive.

\begin{itemize}

    \item\itemparagraph{SD Procedure:} Declare diving flight and note SD as flight type on the aircraft log. At least 1/3 of FPs must be HFPs. \changedin{1B}{1B-apj-21-qa and Flight Rules Summary Page 1}{The rest may be VFPs.}{At least one and at most $2/3$ of the FPs may be VFPs.}

    \item\itemparagraph{SD Altitude Loss:} 1 or 2 altitude levels may be lost per VFP expended.

    \item\itemparagraph{SD Restrictions:} There are no restrictions to maneuvering the aircraft while in a steep dive.

    \item\itemparagraph{SD Accel Points:} \changedin{2A}{2A-steep-dives}{On the first turn of \changedin{1B}{1B-apj-36-errata}{steep diving, the aircraft}{diving flight, an aircraft in a SD} receives 0.5 accel points per altitude level lost; on \changedin{1B}{1B-apj-36-errata}{subsequent turns of continued diving, it}{the second or subsequent consecutive turns of diving flight, an aircraft in a SD} receives 1.0 accel points per altitude level lost.}{An aircraft in a SD receives 1.0 AP per altitude level lost.}

\end{itemize}

}{

A steep dive is a maneuvering dive in which some of the aircraft's acceleration is committed to maneuvering the aircraft instead of increasing speed. A steep dive is the least restrictive type of dive.

\paragraph{SD Procedure.} Declare diving flight and note SD as flight type on the aircraft log sheet. 

\paragraph{SD FPs.} At least one and at most $2/3$ of the FPs may be VFPs. SD VFPs produce an altitude loss of  1 or 2 altitude levels per VFP.

\paragraph{SD Restrictions.} An aircraft in an SD suffers no additional restrictions.

\paragraph{SD APs.} An aircraft in an SD receives 1.0 APs per altitude level lost.

}

\subsection{Unloaded Dives}
\label{rule:unloaded-dives}

\x{
An unloaded dive is used to rapidly gain acceleration. The aircraft dives to match the fall of gravity; this causes weightlessness and eliminates induced drag from the aircraft. Combined with acceleration from gravity and the engine's thrust, the aircraft achieves rapid gains in distance and speed.

\begin{itemize}

    \item\itemparagraph{UD Procedure:} \changedin{2B}{2B-unloaded-dives}{\changedin{2A}{2A-unloaded-dives}{Declare diving flight and note UD on the aircraft log. All FPs are HFPs. All or some (but at least some) of the HFPs must be spent with the aircraft “unloaded”. Each HFP spent while unloaded moves the aircraft forward one hex or hexside and causes it to lose one altitude level.}{Declare level flight and note UD on the aircraft log. One or two FPs may be VFPs and the rest are HFPs. HFPs may be expended with the aircraft loaded or unloaded. VFPs are expended with the aircraft unloaded. The first VFP may only be used after at least half of the HFPs have been expended with the aircraft unloaded. The second VFP may only be used after all of the HFPs have been expended with the aircraft unloaded.}}{Declare diving flight and note UD on the aircraft log. All FPs are HFPs. HFPs may be expended with the aircraft loaded or unloaded. At least half (round down) of the HFPs must be spent with the aircraft unloaded. The aircraft loses one altitude level after half of the HFPs (round down) have been used unloaded. If all of the HFPs are used unloaded, the aircraft loses another altitude level after the last HFP.}

    \item\itemparagraph{UD Limits:} All unloaded \changedin{2A}{2A-unloaded-dives}{HFPs}{FPs} must be expended in a continuous series. They may be spent at the beginning, end or in the middle of the aircraft's flight. The rest of the HFPs may be expended normally for maneuvering purpose. \changedin{2A}{2A-unloaded-dives}{Unloaded HFPs may not be counted toward any turning or maneuvering requirements.}{Unloaded FPs may not be used for turning flight or for preparing for or executing any maneuver except a slide.}

    \item\itemparagraph{UD Restrictions:} An aircraft spending unloaded HFPs may not conduct any attacks, aim, track targets or launch weapons. \addedin{1B}{1B-apj-36-errata}{Recovery Periods also apply to the prohibition from conducting combat actions and radar work following unloaded dives. If you unload at the beginning of turn and then expend sufficient FPs to meet the recovery period after the last unloaded FP is spent, then you may still launch missiles, conduct attacks or radar work that turn.}

     \item\itemparagraph{UD Accel Points:} \changedin{2A}{2A-unloaded-dives}{On the first turn of \changedin{1B}{1B-apj-36-errata}{ an UD, the aircraft}{diving flight, an aircraft in a UD} receives 0.5 accel points per altitude level lost; on \changedin{1B}{1B-apj-36-errata}{subsequent turns of continued diving, it}{the second or subsequent consecutive turns of diving flight, an aircraft in a UD} receives 1.0 accel point per altitude level lost.}{An aircraft in a UD receives 1.0 AP per altitude level lost.}

\end{itemize}

Note: The advantage to unloaded over steep dives is the horizontal distance gained over similar dives.

}{

An unloaded dive is used to accelerate rapidly while maintaining horizontal speed. The aircraft dives to match the fall of gravity, which causes weightlessness and eliminates induced drag. The reduced drag, combined with acceleration from gravity and the engine's thrust, gives the aircraft rapid acceleration. The advantage of unloaded dives over steep dives is the horizontal distance traveled.

\paragraph{UD Procedure.} Declare diving flight and note UD on the aircraft log sheet. 

\paragraph{UD FPs.} All FPs are HFPs. HFPs may be expended with the aircraft loaded or unloaded. At least half (round down) of the HFPs must be spent with the aircraft unloaded. Unloaded HFPs must be expended consecutively. The aircraft loses one altitude level after half of the HFPs (round down) have been used unloaded. If all of the HFPs are used unloaded, the aircraft loses another altitude level after the last HFP.
   
\paragraph{UD Restrictions.} Unloaded FPs may not be used for turning flight or preparing for or executing any maneuver except a slide. An aircraft using unloaded HFPs may not conduct any attacks, aim, track targets, launch weapons, or use its radar until after a recovery period. If an aircraft unloads early in its move and then completes the recovery period before or as its last unloaded FP, it may still launch missiles, conduct attacks, and use its radar in the subsequent phases of the game turn.

\paragraph{UD APs.} An aircraft in a UD receives 1.0 APs per altitude level lost.

}

\subsection{Vertical Dives}
\label{rule:vertical-dives}

\x{
A vertical dive sends an aircraft nearly straight down; altitude is lost quickly and acceleration builds up rapidly.

\begin{itemize}

    \itemaddedin{1B}{1B-apj-21-qa, 1B-apj-22-qa, and 1B-apj-23-errata}{\itemparagraph{VD Prerequisites.} To enter a VD, the aircraft must have used diving flight the turn before (exception: the Half Roll and Dive maneuver allows VDs to be entered from level flight, see rule 13.3.5).}

    \item\itemparagraph{VD Procedure:} Declare diving flight and note VD on the Aircraft Log. If this is the first turn of vertical diving, 1/3 of the FPs must be HFPs and the rest VFPs. If this is the second or subsequent turn of consecutive vertical dives, then no more than 1/3 of the FPs can be HFPs but all can be VFPs.

    \item\itemparagraph{VD Altitude Loss:} In a vertical dive, 2 or 3 altitude levels must be lost for each VFP expended.

    \item\itemparagraph{VD Restrictions:} Aircraft in a VD may not do turns or use maneuvers except for Vertical Rolls. Climbing flight is not allowed on the turn following a vertical dive. Vertical dives must be followed on subsequent game turns by diving flight.

    \item\itemparagraph{VD Recovery:} Due to the difficulty of pulling out of vertical dives, the following applies:
    \begin{enumerate}
        \item[a)] When a steep or unloaded dive immediately follows a vertical one, half the aircraft's \changedin{1B}{1B-apj-23-errata}{speed}{FPs} (round down) must be VFPs or unloaded HFPs as appropriate.
        \item[b)]  In the case of a High Pitch Rate aircraft, only 1/3 has to be VFPs or unloaded HFPs. Exception: A High Pitch Rate aircraft whose start speed after vertically diving is 3.0 or less may use level flight following a vertical dive. \addedin{1B}{1B-original-play-aids}{Other aircraft whose start speed is 2.0 or less may use level flight.}
    \end{enumerate}

    \item\itemparagraph{VD Accel Points:} Aircraft in a vertical dive gain 1 Accel point per altitude level lost.

\end{itemize}
}{

A vertical dive sends an aircraft nearly straight down. It loses altitude quickly and accelerates rapidly.

\paragraph{VD Requirements.} To use a VD during normal flight, the aircraft must have used diving flight on the previous game turn, use a half-roll and dive (rule~\ref{rule:half-rolls-and-dives}), or successfully perform a vertical reverse (rule~\ref{rule:vertical-reverses}). An aircraft can also enter a VD after recovery from stalled or departed flight (rule~\ref{rule:abnormal-flight}) .

\paragraph{VD Procedure.} Declare diving flight and note VD on the aircraft log sheet. 

\paragraph{VD FPs.} If this is the first turn of vertical diving, 1/3 of the FPs must be HFPs and the rest VFPs. If this is the second or subsequent turn of consecutive vertical dives, then no more than 1/3 of the FPs can be HFPs, but all can be VFPs. Each VFP produces a loss of 2 or 3 altitude levels.

\paragraph{VD Restrictions.} An aircraft in a VD may not turn or use maneuvers except for vertical rolls. 

\paragraph{VD APs.} An aircraft in a VD gains 1.0 APs per altitude level lost.

\paragraph{VD Recovery.} Due to the difficulty of pulling out of a VD, an aircraft that ceases to a VD and changes to a different flight type may not climb and must obey the following restrictions:
\begin{itemize}
    \item An aircraft whose start speed is 2.0 or less may use level flight.
    \item An aircraft with a high pitch rate whose start speed is 3.0 or less may use level flight.
    \item An  aircraft with a high pitch rate may use a steep dive in which at least 1/3 (round down) of the FPs are VFPs.
    \item Other aircraft must use a steep dive in which least 1/2 (round down) of the FPs are VFPs.
\end{itemize}

}

\x{
\section{Free Descent}
}{
\section{Free Descent}
\label{rule:free-descent}
}

\x{
\paragraph{Level Flight Free Descents.} An aircraft in level flight may choose free descent and lose one altitude level during the game-turn. \changedin{1B}{1B-apj-36-errata}{This descent}{If a vertical climb was used in the preceding turn, this descent may not occur until after the aircraft has expended HFPs equal to 1/3 of its \changedin{2B}{2B-free-descent}{speed}{HFPs} (according to the 1/3-2/3 chart). Otherwise, it} may be selected to take place after the expenditure of any HFP during the game-turn. No Accel points are received. No other restrictions apply.
}{
\paragraph{Free Descent in Level Flight.} An aircraft in level flight may choose to take a free descent and lose one altitude level during its move. 

If the aircraft used a vertical climb in the preceding turn, this descent may not occur until after it has expended 1/3 (round down) of its HFPs. Otherwise, it may occur after any HFP. 

An aircraft taking a free descent receives no APs and suffers no additional restrictions.
}

\trainingnote{
\centering
\x{
You are now ready to play Training Scenario 2.

The sequence of play is still not required.
}{
You are now ready to play training scenario 2.

The sequence of play is still not required, and you may ignore it.
}
}

\begin{advancedrules}

\deletedin{2B}{2B-half-vfps}{
\section{Using Half VFPs}

\notein{1B}{FH in APJ 36 states “If an A/C has a fractional speed and a 0.5 FP carry available, it must use the carry to create a full HFP or VFP.” I don't see what needs changing in the text of the rules.}

In normal practice, VFPs are spent only as full FPs. When the 1/3-2/3 Table allots half VFPs, the aircraft may either:

\begin{itemize}

    \item Carry the half VFP forward to the next turn as a generic half FP.	

    \item Mate the half VFP to a previously carried half FP to create a full VFP \addedin{1B}{1B-apj-23-errata}{or HFP} and use it.

    \itemdeletedin{1B}{1B-apj-23-errata}{Steal a half VFP from the alloted HFPs to create a full VFP, and carry the remaining half HFP forward to the next turn.}

\end{itemize}
}

\Dx{
\section{Loss of Thrust with Altitude}

Jet engines lose thrust in the thinner air at high altitudes. To reflect this, use the following:

\begin{itemize}

    \item In the VH band, thrust is 2/3 normal (but never less than 0.5).

    \item In the EH and UH bands, thrust is 1/3 normal (but never less than 0.5).

\end{itemize}

The 1/3-2/3 Chart is useful in calculating reduced thrust.

\paragraph{High Altitude Engines.} Some aircraft are noted as having engines specifically designed for high altitude flight. These aircraft ignore this rule.
}

\section{Flight Above Maximum Ceiling}

\x{
Each aircraft has a ceiling (maximum altitude) stated on the MMVC. An aircraft may temporarily ZC or VC above its ceiling (SC cannot carry an aircraft above its ceiling). While above its ceiling, an aircraft is subject to the following risks:

\begin{itemize}

    \item If the aircraft uses any turn rate other than EZ, the aircraft experiences a Maneuvering Departure on a die roll of $4-$.
    
    \item If the aircraft performs a roll maneuver, the aircraft experiences a Maneuvering Departure on a die roll of $4-$.

    \item If the aircraft starts the game turn above its ceiling and selects any power other than Idle, each engine will Flame-out on die roll of $4-$ (apply a modifier of $-1$ for each turn the aircraft starts above its ceiling). Check for Flame-out immediately upon selecting the power setting.

\end{itemize}
}{
An aircraft's ceiling or maximum altitude is given in the MMVC of its ADC.

An aircraft may use a ZC or VC to climb above its ceiling, but a SC cannot carry it above its ceiling. 

While above its ceiling, an aircraft is subject to the following risks:

\begin{itemize}

    \item If the aircraft uses any turn rate other than EZ or performs a roll maneuver, it suffers a maneuvering departure (see rule~\ref{rule:maneuvering-departures}) on a die roll of $4-$.
    
    \item If the aircraft starts the game turn above its ceiling and uses any power other than idle, each engine will flame out (see rule~\ref{rule:engine-flame-outs}) on die roll of $4-$, with a modifier of $-1$ for each consecutive game turn it has started above its ceiling. Check immediately after selecting the power setting.

\end{itemize}
}

\Dx{
\section{Formation Restrictions on Climbs and Dives}

\paragraph{Close Formations.} Aircraft in close formation may only change altitude using non-AB powered sustained climbs, or steep dives of no more than two altitude levels per turn, or by free descents. If these limits are exceeded, the formation automatically breaks down into tactical formation and collisions are possible if more than two aircraft end the turn in the same position.

\paragraph{Tactical Formations.} Aircraft in Tactical formations have no altitude change restrictions.
}

\end{advancedrules}

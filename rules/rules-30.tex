\CX{
\rulechapter{Night and Adverse Weather}
}{
\rulechapter{Night, Sun, and Weather}
}
\label{rule:night-and-weather}

\CX{
This chapter details the effects of various weather phenomena on aircraft flight and combat. This entire chapter is an \emph{ADVANCED RULE}.
}{
This rule describes the impact of night, the Sun, and weather on aircraft flight and combat. This entire chapter is an advanced rule.
}

\begin{advancedrules}

\DX{
Weather is an important consideration in air combat. Clouds offer concealment from visual and IR guided weapons, contrails give away aircraft positions, night or adverse weather conditions can hinder or limit aircraft flight, or the Sun can dazzle a pilot or draw off an IR missile. The following rules address these factors.
}

\CX{
\section{Common Weather}
\silentlyaddedin{1C}{1C-tables}{
    \changedin{1C}{1C-tables}{

\begin{table}[!ht]
\centering
WEATHER GENERATION TABLES

\bigskip
GEOGRAPHIC AREA\\
(roll one die for sky condition)

\medskip
\begin{tabular}{p{2cm}*{4}{p{1.2cm}}}
\hline
Sky Type     &Far North/South    &Middle Regions &Desert Areas &Tropical Areas  \\
\hline
Clear        &1 to 4             &1 to 5         &1 to 8       &1 to 6          \\
Broken       &5 to 7             &6 to 8         &9 to 10      &7               \\
Overcast     &8 to 10            &9 to 10        &NA           &8 to 10         \\
Contrail Alt.&20                 &25             &30           &25              \\
\hline
\end{tabular}

\bigskip

ACTUAL WEATHER\\
(roll die to determine haze and cloud layers)

\medskip
\begin{tabular}{p{1.0cm}*{3}{p{2.2cm}}}
\hline
Die Roll    &Clear      &Broken         &Overcast               \\
\hline
    1	    &LO Hz      &LO Hz, 3 Str.  &LO Hz, 3 Str., 1 Dns.   \\
    2       &ML Hz      &ML Hz, 1 Dns.  &LO Hz, 3 Str., 1 Dns.   \\
    3	    &1 Str.     &ML Hz, 1 Dns.  &LO Hz, 2 Str., 2 Dns.   \\
    4	    &1 Str.     &1 Str., 1 Dns. &1 Str., 3 Dns.         \\
    5	    &2 Str.     &2 Str., 1 Dns. &2 Str., 2 Dns.         \\
    6	    &---        &3 Str,         &2 Str., 2 Dns.          \\
    7       &---        &4 Str.	        &2 Str., 2 Dns.         \\
    8  	    &---        &HI Hz, 1 Str.	&ML Hz., 1 Dns.	        \\
    9	    &---        &LO Hz, 1 Str.  &2 Dns.                 \\
   10	    &---	    &2 Str.	        &1 Dns.	                \\
\hline
\end{tabular}
\end{table}

}{

\begin{twocolumntable}
\tablecaption{table:climate-zones}{Weather Generation: Geographic Area}
\begin{tabularx}{0.7\linewidth}{l*{4}{C}}
\toprule
Sky Type     &Far North/South    &Middle\break Regions &Desert\break Areas &Tropical\break Areas  \\
\midrule
Clear        &\phantom{0}1 to 4\phantom{0} &\phantom{0}1 to 5\phantom{0} &\phantom{0}1 to 8\phantom{0} &\phantom{0}1 to 6\phantom{0} \\
Broken       &\phantom{0}5 to 7\phantom{0} &\phantom{0}6 to 8\phantom{0} &\phantom{0}9 to 10           &7                            \\
Overcast     &\phantom{0}8 to 10\phantom{} &\phantom{0}9 to 10\phantom{} &NA                           &\phantom{0}8 to 10\phantom{} \\
Contrail Alt.&20                 &25             &30           &25              \\
\bottomrule
\end{tabularx}
\end{twocolumntable}

\begin{twocolumntable}
\tablecaption{table:haze-and-clouds}{Weather Generation: Actual Weather}
\begin{tabularx}{0.7\linewidth}{lLLL}
\toprule
Die Roll    &Clear      &Broken         &Overcast               \\
\midrule
    1	    &LO Hz      &LO Hz, 3 Str.  &LO Hz, 3 Str., 1 Dns.   \\
    2       &ML Hz      &ML Hz, 1 Dns.  &LO Hz, 3 Str., 1 Dns.   \\
    3	    &1 Str.     &ML Hz, 1 Dns.  &LO Hz, 2 Str., 2 Dns.   \\
    4	    &1 Str.     &1 Str., 1 Dns. &1 Str., 3 Dns.         \\
    5	    &2 Str.     &2 Str., 1 Dns. &2 Str., 2 Dns.         \\
    6	    &---        &3 Str,         &2 Str., 2 Dns.          \\
    7       &---        &4 Str.	        &2 Str., 2 Dns.         \\
    8  	    &---        &HI Hz, 1 Str.	&ML Hz., 1 Dns.	        \\
    9	    &---        &LO Hz, 1 Str.  &2 Dns.                 \\
   10	    &---	    &2 Str.	        &1 Dns.	                \\
\bottomrule
\end{tabularx}
\end{twocolumntable}

}
}
}{
\section{Flight Conditions}
\label{rule:flight-conditions}

\silentlychangedin{1C}{1C-tables}{

\begin{table}[!ht]
\centering
SUN ANGLE CHART

\medskip
\begin{tabular}{llp{4cm}}
\hline
Die Roll   &Time of Day        &Sun Angle \\
\hline
1          &Dawn	           &0 levels per hex away. \\
2 to 3     &Early Morning      &1 level per two hexes away. \\
4	       &Late Morning	   &1 level per hex away. \\
5 to 6     &Noon	           &Any level below target if within two hexes of target. \\
7	       &Early Afternoon    &1 level per hex away. \\
8 to 9     &Late Afternoon     &1 level per two hexes away. \\
10	       &Dusk	           &0 level per hex away. \\
\hline
\end{tabular}
\end{table}

}{

\CX{
\begin{twocolumntablefloat}
\begin{twocolumntable}
\tablecaption{table:sun-angle}{Sun Angle.}
\begin{tabular}{lll}
\toprule
Die Roll   &Time of Day        &Sun Angle \\
\midrule
1          &Dawn	           &0 levels per hex away. \\
2 to 3     &Early Morning      &1 level per two hexes away. \\
4	       &Late Morning	   &1 level per hex away. \\
5 to 6     &Noon	           &Any level below target if within two hexes of target. \\
7	       &Early Afternoon    &1 level per hex away. \\
8 to 9     &Late Afternoon     &1 level per two hexes away. \\
10	       &Dusk	           &0 level per hex away. \\
\bottomrule
\end{tabular}
\end{twocolumntable}
\end{twocolumntablefloat}
}{
\begin{twocolumntablefloat}
\begin{twocolumntable}
\tablecaption{table:sun-angle}{Sun Position Generation.}
\small
\begin{tabularx}{0.7\linewidth}{lllL}
\toprule
Die Roll   &Time of Day        &Sun Direction&Clutter Altitude Requirement \\
\midrule
1          &Dawn	           &East&Same altitude. \\
2 to 3     &Early Morning      &East&1 level per two whole hexes. \\
4	       &Late Morning	   &East&1 level lower per hex. \\
5 to 6     &Noon	           &Overhead&Lower altitude and within two hexes. \\
7	       &Early Afternoon    &West&1 level lower per hex. \\
8 to 9     &Late Afternoon     &West&1 level lower per two whole hexes. \\
10	       &Dusk	           &West&Same altitude. \\
\bottomrule
\end{tabularx}
\end{twocolumntable}
\end{twocolumntablefloat}

}

}
\changedin{1C}{1C-tables}{

\begin{table}[!ht]
\centering
WEATHER GENERATION TABLES

\bigskip
GEOGRAPHIC AREA\\
(roll one die for sky condition)

\medskip
\begin{tabular}{p{2cm}*{4}{p{1.2cm}}}
\hline
Sky Type     &Far North/South    &Middle Regions &Desert Areas &Tropical Areas  \\
\hline
Clear        &1 to 4             &1 to 5         &1 to 8       &1 to 6          \\
Broken       &5 to 7             &6 to 8         &9 to 10      &7               \\
Overcast     &8 to 10            &9 to 10        &NA           &8 to 10         \\
Contrail Alt.&20                 &25             &30           &25              \\
\hline
\end{tabular}

\bigskip

ACTUAL WEATHER\\
(roll die to determine haze and cloud layers)

\medskip
\begin{tabular}{p{1.0cm}*{3}{p{2.2cm}}}
\hline
Die Roll    &Clear      &Broken         &Overcast               \\
\hline
    1	    &LO Hz      &LO Hz, 3 Str.  &LO Hz, 3 Str., 1 Dns.   \\
    2       &ML Hz      &ML Hz, 1 Dns.  &LO Hz, 3 Str., 1 Dns.   \\
    3	    &1 Str.     &ML Hz, 1 Dns.  &LO Hz, 2 Str., 2 Dns.   \\
    4	    &1 Str.     &1 Str., 1 Dns. &1 Str., 3 Dns.         \\
    5	    &2 Str.     &2 Str., 1 Dns. &2 Str., 2 Dns.         \\
    6	    &---        &3 Str,         &2 Str., 2 Dns.          \\
    7       &---        &4 Str.	        &2 Str., 2 Dns.         \\
    8  	    &---        &HI Hz, 1 Str.	&ML Hz., 1 Dns.	        \\
    9	    &---        &LO Hz, 1 Str.  &2 Dns.                 \\
   10	    &---	    &2 Str.	        &1 Dns.	                \\
\hline
\end{tabular}
\end{table}

}{

\begin{twocolumntable}
\tablecaption{table:climate-zones}{Weather Generation: Geographic Area}
\begin{tabularx}{0.7\linewidth}{l*{4}{C}}
\toprule
Sky Type     &Far North/South    &Middle\break Regions &Desert\break Areas &Tropical\break Areas  \\
\midrule
Clear        &\phantom{0}1 to 4\phantom{0} &\phantom{0}1 to 5\phantom{0} &\phantom{0}1 to 8\phantom{0} &\phantom{0}1 to 6\phantom{0} \\
Broken       &\phantom{0}5 to 7\phantom{0} &\phantom{0}6 to 8\phantom{0} &\phantom{0}9 to 10           &7                            \\
Overcast     &\phantom{0}8 to 10\phantom{} &\phantom{0}9 to 10\phantom{} &NA                           &\phantom{0}8 to 10\phantom{} \\
Contrail Alt.&20                 &25             &30           &25              \\
\bottomrule
\end{tabularx}
\end{twocolumntable}

\begin{twocolumntable}
\tablecaption{table:haze-and-clouds}{Weather Generation: Actual Weather}
\begin{tabularx}{0.7\linewidth}{lLLL}
\toprule
Die Roll    &Clear      &Broken         &Overcast               \\
\midrule
    1	    &LO Hz      &LO Hz, 3 Str.  &LO Hz, 3 Str., 1 Dns.   \\
    2       &ML Hz      &ML Hz, 1 Dns.  &LO Hz, 3 Str., 1 Dns.   \\
    3	    &1 Str.     &ML Hz, 1 Dns.  &LO Hz, 2 Str., 2 Dns.   \\
    4	    &1 Str.     &1 Str., 1 Dns. &1 Str., 3 Dns.         \\
    5	    &2 Str.     &2 Str., 1 Dns. &2 Str., 2 Dns.         \\
    6	    &---        &3 Str,         &2 Str., 2 Dns.          \\
    7       &---        &4 Str.	        &2 Str., 2 Dns.         \\
    8  	    &---        &HI Hz, 1 Str.	&ML Hz., 1 Dns.	        \\
    9	    &---        &LO Hz, 1 Str.  &2 Dns.                 \\
   10	    &---	    &2 Str.	        &1 Dns.	                \\
\bottomrule
\end{tabularx}
\end{twocolumntable}

}

The flight conditions are: whether it is day or night; the position of the Sun during the day; and the presense and altitudes of contrail, haze, and cloud layers.

The flight conditions may be specified by the scenario, agreed on by the players, or generated randomly following the procedures below.

}
\label{rule:clouds}
\label{rule:haze}
\label{rule:stratus-clouds}
\label{rule:dense-clouds}


\AX{
\subsection{Day or Night}

Determine whether the scenario occurs at night by rolling one die and considering the scenario year. The scenario occurs at night on a roll of $1-$ in the 1950s, $2-$ in the 1960s, $3-$ in the 1970s or 1980s, and $4-$ in the 1990s or later.

\subsection{Sun Position}
\label{rule:sun-position}

Determine the time of day of the scenario and the corresponding position of the Sun by rolling one die and consulting Table~\ref{table:sun-angle}. The position has two components: the direction and the angle.

\subsection{Weather}

\paragraph{Climate Types.} The climate types are polar, temperate, tropical, and arid.

The scenario notes may indicate the appropriate climate type. If not, the players should agree on them considering the geographic zone in which the scenario occurs. The climate types should not be interpreted too strictly; for example, the Korean peninsular experiences a continental climate and can be considered polar in winter and temperate in summer.

\paragraph{Contrail Layer.} Determine the lower altitude the contrail layer by rolling one die and adding the base contrail altitude level given by Table~\ref{table:climate-zones} for the appropriate climate type. The higher altitude of the contrail layer is the lower altitude plus 25.

For example, if the climate type is polar and the die roll is 7, the contrail layer extends from altitude level 27 to altitude level 52.

\paragraph{Sky Type.} The sky types are clear, broken, or overcast. Determine the sky type by rolling one die and consulting Table~\ref{table:climate-zones} for the appropriate climate type. 

For example, if the climate type is polar, the sky type is clear on a roll of 1 to 4, broken on a roll of 5 to 7, and overcast on a roll of 8 to 10.

\paragraph{Haze and Cloud Layers.} Determine the number and type of haze and cloud layers by rolling one die and consulting Table~\ref{table:haze-and-clouds} for the appropriate sky type. The result may indicate that there is a haze layer (Hz), and if so, its upper altitude band, and the number of stratus cloud layers (SCL) and dense cloud layers (DCL). 

The altitudes of each haze layer or cloud layer is determined as follows:

\begin{itemize}

\item If a haze layer is present, it extends from the altitude level 0 to the top of the indicated altitude band. For example, “ML Hz” means there is a haze layer in the LO and ML altitude bands, extending from altitude level 0 to altitude level 16.

\IDY[3A-cloud-altitudes]{A stratus cloud layer extends over only one altitude level. For each layer, randomly select one of the four stratus cloud counters and then flip it like a coin to show one side. (The counters show altitudes of 8, 11, 16, 18, 23, 27, 28, and 32.)}

\IAY[3A-cloud-altitudes]{A stratus cloud layer extends over only one altitude level. For each stratus cloud layer, roll two dice to determine its altitude:
\begin{itemize}
\item If the first result is 1 to 5, the altitude is 4 plus the second result.
\item If the first result is 6 to 8, the altitude is 14 plus the second result.
\item If the first result is 9 or 10, the altitude is 24 plus the second result.
\end{itemize}
Thus, stratus cloud layers can be between altitude levels 5 and 34. For example, if the first result is 7 and the second is 2, the altitude is 16.
}

\IDY[3A-cloud-altitudes]{A dense cloud layer can extend over several altitude levels. For each layer, randomly select two of the six dense cloud counters. The first one indicates the lower altitude level, and the second the upper altitude level. (The counters show lower/upper altitudes of 2/25, 3/20, 5/16, 7/15, 9/13, and 11/12.)}

\IAY[3A-cloud-altitudes]{A dense cloud layer may extend over several altitude layers. For each dense cloud layer, roll four dice to determine the lower and upper altitudes:
\begin{itemize}
\item The lower altitude is the sum of half of the first result and half of the second and then rounded down. 
\item The upper altitude is the sum of the third result, half of the fourth, and 10, and then rounded down.
\end{itemize}

Thus, the lower altitude can be between levels 1 and 10, and the upper altitude can be between levels 11 and 25. For example, if the results are 7, 2, 3, and 5, the lower altitude is 4, and the upper altitude is 15.}

\end{itemize}

Overlapping or adjacent dense cloud layers merge to form a single, possibly thicker, dense cloud layer. A stratus layer overlapping a dense cloud layer also merges with the dense cloud layer and is ignored. \CY[3A-cloud-altitudes]{A stratus cloud layer adjacent to a dense cloud layer is retained as a separate cloud layer.}{Stratus cloud layers at the same altitude merge to form a single stratus cloud layer. A stratus cloud layer adjacent to a dense cloud layer or another stratus cloud layer is retained as a separate cloud layer.}

}

\DX{
\paragraph{Contrailing.} Aircraft or missiles at high altitude and high speed can leave contrails in the air. Anytime an aircraft or missile is at a speed of 4.0 or more and within the Contrail Levels, it leaves a contrail.

To determine the contrailing levels, roll one die and add the result to the base contrail level given on \changedin{1C}{1C-tables}{the weather chart}{Table~\ref{table:climate-zones}} for the geographic region of play. The result is the lowest level contrailing occurs. Contrails occur up to 25 levels above the lowest contrail level.

Any aircraft or missile that is contrailing is automatically sighted, assuming no cloud layers are between it and a searching aircraft, out to a range of 150 hexes (90 hexes for missiles) regardless of normal visibility limits.

\changedin{1C}{1C-tables}{
\paragraph{The Weather Table.} The game weather is either given in the scenario notes or can be generated by the weather table. To use the table, roll one die and consult the appropriate geographic area column.  The result will be either, Clear, Broken, or Overcast skies. Roll again under the appropriate heading to determine if Haze and/or any Cloud Layers exist.
}{
\paragraph{Generating Weather.} The game weather is either given in the scenario notes or can be generated using Tables~\ref{table:climate-zones} and \ref{table:haze-and-clouds}. To generate weather, roll one die and consult the appropriate geographic area of Table~\ref{table:climate-zones}. The result will be either, Clear, Broken, or Overcast skies. Roll again under the appropriate heading in Table~\ref{table:haze-and-clouds} to determine if Haze and/or any Cloud Layers exist.
}

\paragraph{Haze Layers.}  \DX{If Haze is indicated for some altitude band, it exists in all levels of that band and down to ground level.} Haze reduces visibility. The maximum visual sighting range to and from aircraft in Haze is twice the target's visibility number in hexes. The sighting range to ground units in Haze is halved.

\paragraph{Cloud Layers.} Two types of Cloud Layers may be called for: Stratus and Dense. More than one layer of each type may be called for. When the Weather Table result includes Cloud Layers determine their exact altitudes as follows:

\begin{itemize}
    \item Stratus Layers: Put all the Stratus information counters in a cup, randomly pick one for each layer called for and flip the counter like a coin. The number on the resulting Face up side is the altitude level the Stratus Layer occupies. Each Stratus Layer is only 1,000 feet thick.

    \item Dense Layers: Put all the Dense information counters in a cup and randomly pick two for each Dense Layer called for. Refer to the “Low Ceiling” side of the first counter drawn and the “High Ceiling” side of the second drawn for each Dense Layer. The indicated numbers are the lowest and highest levels of the Dense Layer. Dense clouds exist in those levels and all levels in between.   \changedin{2B}{2B-merging-cloud-layers}{Dense Layers which overlap simply combine into a larger Dense Layer.}{Dense layers that overlap or are adjacent simply combine into a single, possibly larger, dense layer. A stratus layer that overlaps a dense layer is ignored. A stratus layer that is adjacent to a dense layer is retained as a separate cloud layer.}
\end{itemize}


\paragraph{Stratus Layer Effects.} Aircraft \addedin{2B}{2B-stratus}{or ground units} above or below Stratus may not sight aircraft \addedin{2B}{2B-stratus}{or ground units} on the other side. Aircraft in a Stratus Layer may be sighted and may sight targets above or below but with adverse sighting die roll modifiers (see Sighting Modifiers Tables). Two aircraft in the same Stratus Layer may not sight each other unless they are in the same or adjacent hexes. \addedin{2B}{2B-stratus}{Similarly, aircraft and ground units may not use VAS, IRSTS, TV/IR Optics, Night IR sights, launch IRMs (even with radar assist), launch IR SAMs, launch or track OG/LG SAMs, use laser designators, or use laser spot trackers on targets on the opposite side of Stratus.}

\paragraph{IRM Effects.} IRMs may pursue and attack aircraft in Stratus Layers. However, anytime both target and an IRM end \changedin{2B}{2B-stratus}{a}{the missile's} proportional move in the same Stratus Layer and the missile is not in an adjacent hex it loses seeker lock-up and is removed from play. If, at the end of \changedin{2B}{2B-stratus}{a}{the missile's} proportional move, the target and IRM are on opposite sides of a Stratus Layer, the missile loses seeker lock-up and is removed from play.

\paragraph{Dense Layer Effects.} Aircraft in dense layers may not sight or be sighted. IRMs pursuing aircraft that enter a dense layer may attack them only if they can reach their target during their very next proportional move. If they cannot, they are removed from play.

An aircraft entering a dense layer becomes unsighted unless; it is daytime and an enemy aircraft can establish tailing parameters on it. Only the tailing aircraft can sight it on the following turn. Gun combat is allowed only in daytime and only to the tailing aircraft. In dense clouds collisions remain possible even to tailing aircraft and friendly aircraft not in Close formations.

Dense clouds are considered adverse weather for purposes of pilot disorientation. At night, aircraft under a dense cloud layer are considered in adverse weather. TV/IR optics, illumination flares, and laser designators do not function in dense clouds.

\paragraph{Sighting/Combat Effects of Cloud Layers.} It is easier to see aircraft against white backgrounds. When looking for aircraft whose altitude level is between the sighting aircraft and the clouds apply the silhouette modifiers given in the sighting table.

Bright sunlight reflecting off clouds can distract IR guided missiles. When launching IR missiles at a “lower” target which is above the highest cloud layer in play, add 3 to the missile's launch roll.
}

\silentlydeletedin{1C}{1C-tables}{
    \changedin{1C}{1C-tables}{

\begin{table}[!ht]
\centering
WEATHER GENERATION TABLES

\bigskip
GEOGRAPHIC AREA\\
(roll one die for sky condition)

\medskip
\begin{tabular}{p{2cm}*{4}{p{1.2cm}}}
\hline
Sky Type     &Far North/South    &Middle Regions &Desert Areas &Tropical Areas  \\
\hline
Clear        &1 to 4             &1 to 5         &1 to 8       &1 to 6          \\
Broken       &5 to 7             &6 to 8         &9 to 10      &7               \\
Overcast     &8 to 10            &9 to 10        &NA           &8 to 10         \\
Contrail Alt.&20                 &25             &30           &25              \\
\hline
\end{tabular}

\bigskip

ACTUAL WEATHER\\
(roll die to determine haze and cloud layers)

\medskip
\begin{tabular}{p{1.0cm}*{3}{p{2.2cm}}}
\hline
Die Roll    &Clear      &Broken         &Overcast               \\
\hline
    1	    &LO Hz      &LO Hz, 3 Str.  &LO Hz, 3 Str., 1 Dns.   \\
    2       &ML Hz      &ML Hz, 1 Dns.  &LO Hz, 3 Str., 1 Dns.   \\
    3	    &1 Str.     &ML Hz, 1 Dns.  &LO Hz, 2 Str., 2 Dns.   \\
    4	    &1 Str.     &1 Str., 1 Dns. &1 Str., 3 Dns.         \\
    5	    &2 Str.     &2 Str., 1 Dns. &2 Str., 2 Dns.         \\
    6	    &---        &3 Str,         &2 Str., 2 Dns.          \\
    7       &---        &4 Str.	        &2 Str., 2 Dns.         \\
    8  	    &---        &HI Hz, 1 Str.	&ML Hz., 1 Dns.	        \\
    9	    &---        &LO Hz, 1 Str.  &2 Dns.                 \\
   10	    &---	    &2 Str.	        &1 Dns.	                \\
\hline
\end{tabular}
\end{table}

}{

\begin{twocolumntable}
\tablecaption{table:climate-zones}{Weather Generation: Geographic Area}
\begin{tabularx}{0.7\linewidth}{l*{4}{C}}
\toprule
Sky Type     &Far North/South    &Middle\break Regions &Desert\break Areas &Tropical\break Areas  \\
\midrule
Clear        &\phantom{0}1 to 4\phantom{0} &\phantom{0}1 to 5\phantom{0} &\phantom{0}1 to 8\phantom{0} &\phantom{0}1 to 6\phantom{0} \\
Broken       &\phantom{0}5 to 7\phantom{0} &\phantom{0}6 to 8\phantom{0} &\phantom{0}9 to 10           &7                            \\
Overcast     &\phantom{0}8 to 10\phantom{} &\phantom{0}9 to 10\phantom{} &NA                           &\phantom{0}8 to 10\phantom{} \\
Contrail Alt.&20                 &25             &30           &25              \\
\bottomrule
\end{tabularx}
\end{twocolumntable}

\begin{twocolumntable}
\tablecaption{table:haze-and-clouds}{Weather Generation: Actual Weather}
\begin{tabularx}{0.7\linewidth}{lLLL}
\toprule
Die Roll    &Clear      &Broken         &Overcast               \\
\midrule
    1	    &LO Hz      &LO Hz, 3 Str.  &LO Hz, 3 Str., 1 Dns.   \\
    2       &ML Hz      &ML Hz, 1 Dns.  &LO Hz, 3 Str., 1 Dns.   \\
    3	    &1 Str.     &ML Hz, 1 Dns.  &LO Hz, 2 Str., 2 Dns.   \\
    4	    &1 Str.     &1 Str., 1 Dns. &1 Str., 3 Dns.         \\
    5	    &2 Str.     &2 Str., 1 Dns. &2 Str., 2 Dns.         \\
    6	    &---        &3 Str,         &2 Str., 2 Dns.          \\
    7       &---        &4 Str.	        &2 Str., 2 Dns.         \\
    8  	    &---        &HI Hz, 1 Str.	&ML Hz., 1 Dns.	        \\
    9	    &---        &LO Hz, 1 Str.  &2 Dns.                 \\
   10	    &---	    &2 Str.	        &1 Dns.	                \\
\bottomrule
\end{tabularx}
\end{twocolumntable}

}
}

\section{Sun}
\label{rule:sun}

\DX{

The Sun has always been a factor in air combat. Pilots often try to maneuver so that enemy pilots or AAA gunners are dazzled by the glare. The Sun may also draw off IR guided missiles.

\paragraph{Sun Position.} The sun is always considered to be off map. A Sun Arc is defined from each aircraft extending away from the Sun's position, East in the morning, West in the Evening. Think of the Sun arc as where the aircraft's shadow would go. Any units in the Sun arc and in the Sun Angle altitudes of a target aircraft, will suffer Sun Clutter Effects.

\paragraph{Sun Arc.} The Sun arc equals a limited radar arc in size. 

\addedin{1B}{1B-apj-36-errata}{The Sun arc is not well-defined for some cases. For example, consider the Sun located to the side of an aircraft on a hexside; a limited arc is only defined off the nose or tail. In that case consider the A/C shifted into the hex closest to the Sun and facing the Sun.}

\DX{
\paragraph{The Sun Angle.} To be affected by Sun Clutter, a unit must be in some vertical position relative to the target which depends on the time of day. The Sun angle is defined in terms of an Altitude level per hex away ratio. Mutually agree on, or roll the die to determine a time of day. Refer to \changedin{1C}{1C-tables}{the table below}{Table~\ref{table:sun-angle}} to get the Sun Angle ratio.

\silentlychangedin{1C}{1C-tables}{

\begin{table}[!ht]
\centering
SUN ANGLE CHART

\medskip
\begin{tabular}{llp{4cm}}
\hline
Die Roll   &Time of Day        &Sun Angle \\
\hline
1          &Dawn	           &0 levels per hex away. \\
2 to 3     &Early Morning      &1 level per two hexes away. \\
4	       &Late Morning	   &1 level per hex away. \\
5 to 6     &Noon	           &Any level below target if within two hexes of target. \\
7	       &Early Afternoon    &1 level per hex away. \\
8 to 9     &Late Afternoon     &1 level per two hexes away. \\
10	       &Dusk	           &0 level per hex away. \\
\hline
\end{tabular}
\end{table}

}{

\CX{
\begin{twocolumntablefloat}
\begin{twocolumntable}
\tablecaption{table:sun-angle}{Sun Angle.}
\begin{tabular}{lll}
\toprule
Die Roll   &Time of Day        &Sun Angle \\
\midrule
1          &Dawn	           &0 levels per hex away. \\
2 to 3     &Early Morning      &1 level per two hexes away. \\
4	       &Late Morning	   &1 level per hex away. \\
5 to 6     &Noon	           &Any level below target if within two hexes of target. \\
7	       &Early Afternoon    &1 level per hex away. \\
8 to 9     &Late Afternoon     &1 level per two hexes away. \\
10	       &Dusk	           &0 level per hex away. \\
\bottomrule
\end{tabular}
\end{twocolumntable}
\end{twocolumntablefloat}
}{
\begin{twocolumntablefloat}
\begin{twocolumntable}
\tablecaption{table:sun-angle}{Sun Position Generation.}
\small
\begin{tabularx}{0.7\linewidth}{lllL}
\toprule
Die Roll   &Time of Day        &Sun Direction&Clutter Altitude Requirement \\
\midrule
1          &Dawn	           &East&Same altitude. \\
2 to 3     &Early Morning      &East&1 level per two whole hexes. \\
4	       &Late Morning	   &East&1 level lower per hex. \\
5 to 6     &Noon	           &Overhead&Lower altitude and within two hexes. \\
7	       &Early Afternoon    &West&1 level lower per hex. \\
8 to 9     &Late Afternoon     &West&1 level lower per two whole hexes. \\
10	       &Dusk	           &West&Same altitude. \\
\bottomrule
\end{tabularx}
\end{twocolumntable}
\end{twocolumntablefloat}

}

}
}

For Example: To be affected by Sun Clutter in the early morning, an aircraft six levels below a target would have to be West of the target, in its Sun Arc and either 12 or 13 hexes away. Sun Clutter is only possible if the involved aircraft are all above the highest cloud layers and the target is not in Haze.

\paragraph{Sun Clutter Effects.} Units in Sun Clutter may not be used to visually search for or padlock the target. If an enemy ends a turn with all opposing aircraft in its Sun Arc, it becomes unsighted.

IR missiles launched in target's Sun Clutter add 3 to the launch roll. IR missiles ending a proportional move or game-turn in target's Sun Clutter must roll a die to see if they are decoyed by the Sun. Use the Flare Vulnerability number +3 as the die roll or less needed.

AAA Units firing from target's Sun Clutter, add 1 to their hit die rolls\addedin{2B}{2B-aaa-sun-clutter}{, unless they are stacked with a FCR or have an integral W-type FCR}.

}{

The Sun has always been a factor in air combat. Pilots often try to maneuver so that enemy pilots or AAA gunners are dazzled by the glare. The Sun may also draw off IR-guided missiles.

\paragraph{Sun Clutter.} 
A searcher or attacker is said to be in its target's \emph{Sun clutter} if the target is in front of or close to the Sun. Thus, the geometry of Sun clutter depends not only on the geometry of the searcher or attacker and its target but also on the presence and position of the Sun (see rule \ref{rule:flight-conditions}).

More precisely, a searcher or attacker is in the Sun clutter of its target if:

\begin{itemize}
\item It is day.
\item The searcher or attacker is above any haze and cloud layers present.
\item If the Sun is overhead, the searcher or attacker is at a lower altitude than the target, and the horizontal range is 2 or less.
\item If the Sun is not overhead, the searcher or attacker is within a limited arc that originates on the target and faces directly away from the Sun (so to the west for the dawn/morning and to the east for the afternoon/dusk) and satisfies the altitude requirement. 

At dawn and dusk, the altitude requirement is that the searcher or attacker has to have the same altitude as the target. In the morning and afternoon, the altitude requirements are given in terms of the number of altitude levels that the searcher or attacker has to be below the target for each hex or full two hexes of horizontal range.

A limited arc facing east or west is not defined for hex sides that are not oriented east-west. In these cases, consider the limited arc to originate from the center of the adjacent hex closest to the Sun.

\end{itemize}

For example, to be affected by Sun clutter in the early morning, an aircraft six levels below a target would have to be above any haze and cloud layers present, be west of the target, be in a limited arc centered on the target, and oriented to the west, and be at a horizontal range of either 12 or 13.

\paragraph{Sun Clutter Effects.} Aircraft, missile, or ground units in the Sun clutter of a target suffer the following restrictions:
\begin{itemize}
\item 
Aircraft cannot padlock or participate in searches for the target (see rule \ref{rule:sighting-aircraft-and-missiles}). If a target begins the visual sighting phase in the Sun clutter of all opposing aircraft, it will be unsighted for the rest of the game turn.
\item
Ground units cannot sight the target.
\item 
IRMs and IR SAMs launched at the target will have a $+3$ modifier to their launch roll (see rule \ref{rule:irm-launch-modifiers}
).
\item
IRMs and IR SAMs that end a proportional move in their target’s Sun clutter may be decoyed by the Sun. Roll one die and subtract 3. If the result is less than or equal to the missile’s flare vulnerability number, the missile ceases to track the target and is removed from play (see rule \ref{rule:irm-tracking-requirements}).
\item
AAA units firing on the target with aimed fire suffer a $+1$ modifier to their hit die roll, unless they are stacked with a FCR or have an integral W-type FCR.
\end{itemize}
}

\AX{
\section{Contrail Layers}
\label{rule:contrailing}
\label{rule:contrail-layers}

\paragraph{Contrails.} Aircraft or missiles at high altitude and high speed can leave contrails in the air. If an aircraft or missile is at a speed of 4.0 or more and within the contrail layer, it generates a contrail.

\paragraph{Contrail Effects.} 
The only effects of contrails are on the visual sighting of aircraft and missiles (see rule \ref{rule:sighting-aircraft-and-missiles}):
\begin{itemize}
\item
During the day, a contrailing aircraft can be sighted automatically to a sighting range of 150.

\item
During the day, a contrailing missile can be sighted automatically to a sighting range of 90.

\item
At night, a contrailing aircraft or missile that is \emph{above the highest cloud layer present} can be automatically sighted to a sighting range of 24. 

\item At night, a contrailing aircraft that is \emph{not above the highest cloud layer present} can be sighted normally to a sighting range of 6.
\end{itemize}

In all cases, the usual effects of haze and cloud layers on sighting apply.
}

\AX{
\section{Haze and Cloud Layers}
\label{rule:haze-and-cloud-layers}

Haze and cloud layers can absorb or scatter visible and infrared light and so have negative impacts on visual sighting and on sensors or weapons that operate with visible or infrared light.

In particular, haze layers partially block and scatter visible light and reduce contrast, but have negligible effect on infrared light, stratus cloud layers largely block both visible and infrared light, and dense cloud layers almost completely block both visible and infrared light.

In the game, haze and cloud layers have no impact on radar or radar-guided weapons.

\paragraph{Haze Layers.}
\label{rule:haze-layers}

The effects of haze layers are:

\begin{itemize}

\item\itemparagraph{Sighting Range.} If either the searcher or target or both are in a haze layer, the sighting range for visual sighting and identification (see rule~\ref{rule:visual-sighting}) is considered to be twice the normal range. This applies to normal visual sighting (air-to-air, air-to-ground, and ground-to-air) and to the use of VAS. It does not apply to the use of IRSTS or TV/IR optics.

\item\itemparagraph{Individual Sighting.} Air-to-air attacks with visually aimed weapons require the target to be individually sighted by the attacker at the start of the attacker's movement (see rule \ref{rule:individual-sighting}). Exceptionally and for this purpose only, if a potential target moves first and enters a haze layer , the sighting range is considered to be the normal range and not twice the normal range.

\item\itemparagraph{Adverse Conditions.} At night, an aircraft in a haze layer is considered to be flying in adverse conditions (see rule \ref{rule:adverse-conditions}).

\end{itemize}

\paragraph{Cloud Layers.}
\label{rule:common-cloud-layers}

The common effects of stratus and dense cloud layers are:

\begin{itemize}

\item\itemparagraph{Visual Sighting Modifiers.} Aircraft between a cloud layer and the searcher are considered to be seen against clouds for determining the sighting modifier from Table~\ref{table:sighting-position-modifiers}.

\item\itemparagraph{IRM Launch Modifier.} During the day, if an IRM missile is launched from above the highest cloud layer at a lower target with a horizontal range that is less than twice the difference in altitude levels, a launch roll modifier of $+3$ applies. This modifier represents the effect of the reflection of the Sun off the highest cloud layer.

\end{itemize}

Additional specific effects of stratus and dense cloud layers are noted below.

\paragraph{Stratus Cloud Layers.}
\label{rule:stratus-cloud-layers}

The additional effects of stratus cloud layers are:

\begin{itemize}

\item\itemparagraph{Blocking Sight Lines.} Stratus cloud layers block sight lines and limit visual sighting as follows:

\begin{itemize}
\item
Aircraft and ground units above or below a stratus cloud layer may not visually sight targets beyond that layer.

\item An aircraft in a stratus cloud layer may sight targets outside the layer, albeit with the modifiers discussed below.

\item An aircraft or missile in a stratus cloud layer may sighted by aircraft and ground units outside the layer, albeit with the modifiers discussed below.

\item An aircraft in a stratus layer can sight other aircraft and missiles \emph{in the same layer} to a maximum range of 1.
\end{itemize}

The same limitations apply to the use of VAS, IRSTS, TV/IR optics, laser designators, laser spot trackers, IRM and IR/OG/LG SAM launches, OG/LG SAM tracking, AAA units not stacked with a FCR or with integral W-type FCR, and RS/RS/BG/BS weapons. 

In particular, aircraft above or below a stratus cloud layer may not launch IRMs at targets beyond that layer, even with radar assist (see rule \ref{rule:irm-radar}).

\item\itemparagraph{Visual Sighting Modifiers.} Stratus cloud layers give the following sighting modifiers:
\begin{itemize}

\item Aircraft in a stratus cloud layer are considered to be seen against clouds for determining the sighting modifier from Table~\ref{table:sighting-position-modifiers}.

\item
Aircraft in a stratus cloud layer can sight other aircraft and missiles with a modifier of $+2$.

\item Aircraft and missiles in a stratus cloud layer can be sighted by other aircraft with a modifier of $+3$. 

\end{itemize}
When an aircraft in a stratus cloud layer is searching for a target in the same layer, all of these modifiers apply.

\item\itemparagraph{Missile Tracking.} IRMs and IR/OG/LG SAMs may track and attack aircraft in stratus cloud layers. 

However, if both the missile and the target are in the same layer, then at the end of each of the missile’s proportional moves the range to the target must be at most 1 otherwise the missile fails to track and is removed. 

If, at the end of any of the missile’s proportional moves, the target and missile (for IRMs and IR SAMs) or the target and launching unit (for OG/LG SAMs) are on opposite sides of a stratus cloud layer, the missile fails to track and is removed from play.


\item\itemparagraph{Individual Sighting.} Air-to-air attacks with visually aimed weapons require the target to be individually sighted by the attacker at the start of the attacker's movement (see rule \ref{rule:individual-sighting}). Exceptionally and for this purpose only, if a potential target moves first and crosses to the other side of a stratus cloud layer or enters the same layer as the attacker, the stratus cloud layer is not considered to block visual sighting.

\item\itemparagraph{Adverse Conditions.} At night, an aircraft in a stratus cloud layer is considered to be flying in adverse conditions (see rule \ref{rule:adverse-conditions}).

\end{itemize}

\paragraph{Dense Cloud Layers.}
\label{rule:dense-cloud-layers}

The additional effects of dense cloud layers are:

\begin{itemize}

\item\itemparagraph{Blocking Sight Lines.} Dense cloud layers block sight lines and limit visual sighting as follows:

\begin{itemize}
\item
Aircraft and ground units in a dense cloud layer may not visually sight targets.

\item 
Aircraft and ground units in a dense cloud layer may not be visually sighted.
\end{itemize}

As an exception, during the day, an aircraft being tailed (see rule \ref{rule:tailing}) that enters a dense cloud layer continues to be sighted by the tailing aircraft for as long as the latter continues to tail. The tailing aircraft can potentially attack the aircraft being tailed with guns.

The same limitations apply to the use of VAS, IRSTS, TV/IR optics, laser designators, laser spot trackers, IRM and IR/OG/LG SAM launches, OG/LG SAM tracking, AAA units not stacked with a FCR or with integral W-type FCR, and RG/RS/BG/BS weapons.

In particular, aircraft in dense cloud layer may not launch IRMs or be the target of IRMs, even with radar assist (see rule \ref{rule:irm-radar}).

\item\itemparagraph{Missile Tracking.} IRMs and IR/OG/LG SAMs may not track and attack aircraft in dense cloud layers. 

As an exception, if the target of an IRM or IR SAM enters a dense layer, the missile is allowed to follow during its next proportional move. If it cannot attack during this proportional move, it fails to track and is removed from play.

\item\itemparagraph{Collisions.} As an exception to the usual rule for aircraft collisions (see rule )\ref{rule:aircraft-collisions}), in dense cloud layers collisions are possible between an aircraft being tailed and the aircraft tailing it (see rule \ref{rule:tailing}) and also between two friendly aircraft at the same position and not in a close formation.

\item\itemparagraph{Adverse Conditions.} At night, an aircraft below the highest dense cloud layer is considered to be flying in adverse conditions (see rule \ref{rule:adverse-conditions}).

\end{itemize}
}

\CX{
\section{Night and Adverse Weather Flight}
\label{rule:night-and-adverse-weather-flight}
\label{rule:adverse-conditions}

Night and Adverse weather conditions severely limit what a pilot can or will do with his aircraft. Night flying in Clear Conditions is less restrictive.

\addedin{2B}{2B-generating-night-conditions}{
\paragraph{Determining Night Conditions.}
The scenario notes may indicate night conditions. If not, they may be agreed on by the players or determined randomly following this procedure. Roll one die. The scenario occurs at night on a roll of $1-$ for scenarios in the 1950s, $2-$ in the 1960s, $3-$ in the 1970s or 1980s, and $4-$ in the 1990s or later.
}

\paragraph{Clear Weather Night Flight.} An aircraft above all dense layers, and not in haze or stratus is considered in the Clear. The following limitations apply to Clear Wx night flying:

\begin{itemize}

    \item ET turns, Roll Maneuvers, VIFF Maneuvers, Vertical Climbs and Dives, \changedin{2A}{2A-high-aoa-maneuvers}{and Vertical Reverse Maneuvers}{Vertical Reverse, and high AoA} are allowed only at the risk of Pilot Disorientation.

    \item IRMs may not be fired unless the target is visually sighted or an IRSTS or Radar Assist is used.

    \item Visual sighting ranges for all aircraft is 2 hexes unless the target used AB power or is contrailing in which case the sighting range is 6. Sighting modifiers for paint scheme, relative altitude and smoking do not apply at night.

    \item Missiles may only be sighted on the turn after launch and on each turn a sustainer is burning. The visibility numbers are halved and the Just Launched modifier does not apply.

    \item Missiles or aircraft contrailing above the highest cloud layers are spotted automatically up to 24 hexes away due to moonlight reflections.

    \item Tailing is not allowed, TFF is not allowed unless Terrain Following Technology exists. Note, being equipped with TV/lR optics capability confers Terrain Following-A technology on the aircraft.

    \item Visual Aiming for ground attacks is allowed only if the target is illuminated by parachute flares, or the aircraft has TV/IR optics, or a Laser Spot Tracker with a laser spot.

    \item Only Laser Designator Types B and C may be used at night.

    \item Only Novice or better quality pilots may fly at night.

\end{itemize}

\paragraph{Adverse Weather Flight.} Aircraft in Dense Clouds or below any cloud layers at night or in haze at night are in Adverse Weather. The above restrictions apply as well as the following:

\begin{itemize}

    \item \changedin{2A}{2A-high-aoa-maneuvers}{Viff, Roll, and Vertical Reverse}{VIFF, Roll, Vertical Reverse, and high AoA} Maneuvers are not allowed.

    \item Turns at greater than TT rate risk Pilot Disorientation.

    \item Turn Rates of HT or greater while in TFF flight risk fatal collisions with the ground for Terrain Following-A capable aircraft.  

    \item Turn Rates of BT or greater while in TFF flight risk fatal collisions with the ground for Terrain Following-B capable aircraft.

    \itemdeletedin{1B}{1B-apj-39-qa}{Turn Rates of ET while in TFF flight risk fatal collisions with the ground for Terrain Following-C capable aircraft.}

\end{itemize}

Check for fatal ground collisions each time the aircraft does a facing change at a rate in excess of the safe limit. On a die roll of "1" the aircraft crashes killing the crew.

\paragraph{Pilot Disorientation.} Anytime an aircraft performs an action which risks disorientation or faces while turning at a rate that risks disorientation, roll the die. On a modified "3" or less, the pilot becomes disoriented and the aircraft conducts the rest of its move as if the pilot were GLOC'd. \addedin{1B}{1B-apj-36-errata}{However, disoriented Pilots may bail out or eject normally.}

The disorientation die roll is modified for the following:
\begin{itemize}
    \item $+$ or $-$ pilots Confidence.
    \item $+1$ for Veteran Pilots.
    \item $-1$ per facing in a sustained turn (cumulative).
\end{itemize}

\paragraph{Disorientation Recovery.} Recovery from Disoriented flight requires a die roll as if recovering from disorientated flight with all the appropriate departure recovery modifier.

\paragraph{Night and Adverse Weather Effects on Ground Units.} The following limitations are placed on units attempting to engage aircraft at night and/or in adverse weather:

\begin{itemize}

    \item AAA may not use aimed fire unless stacked with an FCR or having “W” type integral FCR, or if they visually sight the aircraft and can track it long enough to fire (night sighting ranges = 2 or 6 hexes as described above).

    \item AAA may conduct plotted fire but if not radar equipped, each unit may only plot one target hex and altitude before a game begins and may not vary from that.

    \item IR SAMs may only attempt lock-ons against visually sighted targets (night sighting ranges = 2 or 6 as described above).

    \item IR SAMs may operate normally if equipped with Night IR sights against targets in AB power and may attempt lock-ons on others out to six hexes.

    \item OG and LG SAMs may only operate at night if equipped with Night IR sights as above.

    \item Only Radar Guided SAMs and Radar Guided AAA may fire on aircraft in Dense Clouds.

\end{itemize}

\section{Air to Ground Night and Adverse Weather Attacks}

Aircraft need help in locating targets at night. FACs and target marks have been previously discussed. Other aids for night attack include: illumination flares, TV/IR Optics, and Ground Attack Radar.

}{

\section{Night and Adverse Conditions}
\label{rule:night-and-adverse-weather-flight}
\label{rule:adverse-conditions}
\label{rule:night}

Night and adverse conditions severely limit what a pilot can or will do with his aircraft.

\subsection{Night Effects on Aircraft}
\label{rule:aircraft-at-night}

\begin{itemize}

    \item\itemparagraph{Sighting Aircraft.} Aircraft contrailing above the highest cloud layers are spotted automatically to a sighting range of 24 (see rule \ref{rule:contrail-layers}). 
    
    Otherwise, the maximum visual sighting ranges for all aircraft is 2 unless the target used AB power or is contrailing in which case the maximum sighting range is 6. 
    
    Sighting modifiers for paint scheme, relative altitude and smoking do not apply at night.

    \item\itemparagraph{Sighting Missiles.} Missiles contrailing above the highest cloud layers are spotted automatically to a range of 24 hexes (see rule \ref{rule:contrail-layers}). 
    
    Otherwise, missiles may only be sighted on the turn after launch and on each turn a sustainer is burning. The visibility numbers are halved and the modifier for having been launched from a sighted aircraft does not apply.

    \item\itemparagraph{Aiming.} Visual aiming for ground attacks is allowed only if the target is illuminated by parachute flares, or the aircraft has TV/IR optics, or a laser spot tracker with a laser spot.

    \item\itemparagraph{Laser Designators.} Only laser designator types B and C may be used.

\end{itemize}

\subsection{Flying at Night and in Adverse Conditions}

\paragraph{Flying in Clear Night Conditions.}

Clear night conditions exists at night at altitudes outside of stratus layers and above all dense cloud and haze layers. The following restrictions apply to aircraft flying in clear night conditions:

\begin{itemize}

    \item\itemparagraph{Risk of Disorientation.} ET turns, all special maneuvers except slides, vertical climbs, and vertical dives risk pilot disorientation (see advanced rule \ref{rule:disorientation}).

    \item\itemparagraph{Tailing.} Tailing (see rule \ref{rule:tailing}) is not allowed.
    
    \item\itemparagraph{TFF.} TFF is not allowed unless the aircraft has Terrain Following technology or TV/IR optics technology (which confers Terrain Following-A technology).

    \item\itemparagraph{Pilot Quality.} Green pilots (see advanced rule \ref{rule:crew-quality}) may not fly at night.

\end{itemize}

\paragraph{Flying in Adverse Conditions.}

Adverse conditions exist in a dense cloud layer (both during the day and at night) or at night in haze or below the highest dense cloud layer. The following restrictions apply to aircraft in adverse conditions:

\begin{itemize}

    \item\itemparagraph{Forbidden Maneuvers.} All special maneuvers except slides not allowed.

    \item\itemparagraph{Risk of Disorientation.} HT or greater turns, vertical climbs, and vertical dives risk pilot disorientation (see advanced rule \ref{rule:disorientation}).

    \item\itemparagraph{Tailing.} Tailing (see rule \ref{rule:tailing}) is not allowed.

    \item\itemparagraph{TFF.} TFF is not allowed unless the aircraft has Terrain Following technology or TV/IR optics technology (which confers Terrain Following-A technology). Furthermore, turns of HT/BT/ET or greater in TFF risk collisions with the ground for aircraft with TFF technology of type A/B/C, respectively.
    
    Check for fatal ground collisions each time the aircraft changes facing at a rate in excess of the safe limit. On a die roll of 1 the aircraft collides with terrain, killing the crew.

    \item\itemparagraph{Pilot Quality.} Green pilots (see advanced rule \ref{rule:crew-quality}) may not fly in adverse conditions.

\end{itemize}

\subsection{Night Effects on Ground Units} The following limitations are placed on units attempting to engage aircraft at night and/or in adverse weather:

\begin{itemize}

    \item Unless an AAA unit is stacked with a FCR or has an integral W-type FCR, it may only track aircraft for aimed fire within the reduced maximum visual sighting ranges given in rule \ref{rule:aircraft-at-night}. 

    \item Unless an AAA unit is stacked with a FCR or has an integral W-type FCR, it may only conduct plotted fire at one target hex and altitude determined before a game begins.

    \item IR/OG/LG SAM units with night IR sights may operate normally against aircraft in AB power and to a range of to 6 against other aircraft.

    \item IR/OG/LG SAMs units without night IR sights may only attempt lock-ons or launches on targets within the reduced maximum visual sighting ranges given in rule \ref{rule:aircraft-at-night}. 

\end{itemize}
}

\CX{

\section{Air to Ground Night and Adverse Weather Attacks}

Aircraft need help in locating targets at night. FACs and target marks have been previously discussed. Other aids for night attack include: illumination flares, TV/IR Optics, and Ground Attack Radar.


\subsection{Illumination Flares}

Aircraft equipped with illumination flare pods may drop parachute flare clusters which light up the hex they are dropped in and the six adjacent hexes as if it were daytime.

\paragraph{Illumination Flare Pods (IP).} Illumination Flare Pods may be carried on stations capable of carrying EP pods. There are four flare clusters per pod. The parachute flares are distinct from decoy flares and may not be used as such. Parachute flares become effective when within 5 levels of the ground.

\paragraph{Procedure.} Aircraft with FPs may initiate a flare run which counts as the allowed air to ground attack for the turn. In a flare run the aircraft may dispense up to four flare clusters, one per hex, in any hex it passes through in its flight. The aircraft may fly level, climb, or dive, but it must be wings level, not turning or maneuvering when it actually releases flares. It may turn and maneuver between flare releases in a game-turn.

\paragraph{Duration.} Parachute flares last for 10 turns including the turn in which they were launched, or until they hit the ground, whichever occurs first. The flares descend at a rate of one level per odd numbered game-turn, starting the turn they were launched.

\paragraph{Effects.} Ground targets in illuminated hexes may be sighted and attacked normally by aircraft as if it were daytime.

\subsection{TV/IR Optics Capability}

Aircraft with TV/IR Optics technology or that carry Optics pods (OP) are considered to have the following capabilities:

\begin{itemize}

    \item They may visually sight into their 180+ arcs out to a range of 18 hexes (count 2 levels of altitude as one hex) as if it were daylight. They may visually attack sighted targets within that range as if it were daylight. \addedin{2B}{2B-tvir-optics}{This capability may be used during the day and at night. Haze layers do not reduce the maximum visual sighting range for TV/IR optics or OPs.}

    \item If they are designator equipped (via technology, or by carrying an LP pod, or dual capable OP/LP pod), they may opt to visually sight and attack sighted targets as if it were daylight into any single arc that the designator can place a laser spot in instead of their 180+ arcs. As above the range is 18 hexes.

\end{itemize}

\subsection{Radar Bombing}

Aircraft with ground Nav and Attack radars, or with air to air radars of 150+ arc capability, or multi-crewed with radars of 180+ capability, may do radar bombing.

\paragraph{Radar Bombing Options.} Radar bombing allows level bombing attacks with BB class weapons against radar significant targets which have been radar detected. If the radar significant target is locked-onto and the aircraft has computed or advanced bombsights, dive bombing, toss bombing and laydown attacks are allowed as well.

\paragraph{Radar Significant Targets.} The following comprise radar significant targets:

\begin{itemize}

    \item In any terrain: building counters, locomotives, trains, POL sites and bridges.

    \item In Clear terrain or on roads and trails: Vehicle units, AAA and Arty. sites, SAM sites, and radar units.

    \item Naval units at sea, on rivers, or in ports. Docks, Piers, and Dams.

    \item Aircraft on runways, in revetments, and all airport facilities (hangers, shelters, towers).

    \item Isolated urban areas of not greater than 3 hexes in size and not adjacent to built up areas, built up area hexes, and runway hexes.
    
    \item Any hex with a black navigation point triangle in it.

\end{itemize}

\paragraph{Radar Detection Procedure.} If the target counter, or hex is in a line of sight, within the aircraft's radar arc and \changedin{2B}{2B-radar-bombing}{detection}{ground-navigation radar} range, roll one die. On a 7 or less the target is detected. Air to air radars used in the air to ground mode \changedin{2B}{2B-radar-bombing}{halve their maximum detection range}{use half their search range}. Ground Nav and attack radars have no strength ratings and may never be used in an air to air manner. Their lock-on numbers are asterisked as a reminder.

\addedin{2B}{2B-radar-bombing}{
\paragraph{Radar Tracking Procedure.} An aircraft with a ground-attack radar may attempt to lock-on to a detected target. The target must be within the ground-attack radar range. Air to air radars used in the air to ground mode use half their tracking range. The attempt succeeds on a roll equal to the lock-on number or less.
}

\addedin{2B}{2B-radar-bombing}{
\paragraph{Requirements and Restrictions.} A radar cannot simultaneously be used in the air-to-ground mode and air-to-air mode. If it is used in the air-to-ground mode, all detections and lock-ons of aircraft are lost. If it is used in the air-to-air mode, all detections and lock-ons of ground targets are lost. 

The use of radar to detect and track ground targets is governed by the same restrictions as the use of air-to-air radar in normal mode to detect and track aircraft (see rule \ref{rule:air-to-air-radar}). For example, pilot-only aircraft can attempt to detect and lock-on ground targets only of they did not use a HT or harder turn.
}

\paragraph{Radar Bombing Procedure.} Normal aiming is required against the target. Tracking time does not apply. If a lock-on is held, then the bombsight modifier is applied otherwise it is ignored. Attacks are resolved through normal procedures.

\paragraph{Ground Radars and Laser Designators.} Aircraft that have locked up a radar significant target and that have Designators type B or C, may place a laser spot on the target without visually sighting it and may conduct laser guided weapon attacks.

\trainingnote{
\centering 
CONGRATULATIONS!! \\
You have now learned all the rules (it's Miller time)!\\
You may now play all scenarios.
}

}{

\subsection{Illumination Flares}

Aircraft equipped with illumination flare pods (IPs) may drop parachute flare clusters to provide illumination.

\paragraph{Illumination Flare Pods.} Aircraft may carry IPs on any station that can carry an EP. Each IP has four flare clusters.  

\paragraph{Dispensing Flares.} Aircraft with IPs may initiate a flare run. A flare run counts as the allowed air-to-ground attack for the game turn. In a flare run, an aircraft may dispense up to four flare clusters, one per hex, in any hex it passes through in its flight. An aircraft may fly level, climb, dive, turn, and maneuver, but it must not be turning or maneuvering when it actually dispenses flare clusters. It may turn and maneuver between dispensing flare clusters.

\paragraph{Duration.} Parachute flare clusters provide illumination for 10 game turns, including the game turn in which they were dispensed or until they reach ground level, whichever occurs first. They descend one level each odd-numbered game turn, including the game turn on which they were dispensed, if this is odd-numbered.

\paragraph{Effects.} 
Parachute flare clusters illuminate the hex in which they were dispensed and the six surrounding hexes provided they are no more than 5 altitude levels above the ground level. Ground targets in illuminated hexes may be sighted and attacked by aircraft as if it were day. Parachute flare clusters are distinct from decoy flares and may not be used as such.

\subsection{TV/IR Optics and Optics Pods}

TV/IR optics technology and optics pods (OP) use TV cameras sensitive to infrared light to allow aircraft to sight and attack at night or in haze.

\paragraph{Visual Sighting.} Aircraft with TV/IR optics technology or an OP may visually sight ground units and may visually attack sighted ground units in their \arcrange{180}{+} arcs within a range of 18 as if it were day. Haze layers do not reduce the maximum visual sighting range for TV/IR optics or OPs. This capability may be used in the day or at night. 

\paragraph{Use with Laser Designators.} Aircraft also equipped with a laser designator (either an integrated one or one in an LP or dual capable OP/LP pod) may elect to use this capability in any single arc into which they can designate rather than their \arcrange{180}{+} arcs.

\paragraph{Terrain Following Capability.} Aircraft with TV/IR optics technology automatically have type-A terrain-following technology.

\section{Radar Bombing}

Certain aircraft may use their radar to bomb targets that are not visually sighted, albeit with certain restrictions and reduced accuracy.

\paragraph{Air-to-Ground Radar.} An aircraft has a radar with an air-to-ground capability if satisfy one of the following requirements: 
\begin{itemize}
\item Any aircraft with ground-navigation and ground-attack radar.
\item Any aircraft with an air-to-air radar with an arc of \arcrange{150}{+}.
\item Any multi-crewed aircraft with an air-to-air radar with an arc of \arcrange{180}{+}.
\end{itemize}

Ground-navigation and ground-attack radars have no strength ratings and may never be used in an air-to-air manner. Their lock-on numbers are asterisked.

Air-to-air radars have their ranges halved when used for as air-to-ground radars.

\paragraph{Radar-Significant Targets.} The following can potentially be detected by air-to-ground radars:

\begin{itemize}

    \item In any terrain: ground units representing buildings, airport facilities (hangers, shelters, towers), locomotives, trains, POL sites, bridges, docks, piers, and dams

    \item In clear terrain or on roads and trails: ground units representing vehicles, AAA, artillery, SAMs, and radars.

    \item Naval units.

    \item Aircraft on runways and in shelters.

    \item Isolated urban areas of not greater than 3 hexes in size and not adjacent to built-up areas, built-up area hexes, and runway hexes.
    
    \item Any hex with a black navigation point triangle in it.

\end{itemize}

\paragraph{Requirements and Restrictions.} A radar cannot simultaneously be used in the air-to-ground mode and air-to-air mode. If it is used in the air-to-ground mode, all detections and tracks of aircraft are lost. If it is used in the air-to-air mode, all detections and tracks of ground targets are lost. 

The use of radar to detect and track ground targets is governed by the same restrictions as the use of air-to-air radar in normal mode to detect and track aircraft (see rule \ref{rule:air-to-air-radar}). For example, pilot-only aircraft can attempt to detect and lock-on ground targets only of they did not use a HT or harder turn.

\paragraph{Radar Detection Procedure.} If the target unit or hex is in a line of sight, within the aircraft's radar arc and ground-navigation range, roll one die. On a $7-$ the target is detected. Air-to-air radars using in the air-to-ground mode use half of their search range in place of the ground-navigation range.

\paragraph{Radar Tracking Procedure.} Aircraft which also have a ground-attack radar or an air-to-air radar with tracking capability may attempt to track (lock-on) to a detected ground target. The target must be within the ground-attack radar range. Air-to-air radars used in the air-to-ground mode use half their tracking range in place of the ground-attack range. The attempt succeeds on a roll equal to the lock-on number or less.

\paragraph{Radar Bombing Options.} Aircraft may conduct level bombing attacks with BB weapons on detected ground targets. Aircraft with computed or advanced bomb systems may also conduct dive, toss, or laydown bombing attacks with BB weapons on tracked ground targets.

\paragraph{Radar Bombing Procedure.} Attacks are resolved the normal procedures. Normal aiming is required. Modifiers for additional aiming are not applicable. If the target is tracked, then the bomb system modifier is applied, otherwise it is ignored. 

\paragraph{Radar Tracking and Laser Designators.} Aircraft with type-B or type-C laser designators may designate an unsighted but tracked ground target for attacks with laser guided weapons or aircraft with laser spot trackers.

\trainingnote{
\centering 

CONGRATULATIONS!

You have now learned all the rules and may now play all scenarios.
}

}



\end{advancedrules}

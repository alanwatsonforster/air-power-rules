\CX{
\rulechapter{Night and Adverse Weather}
}{
\rulechapter{Night and Weather}
}
\label{rule:night-and-weather}

\CX{
This chapter details the effects of various weather phenomena on aircraft flight and combat. This entire chapter is an \emph{ADVANCED RULE}.
}{
This rule describes the impact of night and weather on aircraft flight and combat. This entire chapter is an advanced rule.
}

\begin{advancedrules}

\DX{
Weather is an important consideration in air combat. Clouds offer concealment from visual and IR guided weapons, contrails give away aircraft positions, night or adverse weather conditions can hinder or limit aircraft flight, or the Sun can dazzle a pilot or draw off an IR missile. The following rules address these factors.
}

\section{Common Weather}
\label{rule:clouds}
\label{rule:haze}
\label{rule:stratus-clouds}
\label{rule:dense-clouds}

\silentlyaddedin{1C}{1C-tables}{
    \changedin{1C}{1C-tables}{

\begin{table}[!ht]
\centering
WEATHER GENERATION TABLES

\bigskip
GEOGRAPHIC AREA\\
(roll one die for sky condition)

\medskip
\begin{tabular}{p{2cm}*{4}{p{1.2cm}}}
\hline
Sky Type     &Far North/South    &Middle Regions &Desert Areas &Tropical Areas  \\
\hline
Clear        &1 to 4             &1 to 5         &1 to 8       &1 to 6          \\
Broken       &5 to 7             &6 to 8         &9 to 10      &7               \\
Overcast     &8 to 10            &9 to 10        &NA           &8 to 10         \\
Contrail Alt.&20                 &25             &30           &25              \\
\hline
\end{tabular}

\bigskip

ACTUAL WEATHER\\
(roll die to determine haze and cloud layers)

\medskip
\begin{tabular}{p{1.0cm}*{3}{p{2.2cm}}}
\hline
Die Roll    &Clear      &Broken         &Overcast               \\
\hline
    1	    &LO Hz      &LO Hz, 3 Str.  &LO Hz, 3 Str., 1 Dns.   \\
    2       &ML Hz      &ML Hz, 1 Dns.  &LO Hz, 3 Str., 1 Dns.   \\
    3	    &1 Str.     &ML Hz, 1 Dns.  &LO Hz, 2 Str., 2 Dns.   \\
    4	    &1 Str.     &1 Str., 1 Dns. &1 Str., 3 Dns.         \\
    5	    &2 Str.     &2 Str., 1 Dns. &2 Str., 2 Dns.         \\
    6	    &---        &3 Str,         &2 Str., 2 Dns.          \\
    7       &---        &4 Str.	        &2 Str., 2 Dns.         \\
    8  	    &---        &HI Hz, 1 Str.	&ML Hz., 1 Dns.	        \\
    9	    &---        &LO Hz, 1 Str.  &2 Dns.                 \\
   10	    &---	    &2 Str.	        &1 Dns.	                \\
\hline
\end{tabular}
\end{table}

}{

\begin{twocolumntable}
\tablecaption{table:climate-zones}{Weather Generation: Geographic Area}
\begin{tabularx}{0.7\linewidth}{l*{4}{C}}
\toprule
Sky Type     &Far North/South    &Middle\break Regions &Desert\break Areas &Tropical\break Areas  \\
\midrule
Clear        &\phantom{0}1 to 4\phantom{0} &\phantom{0}1 to 5\phantom{0} &\phantom{0}1 to 8\phantom{0} &\phantom{0}1 to 6\phantom{0} \\
Broken       &\phantom{0}5 to 7\phantom{0} &\phantom{0}6 to 8\phantom{0} &\phantom{0}9 to 10           &7                            \\
Overcast     &\phantom{0}8 to 10\phantom{} &\phantom{0}9 to 10\phantom{} &NA                           &\phantom{0}8 to 10\phantom{} \\
Contrail Alt.&20                 &25             &30           &25              \\
\bottomrule
\end{tabularx}
\end{twocolumntable}

\begin{twocolumntable}
\tablecaption{table:haze-and-clouds}{Weather Generation: Actual Weather}
\begin{tabularx}{0.7\linewidth}{lLLL}
\toprule
Die Roll    &Clear      &Broken         &Overcast               \\
\midrule
    1	    &LO Hz      &LO Hz, 3 Str.  &LO Hz, 3 Str., 1 Dns.   \\
    2       &ML Hz      &ML Hz, 1 Dns.  &LO Hz, 3 Str., 1 Dns.   \\
    3	    &1 Str.     &ML Hz, 1 Dns.  &LO Hz, 2 Str., 2 Dns.   \\
    4	    &1 Str.     &1 Str., 1 Dns. &1 Str., 3 Dns.         \\
    5	    &2 Str.     &2 Str., 1 Dns. &2 Str., 2 Dns.         \\
    6	    &---        &3 Str,         &2 Str., 2 Dns.          \\
    7       &---        &4 Str.	        &2 Str., 2 Dns.         \\
    8  	    &---        &HI Hz, 1 Str.	&ML Hz., 1 Dns.	        \\
    9	    &---        &LO Hz, 1 Str.  &2 Dns.                 \\
   10	    &---	    &2 Str.	        &1 Dns.	                \\
\bottomrule
\end{tabularx}
\end{twocolumntable}

}
}

\AX{
\subsection{Generating Weather}

The game weather is either given in the scenario notes or can be generated. The following procedure is used to generate weather.

\paragraph{Climate Types.} Decide on the appropriate climate type: polar, temperate, arid, and tropical. These types should not be interpreted too strictly; for example the Korean peninsular experiences a continental climate and can be considered to be polar in winter and temperate in summer.

\paragraph{Contrail Altitudes.} Roll one die and add the base contrail level given by Table~\ref{table:climate-zones} for the appropriate climate type. The result is the lowest altitude level at which contrails can occur. The highest level at which contrails can occur is the lowest level plus 25.

For example, if the climate type is polar and the die roll is 7, contrails can occur from altitude level 27 to altitude level 52.

\paragraph{Sky Type.} Roll one die and determine the sky type from Table~\ref{table:climate-zones} for the appropriate climate type. The sky type can be clear, broken, or overcast.

For example, if the climate type is polar, the sky type is clear on a roll of 1 to 4, broken on a roll of 5 to 7, and overcast on a roll of 8 to 10.

\paragraph{Haze and Cloud Layers.} Roll one die and determine the number and type of haze and cloud layers from Table~\ref{table:haze-and-clouds} for the sky type. The result may indicate that there is haze layer (Hz), and if so its upper altitude band, and the number of stratus cloud layers (SCL) or dense cloud layers (DCL). The altitudes of each haze layer or cloud layer is determined as follows:

\begin{itemize}

\item If haze is present, it extends from the altitude level 0 to the top of the indicated altitude band. For example, “ML Hz” means that there is haze in the LO and ML altitude bands, extending from altitude level 0 to altitude level 16.

\IDY[3A-cloud-altitudes]{A stratus cloud layer extends over only one altitude level. For each layer, randomly select the one of the four stratus cloud counters and then flip it like a coin to show one side. (The counters show altitudes of 8, 11, 16, 18, 23, 27, 28, and 32.)}

\IAY[3A-cloud-altitudes]{A stratus cloud layer extends over only one altitude level. For each stratus cloud layer, roll two dice to determine its altitude:
\begin{itemize}
\item If the first result is 1 to 5, the altitude is 4 plus the second result.
\item If the first result is 6 to 8, the altitude is 14 plus the second result.
\item If the first result is 9 or 10, the altitude is 24 plus the second result.
\end{itemize}
Thus, stratus cloud layers can be between altitude levels 5 and 34. For example, if the first result is 7 and the second is 2, the altitude is 16.
}

\IDY[3A-cloud-altitudes]{A dense cloud layer can extend over several altitude levels. For each layer, randomly select two of the six dense cloud counters. The first one indicates the lower altitude level and the second the upper altitude level. (The counters show lower/upper altitudes of 2/25, 3/20, 5/16, 7/15, 9/13, and 11/12.)}

\IAY[3A-cloud-altitudes]{A dense cloud layer may extend over several altitude layers. For each dense cloud layer, roll four dice to determine the lower and upper altitudes:
\begin{itemize}
\item The lower altitude is the sum of half of the first result and half of the second, and then rounded down. 
\item The upper altitude is the sum of the third result, half of the fourth, and 10, and then rounded down.
\end{itemize}

Thus, the lower altitude can be between levels 1 and 10 and the upper altitude can be between levels 11 and 25. For example, if the results are 7, 2, 3, and 5, the lower altitude is 4 and the upper altitude is 15.}

\end{itemize}

Dense cloud layers that overlap or are adjacent merge to form a single, possibly thicker, dense cloud layer. A stratus layer that overlaps a dense cloud layer also merges with the dense cloud layer and is ignored. \CY[3A-cloud-altitudes]{A stratus cloud layer that is adjacent to a dense cloud layer is retained as a separate cloud layer.}{Two stratus cloud layers with the same altitude are merged to form a single stratus cloud layer. A stratus cloud layer that is adjacent to a dense cloud layer or another stratus cloud layer is retained as a separate cloud layer.}

}

\paragraph{Contrailing.} Aircraft or missiles at high altitude and high speed can leave contrails in the air. Anytime an aircraft or missile is at a speed of 4.0 or more and within the Contrail Levels, it leaves a contrail.

To determine the contrailing levels, roll one die and add the result to the base contrail level given on \changedin{1C}{1C-tables}{the weather chart}{Table~\ref{table:climate-zones}} for the geographic region of play. The result is the lowest level contrailing occurs. Contrails occur up to 25 levels above the lowest contrail level.

Any aircraft or missile that is contrailing is automatically sighted, assuming no cloud layers are between it and a searching aircraft, out to a range of 150 hexes (90 hexes for missiles) regardless of normal visibility limits.

\DX{
\changedin{1C}{1C-tables}{
\paragraph{The Weather Table.} The game weather is either given in the scenario notes or can be generated by the weather table. To use the table, roll one die and consult the appropriate geographic area column.  The result will be either, Clear, Broken, or Overcast skies. Roll again under the appropriate heading to determine if Haze and/or any Cloud Layers exist.
}{
\paragraph{Generating Weather.} The game weather is either given in the scenario notes or can be generated using Tables~\ref{table:climate-zones} and \ref{table:haze-and-clouds}. To generate weather, roll one die and consult the appropriate geographic area of Table~\ref{table:climate-zones}. The result will be either, Clear, Broken, or Overcast skies. Roll again under the appropriate heading in Table~\ref{table:haze-and-clouds} to determine if Haze and/or any Cloud Layers exist.

}

\paragraph{Haze Layers.}  If Haze is indicated for some altitude band, it exists in all levels of that band and down to ground level. Haze reduces visibility. The maximum visual sighting range to and from aircraft in Haze is twice the target's visibility number in hexes. The sighting range to ground units in Haze is halved.


\paragraph{Cloud Layers.} Two types of Cloud Layers may be called for: Stratus and Dense. More than one layer of each type may be called for. When the Weather Table result includes Cloud Layers determine their exact altitudes as follows:

\begin{itemize}
    \item Stratus Layers: Put all the Stratus information counters in a cup, randomly pick one for each layer called for and flip the counter like a coin. The number on the resulting Face up side is the altitude level the Stratus Layer occupies. Each Stratus Layer is only 1,000 feet thick.

    \item Dense Layers: Put all the Dense information counters in a cup and randomly pick two for each Dense Layer called for. Refer to the “Low Ceiling” side of the first counter drawn and the “High Ceiling” side of the second drawn for each Dense Layer. The indicated numbers are the lowest and highest levels of the Dense Layer. Dense clouds exist in those levels and all levels in between.   \changedin{2B}{2B-merging-cloud-layers}{Dense Layers which overlap simply combine into a larger Dense Layer.}{Dense layers that overlap or are adjacent simply combine into a single, possibly larger, dense layer. A stratus layer that overlaps a dense layer is ignored. A stratus later that is adjacent to a dense layer is retained as a separate cloud layer.}
\end{itemize}
}

\paragraph{Stratus Layer Effects.} Aircraft above or below Stratus may not sight aircraft on the other side. Aircraft in a Stratus Layer may be sighted and may sight targets above or below but with adverse sighting die roll modifiers (see Sighting Modifiers Tables). Two aircraft in the same Stratus Layer may not sight each other unless they are in the same or adjacent hexes.

\paragraph{IRM Effects.} IRMs may pursue and attack aircraft in Stratus Layers. However, anytime both target and an IRM end a proportional move in the same Stratus Layer and the missile is not in an adjacent hex it loses seeker lock-up and is removed from play. If, at the end of a proportional move, the target and IRM are on opposite sides of a Stratus Layer, the missile loses seeker lock-up and is removed from play.

\paragraph{Dense Layer Effects.} Aircraft in dense layers may not sight or be sighted. IRMs pursuing aircraft that enter a dense layer may attack them only if they can reach their target during their very next proportional move. If they cannot, they are removed from play.

An aircraft entering a dense layer becomes unsighted unless; it is daytime and an enemy aircraft can establish tailing parameters on it. Only the tailing aircraft can sight it on the following turn. Gun combat is allowed only in daytime and only to the tailing aircraft. In dense clouds collisions remain possible even to tailing aircraft and friendly aircraft not in Close formations.

Dense clouds are considered adverse weather for purposes of pilot disorientation. At night, aircraft under a dense cloud layer are considered in adverse weather. TV/IR optics, illumination flares, and laser designators do not function in dense clouds.

\paragraph{Sighting/Combat Effects of Cloud Layers.} It is easier to see aircraft against white backgrounds. When looking for aircraft whose altitude level is between the sighting aircraft and the clouds apply the silhouette modifiers given in the sighting table.

Bright sunlight reflecting off clouds can distract IR guided missiles. When launching IR missiles at a “lower” target which is above the highest cloud layer in play, add 3 to the missile's launch roll.

\silentlydeletedin{1C}{1C-tables}{
    \changedin{1C}{1C-tables}{

\begin{table}[!ht]
\centering
WEATHER GENERATION TABLES

\bigskip
GEOGRAPHIC AREA\\
(roll one die for sky condition)

\medskip
\begin{tabular}{p{2cm}*{4}{p{1.2cm}}}
\hline
Sky Type     &Far North/South    &Middle Regions &Desert Areas &Tropical Areas  \\
\hline
Clear        &1 to 4             &1 to 5         &1 to 8       &1 to 6          \\
Broken       &5 to 7             &6 to 8         &9 to 10      &7               \\
Overcast     &8 to 10            &9 to 10        &NA           &8 to 10         \\
Contrail Alt.&20                 &25             &30           &25              \\
\hline
\end{tabular}

\bigskip

ACTUAL WEATHER\\
(roll die to determine haze and cloud layers)

\medskip
\begin{tabular}{p{1.0cm}*{3}{p{2.2cm}}}
\hline
Die Roll    &Clear      &Broken         &Overcast               \\
\hline
    1	    &LO Hz      &LO Hz, 3 Str.  &LO Hz, 3 Str., 1 Dns.   \\
    2       &ML Hz      &ML Hz, 1 Dns.  &LO Hz, 3 Str., 1 Dns.   \\
    3	    &1 Str.     &ML Hz, 1 Dns.  &LO Hz, 2 Str., 2 Dns.   \\
    4	    &1 Str.     &1 Str., 1 Dns. &1 Str., 3 Dns.         \\
    5	    &2 Str.     &2 Str., 1 Dns. &2 Str., 2 Dns.         \\
    6	    &---        &3 Str,         &2 Str., 2 Dns.          \\
    7       &---        &4 Str.	        &2 Str., 2 Dns.         \\
    8  	    &---        &HI Hz, 1 Str.	&ML Hz., 1 Dns.	        \\
    9	    &---        &LO Hz, 1 Str.  &2 Dns.                 \\
   10	    &---	    &2 Str.	        &1 Dns.	                \\
\hline
\end{tabular}
\end{table}

}{

\begin{twocolumntable}
\tablecaption{table:climate-zones}{Weather Generation: Geographic Area}
\begin{tabularx}{0.7\linewidth}{l*{4}{C}}
\toprule
Sky Type     &Far North/South    &Middle\break Regions &Desert\break Areas &Tropical\break Areas  \\
\midrule
Clear        &\phantom{0}1 to 4\phantom{0} &\phantom{0}1 to 5\phantom{0} &\phantom{0}1 to 8\phantom{0} &\phantom{0}1 to 6\phantom{0} \\
Broken       &\phantom{0}5 to 7\phantom{0} &\phantom{0}6 to 8\phantom{0} &\phantom{0}9 to 10           &7                            \\
Overcast     &\phantom{0}8 to 10\phantom{} &\phantom{0}9 to 10\phantom{} &NA                           &\phantom{0}8 to 10\phantom{} \\
Contrail Alt.&20                 &25             &30           &25              \\
\bottomrule
\end{tabularx}
\end{twocolumntable}

\begin{twocolumntable}
\tablecaption{table:haze-and-clouds}{Weather Generation: Actual Weather}
\begin{tabularx}{0.7\linewidth}{lLLL}
\toprule
Die Roll    &Clear      &Broken         &Overcast               \\
\midrule
    1	    &LO Hz      &LO Hz, 3 Str.  &LO Hz, 3 Str., 1 Dns.   \\
    2       &ML Hz      &ML Hz, 1 Dns.  &LO Hz, 3 Str., 1 Dns.   \\
    3	    &1 Str.     &ML Hz, 1 Dns.  &LO Hz, 2 Str., 2 Dns.   \\
    4	    &1 Str.     &1 Str., 1 Dns. &1 Str., 3 Dns.         \\
    5	    &2 Str.     &2 Str., 1 Dns. &2 Str., 2 Dns.         \\
    6	    &---        &3 Str,         &2 Str., 2 Dns.          \\
    7       &---        &4 Str.	        &2 Str., 2 Dns.         \\
    8  	    &---        &HI Hz, 1 Str.	&ML Hz., 1 Dns.	        \\
    9	    &---        &LO Hz, 1 Str.  &2 Dns.                 \\
   10	    &---	    &2 Str.	        &1 Dns.	                \\
\bottomrule
\end{tabularx}
\end{twocolumntable}

}
}

\section{Sun}
\label{rule:sun}

The Sun has always been a factor in air combat. Pilots often try to maneuver so that enemy pilots or AAA gunners are dazzled by the glare. The Sun may also draw off IR guided missiles.

\paragraph{Sun Position.} The sun is always considered to be off map. A Sun Arc is defined from each aircraft extending away from the Sun's position, East in the morning, West in the Evening. Think of the Sun arc as where the aircraft's shadow would go. Any units in the Sun arc and in the Sun Angle altitudes of a target aircraft, will suffer Sun Clutter Effects.

\paragraph{Sun Arc.} The Sun arc equals a limited radar arc in size. 

\addedin{1B}{1B-apj-36-errata}{The Sun arc is not well-defined for some cases. For example, consider the Sun located to the side of an aircraft on a hexside; a limited arc is only defined off the nose or tail. In that case consider the A/C shifted into the hex closest to the Sun and facing the Sun.}

\paragraph{The Sun Angle.} To be affected by Sun Clutter, a unit must be in some vertical position relative to the target which depends on the time of day. The Sun angle is defined in terms of an Altitude level per hex away ratio. Mutually agree on, or roll the die to determine a time of day. Refer to \changedin{1C}{1C-tables}{the table below}{Table~\ref{table:sun-angle}} to get the Sun Angle ratio.

\silentlychangedin{1C}{1C-tables}{

\begin{table}[!ht]
\centering
SUN ANGLE CHART

\medskip
\begin{tabular}{llp{4cm}}
\hline
Die Roll   &Time of Day        &Sun Angle \\
\hline
1          &Dawn	           &0 levels per hex away. \\
2 to 3     &Early Morning      &1 level per two hexes away. \\
4	       &Late Morning	   &1 level per hex away. \\
5 to 6     &Noon	           &Any level below target if within two hexes of target. \\
7	       &Early Afternoon    &1 level per hex away. \\
8 to 9     &Late Afternoon     &1 level per two hexes away. \\
10	       &Dusk	           &0 level per hex away. \\
\hline
\end{tabular}
\end{table}

}{

\CX{
\begin{twocolumntablefloat}
\begin{twocolumntable}
\tablecaption{table:sun-angle}{Sun Angle.}
\begin{tabular}{lll}
\toprule
Die Roll   &Time of Day        &Sun Angle \\
\midrule
1          &Dawn	           &0 levels per hex away. \\
2 to 3     &Early Morning      &1 level per two hexes away. \\
4	       &Late Morning	   &1 level per hex away. \\
5 to 6     &Noon	           &Any level below target if within two hexes of target. \\
7	       &Early Afternoon    &1 level per hex away. \\
8 to 9     &Late Afternoon     &1 level per two hexes away. \\
10	       &Dusk	           &0 level per hex away. \\
\bottomrule
\end{tabular}
\end{twocolumntable}
\end{twocolumntablefloat}
}{
\begin{twocolumntablefloat}
\begin{twocolumntable}
\tablecaption{table:sun-angle}{Sun Position Generation.}
\small
\begin{tabularx}{0.7\linewidth}{lllL}
\toprule
Die Roll   &Time of Day        &Sun Direction&Clutter Altitude Requirement \\
\midrule
1          &Dawn	           &East&Same altitude. \\
2 to 3     &Early Morning      &East&1 level per two whole hexes. \\
4	       &Late Morning	   &East&1 level lower per hex. \\
5 to 6     &Noon	           &Overhead&Lower altitude and within two hexes. \\
7	       &Early Afternoon    &West&1 level lower per hex. \\
8 to 9     &Late Afternoon     &West&1 level lower per two whole hexes. \\
10	       &Dusk	           &West&Same altitude. \\
\bottomrule
\end{tabularx}
\end{twocolumntable}
\end{twocolumntablefloat}

}

}

For Example: To be affected by Sun Clutter in the early morning, an aircraft six levels below a target would have to be West of the target, in its Sun Arc and either 12 or 13 hexes away. Sun Clutter is only possible if the involved aircraft are all above the highest cloud layers and the target is not in Haze.

\paragraph{Sun Clutter Effects.} Units in Sun Clutter may not be used to visually search for or padlock the target. If an enemy ends a turn with all opposing aircraft in its Sun Arc, it becomes unsighted.

IR missiles launched in target's Sun Clutter add 3 to the launch roll. IR missiles ending a proportional move or game-turn in target's Sun Clutter must roll a die to see if they are decoyed by the Sun. Use the Flare Vulnerability number +3 as the die roll or less needed.

AAA Units firing from target's Sun Clutter, add 1 to their hit die rolls.

\section{Night and Adverse Weather Flight}
\label{rule:night-and-adverse-weather-flight}

Night and Adverse weather conditions severely limit what a pilot can or will do with his aircraft. Night flying in Clear Conditions is less restrictive.

\paragraph{Clear Weather Night Flight.} An aircraft above all dense layers, and not in haze or stratus is considered in the Clear. The following limitations apply to Clear Wx night flying:

\begin{itemize}

    \item ET turns, Roll Maneuvers, VIFF Maneuvers, Vertical Climbs and Dives, and Vertical Reverse Maneuvers are allowed only at the risk of Pilot Disorientation.

    \item IRMs may not be fired unless the target is visually sighted or an IRSTS or Radar Assist is used.

    \item Visual sighting ranges for all aircraft is 2 hexes unless the target used AB power or is contrailing in which case the sighting range is 6. Sighting modifiers for paint scheme, relative altitude and smoking do not apply at night.

    \item Missiles may only be sighted on the turn after launch and on each turn a sustainer is burning. The visibility numbers are halved and the Just Launched modifier does not apply.

    \item Missiles or aircraft contrailing above the highest cloud layers are spotted automatically up to 24 hexes away due to moonlight reflections.

    \item Tailing is not allowed, TFF is not allowed unless Terrain Following Technology exists. Note, being equipped with TV/lR optics capability confers Terrain Following-A technology on the aircraft.

    \item Visual Aiming for ground attacks is allowed only if the target is illuminated by parachute flares, or the aircraft has TV/IR optics, or a Laser Spot Tracker with a laser spot.

    \item Only Laser Designator Types B and C may be used at night.

    \item Only Novice or better quality pilots may fly at night.

\end{itemize}

\paragraph{Adverse Weather Flight.} Aircraft in Dense Clouds or below any cloud layers at night or in haze at night are in Adverse Weather. The above restrictions apply as well as the following:

\begin{itemize}

    \item Viff, Roll, and Vertical Reverse Maneuvers are not allowed.

    \item Turns at greater than TT rate risk Pilot Disorientation.

    \item Turn Rates of HT or greater while in TFF flight risk fatal collisions with the ground for Terrain Following-A capable aircraft.  

    \item Turn Rates of BT or greater while in TFF flight risk fatal collisions with the ground for Terrain Following-B capable aircraft.

    \itemdeletedin{1B}{1B-apj-39-qa}{Turn Rates of ET while in TFF flight risk fatal collisions with the ground for Terrain Following-C capable aircraft.}

\end{itemize}

Check for fatal ground collisions each time the aircraft does a facing change at a rate in excess of the safe limit. On a die roll of "1" the aircraft crashes killing the crew.

\DX{
\paragraph{Pilot Disorientation.} Anytime an aircraft performs an action which risks disorientation or faces while turning at a rate that risks disorientation, roll the die. On a modified "3" or less, the pilot becomes disoriented and the aircraft conducts the rest of its move as if the pilot were GLOC'd. \addedin{1B}{1B-apj-36-errata}{However, disoriented Pilots may bail out or eject normally.}

The disorientation die roll is modified for the following:
\begin{itemize}
    \item $+$ or $-$ pilots Confidence.
    \item $+1$ for Veteran Pilots.
    \item $-1$ per facing in a sustained turn (cumulative).
\end{itemize}

\paragraph{Disorientation Recovery.} Recovery from Disoriented flight requires a die roll as if recovering from disorientated flight with all the appropriate departure recovery modifier.
}

\paragraph{Night and Adverse Weather Effects on Ground Units.} The following limitations are placed on units attempting to engage aircraft at night and/or in adverse weather:

\begin{itemize}

    \item AAA may not use aimed fire unless stacked with an FCR or having “W” type integral FCR, or if they visually sight the aircraft and can track it long enough to fire (night sighting ranges = 2 or 6 hexes as described above).

    \item AAA may conduct plotted fire but if not radar equipped, each unit may only plot one target hex and altitude before a game begins and may not vary from that.

    \item IR SAMs may only attempt lock-ons against visually sighted targets (night sighting ranges = 2 or 6 as described above).

    \item IR SAMs may operate normally if equipped with Night IR sights against targets in AB power and may attempt lock-ons on others out to six hexes.

    \item OG and LG SAMs may only operate at night if equipped with Night IR sights as above.

    \item Only Radar Guided SAMs and Radar Guided AAA may fire on aircraft in Dense Clouds.

\end{itemize}

\section{Air to Ground Night and Adverse Weather Attacks}

Aircraft need help in locating targets at night. FACs and target marks have been previously discussed. Other aids for night attack include: illumination flares, TV/IR Optics, and Ground Attack Radar.

\subsection{Illumination Flares}

Aircraft equipped with illumination flare pods may drop parachute flare clusters which light up the hex they are dropped in and the six adjacent hexes as if it were daytime.

\paragraph{Illumination Flare Pods (IP).} Illumination Flare Pods may be carried on stations capable of carrying EP pods. There are four flare clusters per pod. The parachute flares are distinct from decoy flares and may not be used as such. Parachute flares become effective when within 5 levels of the ground.

\paragraph{Procedure.} Aircraft with FPs may initiate a flare run which counts as the allowed air to ground attack for the turn. In a flare run the aircraft may dispense up to four flare clusters, one per hex, in any hex it passes through in its flight. The aircraft may fly level, climb, or dive, but it must be wings level, not turning or maneuvering when it actually releases flares. It may turn and maneuver between flare releases in a game-turn.

\paragraph{Duration.} Parachute flares last for 10 turns including the turn in which they were launched, or until they hit the ground, whichever occurs first. The flares descend at a rate of one level per odd numbered game-turn, starting the turn they were launched.

\paragraph{Effects.} Ground targets in illuminated hexes may be sighted and attacked normally by aircraft as if it were daytime.

\subsection{TV/IR Optics Capability}

Aircraft with TV/IR Optics technology or that carry Optics pods (OP) are considered to have the following capabilities:

\begin{itemize}

    \item They may visually sight into their 180+ arcs out to a range of 18 hexes (count 2 levels of altitude as one hex) as if it were daylight. They may visually attack sighted targets within that range as if it were daylight.

    \item If they are designator equipped (via technology, or by carrying an LP pod, or dual capable OP/LP pod), they may opt to visually sight and attack sighted targets as if it were daylight into any single arc that the designator can place a laser spot in instead of their 180+ arcs. As above the range is 18 hexes.

\end{itemize}

\subsection{Radar Bombing}

Aircraft with ground Nav and Attack radars, or with air to air radars of 150+ arc capability, or multi-crewed with radars of 180+ capability, may do radar bombing.

\paragraph{Radar Bombing Options.} Radar bombing allows level bombing attacks with BB class weapons against radar significant targets which have been radar detected. If the radar significant target is locked-onto and the aircraft has computed or advanced bombsights, dive bombing, toss bombing and laydown attacks are allowed as well.

\paragraph{Radar Significant Targets.} The following comprise radar significant targets:

\begin{itemize}

    \item In any terrain: building counters, locomotives, trains, POL sites and bridges.

    \item In Clear terrain or on roads and trails: Vehicle units, AAA and Arty. sites, SAM sites, and radar units.

    \item Naval units at sea, on rivers, or in ports. Docks, Piers, and Dams.

    \item Aircraft on runways, in revetments, and all airport facilities (hangers, shelters, towers).

    \item Isolated urban areas of not greater than 3 hexes in size and not adjacent to built up areas, built up area hexes, and runway hexes.
    
    \item Any hex with a black navigation point triangle in it.

\end{itemize}

\paragraph{Radar Detection Procedure.} If the target counter, or hex is in a line of sight, within the aircraft's radar arc and detection range, roll one die. On a 7 or less the target is detected. Air to air radars used in the air to ground mode halve their maximum detection range. Ground Nav and attack radars have no strength ratings and may never be used in an air to air manner. Their lock-on numbers are asterisked as a reminder.

\paragraph{Radar Bombing Procedure.} Normal aiming is required against the target. Tracking time does not apply. If a lock-on is held, then the bombsight modifier is applied otherwise it is ignored. Attacks are resolved through normal procedures.

\paragraph{Ground Radars and Laser Designators.} Aircraft that have locked up a radar significant target and that have Designators type B or C, may place a laser spot on the target without visually sighting it and may conduct laser guided weapon attacks.

\trainingnote{
\centering 
CONGRATULATIONS!! \\
You have now learned all the rules (it's Miller time)!\\
You may now play all scenarios.
}

\end{advancedrules}

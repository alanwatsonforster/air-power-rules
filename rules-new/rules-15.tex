\CX{
\rulechapter{Heat Seeking Missiles}
}{
\rulechapter{Infrared-Guided Missiles}
}
\label{rule:infrared-heat-seeking-missiles}

\CX{
This chapter details the operations of infrared heat-seeking missiles (weapon code IRM). A heat-seeking missile homes on the infrared heat emissions of a target aircraft.
}{
This rule describes the use of infrared-guided or heat-seeking air-to-air missiles (IRMs), which passively target the target aircraft's infrared emissions.

Infrared seekers are classified into early (E), improved (I), modern (M), and advanced (A) types. This progression in capability gives improvements in field-of-view, target aspect, and hit probability.
}

\CX{
\section{IRM Launch Prerequisites}
\label{rule:irm-launch-prerequisites}

To launch IRMs, the firing aircraft must:

\begin{enumerate}
    \item Have a sighted target in the missile seeker's field of view\addedin{1C}{1C-tables}{\ according to Table~\ref{table:irm-seeker-field-of-view}} and be within the launch angle off limits of the missile's seeker head \changedin{1C}{1C-tables}{(see below)}{according to Table~\ref{table:irm-seeker-limits}}.
    \item Obtain a seeker lock-up while within the minimum and maximum ranges given on the MDT under the missile's launch envelope for the type of shot (front, side, or rear).
    \item Not have violated the missile launch restrictions of rule 14.2.
\end{enumerate}

\paragraph{Seeker Head Field Of View (FOV).} IRMs can only lock-up targets in their FOV\addedin{1C}{1C-tables}{\ according to Table~\ref{table:irm-seeker-field-of-view}}. IR seeker heads normally have an FOV equal to a limited radar arc (see \changedin{1C}{1C-figures}{limited radar arc diagram on play aid reference sheet}{Figure~\ref{figure:limited-arcs}}). If turning left or right or banked left or right, the FOV can optionally be considered equal to the firer’s 180{deg} left or right arc as appropriate. \addedin{1B}{1B-apj-23-errata}{The option to treat a limited arc FOV as a left of right 180 arc for missiles mounted on aircraft in left or right banks or turns is also applicable to all other uses of limited arcs in the rules (radar, BRM, SAM, ARM or otherwise).}

\paragraph{IR Uncage Technology FOV.} \label{rule:ir-uncage} Some aircraft have the ability to uncage a missile's IR seeker head allowing it to swivel freely to acquire targets over a wider area. An uncaged IR seeker always has an FOV equal to a 180{deg} radar arc. The act of uncaging missiles is declared in the Aircraft Decisions Phase. The technology section of the ADC indicates whether an aircraft has “IR Uncage” ability or not. Only “I”, “M”, or “A” type seekers can be uncaged.

\addedin{1C}{1C-tables}{
    %!TEX root = ../rules-working.tex
%LTeX: enabled=false

\begin{onecolumntablefloat}
\begin{onecolumntable}
\tablecaption{table:irm-seeker-field-of-view}{IR Seeker Field of View Limits for Launch}
\medskip
\begin{tabularx}{\linewidth}{X}
\toprule
\begin{enumerate}
    \item Regular FOV = As Limited radar arc
    \item Uncaged FOV = 180+ angle-off arcs
    \item Uncaged FOV with Helmet sight = 150+ angle-off arcs
    \item Uncaged FOV with radar assist = lesser of 150+ or radar arc
    \item Uncaged FOV with VAS assist (M, A only) = 180+ arcs
    \item IRSTS Assisted FOV = Same as IRSTS system
    \medskip
    \item[--] If target one of several in unassisted Uncaged FOV a roll of \minusafter{8} is required for seeker lock-on; \minusafter{9} with helmet sights. Modifier of \plus{1} to roll for each aircraft the seeker must look past.
\end{enumerate}
\\
\bottomrule
\end{tabularx}
\end{onecolumntable}
\end{onecolumntablefloat}
}
\begin{onecolumntable}
\tablecaption{table:irm-seeker-limits}{IRM Seeker Limits}
\begin{tabularx}{\linewidth}{clP}
\toprule
Seeker  &Type       &Angle-Off Limits\\
\midrule
E       &Early      &Target's 60 degree arc or less if target used A/B power; target's 30 degree arc or less otherwise.\\
I       &Improved   &Target's 60 degree arc or less at any target power setting.\\
M       &Modern	    &Target's 120 degree arc or less if it used A/B power; target's 90 degree arc or less otherwise.\\
A       &Advanced   &Any of target's angle-off arcs at any target power setting.\\
\bottomrule
\end{tabularx}
\end{onecolumntable}


\paragraph{Seeker Head Launch Angle-Off Limits.} IR seeker heads cannot lock-up a target unless the firer is within the listed target angle-off arcs as given \changedin{1C}{1C-tables}{below by seeker type}{by Table~\ref{table:irm-seeker-limits}}.

\addedin{1B}{1B-apj-23-errata}{IRMs of any type may be launched from any angle-off arc about propellor driven and helicopter targets. If launch or attack modifiers apply to these targets for low heat signature, it will be stated in the scenarios.}

\paragraph{Seeker Head Lock-Ups.} Anytime there is only one target in a seeker's FOV, it is automatically locked-up.

If more than one aircraft, including friendlies, are in a LIMITED FOV, the closest aircraft is automatically locked-up. If several aircraft are equally close, randomly determine by die roll which was actually locked-up AFTER the missile is launched.

If more than one aircraft, including friendlies, are in an UNCAGED FOV, the firing player chooses the intended target and rolls a die. If the target is the nearest aircraft, the lock-up succeeds on a roll of 8 or less. If other aircraft in the FOV are equally near, or if the target is not the closest, then the die roll is modified by a cumulative \plus{1} for every closer or equally near aircraft. If successful, missiles may be fired normally. If the lock-up attempt fails, missiles may not be launched.

\addedin{1b}{1B-apj-27-qa/1B-apj-37-qa}{A separate roll is made for each missile fired after it is successfully launched.}

}{

\section{IRM Launch Requirements}
\label{rule:irm-launch-requirements}
\label{rule:irm-launch-prerequisites}

In addition to the general missile launch requirements in rule~\ref{rule:missile-launches}, launching an IRM requires locking-up the target’s seeker. This, in turn, requires:

\begin{itemize}

    \item The launching aircraft must have individually sighted the target at the start of their movement (rule \ref{rule:individual-sighting}) or be using a radar- or IRSTS-assisted lock-up method (advanced rule~\ref{rule:irm-seeker-lock-up-assistance-methods}). 

    \item The launching aircraft must have the target in the angle-off arc corresponding to the missile seeker’s field-of-view according to Table~\ref{table:irm-seeker-field-of-view}.
 
    \item The launching aircraft must have the target within these vertical limits:
    \begin{itemize}

    \item If the launching aircraft used climbing flight during the game turn of launch, it may not fire at a lower target.

    \item If the launching aircraft used diving flight during the game turn of launch, it may not fire at a higher target.

    \item If the launching aircraft used level flight during the game turn of launch, the difference in altitude between the firer and target may be at most one level for every two full hexes of horizontal range.

    \end{itemize}
    If advanced rule~\ref{rule:irm-realistic-vertical-limits} is being used, it supersedes these simple vertical limits.
    
    \item The launching aircraft must be within the launch angle-off arc of the missile’s seeker according to Table~\ref{table:irm-seeker-limits}.
    
    \item The range to the target must be no less than the minimum range and no more than the maximum range for the appropriate missile’s launch envelope given in Table~\ref{table:missile-data} for the launch angle-off arc. For the \arcrange{60}{-} arc, use the rear envelope, for the \arc{90} and \arc{120} arcs, use the side envelope, and for the \arcrange{150}{+} arc, use the front envelope.
    
    \item The launching aircraft must successfully complete the seeker lock-up procedure described below.
    
\end{itemize}

\paragraph{Seeker Field-of-View Launch Requirements.} 
An IRM seeker can only lock-up on a target in its field of view. These are given in Table~\ref{table:irm-seeker-field-of-view} and depend on whether the seeker is caged or uncaged (see below) and whether lock-up is assisted by other means.

\paragraph{Caged and Uncaged Fields-of-View.} \label{rule:ir-uncage} 
An IRM seeker can be caged (fixed) or uncaged (able to swivel to acquire or track targets over a larger field). The launch field-of-view of a caged seeker is the limited arc, and that of an uncaged seeker is the \arcrange{180}{+} arc. The technology section of the ADC indicates if an aircraft has “IR Uncage” ability. If it does, it can uncage I, M, or A seekers before launch. Seekers are declared to be caged or uncaged in the aircraft decisions phase.

% ISSUE: Treat limited as 180 left/right when banked.


\addedin{1C}{1C-tables}{
    %!TEX root = ../rules-working.tex
%LTeX: enabled=false

\begin{onecolumntablefloat}
\begin{onecolumntable}
\tablecaption{table:irm-seeker-field-of-view}{IR Seeker Field of View Limits for Launch}
\medskip
\begin{tabularx}{\linewidth}{X}
\toprule
\begin{enumerate}
    \item Regular FOV = As Limited radar arc
    \item Uncaged FOV = 180+ angle-off arcs
    \item Uncaged FOV with Helmet sight = 150+ angle-off arcs
    \item Uncaged FOV with radar assist = lesser of 150+ or radar arc
    \item Uncaged FOV with VAS assist (M, A only) = 180+ arcs
    \item IRSTS Assisted FOV = Same as IRSTS system
    \medskip
    \item[--] If target one of several in unassisted Uncaged FOV a roll of \minusafter{8} is required for seeker lock-on; \minusafter{9} with helmet sights. Modifier of \plus{1} to roll for each aircraft the seeker must look past.
\end{enumerate}
\\
\bottomrule
\end{tabularx}
\end{onecolumntable}
\end{onecolumntablefloat}
}
\begin{onecolumntable}
\tablecaption{table:irm-seeker-limits}{IRM Seeker Limits}
\begin{tabularx}{\linewidth}{clP}
\toprule
Seeker  &Type       &Angle-Off Limits\\
\midrule
E       &Early      &Target's 60 degree arc or less if target used A/B power; target's 30 degree arc or less otherwise.\\
I       &Improved   &Target's 60 degree arc or less at any target power setting.\\
M       &Modern	    &Target's 120 degree arc or less if it used A/B power; target's 90 degree arc or less otherwise.\\
A       &Advanced   &Any of target's angle-off arcs at any target power setting.\\
\bottomrule
\end{tabularx}
\end{onecolumntable}


\paragraph{Seeker Angle-Off Launch Requirements.} An IRM seek\-er cannot lock-up a target unless the launching aircraft is within the target angle-off arc given by Table~\ref{table:irm-seeker-limits}. Propeller and helicopter targets can be locked-up from the \arcrange{180}{-} arc (i.e., any direction). Jet targets can be locked-up from arcs that depend on their power setting (rule~\ref{rule:engine-thrust}) and the seeker type.

\paragraph{Seeker Lock-Up Procedure.} If the target is within the seeker’s field-of-view and the launching aircraft satisfies the angle-off requirement, the launching aircraft can attempt to lock-up the target as follows:

\begin{itemize}
    \item If there is only one target in a seeker’s field-of-view, it is automatically locked-up.

    \item If more than one aircraft, including friendlies, is in the field of view of a caged missile, the closest aircraft is automatically locked-up. 
    
    If more than one aircraft is equally close, randomly determine by die roll which one was actually locked-up and therefore is the target \emph{after} the missile is launched. A separate roll is made for each missile launched.

    \item If more than one aircraft, including friendlies, is in the field of view of an uncaged missile, the target is declared, and then a die is rolled, and the lock-up succeeds on a roll of \minusafter{8}. The die roll is modified by  \plus{1} for every aircraft within the field-of-view that is closer than the target or equally close. If the lock-up is succeeds, missiles may be fired normally. If the lock-up attempt fails, missiles may not be launched.

\end{itemize}

}

\AX{
\section{IRM Launch Modifers} 
\label{rule:irm-launch-modifiers}

In addition to the general missile launch modifiers given in rule~\ref{rule:missile-launch-modifiers}, the following modifiers apply to IRMs:

\begin{itemize}
    \item If an IRM is at a target in infrared ground clutter, it suffers a \plus{2} modifier.

    Infrared ground clutter is caused by heat emission from the ground, which interferes with an IRM seeker's ability to track targets. 

    A target is in ground clutter if the launching aircraft is in the LO or ML altitude bands (altitude levels 0--16) and the target is lower by a number of altitude levels that are more the horizontal range divided by two (rounding down).

    For example, if the launching aircraft is at altitude level 12 (in the ML band) and the horizontal range to the target is 7 hexes, the target will be in ground clutter if it is at altitude level 8 or lower.

    \item If an IRM is launched above the highest cloud layer at a lower target during daylight, it suffers a \plus{3} launch modifier (see advanced rule \ref{rule:clouds}).

    \item If an IRM is launched into Sun clutter, it suffers a \plus{3} launch modifier (see advanced rule~\ref{rule:sun}).

    \item If an IRM is launched at a target with an active DDS program, it suffers a launch modifier equal to the smaller of the DDS flare PPL and the missile flare vulnerability rating (see advanced rule~\ref{rule:dds}).
\end{itemize}

}

\CX{
\paragraph{IRM Tracking Requirements.} Once launched, all IRMs have an uncaged FOV. At the end of every game turn, and at the end of each proportional move for the missile, the target must still be in the missile's FOV otherwise the missile loses its lock-up and becomes unguided. At the instant a missile becomes unguided it is removed from play.
}{
\section{IRM Tracking Requirements} 
\label{rule:irm-tracking-requirements}

Once launched, all IRM seekers become uncaged and have a field-of-view equal to the \arcrange{180}{+} arc. 

At the end of every game turn and at the end of each proportional move for the missile, the target must still be in the field of view. Otherwise, the missile loses its lock-up and becomes unguided. At the instant a missile becomes unguided, it is removed from play.

}

\CX{

\section{IRM Countermeasures}

\paragraph{Flare Decoys.} If an aircraft is equipped with an internal DDS, or is carrying a DDS pod, then it can be equipped with flare decoys to be used against attacking missiles. Flares may be dispensed either through automatic programs as described in rule 19, or manually if the aircraft engages an attacking missile (rule 14.6).

\paragraph{Manual Flare Procedure.} When a missile declares an attack, but before it rolls to hit, an aircraft with a DDS system may declare using flares and “pop” one or two flare clusters. For each flare popped, roll one die; if the result of either roll is equal to or less than the missile's flare vulnerability rating, the missile is decoyed and removed from play. When a flare is popped, it is used up even if it is the second of two popped and the first decoyed the missile. When the aircraft has expended all of its flares, it can no longer decoy IRMs.

\paragraph{Flare Program Procedure.} If a DDS program is in use and includes flares, then a die roll modifier equal to the lesser of the Program Protection Level (PPL) of the flares or the flare vulnerability rating of the missile is used as a positive modifier to the attack die roll. Note: A flare PPL also provides a modifier to IRM launch rolls equal to the program's PPL (see rule 19) or the missile's flare vulnerability rating whichever is less.

\paragraph{Ground Clutter.}\label{rule:irm-ground-clutter} Ground Clutter interferes with an IR missile's ability to track targets. Add 2 to the launch roll if the firing aircraft is in the LO or ML altitude band and fires IR missiles at a “lower” target. A lower target by definition, is one more than one altitude level lower than the firer for each two hexes of horizontal range it is away. Example: a target 7 hexes away is lower if it is more than 3 altitude levels below the firer.

If a missile dives in its proportional move (meaning it loses 2 or more altitude levels by any means) to attack target in the LO altitude band, add 2 to the hit die roll. If a missile attacks a target in “T” level flight (rule 20) add 1 to the hit roll. This can be cumulative with the above modifiers.

}{

\section{IRM Countermeasures}
\label{rule:irm-countermeasures}

\paragraph{Flare Decoys.} 
If an aircraft is equipped with an internal DDS (see advanced rule~\ref{rule:dds}) or carries a DDS pod, it can be equipped with flare decoys to defend against IRMs. Decoys may be used in two ways. Any aircraft may elect to dispense them automatically using a DDS protective program (advanced rule~\ref{rule:dds}), and an engaged aircraft (see rule~\ref{rule:engaging-missiles}) may manually dispense decoy clusters against sighted IRMs (see \ref{rule:missile-attacks}). Decoys may be dispensed manually even if an automatic protective program is in effect (see advanced rule~\ref{rule:dds}).

\paragraph{Manual Flare Decoy Procedure.} This procedure is used if an engaged aircraft manually dispenses flare decoy clusters to defend against an attack by an IRM (see rule~\ref{rule:missile-attacks}). Roll the die once for each flare decoy cluster dispensed. If the result is less than or equal to the missile's flare vulnerability rating, the missile is successfully decoyed and cannot attempt to hit the target. It is immediately removed from play. Table~\ref{table:missile-data} gives the flare vulnerability ratings for missiles. If the missile is not successfully defeated by the decoys, it continues with its attack.

For example, consider an AA-2 Atoll attacking an aircraft with a DDS that contains 8 flare clusters. If the aircraft engages the missile, it can manually dispense 2 flare clusters (plus 2 of any other type of decoy in the DDS). The flare vulnerability of the AA-2 is 6, so each flare cluster defeats the missile on a die roll of \minusafter{6}.

\paragraph{Flare Protective Programs.} If a DDS protective program is in use and includes flares, then a die roll modifier is applied to the launch die roll (see rule~\ref{rule:irm-launch-modifiers}) and to the attack die roll (see rule~\ref{rule:irm-attack-modifiers}).
}

\AX{
\section{IRM Attack Modifiers}
\label{rule:irm-attack-modifiers}

In addition to the general missile attack modifiers given by rule~\ref{rule:missile-attacks}, the following modifiers apply to IRMs:

\begin{itemize}
    \item If the target is using afterburner, military, or idle power (see rule~\ref{rule:engine-thrust}), apply a modifier of \minus{3}, \minus{1}, or \plus{1} respectively. 
    
    If the target does not engage the missile but does select idle power, the modifier is \minus{1} on a die roll of \minusafter{4} and \plus{0} otherwise.

    \item If the target is in infrared ground clutter, apply a modifier of \plus{2}. 
    
    A target is in infrared ground clutter if it is in the LO altitude band (altitude levels 0 to 7) and two or more levels below the missile’s altitude at the start of the proportional move in which the missile attacks.

    \item If the target is in terrain-following flight (see rule~\ref{rule:terrain-following-flight}), apply a modifier of \plus{1}.

    \item If the target has an active DDS dispensing flares (see advanced rule~\ref{rule:dds}), apply the lesser of the flare PPL and the missile flare vulnerability rating given in Table~\ref{table:missile-data} as a modifier. 
    
    For example, consider an AIM-9D missile attacking an aircraft using an active DDS. \ref{table:missile-data} gives a flare vulnerability rating of 5. If the target’s flare PPL is 3, the modifier is \plus{3} (limited by the PPL), but if the target’s flare PPL is 6, the modifier is \plus{5} (limited by the flare vulnerability rating).

    \item If the target has an IRM jamming pod (see advanced rule~\ref{rule:infrared-jammers}), apply the modifier that is appropriate for the arc of the missile attack.
\end{itemize}

}

\trainingnote{
\CX{
You may now play all guns and heat seeking missile only combat scenarios! Ignore the AAA, SAM and Ground Unit Interaction Phases of the SOP.
}{
You may now play all scenarios involving only guns and infrared-guided missiles. Ignore the AAA, SAM, and ground unit interaction phases of the sequence of play.
}
}

\begin{advancedrules}



\CX{
\section{Realistic Seeker Head Vertical Field of View Limits}
\label{rule:irm-realistic-vertical-limits}

Rather than just allowing aircraft that climbed or dived to launch at targets an unlimited distance above or below respectively, a more realistic set of seeker head limits can be simulated by using the Radar Vertical Limits Table (see 16.5) to define the vertical FOV limits for caged and uncaged seekers as follows:

\begin{itemize}

    \item Use the limited radar arc Vertical Limits for a caged seeker head\itemaddedin{2B}{2B-irm-vertical-fov}{ and other situations with a limited horizontal arc}.

    \item Use the 180 degree radar arc Vertical Limits Tables described for uncaged IRMs\itemaddedin{2B}{2B-irm-vertical-fov}{ and other situations with a 180 degree horizontal arc}.

    \itemaddedin{2B}{2B-irm-vertical-fov} Use the 150 degree radar arc Vertical Limits Tables for uncaged seekers used with HMS and other situations with a 150 degree horizontal arc.

\end{itemize}
}{
\section{Realistic Seeker Vertical Field-of-View Limits}
\label{rule:irm-realistic-vertical-limits}

The simple vertical limits given above in the launch requirements (rule
\ref{rule:irm-launch-requirements}) can be replaced by more realistic limits using the radar vertical limits rule (advanced rule~\ref{rule:radar-vertical-limits}). Use the limited, \arcrange{180}{+}, or \arcrange{150}{+} vertical limits corresponding to the horizontal field-of-view given in Table~\ref{table:irm-seeker-field-of-view}.

}

\CX{
\section{Helmet-Mounted Sights}
\label{rule:irm-hms}

\paragraph{Helmet Mounted Sights (HMS) Technology.} Modern aircraft can be equipped with helmet mounted sights (see scenario notes or the technology section of the ADC). HMS allows a pilot firing uncaged IR missiles to attempt lock­up against any one sighted enemy aircraft in the firer's 150 to 180 degree arcs (essentially expanding the uncaged missile's FOV). The lock-up succeeds on an unmodified roll of 9 or less regardless of any closer or equally near aircraft. If the lock-up attempt falls, missiles may not be launched.

\section{IRM Seeker Lock-Up Assistance Methods}
\label{rule:irm-seeker-lock-up-assistance-methods}
\label{rule:irm-radar}
\label{rule:irm-vas}
\label{rule:irm-irsts}

\paragraph{Radar Assist.} Uncaged IR missiles may be slaved to an aircraft's radar (declare in the Aircraft Decisions Phase). If the firing aircraft currently has a radar lock-on (see rule 16) to the intended target, the missiles may be automatically locked-up to it without rolling and regardless of how many aircraft are currently in the missile's FOV. IR missiles may be fired at night against otherwise unsighted targets using radar assist. This is an exception to the rule requiring targets to be sighted.

\paragraph{VAS Assist.} Type “M” and “A” seeker head equipped missiles, if uncaged, may be slaved to the VAS system and automatically lock-up a VAS spotted target as above in radar assist. Declare in the Aircraft Decisions Phase.

\paragraph{IRSTS Assist.} Any IR missile may be slaved to an IRSTS system (declare in the Aircraft Decisions Phase) and may automatically lock-up any target the IRSTS system is locked onto as above in radar assist. A Type B IRSTS lock-on allows missiles to be fired at targets in the firer's 180 arcs even if the missile normally would not have an FOV that wide (i.e., non-uncaged missiles). This, along with HMS technology, are the only exceptions to the missile FOV requirements.
}{
\section{Assisted Seeker Lock-Up Methods}
\label{rule:irm-seeker-lock-up-assistance-methods}

The earliest infrared seekers were caged before launch, and so lock-up required the aircraft maneuver to point the missile almost directly at the target. Later generations allowed the seeker to be uncaged before launch and scan a wider field for targets. Various technologies have been developed to assist the lock-up by directing the seeker to search in specific directions, and these are described in this advanced rule.

\paragraph{Helmet-Mounted Sights (HMS) Technology.}\label{rule:irm-hms} Modern aircraft can be equipped with helmet-mounted sights (see the technology section of the ADC). A HMS allows a pilot launching an uncaged IRM to attempt a lock-up against any one sighted enemy aircraft in the launching aircraft's \arcrange{150}{+} arc, significantly expanding the uncaged missile's launch field-of-view. The lock-up succeeds on an unmodified roll of \minusafter{9}, regardless of any closer or equally close aircraft. If the lock-up attempt falls, the missile may not be launched.

\paragraph{Radar-Assisted Seeker Lock-Ups.}\label{rule:irm-radar} In the aircraft decisions phase, an aircraft may tie its uncaged IRMs to its radar (see  rule~\ref{rule:air-to-air-radar}). The launch field-of-view of the missile is the narrower of the field-of-view of the radar and the \arcrange{150}{+} arc. The aircraft may then lock-up its missiles on any target in the launch field of view with a radar lock-on. The lock-ups succeed automatically, regardless of other aircraft in the missile's field-of-view. IRMs may be fired against unsighted targets this way.

\paragraph{VAS-Assisted Seeker Lock-Ups.}\label{rule:irm-vas} In the aircraft decisions phase, an aircraft may tie its uncaged type "M" and "A" IRMs to its VAS (see advance rule~\ref{rule:vas}). The launch field-of-view of the missile is that of the VAS \arcrange{180}{+}. The aircraft may then lock-up its missiles on any target sighted by the VAS, just as for radar-assisted lock-ups.

\paragraph{IRSTS-Assisted Seeker Lock-Ups.}\label{rule:irm-irsts} In the aircraft decisions phase, an aircraft may tie its caged or uncaged IRMs to its IRSTS (see advance rule~\ref{rule:irsts}). The aircraft may then lock-up its missiles on any target for which it has an IRSTS lock-on, just as for radar-assisted lock-ups. The launch field-of-view of the missile is the limited arc for IRSTS-A and the \arcrange{180}{+} arc for IRSTS-B. The wider field for IRSTS-B applies even to caged missiles.
}

\CX{
\section{Expanded and Reduced IRM Envelopes}

\paragraph{Expanded/Reduced Envelopes.} The listed missile envelopes are for fighter sized targets at normal or military power. Larger or hotter targets may be acquired from greater distances, while targets at idle power may be more difficult to acquire. The following rules reflect this:

\begin{itemize}

    \item Any target with a visibility number of 10 or more, or any target using AB power, or which has a fuel usage number greater than 5 for its chosen power setting increases an IRM's existing lock-on envelope by 50\% (round up).

    \item Any target using idle power reduces an IRM's lock-on envelope to 2/3d's normal amount (round up). Exception: for a large (Vis 10+) target in idle, use the normal missile envelope.

\end{itemize}
}{
\section{Expanded and Reduced IRM Envelopes}
\label{rule:expanded-and-reduced-irm-envelopes}

The missile envelopes in Table~\ref{table:missile-data} are for fighter-sized targets using normal or military power. Larger or hotter targets may be acquired from greater distances, while targets at idle power may be more difficult to acquire. The following rules reflect this.

\paragraph{Expanded IRM Envelopes.}
The maximum range of an IRM's enveloped is increased by a factor of 1.5 (rounding up) if:
\begin{itemize}
\item The target has a visibility of 10 or more (see rule \ref{rule:aircraft-data-cards}) and is not using idle power (see rule \ref{rule:engine-thrust}).
\item The target is using afterburner power (see rule \ref{rule:engine-thrust}).
\item The target's fuel usage is 5 or more fuel points (see advanced rule \ref{rule:fuel-consumption}).
\end{itemize}

\paragraph{Reduced IRM Envelopes}
The maximum range of an IRM's enveloped is reduced by a factor of {\twothirds} (rounding up) if:
\begin{itemize}
\item The target is using idle power (see rule \ref{rule:engine-thrust}) and has a visibility of less than 10 (see rule \ref{rule:aircraft-data-cards}).
\end{itemize}

}

\CX{
\paragraph{Out Of Envelope IRM Launches.}\label{rule:irm-out-of-envelope-launches} An IRM may be launched inside its minimum range with a launch roll modifier of \plus{3}, except range 0 launches are not allowed and range 1 launches for IRMs that do not instantly arm automatically fail.

Type “A” seekers may still be launched at extended range, as defined above, at \changedin{2B}{2B-irm-envelopes}{large targets}{targets with visibility of less than 10} or those which are not in AB power or which do not meet the fuel use parameters given above by accepting the \plus{3} out of envelope launch roll modifier. If the target is at idle power, extended range out of envelope shots are those over 2/3rds the listed range up to the original listed range.

Note: With these rules it is possible to lock-up and launch at targets beyond the missile's flight range capabilities resulting in wasted shots.
}{
\section{Out-Of-Envelope IRM Launches}\label{rule:irm-out-of-envelope-launches} 



\paragraph{IRM Launches Inside Minimum Envelope Range.} An IRM may be launched at a target inside its minimum envelope range. However, range 0 launches are not allowed and range 1 launches for IRMs that are not instant-arming automatically fail. The launch die roll has a modifier of \plus{3}.

\paragraph{IRM Launches Beyond Maximum Envelope Range.}
IRMs with type A seekers can lock-up and launch at targets beyond their usual maximum envelope ranges:
\begin{itemize}
\item If a target qualifies for a reduced envelope (see advanced rule~\ref{rule:expanded-and-reduced-irm-envelopes} above), an IRM with a type A seeker may be launched up to the normal maximum envelope range.
\item If a target qualifies for neither a reduced nor an expanded envelope (see advanced rule~\ref{rule:expanded-and-reduced-irm-envelopes} above), an IRM with a type A seeker may be launched at up to 1.5 times the normal maximum envelope range (rounding up)
\end{itemize}
In both cases, the launch die roll has a modifier of \plus{3}. Firing beyond the envelope can result in launches at targets that exceed the missile's range.





}
\end{advancedrules}

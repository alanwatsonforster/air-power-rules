\rulechapter{Formations}
\label{rule:formations}

\begin{advancedrules}

\section{Formation Flying}

A formation is in effect when one or more aircraft fly and/or operate together as a unit while in close proximity to each other. Formations usually allow for better teamwork.

\paragraph{Formation Types.} Two types of formations are possible: Close and Tactical. Close formations are designated by stacking aircraft in the same hex. Tactical formations exist whenever certain spacing conditions are met.

\paragraph{Formation Size.} Formations can be of the following sizes:

\begin{itemize}
    \item Section (or Element): Two aircraft; one is the leader and the other is the wingman.
    \item Division (or Flight): Three or four aircraft. One is the leader, and the others are wingmen. Alternately, a division may consist of two sections.
\end{itemize}

\paragraph{Formation Leaders.} Each formation must have a designated leader. Leaders are chosen at the start of play or indicated in the scenarios.  If not defined in the scenarios, one division leader is allowed for each four jets in play, and one section leader for each two jets. A division leader doubles as a section leader. During play, formations may split or form according to the leaders available.

\paragraph{Loss of Formation Leaders.} Wingmen always begin play as part of a particular formation. If their original formation leader is lost and no other qualified leader in that formation exists, that formation is dissolved. The former wingmen may join other formations by moving into formation parameters on those leaders. They will not get the initiative benefits but do avoid any penalty for not being in formation.

\subsection{Close Formations}
\label{rule:close-formations}

Up to four aircraft may stack together in the same hex at the same altitude as a close formation. The close formation stack is moved as a single entity when it is the formation leader's time to move. All aircraft in the slack fly exactly as the leader does and must maintain the same speed, altitude and facing as the leader.

\paragraph{Forming Close Formations.} A close formation may be formed prior to the beginning of a scenario (during set up), or in the Aircraft Admin Phase of a game turn whenever two, three, or four friendly aircraft end up in the same position with the same exact facing, speed and altitude.

\paragraph{Splitting Close Formations.} A close formation may split up by declaring the intent to detach airplanes from the stack when it moves. Aircraft which detach are left in the starting hex while the rest of the formation executes its flight. The detached aircraft are then flown.

For example, a four-plane division wishes to split into two sections. Two airplanes (one being a section leader) detach and are left in place while the original division leader and his remaining wingman move. The two detached aircraft may now move elsewhere as a close formation, or split up individually.

A close formation may contain aircraft of differing types or configurations. When an aircraft in the formation is unable to match the leader's moves or speeds, it must detach itself from the formation. Close formations may place restrictions on the activities of wingmen and on the maneuverability of the leader.

\subsection{Tactical Formations}

A tactical formation exists anytime a designated formation leader and his wingmen meet the following parameters:

\begin{itemize}
    \item When they are within six hexes of each other,
    \item and within three altitude levels of each other,
    \item and the wingmen’s' facings are no more than 60{\deg} different from the leader's facing (left or right),
    \item and the leader is not in the wingman's blind arc.
\end{itemize}

Tactical formations can be formed or broken at any time during play by simply flying the aircraft into or out of these established parameters. A tactical formation cannot have more than four aircraft in it. There are no maneuver or combat restrictions in a tactical formation.

\section{Formation Restrictions on Turning}

\paragraph{Close Formations.} If a Close Formation consists solely of two aircraft, turns of up to the HT rate may be used. If more than two aircraft are in close formation, only the EZ and TT rates may be used. If these limits are exceeded, the formation automatically breaks down into tactical formation and collisions are possible if more than two aircraft are in the same position at the end of the turn. Tactical formations have no turning restrictions.

\section{Formation Restrictions on Climbs and Dives}

\paragraph{Close Formations.} Aircraft in close formation may only change altitude using non-AB powered sustained climbs, or steep dives of no more than two altitude levels per turn, or by free descents. If these limits are exceeded, the formation automatically breaks down into tactical formation and collisions are possible if more than two aircraft end the turn in the same position.

\paragraph{Tactical Formations.} Aircraft in Tactical formations have no altitude change restrictions.

\section{Formation Restrictions on Gun and Rocket Combat}

\paragraph{Close Formations.} The wingmen aircraft in a close formation may not fire cannons or rockets at air to air targets. They are too busy holding formation with the formation leader.

\paragraph{Tactical Formations.} There are no restrictions on wingmen of Tactical formations.

\section{Formation Effects on Sighting}

\paragraph{Close Formations Restrictions.} Wingmen in close formations may not padlock enemy aircraft, nor be counted for the multiple searching aircraft modifier. They may not be the reference aircraft either. They are concentrating on holding formation on the leader. Aircraft in tactical formations have no sighting restrictions.

\paragraph{Close Formation Effects.} Any sighting attempts against aircraft in a close formation are done treating the close formation as a single entity. The sighting number of the largest aircraft in the formation is used and a \changedin{1E}{AWF}{minus 1}{\minus{1}} modifier is applied to the sighting roll for each two aircraft in the formation. Success indicates all aircraft in the formation are spotted.

\section{Formations and Order of Flight}

\paragraph{Initiative.} All aircraft in close formations use the leader's initiative in place of their own. Wingmen aircraft in tactical formations whose initiative ends up being less than their leader's may add one to their initiative (reflecting teamwork and radio calls). Regular or less quality aircrew who are not in a formation of some sort must subtract one from their Initiative.

\paragraph{Order of Flight.} All aircraft in a Close Formation move with and at the same time as their leader. Their leader's order of flight is determined normally. Aircraft in Tactical Formations move individually with their order of flight determined normally.

\section{Formation Restrictions on Maneuvers}

\paragraph{Close Formations.} A close formation may slide, but may not use any other maneuvers.

\paragraph{Tactical Formations.} Aircraft in tactical formation are not restricted in which maneuvers they may perform.

\section{Formation Effects on Missile Attacks}

\paragraph{Missiles Versus Close Formations.} Heat seeking missiles launched at a close formation, randomly determine which aircraft is the actual target. Radar, laser, or optically guided missiles target aircraft normally (See following missile rules).

\paragraph{Engaging Missiles.} Aircraft that remain in close formation may not engage missiles. If any aircraft wishes to detach to engage a missile, the close formation is automatically nullified and all must move during the engaged aircraft movement phase and all are restricted from performing actions as if each had engaged the missile (they would all initially be unsure of who the real target is).

\section{Formations and Radar Detections}

\paragraph{Radar Searches.} Enemy radar searches are done against the Close formation as a single entity. If the formation contains 3 or 4 aircraft, apply a modifier of \minus{1}. If radar contacted, all aircraft in the close formation are contacted.

\paragraph{Radar Lock-Ons.} For air radar and SAM TTR lock-ons (see Chapter 25 for SAM rules), randomly determine which aircraft in the close formation is locked up. Exception: an aircraft radar of 120+ or 150+ arc ability with a search strength of 40 or more, or a SAM TTR of VF or MW frequency is powerful enough to distinguish individual aircraft in the formation and may choose which aircraft is locked up normally.

\section{Formation Leader Considerations}

\paragraph{Pilot Quality.} A section or division leader must be of Regular or better quality. Aircraft with at least one veteran in the crew do not suffer the initiative penalty for not being in formation. Green pilots must always begin a general scenario game as a member of a close formation.

\end{advancedrules}

\CX{
\rulechapter{Changing A/C Direction}
}{
\rulechapter{Changing Aircraft Direction}
}

\CX{
This chapter discusses the procedures for changing an aircraft's direction by turning.

Aircraft change facing by turning. Facing changes usually occur in 30 degree increments. To execute a turn, an aircraft must first fly a certain distance forward and then it may change facing by 30 degrees to the left or right depending on which way it was turning. The distance it must fly before changing facing is dependent on its speed, altitude band, and rate of turn as described below. An aircraft may change facing as often as possible within a single game turn depending on its selected turn rate.
}{
This rule gives the procedures for changing an aircraft's facing by turning.

An aircraft executes a turn by flying a certain distance forward and then changing facing by 30{\deg}\DY{ (or sometimes 60{\deg} or 90{\deg})} to the left or right, according to the direction of the turn. The distance it must fly before changing facing depends on its speed, altitude band, and turn rate as described below. An aircraft may change facing more than once in a single game turn if its selected turn rate allows this.
}

\section{Turning}
\label{rule:turning}

\addedin{1C}{1C-tables}{
    \begin{table*}
\centering
\caption{Integrated Turn Chart}
\medskip
\begin{tabular}{p{3em}*{12}{r}p{12em}}
\hline
\multicolumn{14}{c}{LO and ML Altitude Bands (1--7 and 8--16)}\\
\hline
\multirow{2}{=}{Turn Rate}&\multicolumn{12}{c}{Start Speed}\\
&1&2&3&4&5&6&7&8&10&12&14&18+&Notes\\
\hline
EZ&60&1&2&3&4&6&8&10&12&14&16&20\\
TT&90&60&1&2&3&4&5&6&8&10&12&14\\
HT&NA&90&60&1&2&2&3&4&6&8&10&12\\
BT&NA&NA&90&60&1&1&2&3&4&6&8&10\\
ET&NA&NA&NA&90&60&1&1&2&3&4&6&8&GLOC possible\\
\hline
\multicolumn{14}{c}{MH Altitude Band (17--25)}\\
\hline
\multirow{2}{=}{Turn Rate}&\multicolumn{12}{c}{Start Speed}\\
&1&2&3&4&5&6&7&8&10&12&14&18+&Notes\\
\hline
EZ&1&2&3&4&6&8&10&12&14&16&18&22\\
TT&60&1&2&3&4&6&7&8&10&12&14&18\\
HT&NA&60&1&2&3&4&5&6&8&10&12&14\\
BT&NA&NA&60&1&2&2&3&4&6&7&10&11\\
ET&NA&NA&NA&60&1&1&2&2&4&5&7&9&GLOC possible\\
\hline
\multicolumn{14}{c}{HI Altitude Band (26--35)}\\
\hline
\multirow{2}{=}{Turn Rate}&\multicolumn{12}{c}{Start Speed}\\
&1&2&3&4&5&6&7&8&10&12&14&18+&Notes\\
\hline
EZ&2&3&4&6&8&10&12&14&16&18&20&24&\multirow[t]{4}{=}{Add 1 prep-move to all maneuvers\deletedin{2A}{2A-snap}{ and to snap turns}.}\\
TT&1&2&3&4&5&6&8&10&12&14&16&20\\
HT&NA&1&2&3&4&5&6&8&9&10&13&16\\
BT&NA&NA&1&2&3&3&4&6&7&8&10&12\\
ET&NA&NA&NA&1&2&2&3&4&5&6&8&10&No more GLOC risk\\
\hline
\multicolumn{14}{c}{VH Altitude Band (36--45)}\\
\hline
\multirow{2}{=}{Turn Rate}&\multicolumn{12}{c}{Start Speed}\\
&1&2&3&4&5&6&7&8&10&12&14&18+&Notes\\
\hline
EZ&2&4&6&8&10&12&14&16&18&20&22&24&\multirow[t]{5}{=}{Add 2 prep-moves to all maneuvers\deletedin{2A}{2A-snap}{ and to snap turns}. Reduce aircraft power to 2/3ds that listed.}\\
TT&1&2&4&6&8&9&10&13&15&17&20&22\\
HT&NA&NA&3&4&6&7&8&10&12&14&17&20\\
BT&NA&NA&NA&3&4&5&6&7&9&11&14&16\\
ET&NA&NA&NA&NA&3&4&5&6&7&8&10&12\\
\hline
\multicolumn{14}{c}{EH and UH Altitude Bands (46--60 and $61+$)}\\
\hline
\multirow{2}{=}{Turn Rate}&\multicolumn{12}{c}{Start Speed}\\
&1&2&3&4&5&6&7&8&10&12&14&18+&Notes\\
\hline
EZ&3&6&8&10&12&14&16&18&20&22&24&28&\multirow[t]{5}{=}{Add 3 prep-moves at EH \& 4 Preps at UH to all maneuvers\deletedin{2A}{2A-snap}{ and to snap Turns}. Reduce aircraft power to 1/3 that listed.}\\
TT&NA&4&6&8&10&12&13&14&16&18&21&24\\
HT&NA&NA&4&6&7&8&10&11&13&15&18&21\\
BT&NA&NA&NA&4&5&6&7&8&10&12&14&18\\
ET&NA&NA&NA&NA&4&5&6&7&9&10&12&14\\
\hline
\tablemedskip
\tablenotes{14}{0.8\linewidth}{
Note:
\begin{enumerate}
    \item Add 2 to all turn requirements if in UH band.
    \item If aircraft of missile speed falls between two columns refer to the one on the left.
    \item NA = not allowed. 60 or 90 = degrees of facing change per FP expended.
\end{enumerate}
}
\end{tabular}
\end{table*}
}

\CX{
Five rates of turn are available to an aircraft. Each rate of turn corresponds to an increasing angle of bank and wing angle of attack. In general, the greater the angle of bank and angle of attack, the quicker the aircraft will turn and the greater the G force the pilot will feel.
}{
\paragraph{Turn Rates.}
Five rates of turn are potentially available to an aircraft. Each rate of turn corresponds to an increasing angle of bank and wing angle of attack. In general, the greater the angle of bank and angle of attack, the quicker the aircraft will turn and the greater the G force the pilot will feel. The turn rates, from lowest to highest, and their codes are:
\begin{itemize}
\item EZ: Easy turn rate
\item TT: Tactical turn rate
\item HT: Hard turn rate
\item BT: Break turn rate
\item ET: Emergency turn rate
\end{itemize}
An aircraft's ADC, speed, altitude, and damage may restrict the available turn rates. 

\relevantadvancedrules{\ref{rule:aircraft-configuration}, \ref{rule:turning-and-minimum-speeds} and \ref{rule:crew-ability-effects}.}
}

\paragraph{Turning Procedure.} 
\CX{
An aircraft may begin turning at any point in its flight. When the first FP is spent to begin turning, the player must announce:

\begin{itemize}
    \item the direction of the turn (Left or Right) and,
    \item the turn rate, either Easy, Tactical, Hard, Break, or Emergency. (EZ, TT, HT, BT, ET respectively).
\end{itemize}

Next consult \changedin{1C}{1C-tables}{the Integrated Turn Charts}{Table~\ref{table:turn}} and cross index the selected turn rate with the aircraft's current speed (rounded down). \notein{1B}{FH in APJ 36 states: “A/C speed, not FPs, is used on the Integrated Turn Charts.” However, the text already states that it is speed. I have not incorporated this.}
Be sure to use the chart corresponding to the altitude band the aircraft is in. If an “NA” is encountered, that combination of turn rate and speed is not allowed. If a number appears, that is the minimum number of FPs that the aircraft must expend in flight before changing facing. If a 60 or 90 appears, that indicates the aircraft may change facing by up to 60 or 90 degrees respectively for each FP expended inflight that game turn. \addedin{2A}{2A-sustained}{A 60 or 90 degree turn is considered one facing change except for purposes of sustained turning decel.}

Changing facing does not cost any FPs; it is the end result of having begun to turn and having used the specified number of FPs (or more) in flight while turning. A turn consists of all the FPs used prior to changing the aircraft's facing plus the act of facing itself. If an aircraft is on a hexside at the time of facing, shift it to the adjacent hex (in the direction the aircraft is turning) and then change its facing as shown \changedin{1C}{1C-figures}{below}{in Figure~\ref{figure:changing-facing}}. 
}{
\CY[3A-turning]{
An aircraft may begin turning at any point in its flight. When it spends its first FP to begin turning, the player must announce:

\begin{itemize}
    \item the direction of the turn (left or right) and,
    \item the turn rate.
\end{itemize}

Next, consult Table~\ref{table:turn} and cross-index the selected turn rate with the aircraft’s speed (rounded down). Use the section corresponding to the aircraft’s current altitude band. If the result is a dash, that combination of turn rate and speed is not allowed. 
If a 60 or 90 appears, the aircraft may change facing by up to 60{\deg} or 90{\deg}, respectively, for each FP expended.
If another number appears, it indicates the minimum number of FPs that the aircraft must expend before changing facing by 30{\deg}.

Changing facing does not cost any FPs; it is the result of having begun to turn and having used the specified number of FPs (or more) in flight while turning. A turn consists of all the FPs used before changing the aircraft’s facing and the act of facing itself. If an aircraft is on a hex side at the time of facing, shift it to the adjacent hex (in the direction the aircraft is turning) and then change its facing as shown in Figure~\ref{figure:changing-facing}.

}{
An aircraft may declare a turn at any point in its flight. When it declares a turn the player must announce:

\begin{itemize}
    \item the direction of the turn (left or right) and,
    \item the turn rate.
\end{itemize}

Next, consult Table~\ref{table:turn} and cross-index the selected turn rate with the aircraft’s speed (rounded down) to determine the turn requirement. Use the section corresponding to the aircraft’s current altitude band. If the result is a dash, that combination of turn rate and speed is not allowed. 
If a 60 or 90 appears, the aircraft may declare and execute up to two or three turns, respectively, for each FP expended.
If another number appears, it indicates the minimum number of FPs the aircraft must expend between declaring and executing the turn.

An aircraft executes a turn by changing its facing by 30{\deg} in the sense of the turn. If an aircraft is on a hex side when it executes a turn, shift it to the adjacent hex (in the direction the aircraft is turning) and then change its facing as shown in Figure~\ref{figure:changing-facing}. 

Executing a turn does not cost any FPs; it is the result of using the specified number of FPs (or more) in flight after declaring the turn.
}

\relevantadvancedrules{\ref{rule:direction-of-bank}.}

}

\changedin{1D}{AWF}{

\begin{FIGURE}
\includegraphics[width=0.7\linewidth]{figures/figure-changing-facing.pdf}
\end{FIGURE}

}{
\begin{FIGURE}

\begin{tikzfigure}{5.333\standardhexwidth}

    \drawhexgrid{0}{0}{4}{3}  

    \drawaircraftcounter{0.50}{0.75}{60}{MiG-21}{}
    \drawaircraftcounter{1.00}{1.50}{60}{MiG-21}{}
    \drawdashedcounter{1.50}{2.25}{60}
    \drawaircraftcounter{1.00}{2.50}{90}{MiG-21}{}

    \drawaircraftcounter{3.00}{0.50}{90}{F-4}{}
    \drawaircraftcounter{3.00}{1.50}{90}{F-4}{}
    \drawdashedcounter{3.00}{2.50}{90}
    \drawaircraftcounter{3.00}{2.50}{120}{F-4}{}
    
    \begin{scope}[shift={(15:0.3)},thick,->]
        \miniathex{0.50}{0.75}{\draw (60:0.05) -- (60:0.4);}
        \miniathex{1.00}{1.50}{\draw (60:0.05) -- (60:0.4);}
    \end{scope}
    \begin{scope}[shift={(135:0.3)},thick,->]
        \miniathex{3.00}{0.50}{\draw (90:0.1) -- (90:0.5);}
        \miniathex{3.00}{1.50}{\draw (90:0.1) -- (90:0.5);}
    \end{scope}

    \miniathex{1.667}{2.5}{ \draw[thick,->] (-0.1,+0.1) arc (30:60:0.6);}

    \miniathex{3.0}{2.5}{ \draw[thick,->] (45:0.45) arc (45:75:0.5);}

\end{tikzfigure}

\CAPTION{figure:changing-facing}{\protect\x{Changing Facing.}{The aircraft move forward two hexes or hexsides and then change facing by 30 degrees to the left. The right aircraft changes facing on a hexside, and so shifts to the adjacent hex in the direction of the turn.}}

\end{FIGURE}
}



\paragraph{Stopping or Changing Turns.} 
\CX{
An aircraft may stop a turn at any point in its flight prior to changing its facing but any FPs that were used toward the aborted turn may not be counted toward any other maneuvering requirements or new turns. The turn rate in use may be changed to a tighter one whenever a facing change is completed or when the turn is stopped and started anew. An aircraft may always expend more FPs than necessary prior to changing facing.
}{
An aircraft may stop a turn at any point in its flight prior to \CY[3A-turn-drag]{changing its facing}{executing the turn and changing its facing}, but any FPs used toward the aborted turn may not be counted toward any other maneuvering requirements or new turns. The turn rate may be changed to a higher one when an aircraft changes facing change or stops turning. An aircraft may always expend more FPs than necessary before changing facing.
}

\AX{
\paragraph{Turning Flight.} An aircraft is considered to be in \emph{turning flight} from the moment it declares a turn to the moment it either completes the turn by changes facing or aborts the turn.
}


\paragraph{Carrying a Turn.} 
\CX{
Sometimes a turn will not or cannot be completed in one game turn. In this case, the turn may be continued into the next game turn. The continuing turn should be noted on the Turn Carry row of the aircraft log with:
\begin{itemize}

    \item the number of FPs expended for the turn so far,

    \item the turn rate (EZ, TT, HT, BT, ET), and

    \item the direction (L, R).

\end{itemize}
}{
Sometimes a turn will not or cannot be completed in one game turn. In this case, the turn may be carried into the next game turn. The carried turn should be noted on the turn carry row of the aircraft log with:
\begin{itemize}

    \item the number of FPs expended for the turn so far,

    \item the turn rate (EZ, TT, HT, BT, or ET), and

    \item the direction (L or R).

\end{itemize}
}

For example, “2BTL” indicates the aircraft has expended 2 FPs on a break turn to the left.

\relevantadvancedrules{\ref{rule:maneuvering-departures}.}

\CX{
\paragraph{Move to Face Requirement.} 
An aircraft must move in order to change facing. If an aircraft ends a game turn with sufficient FPs expended to meet the turning requirements but does not face; it may not change facing in the next game turn until it expends at least one FP in flight. \addedin{1B}{1B-apj-36-errata}{If an aircraft is carrying a turn and the new turning requirement due to the new speed and altitude is equal to or less than what is has already expended, it must still expend one FP before changing facing} It may not face freely and then begin moving; it must move to face.
}{
\CY[3A-turn-drag]{
\paragraph{Move-to-Face Requirement.} 
An aircraft must move to change facing. If an aircraft ends a game turn with sufficient FPs expended to meet the turning requirements but does not face, it may not change facing in the next game turn until it expends at least one FP in flight. If an aircraft is carrying a turn and the new turning requirement due to the new speed and altitude is equal to or less than the number of FPs it has already expended, it must still expend one FP before changing facing. In both cases, it may not change facing and then begin moving; it must move to face.
}{
\paragraph{Move-to-Execute Requirement.} 
An aircraft must move to execute a turn and change facing. If an aircraft ends a game turn with sufficient FPs expended to meet the turning requirements but does not execute the turn, it may not execute the turn in the next game turn until it expends at least one FP in flight. If an aircraft is carrying a turn and the new turning requirement due to the new speed and altitude is equal to or less than the number of FPs it has already expended, it must still expend one FP before executing the turn. In both cases, it may not execute the turn and then begin moving; it must move to execute.
}
}

\CX{
\paragraph{Turn Drag.} Turning creates drag, which causes deceleration. If an aircraft used turning flight in a game turn (whether it changed facing or not), check the Turn Drag Chart on the ADC. At the intersection of the Configuration column and the Turn Rate row is the number of Decel points it receives for making the turn. EZ turns are not listed; there is no Turn Drag for EZ rates (exception; see sustained turning). 

Note: An aircraft is assessed decel points for Turn Drag based on the highest turn rate used in the game turn. Those Turn Drag Decel points are imposed if any turning occurred, whether the aircraft faced once, more than once, not at all, or used different turn rates in the game turn.

\addedin{1B}{1B-apj-22-qa/1B-apj-35-qa/1B-apj-36-errata}{A turn that is carried over from one game-turn to the next must be considered for decel in both game-turns. If it corresponds to the highest turn rate during both game-turns, the decel will be paid twice.}
}{

\CY[3A-turn-drag]{

\paragraph{Turn-Induced Drag.} Turning creates drag, which causes deceleration. If an aircraft used turning flight in a game turn (whether it changed facing or not), consult the turn-drag chart on its ADC. Cross-index the highest turn rate used in the game turn and the configuration to determine the number of turn-drag DPs the aircraft receives. EZ turns are not listed; turns at EZ rates give no drag. 

If an aircraft used different turn rates in a game turn, it only receives turn-drag DPs for the highest turn rate used. 

An aircraft receives turn-drag DPs for turning flight, whether it changes facing once, more than once, or not at all.

A turn carried from one game turn to the next is considered for turn drag in both game turns. If it corresponds to the highest turn rate during both game turns, the corresponding turn-drag DPs will be received in both game turns.

\relevantadvancedrules{\ref{rule:supersonic-speed-effects} and \ref{rule:sustained-turning}.}

}{

\paragraph{Turn Drag.} Turning creates drag, which causes deceleration. Each time an aircraft declares a turn, consult the turn drag chart on its ADC. Cross-index the turn rate and the configuration to determine the number of DPs the aircraft receives. EZ turns are not listed; turns at the EZ rate give no drag.

\relevantadvancedrules{\ref{rule:supersonic-speed-effects}.}

\paragraph{Multiple Turns.}
If the turn requirement is 60 or 90, the aircraft is carrying out a multiple turn. If the turn requirement is 60, the aircraft may declare and execute one or two turns for each FP expended. If the turn requirement is 90, the aircraft may declare and execute one, two, or three turns for each FP expended.

Each executed turn gives a change of facing of 30{\deg}, so in many ways, declaring and executing multiple turns is equivalent to moving one FP and then changing facing by 60{\deg} or 90{\deg}. 

Nevertheless, in these cases the aircraft is actually executing multiple turns each corresponding to a change of facing of 30{\deg}, and this distinction is important for calculating the total drag. That is, declaring and executing two turns (changing facing by a total of 60{\deg}) gives turn drag for two turns, and doing so for three turns (changing facing by a total of 90{\deg}) gives turn drag for three turns.

In the first and last turns of a multiple turn, the aircraft may expend additional FPs (i.e., one or more) before executing the turn.

}

}

\trainingnote{\centering
\CX{
You are now ready to play Training Scenario 1.

The Sequence of Play is not required, ignore it.
}{
You are now ready to play training scenario 1.

The sequence of play is not required, and you may ignore it.
}
}

% ISSUE: Clarify that this is the altitude at the start of the FP (which is relevant, for example, if HTR/C2 takes the aircraft into a different altitude band).

\paragraph{Turning While Climbing or Diving.} 
\CX{
VFPs expended to climb or to dive do count as FPs expended toward turn FP requirements. If an aircraft moves to a new altitude band while turning, use the Turn Chart entry for the altitude band it began the turn in until the aircraft changes facing the first time; then use the chart for the altitude band the aircraft is currently in.
}{
VFPs expended to climb or dive count as FPs expended toward turn FP requirements. 

If an aircraft moves to a new altitude band while turning, use the turn-chart entry for the altitude band in which it expended the first FP being counted toward its current facing requirement. If the aircraft has not yet changed facing, then this will be the altitude band in which it started the turn. If the aircraft has already changed facing and is continuing to turn, this will be the altitude band in which it last changed facing.
}

\begin{advancedrules}

\DY[3A-turn-drag]{

\section{Sustained Turns}
\label{rule:sustained-turning}

Whenever an aircraft changes facing more than once in a game turn, it is performing a sustained turn.

\notein{1B}{AWF: FH in the APJ 36 errata stated that the penalty was for each 30 degrees of facing change, 1B-apj-38-qa stated that this was a change in the 2nd edition rules.}
\CX{
\paragraph{Sustained Turn Drag Penalty.} An aircraft performing a sustained turn receives a drag penalty to 1.0 decel point for each \changedin{2A}{2A-sustained}{change of facing beyond the first}{30 degrees of facing change in the second and subsequent facing changes} in addition to any decel from the Turn Drag Chart. The drag penalty for sustained turning applies even if mixed turn rates or turn rates that normally incur 0 decel points are used.  

Special maneuvers which cause facing changes are not considered for purposes of the sustained turn penalty. \deletedin{2A}{2A-snap}{The snap turn facing change is counter toward determining sustained turn penalties.}
}{
\paragraph{Sustained Turn Drag.} An aircraft performing a sustained turn receives a drag penalty of 1.0 DPs for each 30{\deg} of facing change in the second and subsequent facing changes, in addition to any DPs from turn drag. The aircraft receives sustained turn drag even if it uses mixed turn rates or turn rates that give 0 DPs of turn drag. Special maneuvers that cause facing changes are not considered for purposes of the sustained turn penalty.
}

\changedin{2A}{2A-sustained}{
\paragraph{High Bleed Rate Drag Penalty.} An aircraft noted as having a High Bleed Rate loses speed faster than others in a sustained turn. Such aircraft receive a drag penalty of 2.0 decel points for each change of facing (after the first; in addition to any penalty from the Turn Drag Chart).

As above, the penalty for sustained turning applies even if mixed turn rates or turn rates that normally incur 0 Decel points are used.

}{
\paragraph{Low Bleed Rate.}
\CX{
An aircraft noted as having a low bleed rate loses speed slower than others in a sustained turn and receives a drag penalty of 0.5 DPs (rather than 1.0 DP) for each 30 degrees of facing change in the second and subsequent facing changes.
}{
If an ADC notes that an aircraft has a low bleed rate (LBR), it loses speed slower than others in a sustained turn and receives a sustained drag penalty of 0.5 DPs (rather than 1.0 DPs) for each 30{\deg} of facing change in the second and subsequent facing changes.
}

\paragraph{High Bleed Rate.} 
\CX{
An aircraft noted as having a high bleed rate loses speed faster than others in a sustained turn and receives a drag penalty of 1.5 DPs (rather than 1.0 DP) for each 30 degrees of facing change in the second and subsequent facing changes.
}{
If an ADC notes that an aircraft has a high bleed rate (HBR), it loses speed faster than others in a sustained turn and receives a sustained turn drag of 1.5 DPs (rather than 1.0 DPs) for each 30{\deg} of facing change in the second and subsequent facing changes.
}
}
}

\deletedin{2A}{2A-snap}{
\section{Snap Turning}

\notein{1B}{ISSUE: JDW states in APJ 22 that a snap turn can also be carried out at the ET rate (in order to gain a 60 or 90 degree turn), but I'm not sure how to incorporate this here.}

A snap turn represents an aircraft using its instantaneous maximum angle of attack (as opposed to smoothly increasing its angle of attack). A Snap Turn allows an aircraft to immediately change facing without meeting the normal turning requirements.

\paragraph{Snap Turn Prerequisites} The aircraft must be capable of performing BT turns to do a Snap Turn safely. An aircraft capable of HT turns (but not BT) can Snap Turn, but at a risk of a Maneuvering Departure. If an aircraft cannot perform HT or better turns it may not Snap Turn.

\paragraph{Snap Turn Limits.} An aircraft is limited to one Snap Turn per game turn but may use one at any point in its flight.

\paragraph{Snap Turn Preparatory Moves} If the aircraft is in transonic or supersonic speeds, it must spend 1 HFP in forward flight as a preparatory move before executing a snap turn. If it is not currently wings level (i.e., it is presently turning, use faced, prepping for or executing a maneuver), it must spend 1 HFP in forward flight as a preparatory move prior to executing the snap turn. If both cases apply, then two HFPs must be expended. If performing a Snap Turn at high altitudes, additional HFPs must be expended for preparatory moves as follows:

\begin{itemize}
    \item In the HI band add \plus{1} preparatory HFP.
    \item In the VH band add \plus{2} preparatory HFP.
    \item In the EH band add \plus{3} preparatory HFP.
    \item In the UH+ band add \plus{4} preparatory HFP.
    \itemaddedin{1B}{1B-apj-34-qa}{At supersonic speeds, add \plus{1} preparatory HFP.}
\end{itemize}

\paragraph{Snap Turn FP Costs.} The act of changing facing with a snap turn does cost 1 HFP. \addedin{1B}{1B-apj-22-qa}{A turn rate of HT, BT, or ET can be used.} The aircraft remains in place and changes facing by 30{deg} in the direction of the turn (if the Turn Chart allows 60{deg} or 90{deg} facing, \notein{1B}{AWF: APJ 21 QA has a comment on this, but it is superseded by the following change from the APJ 23 errata.}\changedin{1B}{1B-apj-23-errata}{the snap turn can be up to 60 degrees}{given the aircraft's speed and the turn type in use for this snap, then the snap turn facing can be 30{deg} or 60{deg}}). A snap turn may be used to begin a turn in any turn rate.

\paragraph{Snap Turn Drag} Any time a Snap Turn is used in a game turn, the aircraft incrues decel points as it if did a \changedin{1B}{1B-apj-22-qa}{BT}{BT or ET} turn unless the aircraft's highest allowed turn rate was HT, in which case HT decel + 2.0 is incurred. If an aircraft later turns at the ET turn rate within the game turn, the ET decel would be used instead.

\paragraph{Snap Turn Equivalent to Break Turn.} For purposes of gun combat, weapons launches, and other restrictions, a Snap Turn is equivalent to a \changedin{1B}{1B-apj-22-qa}{BT}{BT or ET} turn. However, if the aircraft also turned at ET rate during the game-turn, the ET restrictions would apply.
}

\CX{
\section{Angle of Bank}
}{
\section{Direction of Bank}
}
\label{rule:direction-of-bank}

\CX{
When an aircraft turns, it banks in the direction of that turn.  If an aircraft first turns left, and then immediately turns right, it must first roll out of the left bank and into a right bank. To reflect this momentary delay, an aircraft must spend 1 FP after the last facing (or after aborting the other turn) before FPs can be used to turn in the opposite direction. The aircraft is reversing its bank while spending this FP; this FP cannot be used for any other maneuvering requirement.
}{
When an aircraft turns, it banks in the direction of that turn. If an aircraft turns in one direction and then in the other, it must roll between the turns to reverse its bank. 

An aircraft must spend 1 FP after the last facing (or after aborting a turn) to reverse its bank before using FPs to turn in the opposite direction. The aircraft cannot use this FP for any turning or maneuvering requirements.
}

\paragraph{High Roll Rate.} 
\CX{
An aircraft noted as having a High Roll Rate can instantly reverse its angle of bank. It is not required to spend the 1 FP to reverse angle of bank. \deletedin{2A}{2A-snap}{\addedin{1B}{1B-apj-23-errata}{It must still pay the one HFP prep to snap turn if not beginning from wings level, or if having just faced, prepped for or executed a maneuver.}}
}{
If an ADC notes that an aircraft has a high roll rate (HRR), it can instantly reverse its bank. It does not need to spend 1 FP to reverse its bank between turns in opposite directions.
}

\paragraph{Low Roll Rate.} 
\CX{
An aircraft noted as having a Low Roll Rate must spend 2 FP after the last facing (or after aborting the other turn) before FPs can be used to turn in the opposite direction. The aircraft is reversing its bank while spending these FPs; the FPs cannot be used for any other maneuvering requirement.
}{
If an ADC notes that an aircraft has a low roll rate (LRR), it must spend 2 FPs after the last facing (or after aborting a turn) before using FPs to turn in the opposite direction. The aircraft cannot use these FPs for any turning or maneuvering requirements.
}

\addedin{1B}{1B-apj-36-errata}{The term “wings level” means an aircraft did not just face, abort a turn, or prep for or complete a maneuver. Thus it reflects an aircraft's readiness to maneuver further. Being “not wings level” is \emph{not} synonymous with being banked.}

\addedin{1B}{1B-apj-36-errata}{Normal and High Roll Rate aircraft do not track angle of bank as such. For example, after coming out of a rolling maneuver they are not banked, but are also not “wings level”. In this situation they are eligible to immediately commence a normal turn\deletedin{2A}{2A-snap}{, but would pay 1 HFP prep for a snap turn.}}

If wings level, a Low Roll Rate aircraft must first spend 1 FP to establish an angle of bank (either left or right) before spending FPs for turns\deletedin{2A}{2A-snap}{ or snap turns}. A Low Roll Rate aircraft may elect to end a flight phase with wings banked, even if no turn carry is in effect. Note BL (Banked Left) or BR (Banked Right) on the Turn Carry line of the Aircraft Log. \deletedin{2A}{2A-snap}{\addedin{1B}{1B-apj-36-errata}{Low Roll Rate aircraft cannot snap turn immediately if aborting a normal turn, after facing from a normal turn, or coming out of a rolling maneuver that leaves them banked. Though already banked, 1 prep HFP must still be expended. If banked the other way, 2 prep HFPs are required. Only if it did not just face, abort a turn, prep, or complete a maneuver, may it use a pre-existing bank attitude to eliminate the banking prep requirement for snap turns.}} Low Roll Rate aircraft must also expend an extra HFP above the normal amount to prep-move for rolling maneuvers.


\paragraph{Rolling Maneuvers and Banks.} 
\CX{
When an aircraft performs a rolling maneuver, it may exit that maneuver banked in any desired direction.
}{
An aircraft may exit a rolling maneuver banked in any desired direction.
}

\section{Turning and Minimum Speeds}
\label{rule:turning-and-minimum-speeds}

\CX{
As the angle of attack on an aircraft's wing increases, so does the speed at which it stalls. To simulate this, no aircraft may utilize turn rates greater than EZ unless their start speed for the turn equals or exceeds their minimum speed plus the amount shown \changedin{1C}{1C-tables}{on the table below}{in Table~\ref{table:turn-minimum-speed}}.
}{
As the angle of attack on an aircraft's wing increases, so does the speed at which it stalls. To simulate this effect, no aircraft may use turn rates greater than EZ unless their start speed for the turn equals or exceeds their minimum speed plus the amount shown in Table~\ref{table:turn-minimum-speed}.
}

\changedin{1C}{1C-tables}{

\begin{tabular}{lll}
\multicolumn{3}{c}{Turn Rate Minimum Speed Requirements}\\
 EZ &=& A/C's Minimum Speed\\
 TT &=& A/C's Minimum Speed + 0.5\\
 HT &=& A/C's Minimum Speed + 1.0\\
 BT &=& A/C's Minimum Speed + 1.5\\ 
 ET &=& A/C's Minimum Speed + 2.0\\
 \end{tabular}

}{

\begin{onecolumntable}
\tablecaption{table:turn-minimum-speed}{Turn Rate Minimum Speed Requirements}
\begin{tabular}{ll}
\toprule
Turn Rate&Minimum Speed Requirement\\
\midrule
 EZ& Minimum Speed\\
 TT& Minimum Speed + 0.5\\
 HT& Minimum Speed + 1.0\\
 BT& Minimum Speed + 1.5\\ 
 ET& Minimum Speed + 2.0\\
\bottomrule
\end{tabular}
\end{onecolumntable}

}

\CX{
If an aircraft has turn carry brought forward to the next game turn, and the new start speed for the aircraft is less than the Minimum Turn Speed required, a Maneuvering Departure automatically occurs. If not using advanced rule 7.7, consider the aircraft to enter regular departed flight.
}{
If an aircraft carries a turn forward and the new start speed is less than that required by Table~\ref{table:turn-minimum-speed}, it suffers a maneuvering departure. (These departures are described in advanced rule~\ref{rule:maneuvering-departures}); if this advanced rule is not being used, the aircraft enters regular departed flight described in rule~\ref{rule:abnormal-flight}.)
}

\DX{
\section{G Induced Loss of Consciousness}
\label{rule:gloc}

Aircrew aboard aircraft which turn too sharply may lose consciousness due to G force effects. This is known as GLOC.

\paragraph{When Can GLOC Occur?} 
When an aircraft turns at the ET rate in the LO, ML, or MH altitude bands its crew may lose consciousness (at higher altitudes, the G-force of these turn rates is insufficient to cause GLOC). Each crewmember must check for GLOC after each facing change at the ET rate.

\addedin{1C}{1C-tables}{
    \begin{table}
\centering

\caption{G-Induced Loss of Consciousness}

\medskip
\begin{tabular}{p{0.75\linewidth}l}
\hline
\multicolumn{2}{c}{Crewmember}\\
\hline
Non-pilot crewmember &$-1$\\
Excellent fitness    &$+1$\\
Poor fitness         &$-1$\\
\hline
\multicolumn{2}{c}{Aircraft}\\
\hline
%\tablerownotein{2A}{2}{2A:snap: snap-turn modifier deleted in 2A.}
\tablerowdeletedin{2A}{2A:snap}{\deletedin{2A}{2A:snap}{Used snap-turn this phase}   &\deletedin{2A}{2A:snap}{$-1$}}
Has canted seat (e.g., F-16)    &$+1$\\
2nd or subsequent GLOC die roll in GLOC cycle (cumulative)&$-1$\\
\hline
\tablemedskip
\tablenotes{2}{0.9\linewidth}{
\begin{itemize}
    \item Check for GLOC if aircraft turned at ET rate while in the LO, ML, or MH altitude bands.
    \item Roll one die after each facing at the the ET rate for each crewmember. A “1” or less indicates he has GLOC'd.
    \item Cycle lasts until no BT/ET turns used in a game-turn.
\end{itemize}
}
\end{tabular}
\end{table}

\begin{table}
\centering

\caption{GLOC/Disoriented Flight}
\small
\medskip
\begin{tabular}{lp{6cm}}
\hline
Die Roll&Aircraft Random Movement\\
&(Based on Current Flight Type)\\
\hline
\multicolumn{2}{c}{Level Flight}\\
\hline
1   &Stay level, no turns.\\
2   &Stay level, TT turn.\\
3   &Stay level, HT turn.\\
4   &Descend one level, TT turn as above.\\
5   &Descend one level, HT turn as above.\\
6   &Maximum sustained climb, EZ turn.\\
7   &Maximum zoom climb, TT turn.\\
8   &Maximum zoom climb, HT turn.\\
9   &Maximum steep dive, HT turn.\\
10  &Half roll and dive, minimum vertical dive, random vertical rolls.\\
\hline
\multicolumn{2}{c}{Climbing Flight}\\
\hline
1   &Maximum sustained climb, EZ turn.\\
2   &Maximum zoom climb, HT turn.\\
3   &Maximum zoom climb, no turns.\\
4   &Maximum zoom climb, TT turn.\\
5   &Minimum vertical climb, no vertical rolls.\\
6   &Maximum vertical climb, random vertical rolls.\\
7   &Level flight, TT turns.\\
8   &Level flight, HT turns.\\
9   &Half roll and dive, minimum steep dive.\\
10  &Half roll and dive, maximum steep dive.\\
\hline
\multicolumn{2}{c}{Diving Flight}\\
\hline
1   &Level flight if able or meet steep dive requirements while exiting vertical dive.\\
2   &As above plus TT turns.\\
3   &As 1 above plus HT turns.\\
4   &Minimum steep dive, no turns.\\
5   &Minimum steep dive, TT turns.\\
6   &Minimum steep dive, HT turns.\\
7   &Maximum steep dive, TT turns.\\
8   &Maximum steep dive, HT turns.\\
9   &Minimum vertical dive, random vertical rolls.\\
10  &Maximum vertical dive, random vertical rolls.\\
\hline
\end{tabular}

\medskip
\begin{minipage}{\linewidth}
\begin{center}
Directions
\end{center}
\begin{itemize}
    \item Expend all remaining FPs via directions above, it is allowed to switch between climbs and dives in mid-moves if required. Randomly determine direction of turns. Random vertical rolls occur on last VFP only, roll for direction and number of facings.
    \item For climbs and dives, use maximum allowed VFPs. A maximum climb/dive means each VFP gains max possible levels. Minimum means each gains least amount possible.
\end{itemize}
\end{minipage}
\end{table}

\begin{table}

\caption{Recovery from GLOC}

\medskip
\begin{minipage}{\linewidth}\begin{itemize}
    \item Automatic during admin phase of 2d game turn following the one in which GLOC occured.
    \item Early recovery possible in admin phase of game turn of GLOC occurence and in the admin phase of the turn following if crewmember has excellent fitness or is ina multi-crew aircraft where other member not GLOC'd. Die roll 4 or less equals early recovery.
\end{itemize}
\end{minipage}

\end{table}

}

\paragraph{GLOC Procedure.} 
Roll the die once for each crewman and apply any required modifiers\addedin{1C}{1C-tables}{\ from Table~\ref{table:gloc-avoidance}}. A result of 1 or less results in GLOC. \deletedin{1C}{1C-tables}{The GLOC roll is modified for the following:

\begin{itemize}
    \item Second or subsequent GLOC roll in a GLOC cycle = \minus{1} (cumulative)
    \item Non-pilot crewmember checking = \minus{1}
    \item Canted seat in aircraft = \plus{1}
    \item Crewmember fitness (variable) = \plus{/-1} (See Chapter 18)
\end{itemize}
}

\paragraph{GLOC Cycle.} The GLOC check cycle continues with increasing probability of unconsciousness until the aircraft does no ET or BT turn rates in a game turn.

\paragraph{GLOC Duration and Recovery.} 
\addedin{1C}{1C-tables}{Table~\ref{table:gloc-recovery} summarizes the procedure for recovery from GLOC.} GLOC normally lasts from the instant it occurs until the affected crewmember recovers. Affected crewmembers automatically recover in the Admin Phase of the second game turn following the one in which GLOC occurred (If a pilot blacked-out during turn 5, recovery would occur in the Admin Phase of turn 7).

\paragraph{Early Recovery.} A crewmember may recover from GLOC early. \notein{1B}{AWF: APJ 21 QA has a comment on this, but it is superseded by the following change from the APJ 23 errata.}\changedin{1B}{1B-apj-23-errata}{Unconscious crewmembers with excellent fitness, or unconscious crewmembers in multi-crew aircraft in which another member is not GLOC'd, are eligible for early recovery.  Check for early recovery during the Admin Phase of the game turn following the one in which GLOC occurred. }{All GLOC'd crewmembers are eligible for early recovery in the admin phase of the turn following the one in which GLOC occurred. Additionally, GLOC'd crew having excellent fitness or in a multi-crew aircraft where another crewman is not GLOC'd are eligible for early recovery in the admin phase of the turn in which GLOC occurred.} To check for early recovery, roll the die: the crewmembers recover on a result of 4 or less (no modifiers apply).

\paragraph{Effects Of GLOC.} 
An individual affected by GLOC is unconscious. The following procedures are followed depending on which crewmember is unconscious:

\begin{itemize}

    \item\itemparagraph{Unconscious Pilot.} The aircraft's flight is randomized. It is controlled by \changedin{1C}{1C-tables}{the GLOC/Disoriented Flight Tables}{Table~\ref{table:gloc-flight}}. Roll on the table once for the present game turn. If the pilot does not recover early, roll once on the table for the following game turns as well.

    \item\itemparagraph{Unconscious Radar Officer (or Weapons Officer or Observer).} The efforts of that crewmember are lost until he recovers. Radar cannot be used, except for Auto Track and Boresight modes. The bombsight is degraded one level (but never to less than manual). Multi-crew spotting and any other multi-crew benefits are lost.

    \item\itemparagraph{Ejection.} If the pilot of a multi-crew aircraft is GLOC'd, and randomized flight of the aircraft will result in an inevitable crash, a conscious crewmember can eject himself and all other crewmembers.

\end{itemize}
}

\section{Maneuvering Departures}
\label{rule:maneuvering-departures}

\CX{
An aircraft may depart controlled flight for reasons other than being stalled. An aircraft may experience maneuvering departures in the following situations:

\begin{itemize}

    \item When its Start speed is insufficient for a carried turn rate.

    \itemdeletedin{2A}{2A-snap}{When performing a risky snap-turn.}
    
    \itemaddedin{2B}{2B-pssm-maneuvering-departures} When it has poor supersonic maneuverability (PSSM), is supersonic, and its carried turn rate is not permitted.  


    \item When an aircraft is above its ceiling and attempts to roll or use higher than EZ turns.

    \item When an aircraft executes a rolling maneuver in the EH or higher altitude bands.

\end{itemize}

A Maneuvering Departure is always automatic in the first case and occurs on a die roll of 4 or less in the latter two.  
}{
An aircraft may depart controlled flight for reasons other than being stalled. An aircraft may experience a maneuvering departure in the following situations:

\begin{itemize}
    \item When its start speed is insufficient for its carried turn rate (see rule \ref{rule:turning-and-minimum-speeds}).
    
    \item When it has poor supersonic maneuverability (PSSM), is supersonic, and its carried turn rate is not permitted (see rule \ref{rule:pssm}).

    \item When it is above its ceiling and attempts to roll or turn at a rate higher than EZ (see advanced rule \ref{rule:flight-above-maximum-ceiling}).

    \item When it executes a rolling maneuver in the EH or higher altitude bands (see rule \ref{rule:rolling-maneuvers-at-eh-or-uh-altitude}).

\end{itemize}

A maneuvering departure is automatic in the first case and occurs on a die roll of \minusafter{4} in the latter two. 
}

\CX{
When a maneuvering departure occurs, use the following procedures for its abnormal flight:

\paragraph{Determine Location Shift.} A maneuvering departure will shift the aircraft's position. Divide the aircraft's remaining FPs (at the instant of maneuvering departure) by 2 and drop any fractions. This result is the number of hexes the aircraft will be moved during the maneuvering departure.

\paragraph{Facing Changes.} Determine the direction and quantity of facing changes as for regular departures. However, initially make only the first facing change, then shift the aircraft the required number of hexes determined as described above, and only then apply the remaining facing changes.

}{
\paragraph{Maneuvering Departure Procedure.} When an aircraft suffers a maneuvering departure, use the following procedures for its abnormal flight. A maneuvering departure will change the aircraft's map location and facing. Divide the aircraft's remaining FPs (at the instant of maneuvering departure) by two and round down to determine the number of hexes the aircraft will move during the maneuvering departure. Determine the direction and number of 30{\deg} facing changes as for regular departures (see rule~\ref{rule:abnormal-flight}). However, initially make only the first 30{\deg} facing change, then move the aircraft the number of hexes determined above, and only then apply the remaining facing changes.

}

\CX{
\addedin{1B}{1B-apj-36-errata}{\paragraph{Altitude Loss.} On the turn in which it occurs, a Maneuvering Departure does not, by itself, impose an altitude loss. The subsequent departed flight most likely will.}
}{
\paragraph{Altitude Loss.} A maneuvering departure does not, by itself, cause an aircraft to lose altitude. The subsequent departed flight most likely will.
}

\paragraph{Subsequent Turns.} On subsequent game turns, a maneuvering departure is treated as regular departed flight.

\DX{
\section{Formation Restrictions on Turning}

\paragraph{Close Formations.} If a Close Formation consists solely of two aircraft, turns of up to the HT rate may be used. If more than two aircraft are in close formation, only the EZ and TT rates may be used. If these limits are exceeded, the formation automatically breaks down into tactical formation and collisions are possible if more than two aircraft are in the same position at the end of the turn. Tactical formations have no turning restrictions.
}
\end{advancedrules}

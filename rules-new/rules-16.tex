\CX{
\rulechapter{Air to Air Radar}
}{
\rulechapter{Air-to-Air Radar}
}
\label{rule:air-to-air-radar}

\DX{

This chapter details the procedures for the use of radar for detection and tracking of other aircraft.

\paragraph{Detection and Lock-On Eligibility.} Any aircraft not currently detected, within radar range, and in the radar arc of the searcher, is eligible to be detected. Count each two levels of altitude as one hex of range. An aircraft that is, or becomes, radar detected is eligible to be locked-on to. Radar guided missiles may only be launched at locked-on targets.

\section{Radar Searches}
\label{rule:normal-mode-searching}

\paragraph{Radar Search Data.} The Radar Data section of the ADC shows two numbers on the search line. The first number is the maximum detection range in hexes; the second number is the radar search strength rating. If a dash exists there, the aircraft has no search capability.

\silentlyaddedin{1C}{1C-figures}{
    \begin{twocolumnfigure}[tbp]

% These are arcs derived directly from the TSOH play aids.

\newcommand{\drawlimitedarcA}[1][]{   
    \draw [yscale=\hexxfactor,#1]
        (-1.600,20.000) --
        (-1.600,10.000) --
        (-1.100, 9.000) --
        (-1.100, 5.000) --
        (-0.600, 4.000) --
        (-0.600, 2.000) --
        (-0.000, 0.333) -- 
        (+0.600, 2.000) --
        (+0.600, 4.000) --
        (+1.100, 5.000) --
        (+1.100, 9.000) --
        (+1.600,10.000) --
        (+1.600,20.000);   
    \draw[yscale=\hexxfactor, <->, transform shape]
        (-1.6,10.5) -- 
        (0,10.5) node [anchor=south] {\minitable{c}{maximum\\width}} --
        (+1.6,10.5);
}
\newcommand{\drawlimitedarcB}[1][]{  
    \draw [yscale=\hexxfactor,#1]
        (-1.6,20.000) --
        (-1.6,11.000) --
        (-1.1,10.000) --
        (-1.1, 6.000) --
        (-0.6, 5.000) --
        (-0.6, 2.000) --
        (+0.0, 0.333) -- 
        (+0.6, 2.000) --
        (+0.6, 5.000) --
        (+1.1, 6.000) --
        (+1.1,10.000) --
        (+1.6,11.000) --
        (+1.6,20.000);  
    \draw[yscale=\hexxfactor, <->, transform shape]
        (-1.6,11.5) -- 
        (0,11.5) node [anchor=south] {\minitable{c}{maximum\\width}} --
        (+1.6,11.5);
}
\newcommand{\drawlimitedarcC}[1][]{  
    \draw [xscale=\hexxfactor,#1]
        (-1.6,20.000) --
        (-1.6, 9.250) --
        (-1.1, 8.500) --
        (-1.1, 5.000) --
        (-0.6, 4.250) --
        (-0.6, 1.250) --
        (-0.0, 0.333) -- 
        (+0.6, 1.250) --
        (+0.6, 4.250) --
        (+1.1, 5.000) --
        (+1.1, 8.500) --
        (+1.6, 9.250) --
        (+1.6,20.000);   
    \draw[xscale=\hexxfactor, <->, transform shape]
        (-1.6,9.75) -- 
        (0,9.75) node [anchor=south] {\minitable{c}{maximum\\width}} --
        (+1.6,9.75);
}

% This figure shows them in the same orientation as in the TSOH play aids.
%\begin{tikzfigure}{0.5\linewidth}
%\begin{scope}[rotate=0]
%    \drawdottedhexgrid{15.0}{15.5}
%    
%    \begin{athex}{1.00}{13.00}
%        \begin{scope}[rotate=-90,thick]
%            \drawlimitedarcA
%        \end{scope}
%        \drawaircraftcounter{0.00}{0.00}{90}{F-4}{}{}
%    \end{athex}
%
%    \begin{athex}{1.00}{8.50}
%        \begin{scope}[rotate=-90,thick,]
%            \drawlimitedarcB
%        \end{scope}
%        \drawaircraftcounter{0.00}{0.00}{90}{F-4}{}{}
%    \end{athex}
%
%    \begin{athex}{1.00}{7.50}
%        \begin{scope}[rotate=-120,thick]
%            \drawlimitedarcC
%        \end{scope}
%        \drawaircraftcounter{0.00}{0.00}{60}{F-4}{}{}
%    \end{athex}
%    
%\end{scope}
%\end{tikzfigure}

\begin{tikzfigure}{0.6\linewidth}
\begin{scope}[rotate=0]

    \drawhexgrid{0}{0}{16}{12}
    \drawpositiongrid{0}{0}{20}{12}
    
    \begin{athex}{7.50}{0.25}
        \begin{scope}[rotate=-30,thick]
            \drawlimitedarcA
        \end{scope}
        \drawaircraftcounter{0.00}{0.00}{60}{F-4}{}{}
    \end{athex}

    \begin{athex}{4.00}{2.00}
        \begin{scope}[rotate=-30,thick]
            \drawlimitedarcB
        \end{scope}
        \drawaircraftcounter{0.00}{0.00}{60}{F-4}{}{}
    \end{athex}

    \begin{athex}{2.00}{0.00}
        \begin{scope}[rotate=0,thick]
            \drawlimitedarcC
        \end{scope}
        \drawaircraftcounter{0.00}{0.00}{90}{F-4}{}{}
    \end{athex}
    
\end{scope}
\end{tikzfigure}

\figurecaption{figure:limited-arcs}{Limited Arcs}

\end{twocolumnfigure}


}

\paragraph{Radar Arcs.} The Radar Data section of the ADC shows a number on the Arcs line. This is aircraft's radar arc. The radar arc for an aircraft is expressed in terms of its angle-off arcs and followed by a plus. Because angle off arcs begin from the tail of the aircraft, the forward direction for an aircraft is 180{\deg}.  Radar arcs include all angle-off arcs higher than the number stated. Thus, the 150+ radar arc includes the right and left 150{\deg} angle-off arcs, and the right and left 180{\deg} angle-off arcs (because they are higher than 150{\deg}). Some aircraft have a “Limited” radar arc. This arc is less than the 180+ radar arc, and is shown in \changedin{1C}{1C-figures}{the Limited Radar Arc diagram of the play aids}{Figure~\ref{figure:limited-arcs}}. \addedin{1B}{1B-apj-28-qa}{In the \changedin{1C}{1C-figures}{diagram}{figure}, only the center point of hexes and hexsides within the limited arc line, as drawn, are within the limited arc.}

\paragraph{Radar Search Procedure.} Aircraft may attempt to detect targets which are in the search aircraft's radar arc and within its maximum detection range. To detect a target, a successful die roll must be made. The actual detection probability depends on the aircraft's search strength rating and the range. Use the following procedure to determine the die roll required:

\addedin{1C}{1C-tables}{
    \Dx{
\begin{twocolumntablefloat}
\begin{twocolumntable}
\tablecaption{table:radar-contact}{Radar Contact}

\begin{tabularx}{0.7\linewidth}{c*{11}{R}}
\toprule
\multicolumn{1}{c}{\multirow{2}{*}{\minitable{c}{Radar\\Strength}}}&
\multicolumn{11}{c}{Die Roll or Less for Contact}\\
&
\multicolumn{1}{c}{10}&
\multicolumn{1}{c}{9}&
\multicolumn{1}{c}{8}&
\multicolumn{1}{c}{7}&
\multicolumn{1}{c}{6}&
\multicolumn{1}{c}{5}&
\multicolumn{1}{c}{4}&
\multicolumn{1}{c}{3}&
\multicolumn{1}{c}{2}&
\multicolumn{1}{c}{1}&
\multicolumn{1}{c}{$0-$}\\
\midrule
\phantom{0}3& 6& 8& 9&10&11&12&13&---&14&---&\phantom{0}15+\\
\phantom{0}6&12&15&18&20&22&24&26&28&29&30&\phantom{0}31+\\
\phantom{0}8&16&20&24&28&30&32&34&36&28&40&\phantom{0}41+\\
10&20&25&30&35&38&40&42&45&48&50&\phantom{0}51+\\
12&24&30&36&42&45&48&51&54&57&60&\phantom{0}61+\\
15&30&38&45&52&56&60&64&68&72&75&\phantom{0}76+\\
18&36&45&54&63&68&72&76&81&86&90&\phantom{0}91+\\
20&40&50&60&70&75&80&85&90&95&100&101+\\
25&50&63&75&87&94&100&106&113&119&125&126+\\
30&60&75&90&105&113&120&128&135&143&150&151+\\
40&80&100&120&140&150&160&170&180&190&200&201+\\
50&100&125&150&175&188&200&213&225&238&250&251+\\
EWR&150&188&225&263&280&300&319&338&356&375&376+\\
\bottomrule
\end{tabularx}

\medskip

\begin{tablenote}{0.7\linewidth}
Above = Maximum Range in Hexes for Each Column

\begin{enumerate}
    \item No Aircraft may contact targets at a range greater than the maximum listed on its ADC.
    \item Regular aircraft radar may not detect or track tgts.\ within 4 levels of the ground unless search is at lower altitude.
    \item If tgt.\ below searcher \& within 10 levels of ground, diff. in alt.\ between the aircraft must be < tgt.'s alt.\ above ground.
    \item Lookdown radar may ignore cases 2 \& 3.  Boresight radar may ignore case 3 against visually sighted targets.
\end{enumerate}
\end{tablenote}
\end{twocolumntable}
\end{twocolumntablefloat}
}

\Dx{

\begin{onecolumntablefloat}
\begin{onecolumntable}
\tablecaption{table:radar-search-modifiers}{Radar Search Modifiers}
\begin{tabularx}{\linewidth}{X}
\toprule
\begin{enumerate}
    \item AJM \# $-$ Air Radar or EWR ECCM.
    \item BJM \# $-$ Air Radar or EWR ECCM.
    \item \changedin{1B}{1B-apj-23-errata and 1B-apj-24-play-aids}{CHAFF PPL Effectiveness No.}{Chaff PPL Eff.\ No.\ $-$ $1/2$ radar ECCM (round \addedin{1B}{1B-apj-36-errata}{product }up).}
    \item Mini-Jammer PPL Effectiveness No.
    \item Aircraft Size Modifier from ADC.
    \itemdeletedin{1B}{1B-apj-23-errata and 1B-apj-24-play-aids}{$+4$ if aircraft has Stealth Technology.}
    \itemaddedin{1B}{1B-apj-23-errata and 1B-apj-24-play-aids}{$-2$ if target aircraft IFF on.}
    \item Tactics Master or Veteran = $-1$ ($-2$ if both).
    \item Novice = $+1$, Green = $+2$
\end{enumerate}
\\
\bottomrule
\end{tabularx}
\end{onecolumntable}
\end{onecolumntablefloat}

}

\Dx{
\begin{onecolumntablefloat}
\begin{onecolumntable}
\addedin{1B}{1B-apj-23-errata, 1B-apj-24-play-aids, and 1B-apj-25-qa}{
\tablecaption{table:radar-lock-on-modifiers}{Radar Lock-On Modifiers}
\begin{tabularx}{\linewidth}{X}
\toprule
\begin{enumerate}
    \item $-2$ if target IFF on
    \item Chaff PPL Eff.\ No.\ $-$ $1/2$ radar ECCM (round up).
    \item Mini-Jammer PPL Effectiveness No.
\end{enumerate}
\\
\bottomrule
\end{tabularx}
}
\end{onecolumntable}
\end{onecolumntablefloat}
}

\Dx{
\begin{onecolumntablefloat}
\begin{onecolumntable}
\tablecaption{table:ew-radar-modifiers}{Air Radar Search and Lock-On Modifiers}
\begin{tabularx}{\linewidth}{lP}
\toprule
ECM Type & Die Roll Modifiers\\
\midrule
CHAFF&Decoy Effectiveness No.\\
Mini-Jammer&Decoy Effectiveness No.\\
AJM&AJM No. $-$ Air Radar ECCM\\
BJM Noise&BJM No. $-$ Air Radar ECCM\\
\multicolumn{2}{l}{Radar using Boresight for lookdown $+2$}\\
\bottomrule
\end{tabularx}
\end{onecolumntable}
\end{onecolumntablefloat}
}

\Dx{
\begin{onecolumntablefloat}
\begin{onecolumntable}
\tablecaption{table:radar-breaking-lock-on}{Breaking Radar Lock-On}
\begin{tabularx}{\linewidth}{X}
\toprule
\addlinespace
Locks are broken when:

\begin{enumerate}
    \item Aircraft stalls, departs, or becomes engaged.
    \item Aircraft does ET turns, Viffs, or does other than vertical rolls.
    \item Aircraft takes a H or C hit, or radar operator GLOCs.
    \item Target cannot be kept in radar arc while illuminating
    \item Target deploys decoys and rolls effectiveness \# or less.
    \item Target employs EW jammers and makes break lock die roll.
\end{enumerate}
\\
\bottomrule
\end{tabularx}
\end{onecolumntable}
\end{onecolumntablefloat}
}

\Dx{
\begin{onecolumntablefloat}
\begin{onecolumntable}
\tablecaption{table:ew-radar-break-lock-rolls}{Air Radar Break Lock Rolls}
\begin{tabular}{lp{15em}}
\toprule
ECM Type&Break Lock Die Roll Number\\
\midrule
CHAFF&Decoy Effectiveness No.\\
MINI-JAMMER&Decoy Effectiveness No.\\
DJM&DJM $-$ Air Radar ECCM\\
\bottomrule
\end{tabular}
\end{onecolumntable}
\end{onecolumntablefloat}
}

\Dx{
\begin{onecolumntablefloat}
\begin{onecolumntable}
\tablecaption{table:radar-search-limitations}{Radar Search Limitations}
\begin{tabularx}{\linewidth}{X}
\toprule
\begin{enumerate}
    \item Pilot Only aircraft may not search if they:
    \begin{enumerate}
        \item \changedin{2A}{2A-snap}{Snap-turned or turned}{Turned} at HT or greater rate.
        \item Fired Guns or made an air to ground attack.
    \end{enumerate}
    \item Multi-crew aircraft may not search if they:
    \begin{enumerate}
        \item \changedin{2A}{2A-snap}{Snap-turned or turned}{Turned} at BT or greater rate.
    \end{enumerate}    
    \item Neither type aircraft may search if they:
    \begin{enumerate}
        \item Are stalled, departed, or engaged.
        \item Performed more than one vertical roll in the turn.
        \item Performed any other roll types or Viff maneuvers.
        \item Did an Unloaded Dive or Damage Control.
        \item Were Hit and “H” or greater damage ensued.
        \item Had their radar operator go into GLOC.
    \end{enumerate}        
\end{enumerate}
\medskip
Note: Boresight and Auto-Track Modes allow maneuver restrictions to be ignored but not attack, damage, or operator GLOC restrictions; these always apply.
\\
\addlinespace
\bottomrule
\end{tabularx}
\end{onecolumntable}
\end{onecolumntablefloat}
}

\Dx{
\begin{twocolumntablefloat}
\begin{twocolumntable}
\tablecaption{table:radar-vertical-limits}{Radar Vertical Limits}
\begin{tabularx}{0.8\linewidth}{c*{6}{C}}
\toprule
\multicolumn{1}{c}{\multirow{2}{*}{\minitable{c}{Type\\Radar}}}&
\multicolumn{1}{c}{\multirow{2}{*}{\minitable{c}{Vertical\\Dive}}}&
\multicolumn{1}{c}{\multirow{2}{*}{\minitable{c}{Steep\\Dive}}}&
\multicolumn{1}{c}{\multirow{2}{*}{\minitable{c}{Level\\Flight}}}&
\multicolumn{1}{c}{\multirow{2}{*}{\minitable{c}{Sust.\\Climb}}}&
\multicolumn{1}{c}{\multirow{2}{*}{\minitable{c}{Zoom\\Climb}}}&
\multicolumn{1}{c}{\multirow{2}{*}{\minitable{c}{Vertical\\Climb}}}\\
\\
\midrule
Limited&$-2$, $-9$&$-0.5$, $-3$&$+0.5$, $-0.5$&$+2$, $+0$&$+4$, $+0.5$&$+9$, $+2$\\
180+&$-1$, $-X$&$-0$, $-5$&$+1$, $-1$&$+3$, $-0.5$&$+5$, $+0$&$+X$, $+1$\\
150+&$-0.5$, $-X$&$-0$, $-8$&$+2$, $-2$&$+4$, $-1$&$+8$, $+0$&$+X$, $+0.5$\\
120+&$-0$, $-X$&$+0.5$, $-X$&$+4$, $-4$&$+6$, $-2$&$+X$, $-0.5$&$+X$, $+0$\\
\bottomrule
\end{tabularx}
\begin{tablenote}{0.8\linewidth}
Note: The radar vertical limits are given as altitude levels per hex or horizontal distance. 
\end{tablenote}

\end{twocolumntable}
\end{twocolumntablefloat}
}


\Dx{
\begin{onecolumntablefloat}
\begin{onecolumntable}
\tablecaption{table:radar-boresight-mode}{Radar Boresight Mode}
\begin{tabularx}{\linewidth}{X}
\toprule
\begin{enumerate}
    \item Radar Arc = Limited.
    \item Max Range = Search Strength No.
    \item Previous contacts and locks lost when mode declared.
    \item Nearest Visually sighted target in aircraft's Limit arc automatically contacted.
    \item Lock-on roll allowed, no mnvr.\ limitations.
\end{enumerate}
\\
\bottomrule
\end{tabularx}
\end{onecolumntable}
\end{onecolumntablefloat}
}


\Dx{
\begin{onecolumntablefloat}
\begin{onecolumntable}
\tablecaption{table:radar-auto-track-mode}{Radar Auto-Track Mode}
\begin{tabularx}{\linewidth}{X}
\toprule
\begin{enumerate}
    \item Radar Arc = 180+ unless normally it's limited; in which case it remains limited.
    \item Max Range = Search Strength No.
    \item Nearest target in radar arc is automatically detected.
    \item If nearest aircraft was a friendly with IFF on, it may be ignored and next nearest is automatically contacted etc.
    \item A visually sighted aircraft in arc may be selected for auto contact if not the closest by rolling 7 or less.
    \item Lock-on roll allowed, no mnvr.\ limitations.
    \item Previous contacts and locks lost when mode declared.
\end{enumerate}
\\
\bottomrule
\end{tabularx}
\end{onecolumntable}
\end{onecolumntablefloat}
}

%%%%%%%%%%%%%%%%%%%%%%%%%%%%%%%%%%%%%%%%%%%%%%%%%%%%%%%%%%%%%%%%%%%%%%%%%%%%%%%

\Ax{

\begin{twocolumntablefloat}
\begin{twocolumntable}
\tablecaption{table:radar-use-summary}{Radar Use Summary}
\small
\begin{tabularx}{0.9\linewidth}{l*{5}{C}}
\toprule
Mode&\multicolumn{3}{c}{Normal}&Boresight&Auto-Track\\
\cmidrule{2-4}
Use&Searching&Tracking Attempt&Maintaining Track&All&All\\
\midrule
Rule
&\ref{rule:normal-mode}
&\ref{rule:normal-mode}
&\ref{rule:normal-mode}
&\ref{rule:boresight-mode}
&\ref{rule:auto-track-mode}\\
\midrule
Turn Limit (pilot only)
&TT or less
&TT or less
&BT or less
&ET or less
&ET or less\\
Turn Limit (radar officer)
&HT or less
&HT or less
&BT or less\\
Slides?&Yes&Yes&Yes&Yes&Yes\\
VRs?&One&One&One&Yes&Yes\\
Other Maneuvers?&No&No&No&Yes&Yes\\
Unloaded FPs?&No&No&Yes&Yes&Yes\\
Preemption?&No&No&No&Yes&Yes\\
TFF (pilot only)?&No&No&No&No&No\\
TFF (radar officer)?&Yes&Yes&Yes&No&No\\
\midrule
Stalled or Departed?&No&No&No&No&No\\
Engaged?&No&No&No&No&No\\
AtA or AtG Attack?&No&No&No&No&No\\
H or C Damage?&No&No&No&No&No\\
Damage Control?&No&No&No&No&No\\
GLOC or Disoriented?&No&No&No&No&No\\
\midrule
Arc&From ADC&From ADC&From ADC&Limited&\arcrange{180}{+}\\
\midrule
Range&Search&Tracking&Tracking&Tracking&Search\\
&Range&Range& Range&Strength&Strength\\
\midrule
Detection Roll&Table \ref{table:radar-searches}&&&\multicolumn{1}{L}{Automatic for closest sighted target.}&\multicolumn{1}{L}{Automatic for closest target or $7-$ for other sighted target. Friendly aircraft with IFF on are ignored.}\\
\midrule
Tracking Roll&&From ADC&&From ADC&From ADC\\
\bottomrule
\end{tabularx}
\end{twocolumntable}
\end{twocolumntablefloat}

\begin{twocolumntablefloat}[tp]
\begin{twocolumntable}

\tablecaption{table:radar-look-down}{Radar Look-Down Summary}
\small
\begin{tabularx}{0.9\linewidth}{cCCCC}
\toprule
&\multicolumn{2}{c}{No Look-Down}&Limited Look-Down&Full Look-Down\\
\cmidrule{2-3}
\minitable{c}{Target\\Altitude}&
\minitable{c}{Radar\\Altitude}&
\minitable{c}{Radar Altitude\\and Horizontal Range}&
\minitable{c}{Radar\\Altitude}&
\minitable{c}{Radar\\Altitude}\\
\midrule
\phantom{0}0&&&&Any\\
\phantom{0}1&$\le\phantom{0}0$&&$\le\phantom{0}0$&Any\\
\phantom{0}2&$\le\phantom{0}1$&&$\le\phantom{0}3$&Any\\
\phantom{0}3&$\le\phantom{0}2$&&$\le\phantom{0}5$&Any\\
\phantom{0}4&$\le\phantom{0}3$&&$\le\phantom{0}7$&Any\\
\phantom{0}5&$\le\phantom{0}5$&$\le\phantom{0}9$ and range $\le4$&$\le\phantom{0}9$&Any\\
\phantom{0}6&$\le\phantom{0}6$&$\le\phantom{}11$ and range $\le5$&$\le\phantom{}11$&Any\\
\phantom{0}7&$\le\phantom{0}7$&$\le\phantom{}13$ and range $\le6$&$\le\phantom{}13$&Any\\
\phantom{0}8&$\le\phantom{0}8$&$\le\phantom{}15$ and range $\le7$&$\le\phantom{}15$&Any\\
\phantom{0}9&$\le\phantom{0}9$&$\le\phantom{}17$ and range $\le8$&$\le\phantom{}17$&Any\\
\phantom{}10&$\le\phantom{}10$&$\le\phantom{}19$ and range $\le9$&$\le\phantom{}19$&Any\\
\bottomrule
\end{tabularx}
\begin{tablenote}{0.9\linewidth}
The table gives conditions under which a radar can detector or track a target at altitude 10 or less according to its level of look-down technology. For no look-down technology, two conditions are given, and if the radar fulfills either condition, it may search or track. Otherwise, the target is in radar round clutter. All altitudes are above the ground level at the location of the target. See rule \ref{rule:radar-ground-clutter}.
\end{tablenote}


\end{twocolumntable}
\end{twocolumntablefloat}

\begin{twocolumntablefloat}
\begin{twocolumntable}
\tablecaption{table:radar-searches}{Radar Detection Rolls}
\small
\begin{tabularx}{0.9\linewidth}{*{12}{C}}
\toprule
\multicolumn{1}{c}{\multirow[b]{2}{*}{\minitable{c}{Search\\Strength}}}&
\multicolumn{11}{c}{Detection Die Roll}\\
\cmidrule{2-12}
&
\multicolumn{1}{c}{$10-$}&
\multicolumn{1}{c}{$9-$}&
\multicolumn{1}{c}{$8-$}&
\multicolumn{1}{c}{$7-$}&
\multicolumn{1}{c}{$6-$}&
\multicolumn{1}{c}{$5-$}&
\multicolumn{1}{c}{$4-$}&
\multicolumn{1}{c}{$3-$}&
\multicolumn{1}{c}{$2-$}&
\multicolumn{1}{c}{$1-$}&
\multicolumn{1}{c}{$0-$}\\
\midrule
\phantom{0}3& 6& 8& 9&10&11&12&13&---&14&---&\phantom{0}15+\\
\phantom{0}6&12&15&18&20&22&24&26&28&29&30&\phantom{0}31+\\
\phantom{0}8&16&20&24&28&30&32&34&36&28&40&\phantom{0}41+\\
10&20&25&30&35&38&40&42&45&48&50&\phantom{0}51+\\
12&24&30&36&42&45&48&51&54&57&60&\phantom{0}61+\\
15&30&38&45&52&56&60&64&68&72&75&\phantom{0}76+\\
18&36&45&54&63&68&72&76&81&86&90&\phantom{0}91+\\
20&40&50&60&70&75&80&85&90&95&100&101+\\
25&50&63&75&87&94&100&106&113&119&125&126+\\
30&60&75&90&105&113&120&128&135&143&150&151+\\
40&80&100&120&140&150&160&170&180&190&200&201+\\
50&100&125&150&175&188&200&213&225&238&250&251+\\
EWR&150&188&225&263&280&300&319&338&356&375&376+\\
\bottomrule
\end{tabularx}
\begin{tablenote}{0.9\linewidth}
The table gives the detection roll for normal mode searches. In the row corresponding to the radar search strength, move right to the column that is equal to or more than the range, and then move up to determine the detection die roll. See rule \ref{rule:normal-mode}.
\end{tablenote}
\end{twocolumntable}
\end{twocolumntablefloat}

\begin{twocolumntablefloat}
\begin{twocolumntable}
\tablecaption{table:radar-search-and-lock-on-modifiers}{Radar Search and Tracking Modifiers}
\small
\begin{tabularx}{0.9\linewidth}{lLccl}
\toprule
Description&Modifier&Search\tablenotemark{1}&Tracking\tablenotemark{2}&Rule\\
\midrule
\multicolumn{5}{c}{Target}\\
\midrule
Target size modifier&See ADC&Yes&No&\ref{rule:aircraft-data-cards}\\
Boresight mode assuming limited look-down&$+2$&No&Yes&\ref{rule:boresight-mode}\\
Auto-track mode assuming limited look-down&$+2$&No&Yes&\ref{rule:auto-track-mode}\\
\midrule
\multicolumn{5}{c}{ECM}\\
\midrule
Target using IFF&$-2$&Yes&Yes&\ref{rule:iff}\advancedrulemark\\
Target using chaff DDS&+PPL modifier\tablenotemark{3} $-$ half ECCM rating\tablenotemark{4}&Yes&Yes&\ref{rule:dds}\advancedrulemark\\
Target using mini-jammer DDS&+PPL modifier\tablenotemark{3}&Yes&Yes&\ref{rule:dds}\advancedrulemark\\
Target and searcher in BJM arc&+BJM rating $-$ ECCM rating&Yes&No&\ref{rule:radar-jamming}\advancedrulemark\\
Searcher in AJM-C/D arc&+AJM rating $-$ ECCM rating&Yes&No&\ref{rule:radar-jamming}\advancedrulemark\\
\midrule
\multicolumn{5}{c}{Radar Operator}\\
\midrule
Veteran&$-1$&Yes&No&\ref{rule:crew-ability}\advancedrulemark\\
Novice&$+1$&Yes&No&\ref{rule:crew-ability}\advancedrulemark\\
Green&$+2$&Yes&No&\ref{rule:crew-ability}\advancedrulemark\\
Tactics master&$-1$&Yes&No&\ref{rule:crew-characteristics}\advancedrulemark\\
\bottomrule
\end{tabularx}
\begin{tablenote}{0.9\linewidth}
\advancedruletext
\tablenotetext{1}{The modifier applies to search attempts if Yes appears in this column.}
\tablenotetext{2}{The modifier applies to tracking attempts if Yes appears in this column.}
\tablenotetext{3}{Table~\ref{table:ew-decoy-ppl-effectiveness} gives the modifier.}
\tablenotetext{4}{Round up one half of the ECCM rating before subtracting it from the PPL modifier.}
\end{tablenote}
\end{twocolumntable}
\end{twocolumntablefloat}

\Ax{
\begin{onecolumntablefloat}
\begin{onecolumntable}
\tablecaption{table:radar-breaking-lock-on}{Breaking Lock-Ons with ECM}
\begin{tabularx}{\linewidth}{lX}
\toprule
ECM&Roll\\
\midrule
Chaff&Effectiveness or less\\
MJM&Effectiveness or less\\
DJM&DJM rating $-$ ECCM rating or less.\\
\bottomrule
\end{tabularx}
\end{onecolumntable}
\end{onecolumntablefloat}
}


\begin{twocolumntablefloat}
\begin{twocolumntable}
\tablecaption{table:radar-vertical-limits}{Radar Vertical Limits}
\begin{tabularx}{0.8\linewidth}{c*{6}{L@{ to }L}}
\toprule
Arc&
\multicolumn{2}{c}{VD}&
\multicolumn{2}{c}{SD or UD}&
\multicolumn{2}{c}{LVL}&
\multicolumn{2}{c}{SC}&
\multicolumn{2}{c}{ZC}&
\multicolumn{2}{c}{VC}\\
\midrule
Limited&$-2$&$-9$&$-0.5$&$-3$&$+0.5$&$-0.5$&$+2$&$+0$&$+4$&$+0.5$&$+9$&$+2$\\
180+&$-1$&$-\infty$&$-0$&$-5$&$+1$&$-1$&$+3$&$-0.5$&$+5$&$+0$&$+\infty$&$+1$\\
150+&$-0.5$&$-\infty$&$-0$&$-8$&$+2$&$-2$&$+4$&$-1$&$+8$&$+0$&$+\infty$&$+0.5$\\
120+&$-0$&$-\infty$&$+0.5$&$-\infty$&$+4$&$-4$&$+6$&$-2$&$+\infty$&$-0.5$&$+\infty$&$+0$\\
\bottomrule
\end{tabularx}
\begin{tablenote}{0.8\linewidth}
Note: The radar vertical limits are given as altitude levels per hex or horizontal distance. 
\end{tablenote}

\end{twocolumntable}
\end{twocolumntablefloat}



}



}

\begin{enumerate}
    \item Enter \changedin{1D}{1D-table}{the Radar Detection Table}{Table~\ref{table:radar-contact}} on the line corresponding to the radar's search strength. Each column on the table lists a range in hexes corresponding to a detection number at the top which is the die roll or less required to contact a target.
    \item Move right across the listed columns until the column whose range number first equals or exceeds the range the target is at is reached.
    \item Roll the die, if the number is less than or equal to the detection number at the top of the column, the target is detected.
\end{enumerate}

\paragraph{Die Roll Modifiers.} Detection die roll modifiers exist for electronic jamming, the presence of chaff and mini-jammer programs, stealth technology, crew quality, and/or aircraft size. These are summarized \changedin{1D}{1D-table}{on the Search Modifiers Table}{in Tables~\ref{table:radar-search-modifiers} and \ref{table:ew-radar-modifiers}}.

\paragraph{Duration of Detection.} Once an aircraft is detected, it remains detected as long as it remains in the search aircraft's radar arc and the searcher remains a “free” aircraft and does not violate the limitations given below in 16.3

\paragraph{Search Limits.} There is no limit to the number of detected targets an aircraft can maintain, however, no more than four die roll attempts for radar detection are allowed per searching aircraft each game turn, and no more than one die roll per eligible target is allowed.

Once an aircraft switches to tracking mode it loses contact with all detected aircraft except the one being tracked through a lock-on (Exception: see Track-While-Scan radars). It may not search again until the lock-on is broken or dropped. \addedin{1B}{1B-apj-35-qa}{If a lock-on attempt fails, the radar stays in search mode and maintains contact with all detected aircraft.}

\paragraph{Search Example:} An aircraft with a search strength of 12 and a maximum detection range of 48 is looking for two aircraft, one 27 hexes away (9 miles) and another 40 hexes (13.3 miles) away. Entering at the strength line we move right stopping at the second column. This column's range equals 30 which is higher than 27. Looking at the top of the column, we see that we need to roll 9 or less to make contact. Continuing further right two columns, we find the range listing of 42 which is higher than 40 and the die roll required is 7 or less.

\paragraph{Electronic Warfare and ECCM.} The effects of jamming are fully described in rule 19, but generally, the presence of jamming will cause modifiers to the detection die roll. Aircraft radars may have an ECCM (electronic counter-countermeasures) rating on the ADC which is used to counter jamming modifiers.

\section{Radar Tracking and Lock-Ons}
\label{rule:radar-tracking-and-lock-ons}

For weapons guidance, an aircraft must refine and concentrate its radar beam on a target for accurate position readings. This is accomplished by switching to a tracking mode and achieving a lock-on.

\paragraph{Radar Tracking Data.} The Radar Data section of the ADC shows two numbers on the track line. The first number is the maximum tracking range; the second number is the radar tracking strength rating. If a dash exists there, the aircraft is not capable of tracking targets. The tracking strength is used to determine the maximum range a locked-on target can be illuminated at for radar missile guidance.

\paragraph{Lock-On Number.} The Radar Data section of the ADC shows a number on the Lock-on line. This is the base chance of a successful lock-on against a detected radar target. Note: this number is also used for gun attack radar ranging.

\paragraph{Radar Lock-On Procedure.} An aircraft may make one lock-on attempt against one detected target per game turn (exception, see Multi-Target Track Technology). Once a target is locked-onto, it remains locked-onto from game turn to game turn unless the lock-on is broken or voluntarily dropped. An aircraft may only have one lock-on at a time unless it has Multi-Target Track Technology.

\paragraph{Procedure.} Roll the die. If the result after applying any modifiers is less than or equal to the lock-on number listed in the radar section of the ADC, the target is locked-onto. \changedin{1B}{1B-apj-25-qa}{

\paragraph{Die Roll Modifiers.} The same modifiers that apply to search rolls, apply to lock-on rolls.}{The lock-on modifiers are given in \changedin{1C}{1C-tables}{the play aids}{in Table~\ref{table:radar-lock-on-modifiers}}.}

\paragraph{Breaking Radar Lock-Ons.} A lock-on will be broken if the tracking aircraft:

\begin{itemize}

    \item stalls, departs, or declares itself engaged.

    \item performs any rolling maneuver (except Vertical Roll) or a Vertical Reverse.

    \item turns at the ET rate.

    \item receives an H or C hit, or is destroyed.

    \item allows the target to leave its radar arc.

    \item voluntarily breaks its lock-on.

\end{itemize}

\addedin{1C}{1C-tables}{These rules are summarized in Table~\ref{table:radar-breaking-lock-on} and \ref{table:ew-radar-break-lock-rolls}.}

\section{Radar Use Limitations}

\paragraph{Radar Limitations.} An aircraft is limited to four detection attempts against eligible radar targets per game turn. An aircraft is limited to one lock-on attempt per game turn unless it has multi-target track technology. An existing radar lock-on must be broken before a new lock-on is attempted unless multi-target track technology exists.

\addedin{1C}{1C-tables}{The limitations on radar searches are given here and summarized in Table~\ref{table:radar-search-limitations}. }A pilot only crewed aircraft may not perform normal searches if:

\begin{itemize}

    \item it turned at greater than HT rate\deletedin{2A}{2A-snap}{ or Snap turned}.

    \item it performed \changedin{1B}{1B-apj-39-qa}{any rolling maneuvers}{more than one vertical roll, any other rolling maneuver}, VIFF Maneuvers, or a Vertical Reverse.

    \item it\deletedin{1B}{1B-apj-35-qa/1B-apj-39-qa}{ vertical climbed, it vertical dived, or} used an unloaded dive\addedin{1B}{1B-apj-36-errata}{ and has not recovered by the end of its move}.

    \item it made an air to air gun \addedin{2B}{2B-radar-requirements}{or rocket} attack or an air to ground attack.

    \item it stalled, departed, or engaged missiles.

    \itemaddedin{1B}{1B-apj-39-qa}{it received H or C damage this game turn.}

    \item it performed any damage control.

    \itemaddedin{1B}{1B-apj-39-qa}{the pilot is GLOC'd.}
    
\end{itemize}

A multi-crewed aircraft may not perform normal search if:

\begin{itemize}

    \item it turned at greater than BT rate\deletedin{2A}{2A-snap}{ or snap turned}.

    \item it performed \changedin{1B}{1B-apj-39-qa}{any rolling maneuvers}{more than one vertical roll, any other rolling maneuver}, VIFF Maneuvers, or a Vertical Reverse.

    \item it\deletedin{1B}{1B-apj-35-qa/1B-apj-39-qa}{ vertical climbed, it vertical dived, or} used an unloaded dive.

    \itemaddedin{1B}{1B-apj-23-errata/1B-apj-39-qa}{it made an air to air gun attack or an air to ground attack.}

    \item it stalled, departed, or engaged missiles.

    \itemaddedin{1B}{1B-apj-39-qa}{it received H or C damage this game turn.}

    \item it performed any damage control.

    \itemaddedin{1B}{1B-apj-39-qa}{the radar operator is GLOC'd.}

\end{itemize}

\addedin{1B}{1B-apj-39-qa}{In a multi-crewed aircraft, the radar operator is the pilot for boresight and auto-track modes and the radar officer (or equivalent) for other modes.}

\paragraph{Look Down Limitations.} \label{rule:radar-ground-clutter} Due to ground clutter, an aircraft may not search for or track targets within four altitude levels of the ground unless it (the searching aircraft) is at lower level than the target or has full Look-Down technology.

An aircraft may not search for targets whose altitude level is within 5 to 10 levels of the ground if it (the searching aircraft) is higher than the targets unless the difference in altitude between the target and the ground is greater than the difference in altitude between the searcher and the target, and the horizontal range is less than the difference in altitude between the target and ground.

For example, if the ground is at level 0, and a target aircraft is at level 6, a higher searcher would have to be no more than 5 levels above the target and within six hexes.

\deletedin{2B}{2B-limited-look-down}{Note: Aircraft with Look-Down Technology ignore this limit. Aircraft with Limited Look-Down Technology ignore the horizontal range aspect of this limit.}

\paragraph{Nose Attitude Limits.} An aircraft which climbs cannot search for or track lower targets, and an aircraft which dives cannot search for or track higher targets. An aircraft which flies level cannot search for and track targets which are more than one altitude level above or below for each two hexes of range away they are. Note: Advanced rule 16.5 introduces more specific limits.

\begin{advancedrules}

\section{Radar System Technologies}

\paragraph{Multi-Target Track Technology.}\label{rule:multi-target-track-technology} Some radars may track more than one aircraft at a time. This is noted in the Technology section of the ADC as “multi-Tgt Track (Number).” The number is the number of targets the aircraft may attempt to lock-onto and/or maintain locked each game-turn.

\paragraph{Track-While-Scan Technology.} \label{rule:track-while-scan-technology} An aircraft with track-while-scan technology does not lose contact with detected targets when it switches to tracking mode. It is also allowed to continue searching for new targets while maintaining or acquiring lock-ons.

\paragraph{Limited Track-While-Scan.} An aircraft with parenthesis around its Track-While-Scan Technology has a Limited Track-While-Scan capability. It can maintain previous contacts while having a lock-on but may not search for new ones.

\paragraph{Look-Down Technology.} \label{rule:look-down-missiles}
An aircraft with Look Down Technology can ignore all Look Down Limitations and may guide look down capable missiles against targets in ground clutter conditions.

\paragraph{Limited Look Down.} An aircraft with parenthesis around its Look-Down Technology indication has a Limited Look-Down capability.  It may search for and lock-onto lower altitude targets within 2 to 10 levels of the ground if the difference in altitude between the target and the ground is more than the difference in altitude between the searcher and the target. They may also guide look down capable missiles against targets as above.

% ISSUE: This differs from the statement about limited look-down technology in the radar search requirements section.

\section{Radar Vertical Limits}
\label{rule:radar-vertical-limits}

\paragraph{Aircraft Nose Attitude.} An aircraft radar arc is limited in its vertical arc as well as its horizontal arc. \changedin{1C}{1C-tables}{The Radar Vertical Limits Table}{Table~\ref{table:radar-vertical-limits}} defines the vertical limits to radar arcs in terms of allowed altitude differences between searcher and target based on searcher's flight profile for the game turn. \addedin{1B}{1B-apj-23-errata}{Aircraft in unloaded dives use the steep dive vertical limits. }The listed UP limit is a factor used to determine the number of levels above the searcher the target can be for each hex away it is. The listed DOWN limit is a factor used to determine the number of levels below the searcher the target can be for each hex away it is. \addedin{1B}{1B-apj-36-errata}{Any fractions that result after using \changedin{1C}{1C-tables}{the Radar Vertical Limits Table}{Table~\ref{table:radar-vertical-limits}} are dropped (i.e., the result does not favor the searcher).}

For example, an aircraft in a zoom climb with a 180+ arc radar can search for or track higher targets up to 5 levels higher for each hex away they are (“\multiplybefore{5}”). It cannot search for lower targets as its Down limit is “\multiplybefore{0}” meaning the allowed down difference per hex of range the target is away is zero. Thus, a target 10 hexes away could be as much as 50 levels above the searcher but not lower. The target could be at the same level.

\section{Special Radar Modes}
\label{rule:special-radar-modes}

\paragraph{Boresight Mode.}\label{rule:boresight-mode} Boresight mode slaves the radar to the gunsight. All aircraft can use boresight mode. \addedin{1B}{1B-apj-35-qa}{Aircraft whose radar has a lock-on capability but no search capability can only use boresight mode to achieve a lock-on.} \addedin{1C}{1C-tables}{Table~\ref{table:radar-boresight-mode} summarizes boresight mode.}

\paragraph{Procedure.} \changedin{2B}{2B-declaring-special-radar-modes}{Announce Boresight mode when the aircraft begins its flight.}{Announce boresight mode in the aircraft decisions phase.} All previous contacts and lock-ons are lost. Normal radar search is not allowed. The aircraft's effective radar arc becomes a Limited arc (regardless of the aircraft's normal radar arc). In the Air Radar Search and Lock-On Phase, the closest visually sighted aircraft within a range equal to the radar's TRACKING strength in hexes in the limited arc is automatically detected. If two or more aircraft are equally close, randomly determine which is detected. Jamming has no effect on this detection. Boresight mode detection occurs even if the aircraft violated the normal radar use maneuver restrictions.

One Boresight Mode lock-on attempt is allowed against the detected target even if the aircraft violated the normal radar use maneuver restrictions. An aircraft without Look-Down Technology may use boresight mode to detect and lock-on to low level visually sighted targets. In this case, the automatic contact and lock-on attempt is allowed, and the lock-on can be maintained as long as the difference between target level and ground level is more than the searcher and target altitude level difference. The lock-on die roll is subject to a Boresight Look Down Modifier of \minus{2}. \addedin{1B}{1B-apj-22-qa/1B-apj-39-qa}{In multi-crew aircraft, the crew quality modifiers for the pilot are used for boresight mode.}

\paragraph{Auto-Track Radar Mode.}\label{rule:auto-track-mode} Auto-Track Radar mode allows an aircraft to automatically detect and then lock-on a target. Only aircraft with Auto-Track Technology may use this mode. \addedin{1C}{1C-tables}{Table~\ref{table:radar-auto-track-mode} summarizes auto-track mode.}

\paragraph{Procedure.} \changedin{2B}{2B-declaring-special-radar-modes}{Announce Auto-Track when the aircraft begins its flight.}{Announce Auto-Track in the aircraft decisions phase.} All previous contacts and lock-ons are lost. Normal radar search is not allowed. The aircraft's effective radar arc becomes the 180{\deg} arc (regardless of the aircraft's normal radar arc). In the Air Radar Search and Lock-On Phase, the closest aircraft at a range of no more than the radar's SEARCH strength in hexes in the 180{\deg} arc is automatically detected.

One Auto-Track Mode lock-on attempt is allowed against the detected target even if the aircraft violated the normal radar use maneuver restrictions. Auto-Track will ignore all friendly aircraft with IFF on. A visually sighted enemy aircraft may be selected for detection and lock-on \addedin{2B}{2B-auto-track-selection}{on a \minusafter{7}} even if it was not the closest as long as it meets the range and 180{\deg} arc requirements. \addedin{1B}{1B-apj-22-qa/1B-apj-39-qa}{In multi-crew aircraft, the crew quality modifiers for the pilot are used for auto-track mode.}

\addedin{1B}{1B-apj-37-qa}{BJMs and AJMs have no impact on auto-track radar detections, but do impact lock-on attempts.}

\section{Formations and Radar Detections}

\paragraph{Radar Searches.} Enemy radar searches are done against the Close formation as a single entity. If the formation contains 3 or 4 aircraft, apply a modifier of \minus{1}. If radar contacted, all aircraft in the close formation are contacted.

\paragraph{Radar Lock-Ons.} For air radar and SAM TTR lock-ons (see Chapter 25 for SAM rules), randomly determine which aircraft in the close formation is locked up. Exception: an aircraft radar of 120+ or 150+ arc ability with a search strength of 40 or more, or a SAM TTR of VF or MW frequency is powerful enough to distinguish individual aircraft in the formation and may choose which aircraft is locked up normally.

\end{advancedrules}
}

%%%%%%%%%%%%%%%%%%%%%%%%%%%%%%%%%%%%%%%%%%%%%%%%%%%%%%%%%%%%%%%%%%%%%%%%%%%%%%%%

\AX{

This rule describes how aircraft use their air-to-air radars to search for and track other aircraft.

Not all aircraft have these capabilities. Some aircraft have no radar whatsoever. Others have navigation and ground attack radars with no air-to-air capability. Others still have radars that only serve to provide ranging for gunsights (see rule \ref{rule:radar-ranging}). 

Table~\ref{table:radar-use-summary} summarizes many aspects of air-to-air radar.

\section{Radar Data} 

The radar and technology sections of the ADC give the characteristics of an aircraft's air-to-air radar. Here we describe in detail the information in the radar section; radar technology is described in rule \ref{rule:radar-technology}.

\paragraph{Radar Arc.} The arcs line shows the arc in which the radar operates when used in normal mode. (The arc may differ in boresight mode and auto-track mode, described in advanced rules \ref{rule:boresight-mode} and \ref{rule:auto-track-mode}.)

\DY[3A-limited-arcs]{

\begin{twocolumnfigure}[tbp]

% These are arcs derived directly from the TSOH play aids.

\newcommand{\drawlimitedarcA}[1][]{   
    \draw [yscale=\hexxfactor,#1]
        (-1.600,20.000) --
        (-1.600,10.000) --
        (-1.100, 9.000) --
        (-1.100, 5.000) --
        (-0.600, 4.000) --
        (-0.600, 2.000) --
        (-0.000, 0.333) -- 
        (+0.600, 2.000) --
        (+0.600, 4.000) --
        (+1.100, 5.000) --
        (+1.100, 9.000) --
        (+1.600,10.000) --
        (+1.600,20.000);   
    \draw[yscale=\hexxfactor, <->, transform shape]
        (-1.6,10.5) -- 
        (0,10.5) node [anchor=south] {\minitable{c}{maximum\\width}} --
        (+1.6,10.5);
}
\newcommand{\drawlimitedarcB}[1][]{  
    \draw [yscale=\hexxfactor,#1]
        (-1.6,20.000) --
        (-1.6,11.000) --
        (-1.1,10.000) --
        (-1.1, 6.000) --
        (-0.6, 5.000) --
        (-0.6, 2.000) --
        (+0.0, 0.333) -- 
        (+0.6, 2.000) --
        (+0.6, 5.000) --
        (+1.1, 6.000) --
        (+1.1,10.000) --
        (+1.6,11.000) --
        (+1.6,20.000);  
    \draw[yscale=\hexxfactor, <->, transform shape]
        (-1.6,11.5) -- 
        (0,11.5) node [anchor=south] {\minitable{c}{maximum\\width}} --
        (+1.6,11.5);
}
\newcommand{\drawlimitedarcC}[1][]{  
    \draw [xscale=\hexxfactor,#1]
        (-1.6,20.000) --
        (-1.6, 9.250) --
        (-1.1, 8.500) --
        (-1.1, 5.000) --
        (-0.6, 4.250) --
        (-0.6, 1.250) --
        (-0.0, 0.333) -- 
        (+0.6, 1.250) --
        (+0.6, 4.250) --
        (+1.1, 5.000) --
        (+1.1, 8.500) --
        (+1.6, 9.250) --
        (+1.6,20.000);   
    \draw[xscale=\hexxfactor, <->, transform shape]
        (-1.6,9.75) -- 
        (0,9.75) node [anchor=south] {\minitable{c}{maximum\\width}} --
        (+1.6,9.75);
}

% This figure shows them in the same orientation as in the TSOH play aids.
%\begin{tikzfigure}{0.5\linewidth}
%\begin{scope}[rotate=0]
%    \drawdottedhexgrid{15.0}{15.5}
%    
%    \begin{athex}{1.00}{13.00}
%        \begin{scope}[rotate=-90,thick]
%            \drawlimitedarcA
%        \end{scope}
%        \drawaircraftcounter{0.00}{0.00}{90}{F-4}{}{}
%    \end{athex}
%
%    \begin{athex}{1.00}{8.50}
%        \begin{scope}[rotate=-90,thick,]
%            \drawlimitedarcB
%        \end{scope}
%        \drawaircraftcounter{0.00}{0.00}{90}{F-4}{}{}
%    \end{athex}
%
%    \begin{athex}{1.00}{7.50}
%        \begin{scope}[rotate=-120,thick]
%            \drawlimitedarcC
%        \end{scope}
%        \drawaircraftcounter{0.00}{0.00}{60}{F-4}{}{}
%    \end{athex}
%    
%\end{scope}
%\end{tikzfigure}

\begin{tikzfigure}{0.6\linewidth}
\begin{scope}[rotate=0]

    \drawhexgrid{0}{0}{16}{12}
    \drawpositiongrid{0}{0}{20}{12}
    
    \begin{athex}{7.50}{0.25}
        \begin{scope}[rotate=-30,thick]
            \drawlimitedarcA
        \end{scope}
        \drawaircraftcounter{0.00}{0.00}{60}{F-4}{}{}
    \end{athex}

    \begin{athex}{4.00}{2.00}
        \begin{scope}[rotate=-30,thick]
            \drawlimitedarcB
        \end{scope}
        \drawaircraftcounter{0.00}{0.00}{60}{F-4}{}{}
    \end{athex}

    \begin{athex}{2.00}{0.00}
        \begin{scope}[rotate=0,thick]
            \drawlimitedarcC
        \end{scope}
        \drawaircraftcounter{0.00}{0.00}{90}{F-4}{}{}
    \end{athex}
    
\end{scope}
\end{tikzfigure}

\figurecaption{figure:limited-arcs}{Limited Arcs}

\end{twocolumnfigure}



The arc may be the word “limited” or an arc range (see rule \ref{rule:ranges-of-angle-off-arcs}). Figure~\ref{figure:limited-arcs} shows the limited arc. In this figure, only the map locations within the limited arc line are within the limited arc. 

The arc ranges \arcrange{180}{+}, \arcrange{150}{+}, \arcrange{120}{+} include the specified arc and all more forward arcs (see rule \ref{rule:ranges-of-angle-off-arcs}). For example, \arcrange{150}{+} includes the \arc{150} and \arc{180} arcs.

Targets on the borderline of a radar arc are considered to be within the arc.

}

\paragraph{Radar Search Capability.} The search line shows the radar's search capability. If this line has two numbers separated by a dash, the radar is capable of normal air-to-air search. The first number is the maximum search range in hexes. The second number is the search strength rating. 

If the line is empty, contains only a dash, or contains “Gr. Nav.” followed by a number in parentheses, the radar is not capable of normal air-to-air search.

\paragraph{Radar Tracking Capability.} The tracking line shows the radar's tracking capability. If this line has two numbers separated by a dash, the radar is capable of air-to-air tracking. The first number is the maximum tracking range in hexes. The second number is the radar tracking strength rating. 

If the line is empty, contains only a dash, or contains “Gr. Att.” followed by a number in parentheses, the radar is not capable of air-to-air tracking.

Some very early radars are capable have a tracking capability but apparently no search capability; they must use boresight mode (see advanced rule \ref{rule:boresight-mode}) to obtain a detection and track.

\paragraph{Radar Lock-On Number.} The lock-on line of the radar data section of the ADC shows the lock-on number. The unmodified die roll for a successful lock-on against a detected radar target is this number of less. The lock-on number is also used to determine the success of radar-ranging for gun and rocket attacks (see rule \ref{rule:radar-ranging}). 

If the lock-on number is marked with an asterisk, the radar is not capable of air-to-air tracking.

\paragraph{Radar ECCM Rating.} The ECCM rating quantifies the resistance of the radar to jamming and other interference.

\section{Radar Use}
\label{rule:radar-use}

\paragraph{Radar Modes.} Radars may be operated in normal, boresight, or auto-track mode. The latter two modes are collectively known as special modes. Searching and tracking in normal mode are described in rules~\ref{rule:normal-mode}. The special modes are described in advanced rules \ref{rule:boresight-mode} and \ref{rule:auto-track-mode}. 

% Switching off a radar prevents it from being detected by RWR-C and RWR-D (see advanced rule \ref{rule:rwr})

The mode in which a radar is being operated is declared in the aircraft decisions phase. The only restriction is that auto-track mode may only be selected by aircraft with auto-track technology (see advanced rule \ref{rule:radar-technology}).

\paragraph{Radar Operator.} In a single-crewed aircraft, the radar operator is always the pilot. In a multi-crewed aircraft, the radar operator is the radar officer (RIO, WSO, observer, navigator, or equivalent) for normal mode and the pilot for special modes.

\paragraph{Radar Outcomes.}

Radars may be used to search for, track, and illuminate other aircraft. Searches and tracking attempts occur in the air-to-air radar phase, with searches occurring before tracking attempts. Illumination is declared in the aircraft decisions phase.

A successful radar search gives a detection or “contact” and provides an indication of position of a target. It gives benefits for visual sighting (rule \ref{rule:sighting-aircraft-and-missiles}
) and the order of flight (rule \ref{rule:order-of-flight}). It can be used to cue a VAS to sight the target (see advanced rule \ref{rule:vas}). It is also a requirement for tracking.

A successful tracking attempt gives a “track” or “lock-on,” which allows a radar to follow a target, either by steering the antenna mechanically in early radars or by steering the beam electronically in modern radars. This in turn allows the tracking aircraft more freedom to maneuver and can increase the radar power directed at the target. In addition, provided other requirements are met, radar ranging for gun attacks and cuing of VAS for sighting automatically succeeds against targets that are being tracked (see advanced rules~\ref{rule:radar-ranging} and \ref{rule:vas}). An aircraft can fire uncaged IRMs at targets it is tracking (see advanced rule \ref{rule:irm-radar}) and use target ID technology to identify them (see advanced rule \ref{rule:target-id-technology}). Finally, tracking is a requirement for illuminating a target. 

Illumination is required for guiding radar-guided missiles (except AHMs with mid-course guidance or in their active phase) to their target (see rule \ref{rule:target-illumination}).

Unlike visual sighting, radar outcomes apply individually to aircraft. For example, if one aircraft has a detection or a track of an enemy aircraft, other friendly aircraft do not automatically share that detection or track.

\paragraph{Common Radar Use Requirements.}
\label{rule:common-radar-use-requirements}

To use its radar for an air-to-air search or tracking attempt, in any radar mode, an aircraft must satisfy these common requirements:

    \begin{itemize}
        \item It must not have carried out an air-to-air gun, air-to-air rocket, or air-to-ground attack.
        \item It must not have been in stalled or departed flight.
        \item It must not have engaged a missile.
        \item It must not have received H or C damage in the current game turn.
        \item It must not have have performed damage control.
        \item Its radar operator must not be suffering from GLOC.
        \item Its radar operator must not be disoriented.
    \end{itemize}
    
Each radar mode may add additional requirements, particularly on maneuvering.

\paragraph{Radar Limits.}
\label{rule:radar-arc-requirements}
The target of an air-to-air search or tracking attempt, in any radar mode, must be within the horizontal and vertical limits of the radar:

    \begin{itemize}

    \item It must be within the horizontal radar arc of the aircraft using radar.

    \item It must satisfy these vertical limits:
    \begin{itemize}
        \item If the aircraft using radar used climbing flight during the current game turn, the target must not be lower.
        \item If the aircraft aircraft using radar used diving flight during the current game turn, the target must not be higher.
        \item If aircraft aircraft using radar used level flight during the current game turn, the difference in altitude between it and the target must not be more than one level for every two full hexes of horizontal range.
    \end{itemize}

    If advanced rule \ref{rule:radar-vertical-limits} on vertical radar limits is being used, it supersedes these simple rules.
    
\end{itemize}

The horizontal and vertical limits of a radar may depend on the mode in which it is being used. Each radar mode may add additional requirements, particularly on range. 

\paragraph{Radar Range.}
The range between the radar and the target is calculated in the normal manner, according to rule \ref{rule:range}, with two altitude levels being equivalent to one hex.

\section{Radar Ground Clutter}
\label{rule:radar-ground-clutter}

Early radars were unable to effectively search and track targets at lower altitudes due to the strong radar return from the ground. This is known as “radar ground clutter” and was deliberately used by aircraft to avoid radar detection. Later radars gained technology that mitigated this problem.

Whether a target is in radar ground clutter depends on the altitude of the target, the relative position the aircraft using radar, the technology of the radar, and in some cases the use of the radar. The relevant technologies are limited look-down technology and full look-down technology (see advanced rule \ref{rule:radar-technology}). Table~\ref{table:radar-look-down} summarizes these rules.

When considering radar ground clutter, all altitudes are with respect to the ground at the location of the target aircraft.


\paragraph{No Look-Down Technology.}
For a radar with no form of look-down technology attempting to \emph{search, maintain a detection, or convert a detection into a track}, a target that is zero to ten levels above the ground is in ground clutter unless one of the following conditions is met:
 

    \begin{itemize}

        \item The target is one to four altitude levels above the ground and radar-using aircraft is lower than the target.
        
        \item The target is five to ten altitude levels above the ground and the radar-using aircraft is lower than or at the same altitude as the target.
        
        \item The target is five to ten altitude levels above the ground, the radar-using aircraft is higher that the target, the difference in altitude between the radar-using aircraft and the target is less than the altitude of the target above the ground, and the horizontal range to the target is less than than the altitude of the target.
        
    For example, consider a target aircraft is at altitude level 6 at a location in which the gorund level is 0. The difference in altitude between the target and the ground is 6 levels. A higher searcher would have to be at most 5 levels above the target (i.e., at altitude levels 7 to 11) and at a horizontal range of at most 5 hexes.

    \end{itemize}

For a radar with no form of look-down technology attempting to \emph{maintain a track}, a target that is zero to ten levels above the ground is in ground clutter unless one of the following conditions is met:
 

    \begin{itemize}

        \item The target is one to four altitude levels above the ground and radar-using aircraft is lower than the target.
        
        \item The target is five to ten altitude levels above the ground.
        
    \end{itemize}

\paragraph{Limited Look-Down Technology.} For a radar with \emph{limited} look-down technology (see rule \ref{rule:limited-look-down-technology}) attempting any radar use, a target that is zero to ten levels above the ground is in ground clutter unless one of the following conditions is met:

    \begin{itemize}

        \item The target one altitude level above the ground and radar-using aircraft is lower than the target.

        \item The target is two to ten altitude levels above the ground and the radar-using aircraft is lower than or at the same altitude as the target.

        \item The target is two to ten altitude levels above the ground, the radar-using aircraft is higher that the target, and the difference in altitude between the radar-using aircraft and the target is less than the altitude of the target above the ground.

    \end{itemize}
    
These are the same condition that apply for aircraft with no look-down technology, except that the requirement on horizontal range is omitted.

\paragraph{Look-Down Technology.} For a radar with \emph{full} look-down technology (see rule \ref{rule:look-down-technology}), a target is never in ground clutter.

\section{Normal Mode}
\label{rule:normal-mode}

In normal mode, an aircraft's radar operator actively interacts with the radar system to search for and track targets. This allows the radar to operate with optimal sensitivity and flexibility, but imposes significant limitations on the maneuvers that the aircraft can perform.

\subsection{Normal Mode Searching}
\label{rule:normal-mode-searching}

Obtaining a detection is the first step in the use of radar to combat enemy aircraft.

\paragraph{Normal Mode Search Requirements.}
\label{rule:radar-search-requirements}

An aircraft with its radar in normal mode can perform a search in the air-to-air radar phase provided:

\begin{itemize}

\item During the entire preceding flight phase, it did not:
\begin{itemize}
        \item Turn at the HT rate or tighter (if the pilot is the radar operator) or the BT rate or tighter (is another crew member is the radar operator).
        \item Prepare for or execute any special maneuvers except slides and at most one vertical roll.
        \item Use an unloaded FP.
        \item Declare a defensive preemption (see rule \ref{rule:defensive-preemptions}).
        \item Use terrain-following flight if it is a single-crewed aircraft.
        \item Violate any of the common radar use requirements (see rule \ref{rule:common-radar-use-requirements}).
\end{itemize}

\item At the end of the preceding flight phase, the target of the search must:
\begin{itemize}
    \item Have a range that is not greater than the search range given in the radar section of the ADC. 
    \item Be within the radar horizontal and vertical limits (see rule \ref{rule:radar-arc-requirements}).
    \item Not be in radar ground clutter (see  \ref{rule:radar-ground-clutter}).
\end{itemize}

\end{itemize}

\paragraph{Normal Mode Search Procedure.}
An aircraft may search for up to four separate undetected targets each game turn, provided it and each target comply with the preceding requirements. 

For each target, in Table~\ref{table:radar-searches} find the row corresponding to the radar search strength, move right to the column that is equal to or more than the range, and then move up to determine the detection die roll. Then roll the die and apply appropriate modifiers. The target is detected if the modified die roll is equal to or less than the detection die roll. Otherwise, it is not. To determine the detection die roll, 

For example, an aircraft with a search strength of 12 and a maximum detection range of 48 is searching for two aircraft, one at a range of 27 hexes (9 miles) and the other at a range of 40 hexes (13.3 miles). In Table~\ref{table:radar-searches}, we see that the detection die roll is \minusafter{9} for the first and \minusafter{7} for the second.

\addedin{1C}{1C-tables}{
    \Dx{
\begin{twocolumntablefloat}
\begin{twocolumntable}
\tablecaption{table:radar-contact}{Radar Contact}

\begin{tabularx}{0.7\linewidth}{c*{11}{R}}
\toprule
\multicolumn{1}{c}{\multirow{2}{*}{\minitable{c}{Radar\\Strength}}}&
\multicolumn{11}{c}{Die Roll or Less for Contact}\\
&
\multicolumn{1}{c}{10}&
\multicolumn{1}{c}{9}&
\multicolumn{1}{c}{8}&
\multicolumn{1}{c}{7}&
\multicolumn{1}{c}{6}&
\multicolumn{1}{c}{5}&
\multicolumn{1}{c}{4}&
\multicolumn{1}{c}{3}&
\multicolumn{1}{c}{2}&
\multicolumn{1}{c}{1}&
\multicolumn{1}{c}{$0-$}\\
\midrule
\phantom{0}3& 6& 8& 9&10&11&12&13&---&14&---&\phantom{0}15+\\
\phantom{0}6&12&15&18&20&22&24&26&28&29&30&\phantom{0}31+\\
\phantom{0}8&16&20&24&28&30&32&34&36&28&40&\phantom{0}41+\\
10&20&25&30&35&38&40&42&45&48&50&\phantom{0}51+\\
12&24&30&36&42&45&48&51&54&57&60&\phantom{0}61+\\
15&30&38&45&52&56&60&64&68&72&75&\phantom{0}76+\\
18&36&45&54&63&68&72&76&81&86&90&\phantom{0}91+\\
20&40&50&60&70&75&80&85&90&95&100&101+\\
25&50&63&75&87&94&100&106&113&119&125&126+\\
30&60&75&90&105&113&120&128&135&143&150&151+\\
40&80&100&120&140&150&160&170&180&190&200&201+\\
50&100&125&150&175&188&200&213&225&238&250&251+\\
EWR&150&188&225&263&280&300&319&338&356&375&376+\\
\bottomrule
\end{tabularx}

\medskip

\begin{tablenote}{0.7\linewidth}
Above = Maximum Range in Hexes for Each Column

\begin{enumerate}
    \item No Aircraft may contact targets at a range greater than the maximum listed on its ADC.
    \item Regular aircraft radar may not detect or track tgts.\ within 4 levels of the ground unless search is at lower altitude.
    \item If tgt.\ below searcher \& within 10 levels of ground, diff. in alt.\ between the aircraft must be < tgt.'s alt.\ above ground.
    \item Lookdown radar may ignore cases 2 \& 3.  Boresight radar may ignore case 3 against visually sighted targets.
\end{enumerate}
\end{tablenote}
\end{twocolumntable}
\end{twocolumntablefloat}
}

\Dx{

\begin{onecolumntablefloat}
\begin{onecolumntable}
\tablecaption{table:radar-search-modifiers}{Radar Search Modifiers}
\begin{tabularx}{\linewidth}{X}
\toprule
\begin{enumerate}
    \item AJM \# $-$ Air Radar or EWR ECCM.
    \item BJM \# $-$ Air Radar or EWR ECCM.
    \item \changedin{1B}{1B-apj-23-errata and 1B-apj-24-play-aids}{CHAFF PPL Effectiveness No.}{Chaff PPL Eff.\ No.\ $-$ $1/2$ radar ECCM (round \addedin{1B}{1B-apj-36-errata}{product }up).}
    \item Mini-Jammer PPL Effectiveness No.
    \item Aircraft Size Modifier from ADC.
    \itemdeletedin{1B}{1B-apj-23-errata and 1B-apj-24-play-aids}{$+4$ if aircraft has Stealth Technology.}
    \itemaddedin{1B}{1B-apj-23-errata and 1B-apj-24-play-aids}{$-2$ if target aircraft IFF on.}
    \item Tactics Master or Veteran = $-1$ ($-2$ if both).
    \item Novice = $+1$, Green = $+2$
\end{enumerate}
\\
\bottomrule
\end{tabularx}
\end{onecolumntable}
\end{onecolumntablefloat}

}

\Dx{
\begin{onecolumntablefloat}
\begin{onecolumntable}
\addedin{1B}{1B-apj-23-errata, 1B-apj-24-play-aids, and 1B-apj-25-qa}{
\tablecaption{table:radar-lock-on-modifiers}{Radar Lock-On Modifiers}
\begin{tabularx}{\linewidth}{X}
\toprule
\begin{enumerate}
    \item $-2$ if target IFF on
    \item Chaff PPL Eff.\ No.\ $-$ $1/2$ radar ECCM (round up).
    \item Mini-Jammer PPL Effectiveness No.
\end{enumerate}
\\
\bottomrule
\end{tabularx}
}
\end{onecolumntable}
\end{onecolumntablefloat}
}

\Dx{
\begin{onecolumntablefloat}
\begin{onecolumntable}
\tablecaption{table:ew-radar-modifiers}{Air Radar Search and Lock-On Modifiers}
\begin{tabularx}{\linewidth}{lP}
\toprule
ECM Type & Die Roll Modifiers\\
\midrule
CHAFF&Decoy Effectiveness No.\\
Mini-Jammer&Decoy Effectiveness No.\\
AJM&AJM No. $-$ Air Radar ECCM\\
BJM Noise&BJM No. $-$ Air Radar ECCM\\
\multicolumn{2}{l}{Radar using Boresight for lookdown $+2$}\\
\bottomrule
\end{tabularx}
\end{onecolumntable}
\end{onecolumntablefloat}
}

\Dx{
\begin{onecolumntablefloat}
\begin{onecolumntable}
\tablecaption{table:radar-breaking-lock-on}{Breaking Radar Lock-On}
\begin{tabularx}{\linewidth}{X}
\toprule
\addlinespace
Locks are broken when:

\begin{enumerate}
    \item Aircraft stalls, departs, or becomes engaged.
    \item Aircraft does ET turns, Viffs, or does other than vertical rolls.
    \item Aircraft takes a H or C hit, or radar operator GLOCs.
    \item Target cannot be kept in radar arc while illuminating
    \item Target deploys decoys and rolls effectiveness \# or less.
    \item Target employs EW jammers and makes break lock die roll.
\end{enumerate}
\\
\bottomrule
\end{tabularx}
\end{onecolumntable}
\end{onecolumntablefloat}
}

\Dx{
\begin{onecolumntablefloat}
\begin{onecolumntable}
\tablecaption{table:ew-radar-break-lock-rolls}{Air Radar Break Lock Rolls}
\begin{tabular}{lp{15em}}
\toprule
ECM Type&Break Lock Die Roll Number\\
\midrule
CHAFF&Decoy Effectiveness No.\\
MINI-JAMMER&Decoy Effectiveness No.\\
DJM&DJM $-$ Air Radar ECCM\\
\bottomrule
\end{tabular}
\end{onecolumntable}
\end{onecolumntablefloat}
}

\Dx{
\begin{onecolumntablefloat}
\begin{onecolumntable}
\tablecaption{table:radar-search-limitations}{Radar Search Limitations}
\begin{tabularx}{\linewidth}{X}
\toprule
\begin{enumerate}
    \item Pilot Only aircraft may not search if they:
    \begin{enumerate}
        \item \changedin{2A}{2A-snap}{Snap-turned or turned}{Turned} at HT or greater rate.
        \item Fired Guns or made an air to ground attack.
    \end{enumerate}
    \item Multi-crew aircraft may not search if they:
    \begin{enumerate}
        \item \changedin{2A}{2A-snap}{Snap-turned or turned}{Turned} at BT or greater rate.
    \end{enumerate}    
    \item Neither type aircraft may search if they:
    \begin{enumerate}
        \item Are stalled, departed, or engaged.
        \item Performed more than one vertical roll in the turn.
        \item Performed any other roll types or Viff maneuvers.
        \item Did an Unloaded Dive or Damage Control.
        \item Were Hit and “H” or greater damage ensued.
        \item Had their radar operator go into GLOC.
    \end{enumerate}        
\end{enumerate}
\medskip
Note: Boresight and Auto-Track Modes allow maneuver restrictions to be ignored but not attack, damage, or operator GLOC restrictions; these always apply.
\\
\addlinespace
\bottomrule
\end{tabularx}
\end{onecolumntable}
\end{onecolumntablefloat}
}

\Dx{
\begin{twocolumntablefloat}
\begin{twocolumntable}
\tablecaption{table:radar-vertical-limits}{Radar Vertical Limits}
\begin{tabularx}{0.8\linewidth}{c*{6}{C}}
\toprule
\multicolumn{1}{c}{\multirow{2}{*}{\minitable{c}{Type\\Radar}}}&
\multicolumn{1}{c}{\multirow{2}{*}{\minitable{c}{Vertical\\Dive}}}&
\multicolumn{1}{c}{\multirow{2}{*}{\minitable{c}{Steep\\Dive}}}&
\multicolumn{1}{c}{\multirow{2}{*}{\minitable{c}{Level\\Flight}}}&
\multicolumn{1}{c}{\multirow{2}{*}{\minitable{c}{Sust.\\Climb}}}&
\multicolumn{1}{c}{\multirow{2}{*}{\minitable{c}{Zoom\\Climb}}}&
\multicolumn{1}{c}{\multirow{2}{*}{\minitable{c}{Vertical\\Climb}}}\\
\\
\midrule
Limited&$-2$, $-9$&$-0.5$, $-3$&$+0.5$, $-0.5$&$+2$, $+0$&$+4$, $+0.5$&$+9$, $+2$\\
180+&$-1$, $-X$&$-0$, $-5$&$+1$, $-1$&$+3$, $-0.5$&$+5$, $+0$&$+X$, $+1$\\
150+&$-0.5$, $-X$&$-0$, $-8$&$+2$, $-2$&$+4$, $-1$&$+8$, $+0$&$+X$, $+0.5$\\
120+&$-0$, $-X$&$+0.5$, $-X$&$+4$, $-4$&$+6$, $-2$&$+X$, $-0.5$&$+X$, $+0$\\
\bottomrule
\end{tabularx}
\begin{tablenote}{0.8\linewidth}
Note: The radar vertical limits are given as altitude levels per hex or horizontal distance. 
\end{tablenote}

\end{twocolumntable}
\end{twocolumntablefloat}
}


\Dx{
\begin{onecolumntablefloat}
\begin{onecolumntable}
\tablecaption{table:radar-boresight-mode}{Radar Boresight Mode}
\begin{tabularx}{\linewidth}{X}
\toprule
\begin{enumerate}
    \item Radar Arc = Limited.
    \item Max Range = Search Strength No.
    \item Previous contacts and locks lost when mode declared.
    \item Nearest Visually sighted target in aircraft's Limit arc automatically contacted.
    \item Lock-on roll allowed, no mnvr.\ limitations.
\end{enumerate}
\\
\bottomrule
\end{tabularx}
\end{onecolumntable}
\end{onecolumntablefloat}
}


\Dx{
\begin{onecolumntablefloat}
\begin{onecolumntable}
\tablecaption{table:radar-auto-track-mode}{Radar Auto-Track Mode}
\begin{tabularx}{\linewidth}{X}
\toprule
\begin{enumerate}
    \item Radar Arc = 180+ unless normally it's limited; in which case it remains limited.
    \item Max Range = Search Strength No.
    \item Nearest target in radar arc is automatically detected.
    \item If nearest aircraft was a friendly with IFF on, it may be ignored and next nearest is automatically contacted etc.
    \item A visually sighted aircraft in arc may be selected for auto contact if not the closest by rolling 7 or less.
    \item Lock-on roll allowed, no mnvr.\ limitations.
    \item Previous contacts and locks lost when mode declared.
\end{enumerate}
\\
\bottomrule
\end{tabularx}
\end{onecolumntable}
\end{onecolumntablefloat}
}

%%%%%%%%%%%%%%%%%%%%%%%%%%%%%%%%%%%%%%%%%%%%%%%%%%%%%%%%%%%%%%%%%%%%%%%%%%%%%%%

\Ax{

\begin{twocolumntablefloat}
\begin{twocolumntable}
\tablecaption{table:radar-use-summary}{Radar Use Summary}
\small
\begin{tabularx}{0.9\linewidth}{l*{5}{C}}
\toprule
Mode&\multicolumn{3}{c}{Normal}&Boresight&Auto-Track\\
\cmidrule{2-4}
Use&Searching&Tracking Attempt&Maintaining Track&All&All\\
\midrule
Rule
&\ref{rule:normal-mode}
&\ref{rule:normal-mode}
&\ref{rule:normal-mode}
&\ref{rule:boresight-mode}
&\ref{rule:auto-track-mode}\\
\midrule
Turn Limit (pilot only)
&TT or less
&TT or less
&BT or less
&ET or less
&ET or less\\
Turn Limit (radar officer)
&HT or less
&HT or less
&BT or less\\
Slides?&Yes&Yes&Yes&Yes&Yes\\
VRs?&One&One&One&Yes&Yes\\
Other Maneuvers?&No&No&No&Yes&Yes\\
Unloaded FPs?&No&No&Yes&Yes&Yes\\
Preemption?&No&No&No&Yes&Yes\\
TFF (pilot only)?&No&No&No&No&No\\
TFF (radar officer)?&Yes&Yes&Yes&No&No\\
\midrule
Stalled or Departed?&No&No&No&No&No\\
Engaged?&No&No&No&No&No\\
AtA or AtG Attack?&No&No&No&No&No\\
H or C Damage?&No&No&No&No&No\\
Damage Control?&No&No&No&No&No\\
GLOC or Disoriented?&No&No&No&No&No\\
\midrule
Arc&From ADC&From ADC&From ADC&Limited&\arcrange{180}{+}\\
\midrule
Range&Search&Tracking&Tracking&Tracking&Search\\
&Range&Range& Range&Strength&Strength\\
\midrule
Detection Roll&Table \ref{table:radar-searches}&&&\multicolumn{1}{L}{Automatic for closest sighted target.}&\multicolumn{1}{L}{Automatic for closest target or $7-$ for other sighted target. Friendly aircraft with IFF on are ignored.}\\
\midrule
Tracking Roll&&From ADC&&From ADC&From ADC\\
\bottomrule
\end{tabularx}
\end{twocolumntable}
\end{twocolumntablefloat}

\begin{twocolumntablefloat}[tp]
\begin{twocolumntable}

\tablecaption{table:radar-look-down}{Radar Look-Down Summary}
\small
\begin{tabularx}{0.9\linewidth}{cCCCC}
\toprule
&\multicolumn{2}{c}{No Look-Down}&Limited Look-Down&Full Look-Down\\
\cmidrule{2-3}
\minitable{c}{Target\\Altitude}&
\minitable{c}{Radar\\Altitude}&
\minitable{c}{Radar Altitude\\and Horizontal Range}&
\minitable{c}{Radar\\Altitude}&
\minitable{c}{Radar\\Altitude}\\
\midrule
\phantom{0}0&&&&Any\\
\phantom{0}1&$\le\phantom{0}0$&&$\le\phantom{0}0$&Any\\
\phantom{0}2&$\le\phantom{0}1$&&$\le\phantom{0}3$&Any\\
\phantom{0}3&$\le\phantom{0}2$&&$\le\phantom{0}5$&Any\\
\phantom{0}4&$\le\phantom{0}3$&&$\le\phantom{0}7$&Any\\
\phantom{0}5&$\le\phantom{0}5$&$\le\phantom{0}9$ and range $\le4$&$\le\phantom{0}9$&Any\\
\phantom{0}6&$\le\phantom{0}6$&$\le\phantom{}11$ and range $\le5$&$\le\phantom{}11$&Any\\
\phantom{0}7&$\le\phantom{0}7$&$\le\phantom{}13$ and range $\le6$&$\le\phantom{}13$&Any\\
\phantom{0}8&$\le\phantom{0}8$&$\le\phantom{}15$ and range $\le7$&$\le\phantom{}15$&Any\\
\phantom{0}9&$\le\phantom{0}9$&$\le\phantom{}17$ and range $\le8$&$\le\phantom{}17$&Any\\
\phantom{}10&$\le\phantom{}10$&$\le\phantom{}19$ and range $\le9$&$\le\phantom{}19$&Any\\
\bottomrule
\end{tabularx}
\begin{tablenote}{0.9\linewidth}
The table gives conditions under which a radar can detector or track a target at altitude 10 or less according to its level of look-down technology. For no look-down technology, two conditions are given, and if the radar fulfills either condition, it may search or track. Otherwise, the target is in radar round clutter. All altitudes are above the ground level at the location of the target. See rule \ref{rule:radar-ground-clutter}.
\end{tablenote}


\end{twocolumntable}
\end{twocolumntablefloat}

\begin{twocolumntablefloat}
\begin{twocolumntable}
\tablecaption{table:radar-searches}{Radar Detection Rolls}
\small
\begin{tabularx}{0.9\linewidth}{*{12}{C}}
\toprule
\multicolumn{1}{c}{\multirow[b]{2}{*}{\minitable{c}{Search\\Strength}}}&
\multicolumn{11}{c}{Detection Die Roll}\\
\cmidrule{2-12}
&
\multicolumn{1}{c}{$10-$}&
\multicolumn{1}{c}{$9-$}&
\multicolumn{1}{c}{$8-$}&
\multicolumn{1}{c}{$7-$}&
\multicolumn{1}{c}{$6-$}&
\multicolumn{1}{c}{$5-$}&
\multicolumn{1}{c}{$4-$}&
\multicolumn{1}{c}{$3-$}&
\multicolumn{1}{c}{$2-$}&
\multicolumn{1}{c}{$1-$}&
\multicolumn{1}{c}{$0-$}\\
\midrule
\phantom{0}3& 6& 8& 9&10&11&12&13&---&14&---&\phantom{0}15+\\
\phantom{0}6&12&15&18&20&22&24&26&28&29&30&\phantom{0}31+\\
\phantom{0}8&16&20&24&28&30&32&34&36&28&40&\phantom{0}41+\\
10&20&25&30&35&38&40&42&45&48&50&\phantom{0}51+\\
12&24&30&36&42&45&48&51&54&57&60&\phantom{0}61+\\
15&30&38&45&52&56&60&64&68&72&75&\phantom{0}76+\\
18&36&45&54&63&68&72&76&81&86&90&\phantom{0}91+\\
20&40&50&60&70&75&80&85&90&95&100&101+\\
25&50&63&75&87&94&100&106&113&119&125&126+\\
30&60&75&90&105&113&120&128&135&143&150&151+\\
40&80&100&120&140&150&160&170&180&190&200&201+\\
50&100&125&150&175&188&200&213&225&238&250&251+\\
EWR&150&188&225&263&280&300&319&338&356&375&376+\\
\bottomrule
\end{tabularx}
\begin{tablenote}{0.9\linewidth}
The table gives the detection roll for normal mode searches. In the row corresponding to the radar search strength, move right to the column that is equal to or more than the range, and then move up to determine the detection die roll. See rule \ref{rule:normal-mode}.
\end{tablenote}
\end{twocolumntable}
\end{twocolumntablefloat}

\begin{twocolumntablefloat}
\begin{twocolumntable}
\tablecaption{table:radar-search-and-lock-on-modifiers}{Radar Search and Tracking Modifiers}
\small
\begin{tabularx}{0.9\linewidth}{lLccl}
\toprule
Description&Modifier&Search\tablenotemark{1}&Tracking\tablenotemark{2}&Rule\\
\midrule
\multicolumn{5}{c}{Target}\\
\midrule
Target size modifier&See ADC&Yes&No&\ref{rule:aircraft-data-cards}\\
Boresight mode assuming limited look-down&$+2$&No&Yes&\ref{rule:boresight-mode}\\
Auto-track mode assuming limited look-down&$+2$&No&Yes&\ref{rule:auto-track-mode}\\
\midrule
\multicolumn{5}{c}{ECM}\\
\midrule
Target using IFF&$-2$&Yes&Yes&\ref{rule:iff}\advancedrulemark\\
Target using chaff DDS&+PPL modifier\tablenotemark{3} $-$ half ECCM rating\tablenotemark{4}&Yes&Yes&\ref{rule:dds}\advancedrulemark\\
Target using mini-jammer DDS&+PPL modifier\tablenotemark{3}&Yes&Yes&\ref{rule:dds}\advancedrulemark\\
Target and searcher in BJM arc&+BJM rating $-$ ECCM rating&Yes&No&\ref{rule:radar-jamming}\advancedrulemark\\
Searcher in AJM-C/D arc&+AJM rating $-$ ECCM rating&Yes&No&\ref{rule:radar-jamming}\advancedrulemark\\
\midrule
\multicolumn{5}{c}{Radar Operator}\\
\midrule
Veteran&$-1$&Yes&No&\ref{rule:crew-ability}\advancedrulemark\\
Novice&$+1$&Yes&No&\ref{rule:crew-ability}\advancedrulemark\\
Green&$+2$&Yes&No&\ref{rule:crew-ability}\advancedrulemark\\
Tactics master&$-1$&Yes&No&\ref{rule:crew-characteristics}\advancedrulemark\\
\bottomrule
\end{tabularx}
\begin{tablenote}{0.9\linewidth}
\advancedruletext
\tablenotetext{1}{The modifier applies to search attempts if Yes appears in this column.}
\tablenotetext{2}{The modifier applies to tracking attempts if Yes appears in this column.}
\tablenotetext{3}{Table~\ref{table:ew-decoy-ppl-effectiveness} gives the modifier.}
\tablenotetext{4}{Round up one half of the ECCM rating before subtracting it from the PPL modifier.}
\end{tablenote}
\end{twocolumntable}
\end{twocolumntablefloat}

\Ax{
\begin{onecolumntablefloat}
\begin{onecolumntable}
\tablecaption{table:radar-breaking-lock-on}{Breaking Lock-Ons with ECM}
\begin{tabularx}{\linewidth}{lX}
\toprule
ECM&Roll\\
\midrule
Chaff&Effectiveness or less\\
MJM&Effectiveness or less\\
DJM&DJM rating $-$ ECCM rating or less.\\
\bottomrule
\end{tabularx}
\end{onecolumntable}
\end{onecolumntablefloat}
}


\begin{twocolumntablefloat}
\begin{twocolumntable}
\tablecaption{table:radar-vertical-limits}{Radar Vertical Limits}
\begin{tabularx}{0.8\linewidth}{c*{6}{L@{ to }L}}
\toprule
Arc&
\multicolumn{2}{c}{VD}&
\multicolumn{2}{c}{SD or UD}&
\multicolumn{2}{c}{LVL}&
\multicolumn{2}{c}{SC}&
\multicolumn{2}{c}{ZC}&
\multicolumn{2}{c}{VC}\\
\midrule
Limited&$-2$&$-9$&$-0.5$&$-3$&$+0.5$&$-0.5$&$+2$&$+0$&$+4$&$+0.5$&$+9$&$+2$\\
180+&$-1$&$-\infty$&$-0$&$-5$&$+1$&$-1$&$+3$&$-0.5$&$+5$&$+0$&$+\infty$&$+1$\\
150+&$-0.5$&$-\infty$&$-0$&$-8$&$+2$&$-2$&$+4$&$-1$&$+8$&$+0$&$+\infty$&$+0.5$\\
120+&$-0$&$-\infty$&$+0.5$&$-\infty$&$+4$&$-4$&$+6$&$-2$&$+\infty$&$-0.5$&$+\infty$&$+0$\\
\bottomrule
\end{tabularx}
\begin{tablenote}{0.8\linewidth}
Note: The radar vertical limits are given as altitude levels per hex or horizontal distance. 
\end{tablenote}

\end{twocolumntable}
\end{twocolumntablefloat}



}



}

\paragraph{Normal Mode Search Modifiers.} Tables~\ref{table:radar-search-and-lock-on-modifiers} and \ref{table:ew-decoy-ppl-effectiveness} summarize the die-roll modifiers for normal mode searches. In detail, they are:
\begin{itemize}
    \item Apply the size modifier from the target’s ADC.
    \item If the target is using IFF, apply a \minus{2} modifier.
    \item If the target is using a chaff DDS, determine the base modifier from Table~\ref{table:ew-decoy-ppl-effectiveness} (using the PPL level to determine the row and the radar arc to determine the column), multiply the radar ECCM rating by one half and round this product up, and then subtract the product from the base modifier. If the result is positive, apply it as a modifier.
    \item If the target is using a mini-jammer DDS, determine and apply the modifier from Table~\ref{table:ew-decoy-ppl-effectiveness}.
    \item If the target is using a BJM in noise mode and the radar is in a jammed arc or if another aircraft is using a BJM in noise mode and both the target and the radar are in a jammed arc, use the BJM rating as a base modifier. From this base modifier, subtract the radar ECCM rating. If the result is positive, apply it as a modifier.
    \item If the target is using an AJM-C or AJM-D (but not an AJM-A or AJM-B) and the radar is in a jammed arc, use the AJM rating as a base modifier. From this base modifier, subtract the radar ECCM rating. If the result is positive, apply it as a modifier.
    \item Apply modifiers for the quality and characteristics of the radar operator (see advanced rules~\ref{rule:crew-ability} and \ref{rule:crew-characteristics}). 
\end{itemize}

\paragraph{Maintaining Normal Mode Detections.} An aircraft using its radar in normal mode maintains detection of a target as long as:
\begin{itemize}
\item The aircraft does not select a special radar mode.
\item The aircraft does not start tracking a target (unless it has some form of track-while-scan technology, described in advanced rule \ref{rule:track-while-scan-technology}).
\item During the flight phase the aircraft satisfies the normal mode search requirements above.
\item At the end of the flight phase, the target satisfies the normal mode search requirements above.
\end{itemize}

There is no limit to the number of detected targets an aircraft can maintain.

\subsection{Normal Mode Tracking}
\label{rule:normal-mode-tracking}

%To employ radar-guided missiles, an aircraft must refine and concentrate its radar beam on a target to determine the position more accurately and increase the reflected energy for semi-active radar-homing missiles. This is accomplished by switching the radar to a tracking mode and achieving a lock-on.

An aircraft may attempt to convert a detection into a track. In addition to the other benefits mentioned above, the maneuvering requirements for continuing to track a target are relaxed compared to those for continuing to detect a target, as the radar automatically adjusts to follow the target.

\paragraph{Normal Mode Tracking Requirements.} An aircraft with its radar in normal mode can attempt to track a target provided:
\begin{itemize}
\item It has a radar detection of the target.
\item The range to the target is no greater than the tracking range given in the radar section of the ADC. 
\end{itemize}

\paragraph{Normal Mode Tracking Procedure.} An aircraft may make one tracking attempt against one detected target per game turn (unless it has multi-target track technology, described in advanced rule \ref{rule:multi-target-track-technology}). Roll the die. If the result after applying any modifiers is less than or equal to the lock-on number given in the radar section of the ADC, the tracking attempt succeeds. 

If the tracking attempt succeeds, the aircraft begins tracking the target and loses all other radar detections (unless it has some form of track-while-scan technology, described in advanced rule \ref{rule:track-while-scan-technology}).

If a tracking attempt fails, the aircraft maintains all previous radar detections.

%Once a target is locked on, it remains locked on from game turn to game turn unless the lock-on is broken or voluntarily dropped. An aircraft may only have one lock-on at a time unless it has Multi-Target Track Technology.

\paragraph{Normal Mode Tracking Modifiers.} Table~\ref{table:radar-search-and-lock-on-modifiers} summarizes the die-roll modifiers for normal mode tracking attempts. The modifiers for tracking are a subset of those for searches. In detail, they are:

\begin{itemize}
    \item If the target is using IFF, apply a \minus{2} modifier.

    \item If the target is using a chaff DDS, determine the base modifier from Table~\ref{table:ew-decoy-ppl-effectiveness} (using the PPL level to determine the row and the radar arc to determine the column), multiply the radar ECCM rating by one half and round this product up, and then subtract the product from the base modifier. If the result is positive, apply it as a modifier.

    \item If the target is using a mini-jammer DDS, determine and apply the modifier from Table~\ref{table:ew-decoy-ppl-effectiveness}.

\end{itemize}

\paragraph{Maintaining a Normal Mode Track.} An aircraft may voluntarily stop tracking a target at any time.
An aircraft using is radar in normal mode continues to track a target as long as:
\begin{itemize}
\item The tracking aircraft does not select a special radar mode.
\item During the flight phase, the tracking aircraft does not:
\begin{itemize}
\item Use the ET turn rate.
\item Prepare for or execute any special maneuvers except slides and at most one vertical roll.
\item Declare a defensive preemption (see rule \ref{rule:defensive-preemptions}).
\end{itemize}
\item At the end of the flight phase, the target:
\begin{itemize}
\item Is within the radar horizontal and vertical limits (see rule \ref{rule:radar-arc-requirements}).
\item Has range that is no greater than the tracking range.
\item Is not in radar ground clutter (see  \ref{rule:radar-ground-clutter}).
\item Does not break the lock using decoys or ECM (see advanced rule \ref{rule:electronic-warfare}) and Table~\ref{table:radar-breaking-lock-on}. 
\end{itemize}
\end{itemize}

\begin{advancedrules}

\section{Radar Technologies}
\label{rule:radar-technology}

A radar's capabilities can be significantly enhanced by certain technologies.

\paragraph{Look-Down Technology.} \label{rule:look-down-technology} \label{rule:look-down-missiles}
If the technology section of an aircraft’s ADC states that it has look-down technology (sometimes referred to as \emph{full} look-down technology to distinguish it from \emph{limited} look-down technology), its radar is unhindered by ground clutter when searching for or tracking targets (see rule \ref{rule:radar-ground-clutter}) and when illuminating for a radar-guided missile with look-down capability (see rule \ref{rule:target-illumination}).

\paragraph{Limited Look-Down Technology.} \label{rule:limited-look-down-technology} If the technology section of an aircraft’s ADC states that it has limited look-down technology (sometimes denoted by parentheses around the words “look-down”), its radar can partially mitigate ground clutter when searching for or tracking targets (see rule \ref{rule:radar-ground-clutter}) and when illuminating for a radar-guided missile with look-down capability (see rule \ref{rule:target-illumination}).


\paragraph{Track-While-Scan Technology.} \label{rule:track-while-scan-technology} If the technology section of an aircraft’s ADC states that it has track-while-scan technology, it does not lose radar detections when it achieves a track and further can perform radar searches and maintain radar detections while tracking a target.

A number in parenthesis indicates the maximum number of detections the aircraft can achieve or maintain while locked on to a target. If no number is given, there is no limit.

% ISSUE: Which excess detections are lost?

\paragraph{Limited Track-While-Scan Technology.} If the technology section of an aircraft’s ADC states that it has \emph{limited} track-while-scan technology (sometimes denoted by parentheses around the words “track-while-scan”), it does not lose radar detections when it achieves a track but cannot perform further radar searches while tracking a target.

\paragraph{Multi-Target Track Technology.}\label{rule:multi-target-track-technology}  If the technology section of an aircraft’s ADC states that it has multi-target track technology, it may track more than one aircraft at a time. A number in parenthese indicates number of targets the aircraft may attempt to track attempt each game turn and the number of tracks it may maintain. 

\paragraph{Auto-Track Technology.}\label{rule:auto-track-technology} If the technology section of an aircraft’s ADC states that it has auto-track technology, it can use auto-track radar mode (see rule \ref{rule:auto-track-mode}).

\paragraph{Target ID Technology.}\label{rule:target-id-technology} If the technology section of an aircraft’s ADC states that it has target ID technology, it can identify a target automatically when it begins its second turn tracking that target (see rule \ref{rule:aircraft-identification}).

\section{Radar Vertical Limits}
\label{rule:radar-vertical-limits}

An aircraft's radar is limited vertically arc as well as horizontally. This advanced rule replaces the simple vertical limits given in the requirements in rule \ref{rule:radar-search-requirements}.

\paragraph{Vertical Limits Procedure.} Table~\ref{table:radar-vertical-limits} defines the vertical limits to radar arcs for each flight type. The two factors given in the table should be multipled by the horizontal range to the target, and rounded down, to give the maximum allowed difference in altitude between the searcher and the target for that arc.

For example, a searching aircraft in a zoom climb with a \arcrange{180}{+} arc has a potential radar target in its horizontal arc at a range of 8 hexes. Table~\ref{table:radar-vertical-limits} gives the factors for this combination of flight type and radar arc as \plus{5} and \plus{0}. Thus, if the target were between 0 and 40 altitude levels higher than the searcher, it would be within the vertical limits. If the target were lower than the searcher or more than 40 altitude levels higher, it would not be within the vertical limits.

\section{Boresight Mode} 
\label{rule:boresight-mode}

Boresight mode locks the radar to the gunsight and, as such, has many limitations compared to normal  mode. On the other hand, it does not have the maneuver restrictions that apply in normal mode.

If the radar has a tracking capability but no search capability, it can only use boresight mode to achieve a track. 

\paragraph{Declaring Boresight Mode.} All aircraft can use boresight mode. An aircraft may be declared to be using boresight mode in the aircraft decisions phase. 

An aircraft that switches from another radar mode to boresight mode immediately loses all previous radar detections and tracks. 

\paragraph{Boresight Mode Restrictions.}
If an aircraft is using boresight mode, normal mode search and tracking attempts are not allowed in the air-to-air radar phase. Instead, the procedures in this rule are followed.

\paragraph{Boresight Mode Arc.} 
In boresight mode, a radar uses the limited arc even if its normal mode radar arc is wider.

\paragraph{Boresight Mode Requirements.} 
An aircraft with its radar in boresight mode can perform a boresight search and tracking attempt in the air-to-air radar phase provided:

\begin{itemize}

\item During the entire preceding flight phase, it:
    \begin{itemize}
    \item Satisfied the common radar use requirements (see rule \ref{rule:common-radar-use-requirements}).
    \item Did not use terrain-following flight.
    \end{itemize}
\item At the end of the preceding flight phase, the target:
\begin{itemize}
    \item Is visually sighted.
    \item Has a range that does not exceed the radar's \emph{tracking strength}. 
    \item It is within the radar horizontal and vertical limits (see rule \ref{rule:radar-arc-requirements}).
    \item Is not in radar ground clutter (see  \ref{rule:radar-ground-clutter}).
\end{itemize}

\end{itemize}

\paragraph{Boresight Mode Search Procedure.}

% ISSUE: Visually sighted enemy? That is, are friendly aircraft ignored?

The closest \emph{visually sighted} aircraft in the \emph{limited} arc with range of no more than the \emph{tracking} strength is automatically detected. If two or more aircraft are equally close, randomly determine which is detected. 

Chaff and jamming do not affect this detection. 

\paragraph{Boresight Mode Tracking Procedure.}

If a target is detected, one tracking attempt is allowed against it. The procedure is the same as that for normal lock-on attempts, except that in  multi-crew aircraft the crew quality modifier for the pilot is used instead of that of the radar officer.

% ISSUE: What are the specific restrictions here for lock-on attemps?

\paragraph{Maintaining Boresight Mode Detections and Tracks.}

If an aircraft has a detection or track in boresight mode, it may choose to select normal mode in the next aircraft decisions phase. If it does so, it maintains the detection or track during the transition. The usual rules for maintaining a detection or track in normal mode then apply.

Alternatively, it may choose to continue to use boresight mode. If it has only a detection, it loses it immediately, but may attempt another detection in the subsequent air-to-air radar phase. If it has a track, it maintains the track as long as:
\begin{itemize}
    \item It satisfies the common radar use requirements (see rule \ref{rule:common-radar-use-requirements}).
    \item It does not used terrain-following flight.
    \item The target is visually sighted.
    \item At the end of the flight phase, the target:
    \begin{itemize}
        \item Has a range that does not exceed the radar's \emph{tracking strength}.
        \item Is within the radar horizontal and vertical limits (see rule \ref{rule:radar-arc-requirements}).
        \item Is not in radar ground clutter (see  \ref{rule:radar-ground-clutter}).
    \end{itemize}
\end{itemize}

\paragraph{Boresight Mode Limited Look-Down.}

When using boresight mode, an aircraft with no look-down technology can elect to be treated as if it had limited look-down technology and thus gain the ability to detect and track aircraft that would otherwise be in ground clutter. This decision is made just before the search procedure. 

If an aircraft uses this option, its tracking attempt die roll is subject to a modifier of \minus{2}. 

\section{Auto-Track Mode}
\label{rule:auto-track-mode} 

Auto-track mode is similar to boresight mode in that it allows an aircraft to automatically detect and then track a target, but it offers better performance and more capabilities. Again, it has the advantage that it does not have the maneuver restrictions that apply in normal mode.

\paragraph{Declaring Auto-Track Mode.} Only aircraft with auto-track technology (see \ref{rule:radar-technology}) can use auto-track mode. Such an aircraft may be declared to be using auto-track mode in the aircraft decisions phase. 

An aircraft that switches from another radar mode to auto-track mode immediately loses all previous radar detections and tracks. 

\paragraph{Auto-Track Mode Restrictions.}
If an aircraft is using auto-track mode, normal mode search and tracking attempts are not allowed in the air-to-air radar phase. Instead, the procedures in this rule are followed.

\paragraph{Auto-Track Mode Arc.} 
In auto-track mode, a radar uses the \arcrange{180}{+} arc even if its normal mode radar arc is wider.

\paragraph{Auto-Track Mode Requirements.} 
An aircraft with its radar in auto-track mode can perform an auto-track search and tracking attempt in the air-to-air radar phase provided:

\begin{itemize}

\item During the entire preceding flight phase, it:
    \begin{itemize}
    \item Satisfied the common radar use requirements (see rule \ref{rule:common-radar-use-requirements}).
    \item Did not use terrain-following flight.
    \end{itemize}
\item At the end of the preceding flight phase, the target:
\begin{itemize}
    \item Has a range that does not exceed the radar's \emph{search strength}.
    \item Is within the radar horizontal and vertical limits (see rule \ref{rule:radar-arc-requirements}).
    \item Is not in radar ground clutter (see  \ref{rule:radar-ground-clutter}).
\end{itemize}

\end{itemize}


\paragraph{Auto-Track Mode Search Procedure.}

One of two procedures may be chosen for the search:
\begin{enumerate}
\item The closest target may be automatically detected. If two or more aircraft are equally close, randomly determine which is detected. Auto-track mode ignores friendly aircraft with their IFF on.
\item Alternatively, any one \emph{sighted} target may be detected on a roll of \minusafter{7} with no modifiers.
\end{enumerate}

Chaff and jamming do not affect this detection. 

\paragraph{Auto-Track Mode Tracking Procedure.}

If a target is detected, one tracking attempt is allowed against it. The procedure is the same as that for normal lock-on attempts, except that in  multi-crew aircraft the crew quality modifier for the pilot is used instead of that of the radar officer.

BJMs and AJMs can impact tracking attempts.

\paragraph{Maintaining Auto-Track Mode Detections and Tracks.}

If an aircraft has a detection or track in auto-track mode, it may choose to select normal mode in the next aircraft decisions phase. If it does so, it maintains the detection or track during the transition. The usual rules for maintaining a detection or track in normal mode then apply.

Alternatively, it may choose to continue to use auto-track mode. If it has only a detection, it loses it immediately, but may attempt another detection in the subsequent air-to-air radar phase. If it has a track, it maintains the track as long as:
\begin{itemize}
    \item It satisfies the common radar use requirements (see rule \ref{rule:common-radar-use-requirements}).
    \item It does not used terrain-following flight.
    \item At the end of the flight phase, the target:
    \begin{itemize}
        \item Has a range that does not exceed the radar's \emph{search strength}.
        \item Is within the radar horizontal and vertical limits (see rule \ref{rule:radar-arc-requirements}).
        \item Is not in radar ground clutter (see  \ref{rule:radar-ground-clutter}).
    \end{itemize}
\end{itemize}


\paragraph{Auto-Track Mode Limited Look-Down.}

When using auto-track mode, an aircraft with no look-down technology can elect to be treated as if it had limited look-down technology and thus gain the ability to detect and track aircraft that would otherwise be in ground clutter. This decision is made just before the search procedure. 

If an aircraft uses this option, its tracking attempt die roll is subject to a modifier of \minus{2}. 


\AX{

\section{Infrared Search and Track Systems}
\label{rule:irsts}

An infrared search and track system (IRSTS) uses infrared television and optical equipment to provide the aircraft crew with the ability to detect and track aircraft using their infrared emissions. IRSTS operation is not detectable by any aircraft’s ECM or RWR gear. 

Although IRSTS operates passively, it is in many ways similar to radar in that it provides detections and tracks of targets, but does not by itself provide sighting. Nevertheless, like radar, it can aid subsequent sighting attempts and be used to fire IRMs on unsighted targets.

\paragraph{IRSTS Types.} Three types of IRSTS exist: 
\begin{itemize}
    \item IRSTS-A (early) covers the limited \CY[3A-combined-arcs]{arc}{combined arc (see rule \ref{rule:combined-arcs})} and successfully locks-on on a roll of \minusafter{6}.
    \item IRSTS-B (modern) covers the \arcrange{180}{+} \CY[3A-combined-arcs]{arc}{combined arcs (see rule \ref{rule:combined-arcs})} and successfully locks-on on a roll of \minusafter{8}.
    \item IRSTS-C (advanced) covers the \arcrange{150}{+} \CY[3A-combined-arcs]{arc}{combined arcs (see rule \ref{rule:combined-arcs})} and successfully locks-on on a roll of \minusafter{8}.
    
\end{itemize}

\paragraph{IRSTS Requirements.}  IRSTS is used in the Radar Search and Lock-On Phase. A pilot-only aircraft may use either radar or IRSTS in this phase, and a multi-crew aircraft may do both. Unlike radar, there are no maneuver-related restrictions on the use of IRSTS.

\paragraph{IRSTS Detection.} The target is automatically detected its range is no more than the maximum detection range, which depends on the targets fuel consumption and aspect. For IRSTS-A/B, the maximum detection range is four times the number of fuel points used by the target at its current power setting (according to the fuel column in the power chart of the target’s ADC). For IRSTS-C, the maximum detection range is six times the number of fuel points. The maximum range is doubled if the IRSTS is used in the target’s \arcrange{30}{-} arc. 

\AY[3A-irsts-weather]{IRSTS systems are not affected by haze (see advanced rule \ref{rule:haze-layers}).}

\paragraph{IRSTS Lock-On.} An IRSTS-A/B may attempt to lock-on either the nearest detected target or the target with the highest fuel point use. For this purpose, the fuel use is doubled if the IRSTS is used in the target’s \arcrange{30}{-} \AY[3A-combined-arcs]{combined} arc. An IRSTS-C may attempt to lock-on up to six of the detected targets, with no restrictions on closeness or fuel usage.

For each target, roll the die. The lock-on succeeds on a roll of \minusafter{6} for an IRSTS-A and \minusafter{8} for an IRSTS-B/C. 

After the lock-on attempts are resolved, if any succeeded, the aircraft loses all other IRSTS detections.

\paragraph{Maintaining IRSTS Detections and Lock-Ons.} IRSTS detections or lock-ons are maintained provided the target satisfies the arc and range requirements for detection at the start of each subsequent Radar Search and Lock-On Phase and, if the aircraft has a single pilot, provided the pilot does not use radar.

\paragraph{Sighting Effects.}
An IRSTS can provide a \emph{detection} of the target aircraft but does not by itself \emph{sight} the target. Nevertheless, an IRSTS can aid in subsequent sighting attempts. 

\begin{itemize}
\item If an searcher has an IRSTS detection or track of a target aircraft, they gain a \minus{1} sighting roll modifier. (This modifier can also be gained by having a radar detector or track or a RWR detection; it is applied only once regardless of the number of different types of detections or tracks the searcher has.)
\item
If an searcher has an IRSTS track of a target aircraft of a target aircraft in their \arcrange{180}{+} \CY[3A-combined-arcs]{arcs}{combined arcs} and also has HUD interface technology (see rule \ref{rule:hud}), they gain an additional \minus{1} sighting roll modifier.  (This modifier can also be gained by having a radar track; it is applied only once regardless of the number of different types of tracks the searcher has.) Furthermore, the maximum sighting range at night for visual searches is doubled (see advanced rule \ref{rule:combat-at-night}).
\end{itemize}

\paragraph{IRSTS Assist for IRMs.} An aircraft may launch IRMs at a locked-on target even if it is not sighted. An IRSTS-B/C allows IRMs to be launched at locked-on targets as if IRM uncage technology were available, even if the fighter does not have that technology.

}

\end{advancedrules}

}


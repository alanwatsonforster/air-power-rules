\CX{

\begin{onecolumntablefloat}
\begin{onecolumntable}
\tablecaption{table:sighting-rules-summary}{Sighting Rules Summary.}
\begin{tabularx}{\linewidth}{p{12em}X}
\toprule
\multicolumn{2}{c}{Maximum Sighting Ranges}\\
\midrule
by Eyeball&$4 \times \mbox{aircraft Vis.\ No.}$\\
by V.A.S. &$6 \times \mbox{aircraft Vis.\ No.}$\\
by V.A.S.\ with radar assist&$10 \times \mbox{aircraft Vis.\ No.}$\\
at Night&2 hexes\\
at Night in A/B&6 hexes\\
\midrule
\multicolumn{2}{c}{Target I.D. Ranges}\\
\midrule
by Eyeball&$2 \times \mbox{aircraft Vis.\ No.}$\\
by V.A.S. &$4 \times \mbox{aircraft Vis.\ No.}$\\
at Night&Same position, facing and speed required.\\
With Tgt.\ I.D. radar technology available&2d turn of Lock\\
\midrule
\multicolumn{2}{c}{Padlocking (PL)}\\
\midrule
\multicolumn{2}{p{24em}}{
One PL allowed per aircraft.\newline
Two PLs if multi-crew aircraft.\newline
No PLs allowed into blind arcs.\newline
No PLs by Novices or Greens.\newline
1 extra PL per Vet.\ or Tac.\ Mstr.\newline
No PLs if in Target Sun Arc.
}\\
\bottomrule
\end{tabularx}
\end{onecolumntable}
\end{onecolumntablefloat}

}
{

\begin{twocolumntablefloat}
\begin{twocolumntable}
\tablecaption{table:maximum-sighting-range}{Maximum Sighting and Identification Ranges.}
\small
\begin{tabularx}{0.7\linewidth}{llLl}
\toprule
Target&Process&Range$^a$&Rule\\
\midrule
\multicolumn{4}{c}{\wbox[l]{Identification during the Day}{Sighting during the Day}}\\
\midrule
Aircraft&Visual&$\mbox{visibility} \times 4$&\ref{rule:sighting-aircraft-and-missiles}\\
\RDY[3A-sighting]{Aircraft&Visual in haze&$\mbox{visibility} \times 2$&\ref{rule:haze-layers}\asteriskmark}
Aircraft&VAS &$\mbox{visibility} \times 6$&\ref{rule:vas}\asteriskmark\\
Aircraft&VAS with radar assist&$\mbox{visibility} \times 10$&\ref{rule:vas}\asteriskmark\\
Aircraft, contrailing&Visual$^b$&150&\ref{rule:contrail-layers}\asteriskmark\\
Aircraft, in same SCL as searcher&Visual&1&\ref{rule:cloud-layers}\\
Missile&Visual&$\mbox{visibility} \times 4$&\ref{rule:sighting-aircraft-and-missiles}\\
Missile, contrailing&Visual$^b$&90&\ref{rule:contrail-layers}\asteriskmark\\
Missile, in same SCL as searcher&Visual&1&\ref{rule:cloud-layers}\\
\midrule
\multicolumn{4}{c}{\wbox[l]{Identification during the Day}{Sighting at Night}}\\
\midrule
Aircraft&Visual&2&\ref{rule:combat-at-night}\asteriskmark\\
Aircraft, using AB&Visual&6&\ref{rule:combat-at-night}\asteriskmark\\
Aircraft&IRSTS lock-on and HUD&Twice normal&\ref{rule:irsts}\asteriskmark\\
Aircraft, contrailing above clouds&Visual$^b$&24&\ref{rule:contrail-layers}\asteriskmark\\
Aircraft, contrailing below clouds&Visual$^b$&6&\ref{rule:contrail-layers}\asteriskmark\\
Aircraft, in same SCL as searcher&Visual&1&\ref{rule:cloud-layers}\\
Missile&Visual&$\mbox{visibility} \times 2$&\ref{rule:combat-at-night}\asteriskmark\\
Missile, contrailing above clouds&Visual$^b$&24&\ref{rule:contrail-layers}\asteriskmark\\
Missile, contrailing below clouds&Visual$^b$&6&\ref{rule:contrail-layers}\asteriskmark\\
Missile, in same SCL as searcher&Visual&1&\ref{rule:cloud-layers}\\
\midrule
\multicolumn{4}{c}{\wbox[l]{Identification during the Day}{Identification during the Day}}\\
\midrule
Aircraft&Visual$^b$&$\mbox{visibility} \times 2$&\ref{rule:limited-intelligence}\asteriskmark\\
Aircraft&VAS$^b$&$\mbox{visibility} \times 4$&\ref{rule:limited-intelligence}\asteriskmark\\
Aircraft&VAS$^c$&$\mbox{visibility} \times 10$&\ref{rule:limited-intelligence}\asteriskmark\\
\midrule
\multicolumn{4}{c}{\wbox[l]{Identification during the Day}{Identification at Night}}\\
\midrule
Aircraft&Visual$^d$&0&\ref{rule:combat-at-night}\asteriskmark\\
\bottomrule
\end{tabularx}
\begin{tablenote}{0.7\linewidth}
\asteriskmark~Advanced rule. $^a$~If searching visually for higher targets during the day, count each 4 levels difference in altitude as 1 hex of range.\AY[3A-sighting]{ If the searcher, target, or both are in haze (see advanced rule \ref{rule:haze-layers}), consider the sighting range to be twice the normal range.} $^b$~Succeeds automatically. $^c$~Succeeds on roll of \minusafter{7}. $^d$~Identification at night requires the same position, facing, and speed as the target.  
\end{tablenote}
\end{twocolumntable}

\vspace{\floatsep}

\begin{onecolumntable}
\small
\tablecaption{table:padlocks}{Padlocks.}
\begin{tabularx}{0.8\linewidth}{Xcl}
\toprule
Crewmember&Padlocks&Rule\\
\midrule
Pilot&1&\ref{rule:sighting-aircraft-and-missiles}\\
Observer&1&\ref{rule:sighting-aircraft-and-missiles}\\
\midrule
Veteran or tactics master&2&\ref{rule:crew-quality}\asteriskmark\\
Novice or green&0&\ref{rule:crew-quality}\asteriskmark\\
Unconscious&0&\ref{rule:gloc}\asteriskmark\\
Disoriented&0&\ref{rule:disorientation}\asteriskmark\\
\bottomrule
\end{tabularx}
\begin{tablenote}{0.8\linewidth}
\asteriskmark~Advanced rule.
\end{tablenote}
\end{onecolumntable}

\end{twocolumntablefloat}





}

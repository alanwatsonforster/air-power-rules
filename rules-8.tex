\section{Changing Aircraft Altitude}

This chapter details the procedures involved in changing aircraft altitudes.

\paragraph{Climbs or Dives Only.} An aircraft may either climb or dive in a turn, but it may not do both.  When climbing or diving, a portion of the aircraft's FPs are spent as Vertical Flight Points (VFPs); the rest are used as Horizontal Flight Points (HFPs). As an aircraft changes altitude during flight, mark the number of altitude levels gained or lost on the Aircraft Log. At the end of the aircraft's flight use this record to determine the aircraft's new start altitude for the next turn.

\paragraph{Altitude Structure.} In the game, the atmosphere is divided into 1,000-foot levels. These altitude levels are further grouped into altitude bands, each several levels thick. Aircraft and missile performance vary within each altitude band. An aircraft's altitude at the beginning of a turn is noted on the Start Altitude line of the aircraft log. A missile's altitude at the beginning of a turn is noted on the Missile Notes line of the launching aircraft's aircraft log. An aircraft may not climb higher than its ceiling. The altitude structure is depicted below:

\begin{table}[h!]
\centering
\begin{tabular}{lll}
\hline
ALTITUDE BAND&CODE&LEVELS\\
\hline
Low             &LO &1 to 7\\
Medium Low      &ML &8 to 16\\
Medium High     &MH &17 to 25\\
High            &HI &26 to 35\\
Very High       &VH &36 to 45\\
Extremely High  &EG &46 to 60\\
Ultra High +    &UH &61 and Higher\\
\hline
\end{tabular}
\end{table}

\paragraph{Vertical Flight Points.} FPs spent to gain or lose altitude levels are called Vertical Flight Points (VFPs). VFPs vary in the amount of altitude increase or decrease provided depending on the specific type of climb or dive used. VFPs are usually available as 1/3 or 2/3 of the total FPs; the 1/3-2/3 Table is a quick reference to the accepted proportions of VFPs to FPs available for most aircraft speeds. VFPs are only spent as full FPs. When the 1/3-2/3 Table provides half VFPs, they are ignored. A single VFP will never gain or lose more than two altitude levels.

\subsection{Climbing Flight}

An aircraft in climbing flight selects one of three types of climb to use: Zoom Climb, Sustained Climb, or Vertical Climb. Each type of climb prescribes how many VFPs and HFPs an aircraft must or may have available to it.

\subsubsection{Zoom Climbs}

A zoom climb is a maneuvering climb in which the aircraft gains altitude more from inertia than wing lift. Some wing lift may be involved but most of it is being applied to aircraft maneuvering instead of altitude gain. In a zoom climb, the player is less restricted than in other climbs. Altitude is gained less efficiently though sometimes at a greater rate.

\begin{itemize}

    \itemparagraph{ZC Procedure:} Declare climbing flight and note ZC as flight type on the Aircraft Log. At least 1/3 of FPs must be HFPs; the remainder may be VFPs.

    \itemparagraph{ZC Altitude Gain:} ZC VFPs produce an altitude gain based on the Climb Capability Chart. Refer to the CCC on the ADC and find the appropriate configuration, power setting, and altitude band.  If the climb capability is 2.0 or less, one VFP produces a gain of 1 altitude level; if the climb capability is more than 2.0, one VFP produces a gain of 2 altitude levels.

    \itemparagraph{ZC Restrictions:} Aircraft in a Zoom Climb may not use the ET turn rate.

    \itemparagraph{ZC Decel Points:} In the first turn of a Zoom Climb, the aircraft receives 1.0 decel point per altitude level gained; on subsequent consecutive turns of ZC, the aircraft receives 1.5 del points per altitude level gained.

\end{itemize}

\subsection{Sustained Climbs}

A sustained climb relies on maximum power and full lift and is the most efficient way to climb over several game turns. Sustained climbs do restrict the aircraft's maneuverability.

\begin{itemize}

    \itemparagraph{SC Procedure:} Declare climbing flight and note SC as flight type on the Aircraft Log.
    
    \itemparagraph{SC Altitude Gain:} SC VFPs produce an altitude gain based on the Climb Capability Chart. Refer to the CCC and find the appropriate configuration, power setting, and altitude band. Three cases may apply:
    \begin{enumerate}
        \item If the climb capability value is a fraction, only 1 VFP is allowed, the rest are HFPs. The VFP gains only the fractional altitude level.
        \item If the value is 1.0 to 1.5, then up to 2/3ds the FPs may be VFPs and the first VFP gains any fraction and the rest gain 1 level each.
        \item If the value is greater than 2.0, then up to 2/3ds the FPs may be VFPs and each may gain 1 or 2 levels.
    \end{enumerate}

    \itemparagraph{SC Prerequisites and Limits:} To use a sustained climb the aircraft must have a start speed at least 1.0 greater than its minimum speed. If the start speed is less than the aircraft's optimum climb speed, the CCC values are halved (retain fractions). Sustained climb decel applies only to an amount of levels gained equal to the CCC value (halved if applicable).

    \itemparagraph{SC Excess Altitude Gain:} Aircraft may expend VFPs to climb more levels than listed or normally allowed if sufficient VFPs are available. However, any levels gained beyond the listed or allowed CCC limits incur decel points as if the aircraft were zoom climbing instead of sustained climbing.

    \itemparagraph{SC Restrictions:} Aircraft in a sustained climb may only use EZ turn rates (snap turning prohibited) and may only use slide maneuvers.

    \itemparagraph{SC Decel Points:} 0.5 Decel points are incurred for each altitude level gained in a sustained climb until the sustained climb limit is reached and then decel is accumulated as if zoom climbing.

\end{itemize}

Example of a SC with excess altitude gain: A MiG-21 with a speed of 6.0 in the LO band, CL configured, and at AB power has a CCC value of 4. It may have up to 4 VFPs and chooses to do so. Since the CCC value is greater than 2.0 it may gain two levels per VFP. The player elects to move as follows; H, H, V+2, V+2, V+1, V+1 (moves forward two hexes and uses the four VFPs to gain 6 altitude levels). 0.5 Decel is incurred for each of the first four levels gained (the amount = to CCC value), and 1.0 decel for each of the last two (the amount exceeding CCC value). Total decel for climbing is 4.0.
    
\subsubsection{Vertical Climbs}

A vertical climb gains altitude quickly but at great cost in energy; no wing lift is involved. The aircraft is coasting upward on power and inertia.

\begin{itemize}

    \itemparagraph{VC Prerequisite:} A Vertical climb may be selected only if the aircraft climbed in the previous game turn. Exception: A High Pitch Rate aircraft (if its current speed is less than 4.0) may declare a Vertical Climb from Level Flight.

    \itemparagraph{VC Procedure and Limits:} Declare climbing flight and note VC as flight type on the aircraft log. On the first turn of VC, exactly 1/3 of FPs must be HFPs; the remainder are VFPs. On the second or subsequent turns of a consecutive VC, no more than 1/3 of FPs may be HFPs (and up to all FPs may be VFPs).

    \itemparagraph{VC Altitude Gain:} All aircraft may gain 1 or 2 altitude levels per VFP regardless of normal climb ability.

    \itemparagraph{VC Restrictions:} No turns or maneuvers except Vertical Rolls are allowed. The aircraft may not dive on the game turn following a vertical climb. Exceptions: A High Pitch Rate aircraft may freely steep dive or use unloaded dives following a vertical climb. A non-High Pitch Rate aircraft may use the Half-Roll Dive and vertical Reverse maneuvers to enter diving flight (see Ch 13) after vertical climbs.

    \itemparagraph{VC Decel Points:} The aircraft receives 2 decel points for each altitude level gained.

\end{itemize}

\subsection{Additional Considerations}

\paragraph{Partial Altitude Gains.} The CCC at times indicates fractional altitude gains. You will have noticed that the climb charts sometimes allow fractional gains in altitude levels. Some aircraft may require more than one turn of climbing flight to gain an altitude level. An aircraft's starting altitude is always the last full altitude level it climbed to.

\paragraph{Altitude Carry.} If an aircraft's total altitude change during flight included a fractional amount, the fraction is carried forward to the next game-turn to be added to any further climbing. Note this on the climb notes line of the log sheet. Fractional gains may be carried and added only as long as the aircraft continues to climb from turn to turn. The moment an aircraft chooses level or diving flight, any fractional climb carry is lost. Climb carry is ignored when determining an aircraft's altitude for spotting, combat or any other purposes.

\paragraph{Supersonic Climbs.} If the advanced rules for supersonic flight are in use, the CCC value is reduced to 2/3ds that listed on the ADC when aircraft are at supersonic speeds.

\subsection{Diving Flight}

An aircraft in diving flight selects one of three types of dive: Steep Dive, Unloaded Dive, or Vertical Dive. Diving flight is handled in a manner similar to climbing flight. For purposes of maintaining dive speeds (rule 6), at least two or more altitude levels must be lost in a game turn through diving flight.

\subsubsection{Steep Dives}

A steep dive is a maneuvering dive in which some of the aircraft's acceleration is committed toward maneuvering the aircraft instead of speeding up. A steep dive is the least restrictive type of dive.

\begin{itemize}

    \itemparagraph{SD Procedure:} Declare diving flight and note SD as flight type on the aircraft log. At least 1/3 of FPs must be HFPs. The rest may be VFPs.

    \itemparagraph{SD Altitude Loss:} 1 or 2 altitude levels may be lost per VFP expended.

    \itemparagraph{SD Restrictions:} There are no restrictions to maneuvering the aircraft while in a steep dive.

    \itemparagraph{SD Accel Points:} On the first turn of steep diving, the aircraft receives 0.5 accel points per altitude level lost; on subsequent turn of continued diving, it receives 1.0 accel points per altitude level lost.

\end{itemize}

\subsubsection{Unloaded Dives}

An unloaded dive is used to rapidly gain acceleration. The aircraft dives to match the fall of gravity; this causes weightlessness and eliminates induced drag from the aircraft. Combined with acceleration from gravity and the engine's thrust, the aircraft achieves rapid gains in distance and speed.

\begin{itemize}

    \itemparagraph{UD Procedure:} Declare diving flight and note UD on the aircraft log. Start from level flight.  All FPs are HFPs. All or some (but at least some) of the HFPs must be spent with the aircraft “unloaded”. Each HFP spent while unloaded moves the aircraft forward one hex or hexside and causes it to lose one altitude level.

    \itemparagraph{UD Limits:} All unloaded HFPs must be expended in a continuous series. They may be spent at the begining, end or in the middle of the aircraft's flight. The rest of the HFPs may be expended normally for maneuvering puropose. Unloaed HFPs may not be counted toward any turning or maneuvering requirements.

    \itemparagraph{UD Restrictions:} An aircraft spending unloaded HFPs may not conduct any attacks, aim, track targets or launch weapons.

     \itemparagraph{UD Accel Points:} On the first turn of an UD, the aircraft receives 0.5 accel points per altitude level lost; on subsequent turns of continued diving it recieves 1.0 accel point per altitude level lost.

\end{itemize}

Note: The advantage to unloaded over steep dives is the horizontal distance gained over similar dives.

\subsubsection{Vertical Dives}

A vertical dive sends an aircraft nearly straight down; altitude is lost quickly and acceleration builds up rapidly.

\begin{itemize}

    \itemparagraph{VD Procedure:} Declare diving flight and note VD on the Aircraft Log. If this is the first turn of vertical diving, 1/3 of the FPs must be HFPs and the rest VFPs. If this is the second or subsequent turn of consecutive vertical dives, then no more than 1/3 of the FPs can be HFPs but all can be VFPs.

    \itemparagraph{VD Altitude Loss:} In a vertical dive, 2 or 3 altitude levels must be lost for each VFP expended.

    \itemparagraph{VD Restrictions:} Aircraft in a VD may not do turns or use maneuvers except for Vertical Rolls. Climbing flight is not allowed on the turn following a vertical dive. Vertical dives must be followed on subsequent game turns by diving flight.

    \itemparagraph{VD Recovery:} Due to the difficulty of pulling out of vertical dives, the following applies:
    \begin{enumerate}
        \item[a)] When a steep or unloaded dive immediately follows a vertical one, half the aircraft's FPs (round down) must be VFPs or unloaded HFPs as appropriate.
        \item[b)]  In the case of a High Pitch Rate aircraft, only 1/3 has to be VFPs or unloaded HFPs. Exception: A High Pitch Rate aircraft whose start speed after vertically diving is 3.0 or less may use level flight following a vertical dive.
    \end{enumerate}

    \itemparagraph{VD Accel Points:} Aircraft in a vertical dive gain 1 Accel point per altitude level lost.

\end{itemize}

\subsection{Free Descent}

\paragraph{Level Flight Free Descent.} An aircraft in level flight may choose free descent and lose one altitude level during the game-turn. This descent may be selected to take place after the expenditure of any one HFP during the game-turn. No Accel points are received. No other restrictions apply.

\trainingnote{
You are now ready to play Training Scenario 2.\\
The sequence of play is still not required.
}

\advancedrules

\subsection{Using Half VFPs}

In normal practice, VFPs are spent only as full FPs. When the 1/3-2/3 Table allots half VFPs, the aircraft may either:

\begin{itemize}

    \item Carry the half VFP forward to the next turn as a generic half FP.	

    \item Mate the half VFP to a previously carried half FP to create a full VFP and use it.

    \item Steal a half VFP from the alloted HFPs to create a full VFP, and carry the remaining half HFP forward to the next turn.

\end{itemize}

\subsection{Loss of Thrust with Altitude}

Jet engines lose thrust in the thinner air at high altitudes. To reflect this, use the following:

\begin{itemize}

    \item In the VH band, thrust is 2/3 normal (but never less than 0.5).

    \item In the EH and UH bands, thrust is 1/3 normal (but never less than 0.5).

\end{itemize}

The 1/3-2/3 Chart is useful in calculating reduced thrust.

\paragraph{High Altitude Engines.} Some aircraft are noted as having engines specifically designed for high altitude flight. These aircraft ignore this rule.

\subsection{Flight Above Maximum Ceiling}

Each aircraft has a ceiling (maximum altitude) stated on the MMVC. An aircraft may temporarily ZC or VC above its ceiling (SC cannot carry an aircraft above its ceiling). While above its celling, an aircraft is subject to the following risks:

\begin{itemize}

    \item If the aircraft uses any turn rate other than EZ, the aircraft experiences a Maneuvering Departure on a die roll of $4-$.
    
    \item If the aircraft performs a roll maneuver, the aircraft experiences a Maneuvering Departure on a die roll of $4-$.

    \item If the aircraft starts the game turn above its ceiling and selects any power other than Idle, each engine will Flame-out on die roll of $4-$ (apply a modifier of $-1$ for each turn the aircraft starts above its ceiling). Check for Flame-out immediately upon selecting the power setting.

\end{itemize}

\subsection{Formation Restrictions on Climbs and Dives}

\paragraph{Close Formations.} Aircraft in close formation may only change altitude using non-AB powered sustained climbs, or steep dives of no more than two altitude levels per turn, or by free descents. If these limits are exceeded, the formation automatically breaks down into tactical formation and collisions are possible if more than two aircraft end the turn in the same position.

\paragraph{Tactical Formations.} Aircraft in Tactical formations have no altitude change restrictions.

\section{In This Game}

This {\rulename} introduces the {\AirPow} game system and its components.

\subsection{The Game System}

The {\AirPow} game system allows you to control one or more jet fighters in scenarios reflecting the actual life and death situations faced by fighter pilots in modern air combat. These rules provide the procedures for simulating jet fighter combat on a game board. As players, you must provide the brains, the strategy, and the tactics that will allow your jets to survive combat and win the various game scenarios.

Note: The {\AirPow} system is a derivative of the game system I used in GDW's {\AirSup} and {\AirStr} games (now out of print), and as a result the data cards and most of the information tables used in the {\AirSup} games are fully compatible with this game system and may be used with it.

\subsection{Game Scale}

\begin{itemize}
    \item Each game turn represents 12 seconds of real time.
    \item Each map hex covers a distance of {\onethird} \x{a}{of a} statute mile.
    \item Each altitude level represents 1,000 feet of height.
    \item Each altitude band \addedin{1B}{JDW in the TSOH errata}{typically }is 8,000 to 10,000 feet thick.
    \item Each aircraft speed point equals 100 mph of speed.
    \item Each aircraft counter represents a single \x{jet}{aircraft}.
\end{itemize}

\subsection{Learning the Game System}

\paragraph{Read The Basic Rules First.} You should read only the rules that appear prior to any “Advanced Rules” header within each chapter. Skip the advanced rules and read on into the next chapter. Keep reading until you are instructed to play a \xlc{Training Scenario}. Take a break, then set up the scenario, and play it out. All of the \xlc{Training Scenarios} are designed for solitaire play. When you have finished, return to where you left off and continue reading. Do this until you've been exposed to all the basic rules and played all \x{}{the }training scenarios.

\paragraph{Read the Advanced Rules If Desired.} Advanced rules are optional in nature and allow you to raise the level of detail and realism contained in the game. Using the advanced rules will increase the complexity of play but rewards you with a more accurate depiction of modern air combat. All, some, or none of the advanced rules may be learned and used as agreed upon by the players. The training scenarios may be replayed with advanced rules for practice.

\paragraph{Be Patient and Have Fun!} Don't expect to learn all the rules in a single seating. Take your time and enjoy yourself. After playing several of the training scenarios, you will begin to get the hang of the system. Don't forget, this is a game. Play it for its entertainment value, and have a good time as you strive to master the intricacies of modern air combat.

\subsection{Game Components}

Like most simulation games, those in this system have the following basic types of components:

\paragraph{1. Game Rules.} This set of rules defines how aircraft, missiles, and ground units move, detect enemies, and fight. It is not necessary to memorize the rules. Play{\xhyphen}aid sheets summarizing the key points of the rules are provided.

Once familiar with the rules, you should be able to play the game referring to the play{\xhyphen}aid charts alone; returning to the rules booklet only to clarify questions. There are two different levels of rules: Basic, and Advanced. Basic rules are all that are necessary to play any scenario in the game.

\paragraph{2. Game Charts.} Game charts distill large amounts of information into easily used tables. Players use the charts to determine specific capabilities of their aircraft and weapons, and to resolve combat. Charts in this system include: Flight and combat rules summaries, weapons data tables, aircraft data cards, and aircraft logsheets.

\paragraph{3. Game Counters.} The die-cut cardboard counters are the game pieces used to represent the aircraft, missiles, and ground units involved in play. Some of the counters represent information (such as target hits) rather than objects. Usually less than 20 pieces will be required for the play of any scenario. The counters shown here represent a fighter and an air to air missile. Each has a distinctive silhouette and color scheme that lets players recognize them easily.

\begin{figure}[h!]
\centering
\changedin{1D}{AWF}{
\includegraphics[width=\linewidth]{figures/figure-counters.pdf}
}{
\includegraphics[width=1.5cm]{figures/counter-f-4.jpg}\hspace{1.5cm}\includegraphics[width=1.5cm]{figures/counter-aam.jpg}
}
\end{figure}

\paragraph{4. \protect\x{Dice}{Die}.} Each game in this system includes one ten-sided die as a random number generator. {\AirPow} uses the die to resolve events of chance during play. The die is marked with ten digits: 0 to 9. When rolled, it produces numbers from 0 to 9; the topmost number on the die is the one read. The 0 is always read as a 10. Thus, when the ten-sided die is rolled, it will produce a number from 1 to 10.

Die usage example: if a game chart indicates that a certain missile will hit its target 80\% of the time when it attacks. A player must use the die and roll 8 or less in order to hit. That die roll represents the 80\% probability of hitting. In this case, a miss would occur if a 9 or 10 was rolled.

\paragraph{5. Game Maps.} The game maps are the surface over which the playing pieces will be moved. Though several are provided, sometimes only a few will be used in a scenario. The hexagon grid on each map provides spaces in which players place and move their counters. Like a chessboard, the hexagons help to clearly define where a counter is. Distance between counters (range) is determined by counting the number of hexes between them.

For example, as each hex represents {\onethird} of a mile of distance, and as the two aircraft counters shown in the diagram below are 6 hexes apart, they are a scale two miles apart. 
\deletedin{1D}{AWF}{\addedin{1B}{JDW in the TSOH errata}{[The two aircraft counters were left off the diagram.]}}
Notice the large shaded hex outlines on the map. These are termed “megahexes”. Each megahex is 5 regular hexes across and allows players to determine long ranges and distances easier by counting in fives. 
\addedin{1B}{JDW in the TSOH errata}{Altitude differences may affect range calculations.}

\begin{figure}[h!]
\centering
\changedin{1D}{AWF}{
\includegraphics[width=\linewidth]{figures/figure-map.pdf}
}{
\input{figures/figure-map}
}
\end{figure}
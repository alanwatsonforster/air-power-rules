\documentclass[10pt]{article}

\input style.tex

\title{Updated Rules Project}
\author{Alan Watson Forster}
\date{12 March 2024}
\runningtitle{Updated Rules Project}

\setcounter{secnumdepth}{0}
%\setlist[enumerate,1]{label = (\alph*)}

\newcommand{\itemtag}[1]{\item \textbf{Change: #1.}\par}

\begin{document}

\twocolumn
\thispagestyle{empty}
\maketitle
\suppressfloats

\section{Aims}

This project aims to produce updated rules for {\AirPow}.

\section{Context}

At the time of writing, there are several sets of rules available:
\begin{itemize}
    \item The original {\AirSup} and {\AirStr} rules from 1987, with extensions in {\DF} and {\EOTG}. I understand that a limited number of copies of {\AirSup} are still available. Nevertheless, these rules are considered obsolete.
    \item The first-edition {\AirPow} rules, with official errata, published in 1992 with {\TSOH}. There are two sets of errata: one page provided with {\TSOH} and four pages published in {\APJ} \#24. Again, I understand that a very limited number of copies are still available.
    \item Several proposals for second-edition rules outlined by J.D.\ Webster mainly in {\APJ} during the 1990s.
    \item The second-edition {\AirPow} rules, edited by Malcolm Pipes and published in 2022 on the \href{https://airpower.groups.io/g/main}{airpower email group} site. The current text is labeled “v2.4”.
\end{itemize}

\section{Rationale}

There is already a set of second-edition rules. Why is the project necessary?
\begin{itemize}
    \item The second-edition rules do not fully incorporate previous errata, in particular Felix Hack’s errata published by JD Webster in {\APJ} 36, and have certain errors.
    \item Where they incorporate errata, it is often inserted literally rather than changing the text to include the sense of the errata.
    \item There is no traceability. Changes have been made, but there is no way to see what changed or why.
    \item There are no internal hyperlinks for cross references.
    \item Many of the diagrams have not been fixed.
    \item I think it would be helpful to incorporate diagrams and tables into the text (and also have them on the play-aid sheets). With printed rules, this separation reduced costs and, to a large degree, was more convenient. When viewed on the screen, these considerations are different.
    \item The text needs copy editing and, in certain places, could benefit from being rewritten or perhaps reorganized.
    \item There are few examples or explanations. Indeed, there is little effort to distinguish examples or explanations in the original text.
    \item The text still refers to pilots, crew, and players using masculine pronouns. That was understandable when the first-edition rules were written, but nevertheless it needs correcting.
\end{itemize}

\section{Versions}

I use “versions” to distinguish my efforts from the other “editions.” There will be several versions of the rules and play aids, each based on the previous one. 

\begin{itemize}
    \item Version 1A will correspond to the original first-edition rules without the official errata.
    \item Version 1B will incorporate the errata from authoritative sources, including the pubished errata, statements by J.D.\ Webster, and Felix Hack’s article in {\APJ} 36.
    \item Version 1C will have corrected and redrawn diagrams and will incorporate these into the body of the text in addition to the play aid sheets.
%    \item Version 1E will have additional minor changes from non-authoritative sources.
    \item Version 2A will incorporate second-edition changes from {\APJ}, email messages from JD Webster, and  Malcolm Pipe’s second-edition rules. Not all changes will be adopted since some seem to have issues.
    \item Version 2B will have additional non-authoritative changes.
    \item Version 2Z aims to clarify the rules while maintaining the meaning. Certain rules may be rewritten for clarity and conciseness while preserving their meaning. Parts of the text may be reorganized.
\end{itemize}

Having multiple versions allow us to more easily review changes. For example, the changes between 1A and 1B do change the rules according to authoritative sources, but the changes between 1B and 1C are mainly concerned with improving the figures.

The versions up to and including 2A will include only minimal changes to the text, and these changes will largely be drawn from authoritative sources. 

% I’m pretty sure about the nature of these versions. I’m less sure about 2Z, and I am considering designating version 2Z as version 3 since it will include significantly rewritten and reorganized text.

\section{Products}

I will produce two PDF files for each version, on showing the changes with respect to the previous version and one clean version.

%(Other versions can be generated from the sources.) Furthermore, there will be two options for version 2Z: with comments and examples and without. The latter option will have the advantage of conciseness. 

%There will also be a PDF file showing all of the changes and their origins. This will give traceability.

\section{Rules Changes}

This section describes the rules changes between different versions.

\subsection{Version 1B}

Version 1B is an update to the text of version 1A according to authoritative sources. The are changes to the sense of some rules compared to version 1A, but the changes to the text are minimal. It incorporates the following errata:

\begin{itemize}
    \itemtag{1B-tsoh-errata} All changes in the single sheet of errata included with TSOH.
    \itemtag{1B-apj-22-damage-tables} The optional advanced damage tables from APJ 22.
    \itemtag{1B-apj-23-errata} All changes in the four sheets of errata published in APJ 23.
    \itemtag{1B-apj-24-play-aids} All changes in the revised play aids published in APJ 24. 
    \itemtag{1B-apj-36-errata} All changes in Felix Hack’s list of errata published in APJ 36, except for the change to sustained turning being assessed per 30 degrees of facing change, which was later clarified as a second-edition change.
    \itemtag{1B-apj-XX-qa} Changes and clarifications from JD Webster’s Q\&A articles in {\APJ} 20, 21, 22, 23, 24, 25, 26, 27, 28, 29, 30, 34, 35, 36, 37, 38, and 39.
    \itemtag{1B-tailing} A correction to the rule on tailing following an email message by JD Webster to the airpower group. 
    \itemtag{1B-credits} Additions to the credits.
\end{itemize}

\subsection{Version 1C}

Version 1C incorporates the following changes. There should be no changes to the sense of the rules compared to version 1B.

\begin{itemize}
    \itemtag{1C-figures} All figures have been redrawn. 
    
    Figures 2 (the game map), 6 (level flight), 11 (angle-off on a hexside) have been corrected according to the comments in corresponding the errata incorporated in 1B. Figure 12 (SSGT) now shows a range of 6 hexes. Figure 12 (Genie scatter) now shows the scatter for a Genie centered on a hex and pointing to a hex corner.
    
    All figures have been incorporated into the body of the rules. Certain figures also continue to appear in the play aids.
    
    Reference to figures now follow the pattern “Figure 1” rather than “Angle-Off Diagrams”.

    \itemtag{1C-tables} All tables have been have been incorporated into the body of the rules and also appear in the play aids. 
    
    Reference to tables now follow the pattern “Table 1” rather than “IRM Seeker Head Table”.

    \itemtag{1C-cover} The rule have a cover featuring a USN photograph of an A-7A. This appears to have been the inspiration for the line drawing on the cover of the {\TSOH} rules.

    \itemtag{1C-credits} Additions to the credits.

\end{itemize}

\subsection{Version 2A}

Version 2A incorporates the following changes. There are changes to the sense of the rules compared to version 1C, but the changes to the text are minimal.

\begin{itemize}

    \itemtag{2A-adc} The first-edition F-4B/C ADC is replaced with the second-edition F-4F ADC from {\APJ}~44.

    \itemtag{2A-idle} An aircraft that selects idle power no longer has its start speed reduced. Instead, the aircraft incurs DPs for normal power plus the additional DPs listed in the ADC. This change is taken from the Origins rules in {\APJ}~39, 41, 44, and 53 and the v2.4 rules. 

    \itemtag{2A-spbr} Similarly, an aircraft that uses speedbrakes no longer loses FPs. Instead, the aircraft incurs DPs up to the value shown in the ADC. If the aircraft is supersonic, the maximum is increased by 1 DP. This change is taken from the Origins rules in {\APJ}~39, 41, 44, and 53 and the v2.4 rules.

    \itemtag{2A-fp-to-dp} The conversion factor from FPs to DPs for the new idle power and speedbrake rules is 2. This change appears in the Origins rules in {\APJ}~41 and the v2.4 rules.

    \itemtag{2A-supersonic-flame-out} An aircraft that selects idle or military power at supersonic speeds automatically and immediately suffers a flame-out. This change appears the 1995 GEnie post by JD Webster and the v2.4 rules.

    \itemtag{2A-cruise} The cruise speed in the ADC is for CL configuration and is 0.5 less for 1/2 configuration and 1.0 less for DT configuration. This change was initially suggested by Guy Acala in {\APJ}~24, and appears in the Origins rules in {\APJ}~44 and 53 and the v2.4 rules.

    \itemtag{2A-snap} Aircraft and missiles can no longer execute snap turns. This change appears in the 1995 GEnie post by JD Webster, the Origins rules in {\APJ}~41, and the v2.4 rules.

    \itemtag{2A-sustained} Sustained turning penalties are now assessed per 30 degrees of facing change for second and subsequent facing changes. This change was included by Felix Hack in the errata in {\APJ}~36 but was clarified as applying to 2nd edition rules by JD Webster in the {\APJ} 38 QA. It was included in the Origins rules in {\APJ}~39, 44, and 53 and the v2.4 rules.

    For aircraft with low and high bleed rates, the sustained turning penalties are 0.5 and 1.5 DP. This change was included in the “Props against Jets” article in {\APJ}~32 and the Origins rules {\APJ}~39, 44, and 53 and the v2.4 rules.

    \itemtag{2A-zoom-climbs} The deceleration for zoom climbs (and sustained climbs that gain more than the CC) is now always 1.0 DP per level. See the 1995 GEnie post by JD Webster, the Origins rules {in \APJ}~44 and 53, and the v2.4 rules.

    \itemtag{2A-super-climbs} Aircraft with a CC of 6.0 or more in a ZC or SC can use one of their VFPs to climb three levels. This change appeared in the 1995 GEnie post by JD Webster and the v2.4 rules.

    \itemtag{2A-steep-dives} The acceleration for steep dives is now always 1.0 AP per level. This change appeared in the 1995 GEnie post by JD Webster, the Origins rules in {\APJ}~44 and 53, and the v2.4 rules.

    \itemtag{2A-unloaded-dives} The unloaded dives rule is changed to follow the Origins rules in {\APJ}~44 and 53 and the v2.4 rules.
    
    The proposed wording is not completely clear, and in an attempt to state the rule more clearly, I have added that the VFPs are unloaded, too. 
    
    Two things cause me to pause. First, with this rule, unloaded dives do not give any advantage over steep dives in terms of acceleration of horizontal distance. Indeed, a steep dive is better since one VFP can lose two altitude levels, whereas in an unloaded dive, each VFP can only lose one altitude level. Second, the rule appears to state that unloaded dives are a form of level flight, in which case other rules need modifying (e.g., an aircraft cannot recover from a vertical dive into an unloaded dive, and unloaded dives do not count as diving for barrel rolls). If unloaded dives are a form of level flight, treating them separately from diving flight would probably be better.

    \itemtag{2A-missile-sighting} The sighting rules are changed so that attempts to sight missiles come before attempts to sight aircraft, and the modifiers are changed. This change appeared in the Origins rules in {\APJ}~44 and 53.
    
    The Origins rules mention that missiles launched from sighted aircraft are no longer sighted, but I can’t find this in the original rules.

    \itemtag{2A-advantage} An aircraft in a vertical climb may disadvantage aircraft at the same level or lower. This change appeared in the Origins rules in {\APJ}~39, 41, 44, and 53 and the v2.4 rules.

    \itemtag{2A-roll-preparatory-fps} Lag and displacement rolls now require preparatory HFPs equal to {\onethird} of the aircraft's speed (rounded down). This change appeared in the 1995 GEnie post by JD Webster and the v2.4 rules.
    
    \itemtag{2A-missile-launches} The missile launch rules are changed so that if the launch roll fails by exactly one and is not an unmodified ten, the missile fails to launch but remains on the rail and can be used in the future. This change appeared in the Origins rules in \APJ~39, 31, 44 and 53.

    \itemtag{2A-missile-flight} Air-to-air missiles always move before their target. This change appeared in the Origins rules in \APJ~44 and 53.

    \itemtag{2A-missile-attacks} Missile attacks occur when the missile has the same position and altitude as the target or when it has the same altitude, at least one FP left in its proportional move, and the target is one of the seven positions immediately in front of the missile. This change appeared in the Origins rules in \APJ~44 and 53.

    \itemtag{2A-credits} Additions to the credits.

\end{itemize}

The rule for pylon drag (suggested originally by Mark Bovankovich in {\APJ}~18 and adopted in the Origins rules in {\APJ}~41, 44, and 53) is not incorporated simply because applying it uniformly would require extensive research and modifications to almost all ADCs.

\subsection{Version 2B}

Version 2B contains unauthoritaive minor changes.

\begin{itemize}
    \itemtag{2B-directions} ENE, ESE, WSW, and WNW are used in preference to NE, SE, SW, and NW. The original directions fall evenly between two 30{\deg} facings, whereas the new  ones are unambiguously closer to one.

    \itemtag{2B-stacking} Collisions are only possible if the aircraft are at the same altitude. They are also possible if four or more aircraft are stacked at the same altitude, even if they are in a close formation.

    \itemtag{2B-collisions} Potential collisions from head-on attacks are resolved after the attack. Other collisions are resolved at the end of the flight phase.

    \itemtag{2B-range} I have written a rule to clarify how to calculate range when counters are on hex sides. Moving from the hex side to either of the two adjacent hexes or four adjacent hex sides counts as half a hex. Determine the horizontal range including half-hexes, then round down. This gives the correct result, for example, when two counters are on hex sides on the oposite sides of a hex.

    \itemtag{2B-ground-fac-marking} The text of the original rules states that ground FACs mark targets in the AAA planning phase. This phase does not exist. The extended sequence of play indicates that this occurs during the visual sighting phase. I have changed the text of the rules to match this.

    \itemtag{2B-ra-speed-limits} RA aircraft at their maximum or dive speed, as appropriate, may only carry forward 1.0 APs.

    \itemtag{2B-military} Aircraft can select 0.0 APs when using military power. This gives no acceleration, but allows them to maintain a steady speed above cruise speed
    
    \itemtag{2B-stall} Aircraft recover from a stall to a wings-level attitude.

    \itemtag{2B-departures} Aircraft lose any carried APs or DPs when they recover from departure.

    \itemtag{2B-low} The low altitude band extends down to level 0, to account for aircraft at level 0 in TFF.
\end{itemize}

\section{Language Changes}

\subsection{Copy Editing}

I will apply the standard rules for capitalization, punctuation, and compound adjectives (e.g., “air-to-air attack”). I will use the Oxford comma.

\subsection{Uniformity}

I will use “AP” and “DP” to refer to acceleration and deceleration points. This follows the uniform use of FP, HFP, and VPF.

\subsection{Replacing Ambiguous Terms}

\begin{itemize}
    \item \itemparagraph{Turn.} The original rules use “turn” to refer to both a game turn and turning flight. I will use “game turn” to refer to a game turn and “turn” to refer to turning flight.

    \item \itemparagraph{Within.} The original rules use “within” regarding ranges and other quantities. This is ambiguous. 
    
    For example, an aircraft must be “within four hexes” to conduct a rocket attack. Is this inclusive (i.e., “at a range of no more than four hexes” or $\textrm{range}\le4$) or exclusive (i.e., “at a range of less than four hexes” or $\textrm{range}<4$)? In the case of rocket attacks, the rocketry table resolves the ambiguity, which shows hit rolls for ranges of 1, 2, 3, and 4 hexes. Thus, here, within is inclusive. It is also apparently inclusive when used in the context of look-down limitations, which are expressed as “within four altitude levels” and “within 5 to 10 levels”.

    In the original rules, within is used in the context of:
    \begin{itemize}
    \item Aircraft horizontal and vertical separations in tactical formations.
    \item AP limits in military and afterburner power.
    \item The maximum range for rocket attacks.
    \item The maximum range for SSGT.
    \item The maximum range for visual sighting.
    \item When considering distances to determine if a line of sight is blocked by terrain.
    \item The maximum horizontal range to gain advantage.
    \item The relative facing of a tailed and tailing aircraft.
    \item The maximum range for defensive preemptions.
    \item The minimum and maximum ranges for missile launch envelopes.
    \item The maximum radar range.
    \item The altitude ranges for look-down limitations.
    \item The altitude difference for ground clutter for BRM/RHM/AHM.
    \item The range from the target at which an AIM-26A can be detonated.
    \item The altitude difference limits for jamming cell formations.
    \item Sun clutter.
    \item Parachute flares.
    \end{itemize}
    In all of these cases, I believe the use is inclusive.

\end{itemize}


\subsection{Gender-Neutral Language}
 
Many of the terms used in the rules are already gender-neutral, and no change is required to these. However, we need to make the following substitutions: 
\begin{itemize}
    \item “they/them/their” for “he/him/his”
    \item “crewmember” for “crewman”
    \item “winger” for “wingman” (cf. section “leader”)
\end{itemize}

\section{Process}

I will typeset the rules using LaTeX on Overleaf and track the source text using GitHub.

The source text for version 1A will be generated by taking Malcolm Pipe’s second-edition text, manually comparing it to the first-edition rules, and making any changes necessary to obtain agreement. (I believe permission to do this is implicit in his statement, “Edit as desired after download.” on 2021-02-05 to the airpower.io group.)

I will use custom LaTeX commands to introduce changes into the source text. These changes will be tagged and the tags described in this document. This gives traceability.

\section{Design}

The {\AirSup} rules were published in GDW’s house style. This used a san serif font, which favors compactness over readability. The first-edition rules followed this style to a large degree.

I want the rules to be readable and beautiful, but I also wish to preserve some of the design heritage of the earlier rules. Therefore, I have maintained a layout with a two-column design, with a line separating the columns, and adapted the header from the GDW editions and the section format from the first-edition rules. However, I have replaced the san serif font with a serif font designed for readability.

After considering several options, I have chosen the \href{https://www.gust.org.pl/projects/e-foundry/tex-gyre/schola}{Schola} font, based on the URW Century Schoolbook font and adapted for LaTeX by the GUST foundry as part of the TeX Gyre project. Schola has a version for math, which is useful for text like “$1 ≤ \CC ≤ 2$”. Of the other TeX Gyre fonts, the Palatino version would also be suitable, but Palatino is heavily used in {\itshape Birds of Prey}, and I wish to maintain a clear distinction.

\section{Rewriting and Reorganization}

This section collects my ideas for rewriting and reorganization.

\begin{itemize}
    \item Move the rules on formations to their own section.
    \item Move the modifications to the rules for aircraft with properties (e.g., LRR, HPR) to their own appendix.
    \item Move the rule on loss of thrust with altitude to the section on speed.
    \item  I think the following sections in particular could benefit from rewriting:
    \begin{enumerate}
        \item Recovery periods.
        \item Visual sighting.
        \item Electronic warefare.
    \end{enumerate}
    
\end{itemize}

\end{document}

